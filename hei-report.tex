% Options for packages loaded elsewhere
\PassOptionsToPackage{unicode}{hyperref}
\PassOptionsToPackage{hyphens}{url}
\PassOptionsToPackage{dvipsnames,svgnames,x11names}{xcolor}
%
\documentclass[
  letterpaper,
  DIV=11,
  numbers=noendperiod]{scrartcl}

\usepackage{amsmath,amssymb}
\usepackage{iftex}
\ifPDFTeX
  \usepackage[T1]{fontenc}
  \usepackage[utf8]{inputenc}
  \usepackage{textcomp} % provide euro and other symbols
\else % if luatex or xetex
  \usepackage{unicode-math}
  \defaultfontfeatures{Scale=MatchLowercase}
  \defaultfontfeatures[\rmfamily]{Ligatures=TeX,Scale=1}
\fi
\usepackage{lmodern}
\ifPDFTeX\else  
    % xetex/luatex font selection
\fi
% Use upquote if available, for straight quotes in verbatim environments
\IfFileExists{upquote.sty}{\usepackage{upquote}}{}
\IfFileExists{microtype.sty}{% use microtype if available
  \usepackage[]{microtype}
  \UseMicrotypeSet[protrusion]{basicmath} % disable protrusion for tt fonts
}{}
\makeatletter
\@ifundefined{KOMAClassName}{% if non-KOMA class
  \IfFileExists{parskip.sty}{%
    \usepackage{parskip}
  }{% else
    \setlength{\parindent}{0pt}
    \setlength{\parskip}{6pt plus 2pt minus 1pt}}
}{% if KOMA class
  \KOMAoptions{parskip=half}}
\makeatother
\usepackage{xcolor}
\usepackage[right=1in,left=1in]{geometry}
\setlength{\emergencystretch}{3em} % prevent overfull lines
\setcounter{secnumdepth}{3}
% Make \paragraph and \subparagraph free-standing
\ifx\paragraph\undefined\else
  \let\oldparagraph\paragraph
  \renewcommand{\paragraph}[1]{\oldparagraph{#1}\mbox{}}
\fi
\ifx\subparagraph\undefined\else
  \let\oldsubparagraph\subparagraph
  \renewcommand{\subparagraph}[1]{\oldsubparagraph{#1}\mbox{}}
\fi


\providecommand{\tightlist}{%
  \setlength{\itemsep}{0pt}\setlength{\parskip}{0pt}}\usepackage{longtable,booktabs,array}
\usepackage{calc} % for calculating minipage widths
% Correct order of tables after \paragraph or \subparagraph
\usepackage{etoolbox}
\makeatletter
\patchcmd\longtable{\par}{\if@noskipsec\mbox{}\fi\par}{}{}
\makeatother
% Allow footnotes in longtable head/foot
\IfFileExists{footnotehyper.sty}{\usepackage{footnotehyper}}{\usepackage{footnote}}
\makesavenoteenv{longtable}
\usepackage{graphicx}
\makeatletter
\def\maxwidth{\ifdim\Gin@nat@width>\linewidth\linewidth\else\Gin@nat@width\fi}
\def\maxheight{\ifdim\Gin@nat@height>\textheight\textheight\else\Gin@nat@height\fi}
\makeatother
% Scale images if necessary, so that they will not overflow the page
% margins by default, and it is still possible to overwrite the defaults
% using explicit options in \includegraphics[width, height, ...]{}
\setkeys{Gin}{width=\maxwidth,height=\maxheight,keepaspectratio}
% Set default figure placement to htbp
\makeatletter
\def\fps@figure{htbp}
\makeatother
\newlength{\cslhangindent}
\setlength{\cslhangindent}{1.5em}
\newlength{\csllabelwidth}
\setlength{\csllabelwidth}{3em}
\newlength{\cslentryspacingunit} % times entry-spacing
\setlength{\cslentryspacingunit}{\parskip}
\newenvironment{CSLReferences}[2] % #1 hanging-ident, #2 entry spacing
 {% don't indent paragraphs
  \setlength{\parindent}{0pt}
  % turn on hanging indent if param 1 is 1
  \ifodd #1
  \let\oldpar\par
  \def\par{\hangindent=\cslhangindent\oldpar}
  \fi
  % set entry spacing
  \setlength{\parskip}{#2\cslentryspacingunit}
 }%
 {}
\usepackage{calc}
\newcommand{\CSLBlock}[1]{#1\hfill\break}
\newcommand{\CSLLeftMargin}[1]{\parbox[t]{\csllabelwidth}{#1}}
\newcommand{\CSLRightInline}[1]{\parbox[t]{\linewidth - \csllabelwidth}{#1}\break}
\newcommand{\CSLIndent}[1]{\hspace{\cslhangindent}#1}

\usepackage{booktabs}
\usepackage{longtable}
\usepackage{array}
\usepackage{multirow}
\usepackage{wrapfig}
\usepackage{float}
\usepackage{colortbl}
\usepackage{pdflscape}
\usepackage{tabu}
\usepackage{threeparttable}
\usepackage{threeparttablex}
\usepackage[normalem]{ulem}
\usepackage{makecell}
\usepackage{xcolor}
\usepackage{colortbl}
\makeatletter
\renewenvironment{table}%
  {\renewcommand\familydefault\sfdefault
   \@float{table}}
  {\end@float}
\makeatother
\KOMAoption{captions}{tableheading}
\makeatletter
\makeatother
\makeatletter
\makeatother
\makeatletter
\@ifpackageloaded{caption}{}{\usepackage{caption}}
\AtBeginDocument{%
\ifdefined\contentsname
  \renewcommand*\contentsname{Table of contents}
\else
  \newcommand\contentsname{Table of contents}
\fi
\ifdefined\listfigurename
  \renewcommand*\listfigurename{List of Figures}
\else
  \newcommand\listfigurename{List of Figures}
\fi
\ifdefined\listtablename
  \renewcommand*\listtablename{List of Tables}
\else
  \newcommand\listtablename{List of Tables}
\fi
\ifdefined\figurename
  \renewcommand*\figurename{Figure}
\else
  \newcommand\figurename{Figure}
\fi
\ifdefined\tablename
  \renewcommand*\tablename{Table}
\else
  \newcommand\tablename{Table}
\fi
}
\@ifpackageloaded{float}{}{\usepackage{float}}
\floatstyle{ruled}
\@ifundefined{c@chapter}{\newfloat{codelisting}{h}{lop}}{\newfloat{codelisting}{h}{lop}[chapter]}
\floatname{codelisting}{Listing}
\newcommand*\listoflistings{\listof{codelisting}{List of Listings}}
\makeatother
\makeatletter
\@ifpackageloaded{caption}{}{\usepackage{caption}}
\@ifpackageloaded{subcaption}{}{\usepackage{subcaption}}
\makeatother
\makeatletter
\@ifpackageloaded{tcolorbox}{}{\usepackage[skins,breakable]{tcolorbox}}
\makeatother
\makeatletter
\@ifundefined{shadecolor}{\definecolor{shadecolor}{rgb}{.97, .97, .97}}
\makeatother
\makeatletter
\makeatother
\makeatletter
\makeatother
\ifLuaTeX
  \usepackage{selnolig}  % disable illegal ligatures
\fi
\IfFileExists{bookmark.sty}{\usepackage{bookmark}}{\usepackage{hyperref}}
\IfFileExists{xurl.sty}{\usepackage{xurl}}{} % add URL line breaks if available
\urlstyle{same} % disable monospaced font for URLs
\hypersetup{
  pdftitle={How Do Household Energy Transitions Work?},
  pdfauthor={Jill Baumgartner (Co-PI); Sam Harper (Co-PI); On behalf of the Beijing Household Energy Transitions Team},
  colorlinks=true,
  linkcolor={blue},
  filecolor={Maroon},
  citecolor={Blue},
  urlcolor={Blue},
  pdfcreator={LaTeX via pandoc}}

\title{How Do Household Energy Transitions Work?}
\author{Jill Baumgartner (Co-PI) \and Sam Harper (Co-PI) \and On behalf
of the Beijing Household Energy Transitions Team}
\date{2024-04-23}

\begin{document}
\maketitle
\ifdefined\Shaded\renewenvironment{Shaded}{\begin{tcolorbox}[sharp corners, boxrule=0pt, borderline west={3pt}{0pt}{shadecolor}, interior hidden, breakable, enhanced, frame hidden]}{\end{tcolorbox}}\fi

\renewcommand*\contentsname{Table of contents}
{
\hypersetup{linkcolor=}
\setcounter{tocdepth}{3}
\tableofcontents
}
\hypertarget{abstract}{%
\subsection*{Abstract}\label{abstract}}

\hypertarget{introduction}{%
\subsubsection*{Introduction}\label{introduction}}

\hypertarget{methods}{%
\subsubsection*{Methods}\label{methods}}

\hypertarget{results}{%
\subsubsection*{Results}\label{results}}

\hypertarget{conclusions}{%
\subsubsection*{Conclusions}\label{conclusions}}

\hypertarget{introduction-1}{%
\section{Introduction}\label{introduction-1}}

China is deploying an ambitious policy to transition up to 70\% of
households in northern China to clean space heating, including a
large-scale roll out across rural and peri-urban Beijing, referred to in
this document as China's Coal Ban and Heat Pump (CBHP) subsidy policy.
To meet this target the Beijing municipal government announced a
two-pronged program that designates coal-restricted areas and
simultaneously offers subsidies to night-time electricity rates and for
the purchase and installation of electric-powered, air-source heat pumps
to replace traditional coal-heating stoves. The policy was piloted in
2015 and, starting in 2016, was rolled out on a village-by-village
basis. The variability in when the policy is applied to each village
allows us to treat the roll-out of the program as a quasi-randomized
intervention. Households may also be differentially affected by this
program due to factors such as financial constraints, preferences and
social capital, and there is uncertainty about whether and how this
intervention may affect indoor and outdoor air pollution, as well as
heating behaviors and health outcomes.

\hypertarget{background}{%
\section{Background}\label{background}}

\hypertarget{context-for-the-policy}{%
\subsection{Context for the policy}\label{context-for-the-policy}}

Beijing has a temperate continental monsoon climate characterized by
cold, dry winters and hot, humid summers. Access to central heating is
limited to urban areas and households in most rural and peri-urban areas
of Beijing have historically heated their homes using coal heaters and
biomass-fueled \emph{kangs} (a traditional Chinese energy technology
that integrates at least four different home functions including
cooking, a bed for sleeping, space heating, and home ventilation).
Household coal burning was a major contributor to indoor and outdoor air
pollution in northern China, especially in winter. In 2015, over 100
million rural households consumed around 200 million tons of coal to
meet over 80\% of northern China's residential space heating demand
(Dispersed Coal Management Research Group 2023). At that time, household
coal-fuelled heaters burned approximately half of the over 400 million
tons of coal used for space heating (Group 2016) and contributed to
roughly 30\% of northern China's wintertime air pollution. In 2013,
exposure to ambient fine particulate matter from coal combustion - from
industry, electricity, and domestic sources - was the largest estimated
contributor to population exposure to PM\textsubscript{2.5} and
contributed to an estimated 366,000 premature deaths annually in China
(Group 2016).

Banning residential coal burning and replacing household coal stoves
with clean heating alternatives was considered a potentially impactful
intervention to improve rural development, reduce PM\textsubscript{2.5}
across the region, and mitigate air pollution-related health impacts. A
number of clean heating options, including electric heat pumps, gas
heaters, pelletized biomass stoves, and electric resistance heaters with
thermal storage, were widely promoted by the Chinese government
(Dispersed Coal Management Research Group 2023). By 2021, over 36
million households in northern China were treated by the policy and an
estimated 21 million additional households are expected to be treated by
2025. Whether this large-scale energy policy yielded air quality and
health benefits remains a critical and unresolved question.

\hypertarget{prior-evidence-on-household-energy-interventions-and-air-pollution}{%
\subsection{Prior evidence on household energy interventions and air
pollution}\label{prior-evidence-on-household-energy-interventions-and-air-pollution}}

Household energy interventions, mostly cooking-related, that replace
solid fuel stoves with cleaner-burning alternatives have been
implemented and studied extensively in countries including China over
the past several decades. While their introduction of more efficient
residential stoves and fuels is expected to reduce air pollution
emissions and subsequent exposures, there is still no consensus about
their effectiveness in achieving health-relevant air pollution
reductions (Quansah et al. 2017). In particular, the effectiveness of
large-scale household energy programs like China's Coal Ban and Heat
Pump (CBHP) subsidy policy have been rarely empirically investigated,
especially at sub-city spatial resolution. In Ireland, county-level
residential coal bans in the 1990s were associated with 40-70\%
decreases in black smoke concentrations in ban-affected areas (Dockery
et al. 2013). In Australia, a wood-burning stove exchange lowered daily
wintertime PM\textsubscript{10} from 44 to 27 µg/m\textsuperscript{3}
(Johnston et al. 2013), and clean energy policies in New Zealand were
associated with 11-36\% reductions in winter PM\textsubscript{10} (Scott
and Scarrott 2011). The few evaluations of the CBHP policy reported
small decreases in outdoor PM\textsubscript{2.5} (-7 to -2.4
µg/m\textsuperscript{3}) in municipalities or prefectures in the policy
compared with neighboring areas not affected by the policy (Niu et al.
2024; Song et al. 2023; Tan et al. 2023; Yu et al. 2021), and a recent
modeling study estimated 36\% lower personal exposure to
PM\textsubscript{2.5} based on household-reported changes in fuel use
(Meng et al. 2023). However, none of these studies included field-based
measurements of air pollution or personal exposures, which are known to
differ considerably from modeled estimates (Thompson et al. 2019), and
few accounted for secular changes in air quality over time, limiting any
conclusions about the causal effect of the CBHP policy on air quality.

\hypertarget{prior-evidence-on-clean-energy-interventions-and-cardiovascular-outcomes}{%
\subsection{Prior evidence on clean energy interventions and
cardiovascular
outcomes}\label{prior-evidence-on-clean-energy-interventions-and-cardiovascular-outcomes}}

Most previous health assessments of household energy interventions have
focused on cookstoves instead of heating, and randomized trials of less
polluting cookstoves generally indicate a cardiovascular benefit. In
older Guatemalan women, a chimney stove intervention lowered exposure to
air pollution and reduced the occurrence of nonspecific ST-segment
depression (McCracken et al. 2011). Randomized trials in Guatemala,
Nigeria, and Ghana also showed reductions in blood pressure (systolic
range: −3.7 to −1.3 mmHg) in women assigned to gas, ethanol, or improved
combustion biomass stoves. In contrast, a recent multi-country
randomized trial found little evidence for a protective effect of gas
stoves on gestational blood pressure (Ye et al. 2022) despite large
reductions (\textasciitilde66\% lower) in exposure to
PM\textsubscript{2.5}(Johnson et al. 2022).

The few population-based evaluations of large-scale residential energy
policies also suggest a cardio-respiratory benefit of clean energy
transition. Residential wood-burning bans were associated with
reductions in cardiovascular hospitalizations (-7\%) in California (Yap
and Garcia 2015) and with reduced cardiovascular (-17.9\%) and
respiratory (−22.8\%) mortality in Australia (Johnston et al. 2013),
though neither study fully controlled for possible secular improvements
in health that were unrelated to the policy. Most relevant to our study
are two quasi-experimental assessments of coal replacement policies. In
Ireland, reductions in respiratory not but cardiovascular mortality were
observed following a coal ban (Dockery et al. 2013). A multi-city study
of Chinese adults in cities where the CBHP policy was piloted compared
with adults in cities not in the pilot observed small decreases in
chronic lung diseases (-3.0 to -1.1\%) but no change in
physician-diagnosed cardiovascular diseases, potentially due to the
short (one-year) post-policy evaluation period or confounding by other
unmeasured city-wide air quality or health-related policies (Wen et al.
2023).

Though household air pollution is a well-established health risk factor,
which energy interventions can reduce air pollution exposures, improve
health, and are scalable and sustainable remains a critical and
unanswered question. In a recent Official American Thoracic Society
Statement, for example, the committee could not reach a consensus on
whether previously studied household energy interventions (including
gas, ethanol, solar, and improved biomass stoves) improved health
outcomes that included blood pressure (with 55\% saying no and 45\%
saying yes) (Harrison et al. Approved February 2024).

\hypertarget{assessing-dynamic-and-heterogeneous-treatment-effects}{%
\subsection{Assessing dynamic and heterogeneous treatment
effects}\label{assessing-dynamic-and-heterogeneous-treatment-effects}}

Since 2015, thousands of villages across Beijing and northern China
entered the CBHP policy prior to the start of the heating season each
year. Given the many behavioral, social, or economic factors that might
affect both new heater use and coal stove suspension (e.g., energy
prices and availability, wintertime temperature, COVID-19 pandemic, user
preferences), it is possible that the effect of the policy on air
pollution and health may be dynamic over time and/or heterogeneous
across treatment cohorts. Thus, it may be important to study both the
overall and group-time effects of the policy.

\hypertarget{evaluating-the-mechanisms-through-which-policies-may-affect-health-outcomes.}{%
\subsection{Evaluating the mechanisms through which policies may affect
health
outcomes.}\label{evaluating-the-mechanisms-through-which-policies-may-affect-health-outcomes.}}

With several exceptions (Alexander et al. 2018; Gould et al. 2023;
McCracken et al. 2007; McCracken et al. 2011), decades of household
energy intervention studies showing limited or no health benefit
demonstrate the complexity of evaluating interventions on exposures such
as cooking or space heating that are central to daily life (Ezzati and
Baumgartner 2017) (Harrison et al. Approved February 2024). Energy
interventions and policies, particularly those implemented at the
household- or village-scales, can produce multiple behavioral,
environmental, and health-related changes, making it important to
investigate the mechanisms through which such policies exert their
health impacts (Dominici et al. 2014). The health benefits achievable
with transition from traditional coal stoves to a new electric home
heating system, for example, may be influenced by factors including
outdoor air quality (Lai 2019), the desirability and usage patterns of
new and traditional stoves (Ezzati and Baumgartner 2017), indoor
temperature (Lewington et al. 2012), or behaviors including physical
activity (Lindemann et al. 2017). Only recently were these mediating
factors considered in health assessments of household energy
interventions, and rarely in a comprehensive or formalized way
(Rosenthal et al. 2018). Understanding these mechanisms can provide
valuable insights into the success (or failure) of clean energy programs
or policies like the CBHP policy in meeting their air quality and health
goals, and may answer questions that can inform the design of more
effective future energy interventions (Harrison et al. Approved February
2024). For example, is there successful uptake of the intervention or
policy? Does the policy lead to heating behavior changes that result in
colder homes and thus offsets any cardiovascular-enhancing effects of
improved air quality? Answers to these questions are facilitated by the
analysis of mediating pathways.

\hypertarget{specific-aims-and-overarching-approach}{%
\section{Specific Aims and Overarching
Approach}\label{specific-aims-and-overarching-approach}}

This study used three data collection campaigns in winter 2018/19,
winter 2019/20, and winter 2021/22, as well as a partial campaign in
winter 2020/21 to advance the following aims:

\begin{enumerate}
\def\labelenumi{\arabic{enumi}.}
\item
  Estimate how much of the CBHP policy's overall effect on health,
  including respiratory symptoms and cardiovascular outcomes (blood
  pressure, blood inflammatory and oxidative stress markers), can be
  attributed to its impact on changes in PM\textsubscript{2.5};
\item
  Quantify the impact of the policy on outdoor air quality and personal
  air pollution exposures, and specifically the source contribution from
  household coal burning;
\item
  Quantify the contribution of changes in the chemical composition of
  PM\textsubscript{2.5} from different sources to the overall effect on
  health outcomes.
\end{enumerate}

\hypertarget{study-design-and-methods}{%
\section{Study Design and Methods}\label{study-design-and-methods}}

\hypertarget{study-area}{%
\subsection{Study area}\label{study-area}}

Beijing is the capital of China (pop. 21.9 million in 2020) and covers a
large geographic area (\textasciitilde16,000 km2) that includes a highly
developed and densely-populated urban core that is surrounded by several
satellite towns and peri-urban and rural villages in the periphery.
Beijing winters begin in early November and tend to be cold, dry, and
windy with the lowest temperatures mostly often occurring in January
(-3°C, on average), thus requiring space heating (An et al. 2021). Most
urban areas of Beijing are connected to a central heating grid that
supplies home heating from central locations, whereas rural and many
peri-urban areas have historically relied on individual space heating
units that, prior to 2015, were largely fueled by unprocessed raw coal
(Duan et al. 2014).

\hypertarget{location-and-participant-recruitment-and-enrolment}{%
\subsection{Location and participant recruitment and
enrolment}\label{location-and-participant-recruitment-and-enrolment}}

Between December 2018 and January 2019 we recruited 50 villages across 4
administrative districts (Fangshan, Huairou, Mentougou, and Miyun) in
the Beijing municipality in northern China. The villages predominately
used coal for heating at the time of enrollment and were eligible for
and not currently participating in the CBHP policy. Roughly half of the
villages were expected to enter into the policy during our study
(Figure~\ref{fig-cbhp-map}). We used local guides in each village to
help determine a roster of households that were not vacant during the
winter months, from which we randomly selected households to recruit for
participation.

\begin{figure}[H]

{\centering \includegraphics[width=0.7\textwidth,height=\textheight]{images/policy-implementation-map.png}

}

\caption{\label{fig-cbhp-map}Map of village implementation of CBHP
policy}

\end{figure}

We recruited approximately 20 households in each village and randomly
selected one eligible person from each household to participate.
Household members were eligible to participate if they were over 40
years old, lived in the study villages, were not planning to move out of
the village in the next year, and were not on current immunotherapy or
treatment with corticosteroids. Research staff introduced the study and
its measurements to an eligible adult in each household and answered any
questions related to the study. In follow-up visits to the study
villages, staff first approached households with participants from an
earlier campaign. If previous participants were not at home or refused
to participate, staff first tried to randomly recruit an eligible
participant from the same household. If there was not another eligible
or willing participant in the household, we randomly selected and
recruited a participant from a new household using the village roster.
All participants provided written informed consent prior to joining the
study. The study protocols were approved by research ethics boards at
Peking University (IRB00001052-18090), Peking Union Medical College
Hospital (HS-3184) and McGill University (A08-E53-18B).

\hypertarget{data-collection-overview}{%
\subsection{Data Collection Overview}\label{data-collection-overview}}

We conducted study measurements over four consecutive winter seasons in
2018-19, 2019-20, 2020-21, and 2021-22 (referred to hereafter as Season
1 {[}S1{]}, S2, S3 and S4, respectively). Field data collection was
conducted by \textasciitilde20 trained staff members who traveled to
participants' homes to conduct tablet-based household and individual
questionnaires, measure participant blood pressure, and distribute
temperature sensors (for measurement of indoor temperature and stove
use) and air pollution monitors in all 50 study villages in S1, S2, and
S4. Anthropometrics (height, weight, and waist circumference),
measurement of airway inflammation, and whole blood samples were
obtained no more than a month later at a village clinic in S1 and S2. In
S3, which was during the height of the pandemic, we limited household
measurements to indoor air quality and sensor-based measurement of
indoor temperature and stove use in 41 villages, including all 17
treated villages and 24 untreated (control) villages, prior to
COVID-related travel restrictions that halted field data collection. In
S4, which also occurred during the COVID-19 pandemic, we returned to
conducting individual-level assessments. However, unlike in S1 and S2,
anthropometric measurements and airway inflammation were assessed in
participant homes rather than clinics to avoid group contact and blood
samples were not collected. Outdoor (community) air pollution was
measured throughout the study period.

\hypertarget{air-pollution}{%
\subsubsection{Air Pollution}\label{air-pollution}}

\hypertarget{outdoor-air-pollution}{%
\paragraph{Outdoor air pollution}\label{outdoor-air-pollution}}

In each village, two sensors for particulate matter air pollution were
set up to measure outdoor (community) PM\textsubscript{2.5} at different
locations in each village. One sensor was placed near the center of the
village, and the other was placed no less than 500m away from the
centrally-located sensor. Sensors were placed at least 1.5m above the
ground and not in a location within sight of a visible point source of
PM\textsubscript{2.5}.

We collected filter-based community PM\textsubscript{2.5} samples to
calibrate the sensor-based PM\textsubscript{2.5} measurements as well as
to conduct analysis of chemical composition for source apportionment.
Ultrasonic Personal Aerosol Samplers (UPAS, Access Sensor Technologies,
Fort Collins, CO, USA) were used to collect filter-based
PM\textsubscript{2.5} samples with a flow rate of 1.0 L/min (Volckens et
al. 2017). Samplers housed 37mm PTFE filters (VWR, 2.0-μm pore size) and
were equipped with a cyclone inlet with a 2.5μm cut point designed to
perform under the sampling flow rate. For community measurements, a UPAS
was co-located with each PM\textsubscript{2.5} sensor in each village in
rotation. Every week, the used filters were removed and replaced with a
new filter. In total, we successfully collected 126, 371, and 289
filter-based, community outdoor PM\textsubscript{2.5} samples in Seasons
1, 2, and 4, respectively. Field blank filters were collected at a rate
of \textasciitilde10\%, subject to the same field conditions as samples.
To support post-sampling determination of organic carbon (OC) and
elemental carbon (EC) fractions of PM\textsubscript{2.5} mass, quartz
filters were co-located with a subset of Teflon filter samples collected
outdoors. Quartz filter-based PM\textsubscript{2.5} samples were
collected using UPAS operating with a flow rate of 1.0 L/min. UPASs
housed 37 mm quartz filters (VWR, 2.0-μm pore size) and were equipped
with a cyclone inlet with a 2.5 μm cut point designed to perform under
the corresponding sampling flow rate. All quartz fiber filters were
baked at 550 °C for a minimum of 8 h to remove organic impurities prior
to sample collection. PM\textsubscript{2.5} samples collected on quartz
filters were analyzed using established thermo-optical methods for
quantifying elemental carbon (EC) and organic carbon (OC) to, then,
calibrate the colorimetric analysis of EC and OC on Teflon filters. In
Season 2, 23 quartz-based outdoor PM\textsubscript{2.5} samples and 3
field blanks were collected. In Season 4, 11 quartz-based outdoor
PM\textsubscript{2.5} samples and 3 field blanks were collected.

For PM\textsubscript{2.5} sensor calibration and quality control, all PM
sensors were co-located with a reference-grade PM\textsubscript{2.5}
instrument (Model 5030 Synchronized Hybrid Ambient Realtime Particulate
(SHARP) Monitor, Thermo Fisher Scientific, United States) on the rooftop
of a building at Peking University campus and/or the Tapered Element
Oscillating Microbalance (TEOM, Thermo Scientific™ 1405 TEOM™) at the
Chinese Academy of Sciences University campus for 7 to 10 days before
and after each field campaign (Figure~\ref{fig-calibration}).
Sensor-measured PM\textsubscript{2.5} concentrations were highly
correlated with those measured by the reference instruments (Spearman
correlation coefficients (rho) \textgreater0.75 in each pre- and
post-calibration).

\begin{figure}[H]

{\centering \includegraphics[width=0.8\textwidth,height=\textheight]{images/sensor-calibration.png}

}

\caption{\label{fig-calibration}Calibration of real-time sensors against
a reference monitor at University of the Chinese Academy of Sciences.}

\end{figure}

\hypertarget{indoor-pm2.5}{%
\paragraph{\texorpdfstring{Indoor
PM\textsubscript{2.5}}{Indoor PM2.5}}\label{indoor-pm2.5}}

In the second and fourth field seasons (i.e., S2 and S4), we randomly
selected six households from the 20 recruited in each village to measure
indoor concentrations of PM\textsubscript{2.5}. In Season 4, we aimed to
monitor indoor PM\textsubscript{2.5} in the same households where we
measured indoor PM\textsubscript{2.5} in Season 2. If a household
dropped out of the project or declined indoor PM\textsubscript{2.5}
monitoring, we then recruited another household already enrolled in this
study to measure indoor PM\textsubscript{2.5}. In total, indoor
measurements were conducted in 300 households in both Season 2 and
Season 4 (Table~\ref{tbl-pm-sample}).

\hypertarget{tbl-pm-sample}{}
\begin{table}
\caption{\label{tbl-pm-sample}Household recruitment for overall and indoor air quality measurements. }\tabularnewline

\centering
\begin{tabular}{lrrrrrr}
\toprule
\multicolumn{1}{c}{ } & \multicolumn{3}{c}{Overall} & \multicolumn{3}{c}{Indoor} \\
\cmidrule(l{3pt}r{3pt}){2-4} \cmidrule(l{3pt}r{3pt}){5-7}
Sample & Season 1 & Season 2 & Season 4 & Season 1 & Season 2 & Season 4\\
\midrule
New recruitment & 977 & 196 & 68 & 0 & 300 & 52\\
Households from Season 1 & \textbackslash{} & 866 & 780 & \textbackslash{} & 0 & 0\\
Households from Season 2 & \textbackslash{} & \textbackslash{} & 162 & \textbackslash{} & \textbackslash{} & 248\\
Total recruitment & 977 & 1062 & 1010 & 0 & 300 & 300\\
\bottomrule
\end{tabular}
\end{table}

Time-resolved indoor PM\textsubscript{2.5} concentrations were measured
using the same commercially available sensor (PMS7003 Plantower, Zefan,
Inc.) as was used for outdoor sensor-based PM\textsubscript{2.5}
measurements and recorded PM\textsubscript{2.5} concentrations every 1
min. The sensor was placed on a table in a room where participants
reported spending most of their time when awake, e.g., a living room or
bedroom. Indoor PM\textsubscript{2.5} sensors were deployed between late
November and mid January within field seasons (i.e., S2 and S4),
depending on the village and household visit schedule. The measurement
continued from the time of deployment until sensors were recollected
from homes in late April to capture the full heating season.

We randomly selected three households from the six in which we deployed
PM\textsubscript{2.5} sensors to co-locate a filter-based
PM\textsubscript{2.5} sampler with the PM\textsubscript{2.5} sensor. We
collected a 24-h PM\textsubscript{2.5} filter sample at the first 24-h
of indoor PM\textsubscript{2.5} sensor measurements. Filter-based
PM\textsubscript{2.5} samples were collected using Ultrasonic Personal
Aerosol Samplers (UPAS, Access Sensor Technologies) or Personal Exposure
Monitors (PEMs, Apex Pro) operating with flow rates of 1.0 and 1.8
L/min, respectively. Both samplers housed 37 mm PTFE filters (VWR,
2.0-μm pore size) and were equipped with a cyclone inlet with a 2.5 μm
cut point designed to perform under the corresponding sampling flow
rate. After 24-h, the samplers were retrieved and loaded with new
filters for measurements in other villages, once the previous sample
filters were removed and stored for later analysis. In total, we
successfully collected 149 and 148 indoor PM\textsubscript{2.5} filter
samples in S2 and S4, respectively.

As with the community outdoor air sampling, to support post-sampling
determination of organic carbon (OC) and elemental carbon (EC) fractions
of PM\textsubscript{2.5} mass, quartz filters were co-located with a
subset of Teflon filter samples collected in homes. Filter-based
PM\textsubscript{2.5} samples were collected using Personal Exposure
Monitors (PEMs, Apex Pro) operating with flow rates of 1.8 L/min. PEMs
housed 37 mm quartz filters (VWR, 2.0-μm pore size) and were equipped
with a cyclone inlet with a 2.5 μm cut point designed to perform under
the corresponding sampling flow rate. All quartz fiber filters were
baked at 550 °C for a minimum of 8 h to remove organic impurities prior
to sample collection. PM\textsubscript{2.5} samples collected on quartz
filters were analyzed using established thermo-optical methods for
quantifying elemental carbon (EC) and organic carbon (OC) to, then,
calibrate the colorimetric analysis of EC and OC on Teflon filters. In
Season 2, 71 quartz-based indoor PM\textsubscript{2.5} samples and 14
field blanks were successfully collected. In Season 4, indoor
PM\textsubscript{2.5} samples for gravimetric analysis had to be
collected on two types of PTFE sample media (Zefluor and Teflo filters),
due to discontinuation of manufacturing of the Zefluor filter media. To
ensure that quartz filters were deployed with both types of Teflon-based
filter media, 73 quartz-based indoor PM\textsubscript{2.5} samples were
collected concurrently with Zefluor samples, and 47 quartz indoor
PM\textsubscript{2.5} samples were collected alongside Teflo samples.
For indoor quartz PM\textsubscript{2.5} mass sampling in S4, 18 field
blanks were collected.

\hypertarget{personal-exposure-to-pm2.5-and-black-carbon}{%
\paragraph{\texorpdfstring{Personal exposure to PM\textsubscript{2.5}
and black
carbon}{Personal exposure to PM2.5 and black carbon}}\label{personal-exposure-to-pm2.5-and-black-carbon}}

To measure personal exposure we used two types of samplers: Personal
Exposure Monitors (PEMs, Apex Pro; Casella, UK) and Ultrasonic Personal
Aerosol Samplers (UPAS, Access Sensor Technologies, Fort Collins, CO,
USA). PEMs actively sampled air at a flow rate of 1.8 L/min, and UPAS
sampled air at 1.0 L/min (Volckens et al. 2017). Both samplers housed 37
mm PTFE filters (VWR, 2.0-μm pore size) and were equipped with a cyclone
inlet with a 2.5 μm cutpoint. Sampler flow rates were calibrated the
night before deployment and measured immediately after the sampling
period. Only 2\% of the post-sampling measurements deviated from the
target flow rate by greater than +/-10\%. Participants were instructed
to wear a small waistpack (for the PEM and sampling pump) or an arm band
or cross-body sling (for the UPAS) for 24 hours, which they could remove
from their body and place within 2 meters while sleeping, sitting, or
bathing. Field blanks for personal air pollution exposure measurements
were collected at a rate of \textasciitilde10\% in each village.

\hypertarget{gravimetric-analyses-of-ptfe-filter-based-pm2.5-samples}{%
\paragraph{\texorpdfstring{Gravimetric analyses of PTFE filter-based
PM\textsubscript{2.5}
samples}{Gravimetric analyses of PTFE filter-based PM2.5 samples}}\label{gravimetric-analyses-of-ptfe-filter-based-pm2.5-samples}}

All filters were placed in individually labeled cases, sealed in plastic
bags, and then transported to a field laboratory and immediately stored
in a -20°C freezer. Following completion of the field sampling campaign,
the samples and blanks were transported to Colorado State University,
where they were stored in a -20°C freezer prior to gravimetric and
chemical analysis of PM\textsubscript{2.5}.

All filters were placed in an environmentally-controlled equilibration
chamber (21-22 °C, 30-34\% relative humidity) for at least 24 hours
before tare and gross weighing. Before each weight was taken, filters
were discharged by a polonium-210 strip. Filters were weighed on a
microbalance (Mettler Toledo Inc., XS3DU, USA) with 1-μg resolution in
triplicate or more, until the differences among three weights were less
than 3 μg. The average of three readings was used to determine filter
mass, which was then blank-corrected using the median value of blank
filters {[}3 μg for UPAS-collected filters (53\% of samples); 33 μg for
PEM-collected filters (47\% of filter samples){]}, and
PM\textsubscript{2.5} concentrations were calculated by dividing the
mass by the sampled air volume.

\hypertarget{adjusting-sensor-based-pm2.5-using-filter-based-gravimetric-measurements}{%
\paragraph{\texorpdfstring{Adjusting sensor-based PM\textsubscript{2.5}
using filter-based gravimetric
measurements}{Adjusting sensor-based PM2.5 using filter-based gravimetric measurements}}\label{adjusting-sensor-based-pm2.5-using-filter-based-gravimetric-measurements}}

We established linear regression models between the filter-based
PM\textsubscript{2.5} mass concentrations (i.e., the `gold standard'
reference concentrations) and the sensor-based PM\textsubscript{2.5}
concentrations averaged over the same sampling period as the
filter-based samples. The slopes of the models were used as the
adjustment factors for the sensor-based PM\textsubscript{2.5}
concentrations. Separate regression models were conducted for indoor and
outdoor sensors and for each season given the sensitivity of the sensors
to relative humidity, temperature, and particle sources, which may
differ for indoor versus outdoor conditions and across seasons. In
Season 3, where only sensor-based measurements were conducted for indoor
PM\textsubscript{2.5} to avoid direct contact with household members
during the COVID-19 pandemic, we applied an adjustment factor developed
from a linear regression model that incorporated data from both Season 2
and Season 4.

The PM sensors were also evaluated before and after each season to
identify any sensors that needed further repair or replacement. The
PM\textsubscript{2.5} sensors underwent a calibration process that began
with synchronization to real-time PM\textsubscript{2.5} monitors at
Peking University (PKU) campus. This pre- and post-season calibration
included a week-long session using the Beta Attenuation Monitor (BAM)
alongside daily 24-hour filter samples. During this time, approximately
240 sensors were placed on the rooftop of the College of Urban and
Environmental Sciences building, each recording data every minute. A
similar approach was taken at the University of Chinese Academy of
Sciences (UCAS) campus, where around 400 PM sensors were installed on
the rooftop of the Environmental Monitoring Site of the College of
Resources and Environment, with data logging at one-minute intervals.
Daily collections of 24-hour Zeflour (Teflon) and quartz filter samples
accompanied the sensors' measurements to ensure accuracy. The
calibration process was repeated post-fieldwork to account for any
potential shifts or discrepancies in sensor performance. This approach
aimed to maintain consistent and accurate measurements from the PM
sensors throughout the study.

\hypertarget{chemical-analysis-of-pm-mass}{%
\paragraph{Chemical analysis of PM
mass}\label{chemical-analysis-of-pm-mass}}

We analyzed the chemical composition of community outdoor and personal
exposure PM\textsubscript{2.5} samples from each season to quantify the
individual components and species. PM\textsubscript{2.5} component
concentrations were determined within each community by dividing the
quantified component mass by the sampled air volume, after correcting
for field blanks collected in the corresponding season.

Elemental analysis of PM\textsubscript{2.5} mass was performed using a
Thermo Scientific Quant'X Evo energy-dispersive X-ray fluorescence
(EDXRF) spectrometer with Wintrace software version 10.3 using standard
methods (International 2009). Quantitative mass concentrations of 22
individual elements (Mg, Al, Si, S, K, Ca, Ti, Cr, Mn, Fe, Ni, Cu, Zn,
Ga, As, Se, Cd, In, Sn, Sb, Te, I) were determined empirically using
linear standard curves. Standard curves were generated from commercial,
single and dual element, thin film standards from MicroMatter
Technologies Inc.~(Montreal, Canada) in addition to blank films. The
quality of the analysis method was evaluated by analyzing a National
Institute of Standards and Technology (NIST) standard reference material
(SRM) 2783 Air particulate on filter media (Gaithersburg, MD, USA).
Elements for which at least 80\% of PM\textsubscript{2.5} mass samples
yielded quantifiable element mass were included for positive matrix
factorization and source analysis and apportionment. Those elements
were: Si, Mg, Fe, S, Ca, Al, K, Pb.

For analysis of water-soluble ions, a portion of each PTFE filter was
extracted in 15 mL deionized water (DI Water) in a Nalgene Amber HDPE
bottle using sonication without heat for 40 min. The extracts were
filtered to ensure that insoluble particles were removed using a 0.2 μm
PTFE syringe filter. Water-soluble ions were measured using a dual
channel Dionex ICS-3000 ion chromatography system. Specifically, a
Dionex IonPac CS12A analytical (3 × 150 mm) column with eluent of 20 mM
methanesulfonic acid at a flow rate of 0.5 mL/min was used to measure
cations (Ca2+, Mg2+, Na+, NH4+, K+), while a Dionex IonPac AS14A
analytical (4 × 250 mm) column with an eluent of 1 mM sodium
bicarbonate/8 mM sodium carbonate at a flow rate of 1 mL/min was used to
measure anions (SO42−, NO3−, Cl−) (Sullivan et al., 2008).

Organic (OC) and elemental carbon (EC) on PTFE filters were measured
using an optical color space sensing system. The CIE-Lab color space
optical sensing system measures the optical properties of the
PM\textsubscript{2.5} samples, and these properties are used to develop
the EC and OC predictive models. The CIE-Lab color system is a
color-opponent space that includes all of the color models, with
dimension L* for lightness and a* and b* for the color-opponent
dimensions. More information about the CIE Lab color space system, its
formulation, and its specific application to the analysis of OC and EC
fractions of fine particulate matter pollution is provided in Khuzestani
et al. (Khuzestani et al. 2017). Briefly, all the Teflon (PTFE) and
quartz filters collected were analyzed using the i1Pro Colorimeter
(X-Rite, INC. Grand Rapids, MI). The colorimeter sensor was placed
directly over the quartz and Teflon filters, and the color components
were measured under the D65 instrument internal illumination light
source. Each filter sample was analyzed in triplicate, and the average
value of each color coordinate was applied as the optical property of
the sample (Olson et al. 2016). CIE Standard Illuminant D65 simulates
average midday light and is a commonly used standard illuminant, as
defined by the International Commission on Illumination (CIE). The
CIE-Lab color space response variables were used in separate random
forest models for EC and OC.

The reference measurements for the random forest model development were
EC and OC determined from quartz filters collected indoors and outdoors
(as described above). PM\textsubscript{2.5} samples collected on quartz
filters were analyzed for OC and EC using a Sunset Laboratory OC/EC Lab
instrument (Sunset Laboratories, Inc., MODEL, USA) according to the
default Sunset Analyzer protocol. A section of each quartz filter
underwent a combined thermal desorption-optical transmittance
measurement based on NIOSH methods 5040 to differentiate and quantify
the EC and OC components in mass. For the thermal desorption component,
the sample is oxidized twice, according to a strict temperature regime.
The first oxidation stage thermally removes OC in a mobile phase of pure
helium gas to be converted from carbon dioxide (CO2) to methane (CH4)
gas and measured by a flame ionization detector (FID). The second
oxidation stage proceeds in a mixture of helium and oxygen to oxidize
EC, which is also quantified by the FID. The FID is internally
calibrated with methane, and external quality control checks are made
with sucrose standards. To correct for the potential production of EC by
OC pyrolysis during the first heating stage, light transmission from a
laser through the filter section was monitored throughout analysis.
Reduced light transmittance corresponds to EC generated by the
laboratory analysis.

Following gravimetric analysis, all PTFE filters were also analyzed for
black carbon (BC) using an optical transmissometer data acquisition
system (SootScan\^{}TM OT21 Optical Transmissometer; Magee Scientific,
Berkeley, CA, USA). Light attenuation through each filter was measured
before and after sampling in the field. To calculate BC mass, the
difference between the pre- and post- light attenuation was converted to
a mass surface loading using the classical Magee mass absorption
cross-sections of 16.6 m2/g for the 880 nm channel optical BC (Ahmed et
al. 2009). BC concentrations were calculated by multiplying surface
loadings by the sampled surface area of the filters (8.6 cm2 for
UPAS-collected filters; 7.1 cm2 for PEM-collected filters), correcting
for the field blank mass using the median value of blanks (0.31 μg for
UPAS-collected filters; 0.01 μg for PEM-collected filters), and finally
dividing by the sampled air volume.

\hypertarget{outdoor-and-indoor-household-air-temperature}{%
\subsubsection{Outdoor and indoor (household) air
temperature}\label{outdoor-and-indoor-household-air-temperature}}

Hourly outdoor temperature and relative humidity data were obtained from
the extensive network of meteorological
\href{http://beijingair.sinaapp.com}{stations} in Beijing. We measured
indoor temperature in all participant homes prior to blood pressure
measurement. In a random 75\% subsample of households in each campaign,
we also placed a real-time temperature sensor (iButton DS1921G-F5;
Thermochron, Maxim Inc., USA) in the room where participants reported
spending most of their daytime hours when indoors. Sensors were
wall-mounted at a standardized height (\textasciitilde1.5 to 2 meters),
away from major heating sources, windows, and doors, and were programmed
to log a temperature reading every 125 minutes for up to 4 months to
capture the full winter period and early spring weeks when heating may
still intermittently occur. Prior to the start of each campaign, we
co-located all of the sensors and measured temperature over two days and
compared the readings. Sensors recording values \textgreater1C from the
group median value were excluded from data collection.

\hypertarget{objective-measurement-of-household-stove-use-using-sensors}{%
\subsubsection{Objective measurement of household stove use using
sensors}\label{objective-measurement-of-household-stove-use-using-sensors}}

Following methods used in a previous intervention evaluation study in
rural China (Clark et al. 2017), we objectively measured household
heating stove use in a random sample of households selected, also at
random, for either short- or long-term measurement. We measured
short-term (24 h) stove use for all household heating stoves in 315 and
227 households in seasons 2 and 3, respectively. Long-term stove use was
assessed in 324, 273, and 585 homes in S2, S3, and S4, respectively, for
a period of approximately 6 months. We measured stove use using the same
real-time temperature data loggers used for indoor temperature (iButton
DS1921G-F5; Thermochron, Maxim Inc., USA). Field staff placed the
sensors on stoves and programmed them to record surface temperature
every 125 minutes, a timing decision based on pilot assessments showing
that shorter time intervals did not change the number of heating events
detected. Sensors were on the surfaces of biomass and coal-fuelled
stoves and radiators. For heat pumps, sensors were placed on the heat
exchanger coil on air-to-air units and on the radiator of air-to-water
units.

The number and duration of stove combustion events were identified from
the temperature data using criteria defined based on the observed
changes in the peak shape of the time series temperature curves (i.e.,
changes in the slope or in absolute temperature compared with the indoor
ambient temperature). This approach was specific to heating stoves but
developed based on stove use identification for cookstoves in previous
studies by us and others (Clark et al. 2017; Ruiz-Mercado et al. 2013;
Snider et al. 2018). We developed separate criteria for each stove since
heating patterns varied by stove. These criteria were coded into
stove-specific algorithms (using R Studio) to systematically identify
the number and duration of heating events across households. A random
15\% of stove use temperature files were sampled with respect to the
stove type and measurement duration (short-term/24 h or
long-term/\textasciitilde6 mo), and manually coded to develop the
criteria. The number and duration of heating events were identified by
the algorithms in the remaining 85\% of files. We compared heating
periods identified manually with those identified by the algorithm to
check for systematic differences and possible overfitting.

\hypertarget{questionnaires}{%
\subsubsection{Questionnaires}\label{questionnaires}}

Field staff administered household and individual-level questionnaires
to assess household demographic information and educational attainment,
household assets, house structure, stove and fuel use patterns
(including a complete roster of heating methods and their contributions
in each room), and individual health behaviors including exercise
frequency, smoking, alcohol consumption, medication use, and
clinician-diagnosed health conditions. We used Surveybe
computer-assisted personal interview (CAPI) software to collect survey
data via handheld electronic tablets. Questions were read to
participants in Mandarin-Chinese, and their responses were recorded into
tablets.

Prior to the start of data collection, all questions were translated
from English into Chinese and then back-translated to English for
quality assurance. Many questions were adapted from previous field
studies of household energy and blood pressure conducted in rural
Beijing or other rural sites in China (Baumgartner et al. 2018; Yan et
al. 2020), and all questions were iteratively tested with staff and
adapted prior to implementation. Prior to each campaign in this study,
the questionnaire and other study measurements were tested in 12
households located in a Beijing village that was eligible for our study
but was instead selected for testing. We used the test village to assess
whether the questions were understandable and interpreted as intended
and to identify any problems with the study measurements or their
implementation. Study protocols were subsequently adapted prior to the
start of data collection.

In addition to household and individual participant questionnaires, we
also conducted village surveys with one representative from each village
committee to understand how the policy was implemented in that village
and to inquire about any other rural development or health programs
being implemented in the village. Committee members answered questions
about committee and villager interest in the policy and, for in treated
villages, assignment versus application to the policy, any home or
village renovations required by the upper-level government prior to heat
pump installation, decision-making for the type and brand of heating
technology, level of subsidies provided for heaters and electricity, and
technical and logistic guidance to villagers.

\hypertarget{blood-pressure}{%
\subsubsection{Blood pressure}\label{blood-pressure}}

Following 5 min of quiet rest, at least three brachial and central
systolic (bSBP/cSBP) and diastolic (bDBP/cDBP) blood pressures (BPs)
were taken by trained staff at 1 min apart on the participant's
supported right arm. We used an automated oscillometric device (BP+;
Uscom Ltd, New Zealand) that estimates central pressures from the
brachial cuff pressure fluctuations. Central pressures were previously
validated against invasive cBP measurements in previous studies
(Costello et al. 2015; Lowe et al. 2009). The BP devices were factory
calibrated by the manufacturer prior to the start of the first and
fourth campaigns. Up to five measurements were taken if the difference
between the last two was \textgreater5 mmHg or staff were unable to
obtain a reading. The BP measurements were conducted in the
participant's home and staff were trained to follow strict quality
control procedures, including use of an appropriately sized cuff,
correct positioning of the arm, both feet on the ground, and ensuring 5
min of quiet rest before measurement. Details are described in the
standard operating procedures (SOP): https://osf.io/gmka5. The average
of the final two measurements was used for statistical analysis unless
only one BP measurement was obtained (n = 13 observations), in which
case, a single measurement was used. The time of day, day of the week,
and indoor temperature prior to BP measurement were also recorded.

\hypertarget{multiple-imputation-for-covariates-in-analyses-with-bp-outcomes}{%
\paragraph{Multiple imputation for covariates in analyses with BP
outcomes}\label{multiple-imputation-for-covariates-in-analyses-with-bp-outcomes}}

Blood pressure was measured at household visits but several key
covariates like waist circumference, height, and weight were measured at
the clinic visits in S1 and S2. Thus, we were missing covariate
information for individuals who were unable to attend the clinic visits
(\textasciitilde15-20\% of participants in each campaign). Multiple
imputation with chained equations (MICE) was conducted to impute missing
covariate data for individuals who participated in the household visit
but not the clinic visit in analyses with BP outcomes in order to retain
observations with BP measurements that would have otherwise been dropped
in adjusted models using complete-case analysis. Imputation was
performed with the `MICE' package (van Buuren and Groothuis-Oudshoorn
2011) in R (m = 30 imputation datasets, with 30 iterations each), and
the DiD analysis was conducted for each of the 30 datasets. We then used
Rubin's Rules to combine point estimates and standard errors while
accounting for both within- and between-dataset variances (Rubin 1987).

\hypertarget{self-reported-respiratory-symptoms-and-airway-inflammation}{%
\subsubsection{Self-reported respiratory symptoms and airway
inflammation}\label{self-reported-respiratory-symptoms-and-airway-inflammation}}

During questionnaire assessment, participants were asked about chronic
airway symptoms including cough, phlegm, wheeze, and tightness in the
chest using questions validated for use in Mandarin-Chinese and
developed from the standard questionnaires on COPD (Medical Research
Council/International Union Against Tuberculosis and Lung Disease) and
asthma (International Study of Asthma and Allergies in Childhood). The
Mandarin-Chinese questions were extensively piloted with rural and
peri-urban Beijing residents to ensure that the health terminology and
symptom time patterns were adequate and understandable to the local
population.

In a \textasciitilde25\% random subsample of participants, we also
measured fractional exhaled nitric oxide (FeNO), a non-invasive and
established marker of airway inflammation, using a portable handheld
device (Aerocrine, Solna, Sweden) fit with a NIOX VERO® sensor,
following ATS recommendations and guidelines (ATS/ERS 2005). Briefly,
FeNO measurement was performed with participants in a standing position.
They inhaled NO-free air through a mouthpiece with an NO-scrubber
attached, followed by controlled expiration for 10 s through the
mouthpiece at 50±5 mL/s. A nose clip was used to avoid nasal inhalation,
and accurate flow rate was achieved using visual and auditory cues
generated by the device. Detailed methods are provided in our previous
study of air pollution and FeNO in Beijing adults (Shang et al. 2020).
At least two measurements were obtained for each participant.

\hypertarget{blood-inflammatory-and-oxidative-stress-markers}{%
\subsubsection{Blood inflammatory and oxidative stress
markers}\label{blood-inflammatory-and-oxidative-stress-markers}}

Trained nurses collected 20 ml of whole blood in a labeled vacutainer
via venipuncture using standard techniques (Tuck et al. 2009). Details
are described in our published \href{https://osf.io/zwpfg}{SOP}.
Briefly, fasting blood samples were collected by experienced
phlebotomists (nurses) in the morning and stored at 4-10°C prior to
centrifugation. Two serum aliquots from each participant were then
placed in a -30°C freezer for temporary storage. Collection-to-storage
time was \textless4 hrs for all samples in both campaigns where blood
samples were collected. Within 3-5 days of collection, the samples were
transported in styrofoam containers with dry ice to a -80°C freezer with
a backup generator and alarm system at Peking University.

The first aliquot was analyzed for glucose and a complete lipid profile
within two months of collection, and results were communicated to
participants. The second aliquot was stored in the -80°C freezer for
analysis of biomarkers of systemic inflammation {[}C-reactive protein
(CRP), interleukin-6 (IL-6), tumour necrosis factor alpha
(TNF-\(\alpha\)) and oxidative stress {[}8-hydroxy-2'-deoxyguanosine
(8-OHdG) and malondialdehyde (MDA){]} at the University of the Chinese
Academy of Sciences between July and September of 2023. These biomarkers
were selected because they are associated with the development of
cardiovascular disease and events (e.g., Danesh 2008; Pearson 2003;
Ridker 2000; 2001; ERF 2012), and both acute and longer-term exposures
to air pollution have been associated with changes in inflammatory and
oxidative stress markers (e.g., Pope 2004; Rückerl 2007; Rich 2012;
Kipen 2010; Huang 2012).

We followed standard methods for analysis (FDA Guidance, 2018). For
inflammatory markers (IL-6, TNF-𝛼, CRP), the optic densities (OD) of all
samples were measured using an automated ELISA reader. Every plate had 8
standard samples used to generate a standard curve that related OD and
standard inflammatory marker concentration. A standard curve for each
microplate was generated by a computer software program based on a
4-parameter method. Each plate included at least 3 control samples to
ensure the stability of standard curves. All samples, standards, and
controls were measured in duplicate, and the average was used for
statistical analysis. For oxidative stress biomarkers (MDA and 8-OHdG),
the chromatographic peak areas of all samples were measured using HPLC
with UV detector and HPLC-MS/MS. Every plate had 7 standard samples used
to generate a standard curve that related peak area and concentration of
each standard oxidative stress marker. A standard curve for each plate
was generated using a computer software program based on a linear
method. Each plate included at least 3 control samples to ensure the
stability of standard curves. Standards and controls were measured in
duplicate and samples were measured once due to high precision in a
pilot study (Food and Drug Administration 2018).

\hypertarget{anthropometric-measurements.}{%
\subsubsection{Anthropometric
measurements.}\label{anthropometric-measurements.}}

Body weight, height, and waist circumference were measured at the clinic
visit in the first two campaigns and in participant homes in the last
campaign. Weight was measured in light indoor clothing without shoes in
kilograms to one decimal place, using standing scales supported on a
steady surface. The scales were calibrated prior to the start of each
campaign, and the same staff member stepped on the scale each morning to
ensure that it was functioning properly. Height was measured without
shoes in centimeters to one decimal place with a stadiometer. Waist
circumference was measured without clothing obstruction at one
centimeter above the participant's navel at minimal respiration in
centimeters to one decimal place. The measuring tape was replaced at the
start of each campaign to avoid stretching.

\hypertarget{measuring-policy-impacts}{%
\subsection{Measuring policy impacts}\label{measuring-policy-impacts}}

To understand how Beijing's policy works we used a
difference-in-differences (DiD) design (Callaway 2020), leveraging the
staggered rollout of the policy across multiple villages to estimate its
impact on health outcomes and understand the mechanisms through which it
works. Simple comparisons of treated and untreated (i.e., control)
villages after the CBHP policy has been implemented are likely to be
biased by unmeasured village-level characteristics (e.g., migration,
average winter temperature) that are associated with health outcomes.
Similarly, comparisons of only treated villages before and after
exposure to the program are susceptible to bias by other factors
associated with changes in outcomes over time (i.e., secular trends,
impacts of the COVID-19 pandemic). By comparing the \emph{changes} in
outcomes among treated villages to the \emph{changes} in outcomes among
untreated villages, we can control for any unmeasured time-invariant
characteristics of villages as well as any general secular trends
affecting all villages that are unrelated to the policy.

\begin{figure}[H]

{\centering \includegraphics[width=0.5\textwidth,height=\textheight]{hei-report_files/figure-pdf/fig-didfig-1.pdf}

}

\caption{\label{fig-didfig}Stylized example of
difference-in-differences}

\end{figure}

The DiD design compares outcomes before and after an intervention in a
treated group relative to the same outcomes measured in a control group.
The control group trend provides the crucial ``counterfactual'' estimate
of what would have happened in the treated group had it not been
treated. By comparing each group to itself, this approach helps to
control for both measured and unmeasured fixed differences between the
treated and control groups. By measuring changes over time in outcomes
in the control group unaffected by the treatment, this approach also
controls for any unmeasured factors affecting outcome trends in both
treated and control groups. This is important since there are often many
potential factors affecting outcome trends that cannot be disentangled
from the policy if one only studies the treated group (as in a
traditional pre-post design).

The canonical DiD design (Card and Krueger 1994) compares two groups
(treated and control) at two different time periods (pre- and
post-intervention, Figure~\ref{fig-didfig}). In the first time period
both groups are untreated, and in the second time period one group is
exposed to the intervention. If we assume that the differences between
the groups would have remained constant in the absence of the
intervention (parallel trends assumption), then an unbiased estimate of
the impact of the intervention in the post period can be calculated by
subtracting the pre-post difference in the untreated group from the
pre-post difference in the treated group.

However, when multiple groups are treated at different time periods, the
most common approach has been to use a two-way fixed effects model to
estimate the impact of the intervention which controls for secular
trends and differences between districts. However, recent evidence
suggests that the traditional two-way fixed effects estimation of the
treatment effect may be biased in the context of heterogeneous treatment
effects (Callaway and Sant'Anna 2021; Goodman-Bacon 2021).

\hypertarget{measuring-pathways-and-mechanisms}{%
\subsection{Measuring pathways and
mechanisms}\label{measuring-pathways-and-mechanisms}}

To estimate how much of the CBHP intervention may work through different
mechanisms, we used causal mediation analysis. Causal approaches to
mediation attempt to discern between, and clarify the necessary
assumptions for identifying, different kinds of mediated effects. Taking
as an example the DAG in Figure~\ref{fig-dag1}, with \(T\) as the
policy, \(M\) as PM\textsubscript{2.5}, and \(Y\) as systolic blood
pressure, we can define the controlled direct effect (\(CDE\)) as the
effect of the CBHP policy on systolic blood pressure if we fix the value
of PM\textsubscript{2.5} to a certain reference level for the entire
population. For example, we can estimate the impact of the policy on
health outcomes while holding PM\textsubscript{2.5} at a uniform level
of average background exposure, or some other hypothetical level.

\begin{figure}[H]

{\centering \includegraphics[width=0.6\textwidth,height=\textheight]{hei-report_files/figure-pdf/fig-dag1-1.pdf}

}

\caption{\label{fig-dag1}Hypothetical Directed Acyclic Graph showing
direct and indirect effects with outcome (\(Y\)), pre-treatment
covariates (\(X\)), policy (\(T\)), multiple mediators
(\(M_{1},M_{2}\)), as well as covariates for the mediators (\(W\)).}

\end{figure}

Although other mediated effects such as ``natural'' direct and indirect
effects are theoretically estimable (VanderWeele 2015), they involve
challenging ``cross-world'' assumptions that are difficult to anchor in
policy (Naimi et al. 2014). Other approaches to mechanisms have focused
on principal stratification (e.g., Zigler et al. 2016), although
conceptual difficulties with identifying the (unverifiable) principal
strata make it challenging for questions of mediation. Because
controlled direct effects are considered more directly policy relevant
for public health, we focus on estimating these mediated quantities.

\hypertarget{data-analysis}{%
\section{Data Analysis}\label{data-analysis}}

To understand how the policy's impact on health may be mediated by
different potential mediators, we need to estimate first the total
effect of the policy on the outcomes, as well as the \(CDE\)s with
adjustment for potential mediators. As discussed above, in order for the
mediators to `explain' the total effects of the policy on health, the
policy should affect the mediators, and the mediators should also affect
the outcomes.

\hypertarget{total-effect}{%
\subsection{Total Effect}\label{total-effect}}

To estimate the total effect of the policy we used a DiD analysis that
accommodates staggered treatment rollout. To allow for heterogeneity in
the context of staggered rollout we used `extended' two-way fixed
effects (ETWFE) models (Wooldridge 2021) to estimate the total effect of
the CBHP policy. The mean outcome (replaced by a suitable link function
\(g(\cdot)\) for binary or count outcomes) was defined using a set of
linear predictors:

\begin{equation}\protect\hypertarget{eq-etwfe}{}{Y_{ijt}=g(\mu_{ijt}) = \alpha + \sum_{r=q}^{T} \beta_{r} d_{r} + \sum_{s=r}^{T} \gamma_{s} fs_{t}+ \sum_{r=q}^{T} \sum_{s=r}^{T} \tau_{rt} (d_{r} \times fs_{t}) + \varepsilon_{ijt}}\label{eq-etwfe}\end{equation}

where \(Y_{ijt}\) is the outcome for individual \(i\) in village \(j\)
at time \(t\), \(d_{r}\) represent treatment cohort dummies, i.e., fixed
effects for cohorts of villages that were first exposed to the policy at
the same time \(q\) (e.g., in 2019, 2020, or 2021), \(fs_{t}\) are time
fixed effects corresponding to different winter data collection
campaigns (2018-19, 2019-20, or 2021-22), and \(\tau_{rt}\) are the
cohort-time \emph{ATTs}. The ETWFE and other approaches that allow for
several (potentially heterogenous) treatment effects may also be
averaged to provide a weighted \(ATT\). Several potential possibilities
are feasible, including weighting by treatment cohorts or time since
policy adoption (Goin and Riddell 2023).

\hypertarget{mediation-analysis}{%
\subsection{Mediation Analysis}\label{mediation-analysis}}

As noted above, with respect to the mediation analysis we are chiefly
interested in the \(CDE\), which can be derived by adding relevant
mediators \(M\) to this model. If we also allow for exposure-mediator
interaction and potentially allow for adjustment for confounders \(W\)
of the mediator-outcome effect, we can extend equation
Equation~\ref{eq-etwfe} as follows:

\begin{equation}\protect\hypertarget{eq-etwfem}{}{
\begin{aligned}
Y_{ijt}=g(\mu_{ijt}) = \alpha + \sum_{r=q}^{T} \beta_{r} d_{r} + \sum_{s=r}^{T} \gamma_{s} fs_{t}+ \sum_{r=q}^{T} \sum_{s=r}^{T} \tau_{rt} (d_{r} \times fs_{t}) \\ + \delta M_{it} + \sum_{r=q}^{T} \sum_{s=r}^{T} \eta_{rt} (d_{r} \times fs_{t} \times M_{it}) + \zeta \mathbf{W} + \varepsilon_{ijt}
\end{aligned}
}\label{eq-etwfem}\end{equation}

where now \(\delta\) is the conditional effect of the mediator \(M\) at
the reference level of the treatment (again, represented via the series
of group-time interaction terms), and the collection of \(\eta\) terms
are coefficients for the product terms allowing for mediator-treatment
interaction. Finally, \(\zeta\) is a vector of coefficients for the set
of confounders contained within \(\mathbf{W}\).

As noted above, in the staggered DiD framework that allows for
heterogeneity, we do not have a single treatment effect but a collection
of group-time treatment effects that may be averaged in different ways.
This extends to the estimation of the \(CDE\), in which case we will
also have several \(CDE\)s that can be averaged to make inferences about
the extent to which the policy's impact is mediated by
\emph{PM\textsubscript{2.5}}. Based on the setup in
Equation~\ref{eq-etwfem} the \(CDE\) is estimated as:
\(\delta + \eta_{rt}MT\). In the absence of interaction between the
exposure and the mediator (i.e., \(\eta_{rt}=0\)) the \(CDE\) will
simply be the estimated treatment effects
\(\sum_{r=q}^{T} \sum_{s=r}^{T} \tau_{rt}\), i.e., the effect of the
policy holding \(M\) constant. For a valid estimate of the \(CDE\) we
must account for confounding of the mediator-outcome effect, represented
by \(W\) in the equation above. The inclusion of baseline measures of
both the outcome and the proposed mediators inherent in our DiD strategy
help to reduce the potential for unmeasured confounding of the
mediator-outcome effect (Keele et al. 2015). Given the large number of
outcomes of interest in this study, as well as the potential for
heterogeneous treatment effects, we limited the mediation analysis to
health outcomes for which we observed a total effect of the CBHP policy.

\hypertarget{identification-of-potential-confounders-and-model-covariates}{%
\subsection{Identification of potential confounders and model
covariates}\label{identification-of-potential-confounders-and-model-covariates}}

DiD is a strong analytical approach that already minimizes the risk of
confounding, where cohort-fixed effects control for measured and
unmeasured time-constant factors that may differ between treatment
cohorts (e.g., genetics, altitude), and time-fixed effects control for
secular trends that affect all treatment cohorts similarly over the
study period (e.g., background improvements in ambient air quality or
household transition to more efficient heating).

For models estimating the effect of the policy on health outcomes, we
used directed acyclic graphs (DAGs) to identify potential time-varying
causes of both treatment by the policy and our study outcome(s) that
could differ between treatment groups, and adjusted for those potential
confounders in the regression models. For the mediation analysis, we
identified potential mediator-outcome confounders using the same
approach. These variables were identified from the relevant
peer-reviewed literature and our team's substantive knowledge about the
CBHP policy. In the multivariable models, we also adjusted for strong
predictors of the outcome that were not affected by treatment, and thus
not confounders, to improve model precision. The covariates included in
each of the models for cardiovascular and respiratory health outcomes
are provided in the tables.

For air pollution outcomes, we considered the following covariates:
village population and total number of households in the village;
temperature, relative humidity, wind direction, wind speed, boundary
layer height; home area and home area heated; home insulation; smoking
status of participant and whether or not they lived with a smoker;
whether or not the household reported using wood (i.e., biomass) for
household energy activities, and if so, self-reported quantity of wood.
Potential non-linearity between continuous covariates and our study
outcomes were evaluated using cubic splines. Ultimately, the following
covariates were included in the final DiD models for outdoor, indoor,
and personal exposures to air pollution, based on whether measurable
changes in the covariate over time were observed. For the final adjusted
DiD model for personal exposure `mixed combustion' source contributions,
the following covariates were included: temperature (represented by a
spline with 2 degrees of freedom); participant smoking status; and
whether or not the household reported using biomass fuel. For the final
adjusted DiD model for outdoor (community) `mixed combustion' source
contributions, the following covariates were included: total number of
households in the village; village population; and ambient relative
humidity (represented by a spline with 2 degrees of freedom).

\hypertarget{results-1}{%
\section{Results}\label{results-1}}

We retained all 50 study villages during this four-year longitudinal
assessment of village treatment by the CBHP policy, though we were only
able to visit 41 villages in winter 2020-21 (S3) and were limited to
village and household-level measurements of air quality, indoor
temperature, and stove use in that campaign due to travel restrictions
during the COVID pandemic.

By S2, S3, and S4 there were a cumulative total of 10, 17, and 20 (out
of 50 total) study villages treated by the CBHP policy, respectively.
All of the treated villages in our study selected to install
electric-powered air-source heat pumps with 200 RMB per meter square (up
to 24,000 RMB) in subsidies and were also provided with 80\% night-time
electricity subsidies up to 10,000kWh per heating season. To limit coal
use, villages enrolled in the policy were no longer allowed to place
orders for subsidized coal with the district-level governments that
manage the procurement and distribution of coal for residential heating
in Beijing. In addition, village committee leaders in treated villages
reported feeling accountable to the Environmental Protection Department
for limited coal-related air pollution, and were motivated to encourage
residents to not burn coal. Some villages were equipped with government
air pollution monitors and the Environmental Protection Department
conducted village inspections and issued warnings about coal burning.
Households burning coal in treated villages were at risk of losing their
electricity subsidy.

Appendix Figure~\ref{fig-flowchart} shows the participation of villages,
households, and participants across the four waves of data collection.
We conducted measurements in over 1000 participants in each of the three
measurement campaigns that included individual-level measurements. In
total, we enrolled 1432 participants into the study, of which 630 (43\%)
participated in all three campaigns, 443 (31\%) participated in two
campaigns and 365 (25\%) participated in a single campaign. We did not
observe any notable differences in demographic characteristics or health
behaviors between participants who contributed to a different number of
campaigns (Table~\ref{tbl-each-campaign}) or between participants in
each of the three campaigns with individual measurements
(Table~\ref{tbl-diff-campaign}).

\hypertarget{tbl-each-campaign}{}
\begin{table}
\caption{\label{tbl-each-campaign}Demographic and health characteristics of participants in each study
campaign. }\tabularnewline

\centering
\begin{tabular}{l>{\centering\arraybackslash}p{2.5cm}>{\centering\arraybackslash}p{2.5cm}>{\centering\arraybackslash}p{2.5cm}}
\toprule
\textbf{Characteristic} & \textbf{Campaign 1 (2018-19) N=1003} & \textbf{Campaign 2 (2019-20) N=1110} & \textbf{Campaign 4 (2021-22) N=1028}\\
\midrule
Female, n (\%) & 580 (57.8) & 653 (58.8) & 612 (59.5)\\
Current smoker, n (\%) & 257 (25.6) & 295 (26.6) & 265 (25.8)\\
Any smoke exposure, n (\%) & 788 (78.6) & 897 (80.8) & 843 (82)\\
Age in years, Mean (SD) & 60.7 (9.2) & 61.4 (9.1) & 63.1 (9)\\
BMI in kg/m\textsuperscript{2}, Mean (SD) & 26.1 (3.7) & 25.7 (3.5) & 26.1 (4)\\
Waist circumference in cm, Mean (SD) & 86.8 (10.2) & 87.4 (9.4) & 91.4 (10.7)\\
\bottomrule
\end{tabular}
\end{table}

\hypertarget{tbl-diff-campaign}{}
\begin{table}
\caption{\label{tbl-diff-campaign}Demographic and health characteristics of participants who contributed
to different numbers of campaigns. }\tabularnewline

\centering
\begin{tabular}{l>{\centering\arraybackslash}p{2.5cm}>{\centering\arraybackslash}p{2.5cm}>{\centering\arraybackslash}p{2.5cm}}
\toprule
\textbf{Characteristic} & \textbf{1 Campaign N=365} & \textbf{2 Campaigns N=443} & \textbf{3 Campaigns N=630}\\
\midrule
Female, n (\%) & 211 (57.8) & 253 (57.1) & 370 (58.7)\\
Current smoker, n (\%) & 110 (30.1) & 117 (26.4) & 161 (25.6)\\
Any smoke exposure, n (\%) & 288 (78.9) & 360 (81.3) & 498 (79)\\
Age in years, Mean (SD) & 59.9 (9.2) & 60.5 (8.8) & 61.3 (9.1)\\
BMI in kg/m\textsuperscript{2}, Mean (SD) & 26.3 (3.6) & 25.8 (3.5) & 26.1 (3.7)\\
Waist circumference in cm, Mean (SD) & 90.3 (9.8) & 86.5 (10) & 86.9 (10.4)\\
\bottomrule
\end{tabular}
\end{table}

\hypertarget{description-of-study-sample}{%
\subsection{Description of study
sample}\label{description-of-study-sample}}

\hypertarget{tbl-table1}{}
\begin{table}
\caption{\label{tbl-table1}Descriptive statistics for selected demographic, health, and
environmental measures at baseline, by treatment status }\tabularnewline

\centering\centering
\fontsize{9}{11}\selectfont
\begin{tabular}[t]{rrrrrrr}
\toprule
\multicolumn{1}{c}{ } & \multicolumn{2}{c}{Never treated (N=603)} & \multicolumn{2}{c}{Ever treated (N=400)} & \multicolumn{2}{c}{ } \\
\cmidrule(l{3pt}r{3pt}){2-3} \cmidrule(l{3pt}r{3pt}){4-5}
  & Mean & Std. Dev. & Mean & Std. Dev. & Diff. in Means & Std. Error\\
\midrule
\textbf{Demographics:} & \textbf{} & \textbf{} & \textbf{} & \textbf{} & \textbf{} & \textbf{}\\
Age (years) & 59.9 & 9.4 & 60.4 & 9.2 & 0.5 & 0.6\\
Female (\%) & 59.5 & 49.1 & 59.1 & 49.2 & -0.4 & 3.2\\
No education (\%) & 11.5 & 31.9 & 12.3 & 32.9 & 0.9 & 2.1\\
Primary education (\%) & 75.5 & 43.0 & 77.6 & 41.7 & 2.1 & 2.8\\
Secondary+ education (\%) & 12.6 & 33.2 & 9.8 & 29.7 & -2.9 & 2.0\\
\textbf{Health measures:} & \textbf{} & \textbf{} & \textbf{} & \textbf{} & \textbf{} & \textbf{}\\
Never smoker (\%) & 61.9 & 48.6 & 59.5 & 49.1 & -2.4 & 3.2\\
Former smoker (\%) & 11.9 & 32.4 & 15.1 & 35.8 & 3.2 & 2.2\\
Current smoker (\%) & 26.2 & 44.0 & 25.4 & 43.6 & -0.8 & 2.8\\
Never drinker (\%) & 55.9 & 49.7 & 52.5 & 50.0 & -3.4 & 3.2\\
Occasional drinker (\%) & 26.0 & 43.9 & 25.5 & 43.6 & -0.5 & 2.8\\
Daily drinker (\%) & 17.8 & 38.3 & 21.9 & 41.4 & 4.1 & 2.6\\
Systolic (mmHg) & 131.4 & 16.8 & 128.7 & 14.3 & -2.7 & 1.0\\
Diastolic (mmHg) & 82.7 & 11.6 & 82.1 & 11.3 & -0.6 & 0.8\\
Waist circumference (cm) & 87.7 & 10.5 & 85.4 & 9.5 & -2.3 & 0.8\\
Body mass index (kg/m2) & 26.3 & 3.7 & 25.8 & 3.6 & -0.5 & 0.3\\
Frequency of coughing (\%) & 18.7 & 39.0 & 19.7 & 39.8 & 1.0 & 2.6\\
Frequency of wheezing (\%) & 6.2 & 24.2 & 6.6 & 24.8 & 0.3 & 1.6\\
Shortness of breath (\%) & 29.2 & 45.5 & 34.3 & 47.5 & 5.1 & 3.0\\
Chest trouble (\%) & 11.6 & 32.0 & 14.1 & 34.9 & 2.5 & 2.2\\
Any respiratory problem (\%) & 50.6 & 50.0 & 54.3 & 49.9 & 3.7 & 3.2\\
\textbf{Environmental measures:} & \textbf{} & \textbf{} & \textbf{} & \textbf{} & \textbf{} & \textbf{}\\
Temperature (°C) & 13.8 & 3.6 & 13.5 & 3.3 & -0.3 & 0.2\\
Personal PM2.5 (ug/m3) & 150.2 & 300.3 & 103.8 & 107.3 & -46.3 & 19.1\\
\bottomrule
\end{tabular}
\end{table}

Table~\ref{tbl-table1} shows the distribution of selected demographic,
health, and environmental characteristics from the baseline survey,
prior to any villages being enrolled in the CBHP policy. We provide
means and standard deviations separately for villages that eventually
enter into the policy with those that never do so. As noted above,
although our DiD identification strategy allows for fixed differences
between treated and untreated villages, overall the differences at
baseline are generally small and the groups seem well balanced on most
measures, with the exception of personal exposure to
PM\textsubscript{2.5}, which was lower in villages that were eventually
treated.

\hypertarget{summary-of-pm-and-bc-measurements}{%
\subsection{Summary of PM and BC
measurements}\label{summary-of-pm-and-bc-measurements}}

At baseline, fine particulate matter (PM\textsubscript{2.5}) and black
carbon (BC) concentrations were higher, on average, for personal
exposures compared with outdoor concentrations. From Season 2 onward,
with the inclusion of indoor air pollution measurements, personal
exposure air pollution concentrations were still higher than indoor or
outdoor concentrations, with indoor levels being higher than outdoors
(Table~\ref{tbl-pm-season}). This trend (personal \textgreater{} indoor
\textgreater{} outdoor) was observed among households in treated and
untreated villages. Personal, indoor, and outdoor geometric mean (95\%
confidence interval) concentrations of PM\textsubscript{2.5} were 72
(65, 80), 45 (39, 53), and 31 (28, 35){[}a{]}, respectively, and
elevated relative to health-based guidelines. The current World Health
Organization (WHO) guidelines state that annual average concentrations
of PM\textsubscript{2.5} should not exceed 5 µg/m\textsuperscript{3},
while 24-hour average exposures should not exceed 15
µg/m\textsuperscript{3} for more than 3 to 4 days per year (Organization
2021). Interim targets have been set to support the planning of
incremental milestones toward cleaner air, particularly for cities,
regions, and countries with higher air pollution levels. For
PM\textsubscript{2.5}, the four interim (IT) targets for annual and 24-h
means are: IT-1: 35 and 75 µg/m\textsuperscript{3}; IT-2: 25 and 50
µg/m\textsuperscript{3}; IT-3: 15 and 37.5 µg/m\textsuperscript{3}; and
IT-4: 10 and 25 µg/m\textsuperscript{3} (Organization 2021). In our
study, baseline personal exposures to PM\textsubscript{2.5} fell between
IT-1 and IT-2, indicating considerable opportunity for air quality
improvement from intervention.

\hypertarget{tbl-pm-season}{}
\begin{table}
\caption{\label{tbl-pm-season}Arithmetic and geometric means for air pollutant concentrations
(micrograms per cubic meter) by season. }\tabularnewline

\centering\begingroup\fontsize{9}{11}\selectfont

\begin{tabular}{lllcccccccc}
\toprule
\multicolumn{3}{c}{ } & \multicolumn{2}{c}{Season 1} & \multicolumn{2}{c}{Season 2} & \multicolumn{2}{c}{Season 3} & \multicolumn{2}{c}{Season 4} \\
\cmidrule(l{3pt}r{3pt}){4-5} \cmidrule(l{3pt}r{3pt}){6-7} \cmidrule(l{3pt}r{3pt}){8-9} \cmidrule(l{3pt}r{3pt}){10-11}
 &  &  & Est. & CI & Est. & CI & Est. & CI & Est. & CI\\
\midrule
\addlinespace[0.3em]
\multicolumn{11}{l}{\textbf{Personal}}\\
 &  & Mean & 117 & {}[105, 129] & 97 & {}[87, 107] &  &  & 84 & {}[72, 97]\\
\cmidrule{3-11}
 & \multirow[t]{-2}{*}{\raggedright\arraybackslash 24h PM2.5} & GM & 72 & {}[65, 80] & 59 & {}[53, 65] &  &  & 47 & {}[42, 52]\\
\cmidrule{2-11}
 &  & Mean & 4 & {}[3.5, 4.4] & 3.5 & {}[2.7, 4.2] &  &  & 3.7 & {}[2.9, 4.5]\\
\cmidrule{3-11}
\multirow[t]{-4}{*}[1\dimexpr\aboverulesep+\belowrulesep+\cmidrulewidth]{\raggedright\arraybackslash Filter-derived} & \multirow[t]{-2}{*}{\raggedright\arraybackslash 24h BC} & GM & 2.6 & {}[2.4, 2.8] & 1.9 & {}[1.7, 2.1] &  &  & 1.7 & {}[1.5, 1.9]\\
\cmidrule{1-11}
\addlinespace[0.3em]
\multicolumn{11}{l}{\textbf{Indoor}}\\
 &  & Mean &  &  & 94 & {}[84, 104] & 84 & {}[75, 94] & 67 & {}[60, 75]\\
\cmidrule{3-11}
\multirow[t]{-2}{*}{\raggedright\arraybackslash Sensor-derived} & \multirow[t]{-2}{*}{\raggedright\arraybackslash Seasonal PM2.5} & GM &  &  & 71 & {}[65, 78] & 63 & {}[57, 70] & 47 & {}[42, 52]\\
\cmidrule{1-11}
 &  & Mean &  &  & 69 & {}[59, 79] &  &  & 59 & {}[49, 69]\\
\cmidrule{3-11}
 & \multirow[t]{-2}{*}{\raggedright\arraybackslash 24h PM2.5} & GM &  &  & 45 & {}[39, 53] &  &  & 33 & {}[27, 40]\\
\cmidrule{2-11}
 &  & Mean &  &  & 2.3 & {}[1.8, 2.8] &  &  & 2.8 & {}[2.1, 3.4]\\
\cmidrule{3-11}
\multirow[t]{-4}{*}[1\dimexpr\aboverulesep+\belowrulesep+\cmidrulewidth]{\raggedright\arraybackslash Filter-derived} & \multirow[t]{-2}{*}{\raggedright\arraybackslash 24h BC} & GM &  &  & 1.6 & {}[1.3, 2.0] &  &  & 1.6 & {}[1.3, 1.9]\\
\cmidrule{1-11}
\addlinespace[0.3em]
\multicolumn{11}{l}{\textbf{Outdoor}}\\
 &  & Mean & 47 & {}[45, 48] & 55 & {}[54, 56] & 23 & {}[22, 23] & 33 & {}[32, 34]\\
\cmidrule{3-11}
\multirow[t]{-2}{*}{\raggedright\arraybackslash Sensor-derived} &  & GM & 36 & {}[35, 37] & 40 & {}[39, 41] & 33 & {}[32, 34] & 22 & {}[22, 23]\\
\cmidrule{1-1}
\cmidrule{3-11}
 &  & Mean & 38 & {}[34, 42] & 38 & {}[34, 41] &  &  & 26 & {}[24, 28]\\
\cmidrule{3-11}
 & \multirow[t]{-4}{*}{\raggedright\arraybackslash Seasonal PM2.5} & GM & 33 & {}[29, 36] & 30 & {}[28, 32] &  &  & 22 & {}[21, 24]\\
\cmidrule{2-11}
 &  & Mean & 1.5 & {}[1.3, 1.6] & 1.4 & {}[1.3, 1.5] &  &  & 1.2 & {}[1.1, 1.2]\\
\cmidrule{3-11}
\multirow[t]{-4}{*}[1\dimexpr\aboverulesep+\belowrulesep+\cmidrulewidth]{\raggedright\arraybackslash Filter-derived} & \multirow[t]{-2}{*}{\raggedright\arraybackslash Seasonal BC} & GM & 1.3 & {}[1.1, 1.4] & 1.1 & {}[1.0, 1.2] &  &  & 1 & {}[0.9, 1.1]\\
\bottomrule
\multicolumn{11}{l}{\rule{0pt}{1em}Note: Est. = Estimate, CI = 95\% CI, GM = Geometric Mean}\\
\end{tabular}
\endgroup{}
\end{table}

We also present the geometric and arithmetic means (and 95\% confidence
intervals) for PM\textsubscript{2.5} and BC in each seasonal measurement
campaign (Table~\ref{tbl-pm-season}). Season 3 (2020/2021) was a partial
campaign that took place over a time period impacted by the COVID-19
pandemic and did not involve filter-based air pollution sample
collection.

\hypertarget{policy-uptake}{%
\subsection{Policy uptake}\label{policy-uptake}}

Each year of the study, participants reported the types of fuels and
stoves and the amount of fuel used for space heating in winter. Based on
these data, heating energy types were classified into four categories:
exclusive use of a heat pump (`Heat pump exclusively'), use of a heat
pump and a biomass-fueled kang (`Heat pump with kang'), use of solid
fuel heater with an electric heating devices other than heat pumps
(`Solid fuel with electric heating appliances'), and exclusive use of
solid fuel (`Solid fuel stove exclusively'). In villages treated by the
policy, Figure~\ref{fig-sankey} shows meaningful transitions from solid
fuel to electric-powered heat pumps for all treatment cohorts. For
example, the proportion of households in the group treated in 2019 (S2)
using heat pumps increased from 3\% in S1 to 93\% in S2 and 96\% in S4.
Conversely, use of coal stoves decreased from 97\% in S1 to 8\% in S2
and 3\% in S4. We observed similar stove use transitions for households
in villages treated in 2020 (S3). In the three villages treated in 2021,
we observed overall less exclusive use of the heat pump and a slightly
larger proportion of households continuing to use coal.

\begin{figure}[H]

{\centering \includegraphics[width=1\textwidth,height=\textheight]{images/Sankey diagram_2.png}

}

\caption{\label{fig-sankey}Transitions to different energy sources
across study seasons}

\end{figure}

We also observed a substantial decline in the amount of self-reported
coal used in villages treated by CBHP policy (Appendix
Figure~\ref{fig-afig-coal}), though the reduction in coal use was
smaller with each subsequent treatment cohort (Appendix
Table~\ref{tbl-fuel-did}). Biomass (i.e., wood logs/twigs or charcoal),
usually burned in kangs for both cooking and space heating, was not
expressly targeted by the CBHP policy. We observed declines in
self-reported biomass use in villages treated in 2019 and 2020 but there
was a small increase in biomass consumption in the cohort treated last
(2021).

In never treated villages, we also observed a transition from solid fuel
to clean energy over the four year study but it was much slower than in
treated villages. The proportion of households that reported using
electric heat pumps increased from 5\% in S1 to 10\% in S2 and 25\% in
S4, and those who adopted heat pumps tended to use them exclusively.
Commensurately, the reported expenditures on electricity increased
gradually over time in the untreated villages (Appendix
Figure~\ref{fig-afig-coal}). The percentage of untreated households
using solid fuel with other types of electric devices remained
relatively stable, ranging from 64\% to 70\% across campaigns.
Self-reported use of biomass also remained stable, at approximately one
ton of fuel each winter, whereas exclusive use of solid fuel decreased
from 30\% in S1 to 7\% in S4.

\hypertarget{aim-1-policy-impacts-and-potential-mediation}{%
\subsection{Aim 1: Policy impacts and potential
mediation}\label{aim-1-policy-impacts-and-potential-mediation}}

\hypertarget{impact-of-policy-on-potential-mediators}{%
\subsubsection{Impact of policy on potential
mediators}\label{impact-of-policy-on-potential-mediators}}

In estimating the treatment effect on indoor and outdoor air pollution,
we evaluated both 24-h mean values (specifically, the same 24-h period
when personal exposure samples were collected in each village) and
seasonal mean values (with `season' defined from Jan.~15th to Mar.~15th)
of PM\textsubscript{2.5} data collected in each village. For estimating
the treatment effect on personal exposure to PM\textsubscript{2.5} and
black carbon (BC), the results from the filter-based measurements that
were collected for a 24-h period were used for analysis. We estimated
the basic ETWFE models for outdoor, personal, and indoor measures of air
pollution. ETWFE models were further adjusted for covariates, including
temperature, relative humidity, wind speed, boundary layer height, wind
direction, and the mean quantity of wood burned in each village (for
outdoor measures of air pollution); outdoor temperature, dewpoint,
household smoking status, and the number of residents in each household
(for indoor measures of air pollution); and outdoor temperature,
dewpoint, household smoking status, and the number of residents in each
household (for personal measures of air pollution).

\hypertarget{tbl-did-med}{}
\begin{table}
\caption{\label{tbl-did-med}Treatment effect on outdoor and indoor PM\textsubscript{2.5}, personal
exposure to PM\textsubscript{2.5} and black carbon, and measures of
indoor temperature. Outdoor and indoor PM\textsubscript{2.5} were
derived from sensor measurements after being adjusted based on
co-located gravimetric PM\textsubscript{2.5} measurements. 24h indicates
the mean PM\textsubscript{2.5} concentrations during the 24 hours when
personal exposure samples were collected in each village. `Seasonal'
indicates the seasonal mean PM\textsubscript{2.5} concentrations in each
village, from Jan. 15th to Mar. 15th. }\tabularnewline

\centering
\begin{threeparttable}
\begin{tabular}{llcccc}
\toprule
\multicolumn{2}{c}{ } & \multicolumn{2}{c}{DiD} & \multicolumn{2}{c}{Adjusted DiD\textsuperscript{a}} \\
\cmidrule(l{3pt}r{3pt}){3-4} \cmidrule(l{3pt}r{3pt}){5-6}
  &   & ATT & (95\% CI) & ATT & (95\% CI)\\
\midrule
\addlinespace[0.3em]
\multicolumn{6}{l}{\textbf{Air pollution (µg/m\textsuperscript{3})}}\\
\hspace{1em} & PM2.5 & -2.09 & (-29.38, 25.2) & 1.95 & (-23.34, 27.23)\\
\cmidrule{2-6}
\multirow[t]{-2}{*}[1\dimexpr\aboverulesep+\belowrulesep+\cmidrulewidth]{\raggedright\arraybackslash Personal} & Black carbon & -0.46 & (-1.73, 0.81) & -0.43 & (-1.67, 0.81)\\
\cmidrule{1-6}
\hspace{1em} & Daily & -19.10 & (-60.56, 22.35) & -14.20 & (-53.94, 25.54)\\
\cmidrule{2-6}
\multirow[t]{-2}{*}[1\dimexpr\aboverulesep+\belowrulesep+\cmidrulewidth]{\raggedright\arraybackslash Indoor} & Seasonal & -35.11 & (-59.36, -10.85) & -36.19 & (-60.74, -11.65)\\
\cmidrule{1-6}
\hspace{1em} & Daily & -0.11 & (-5.86, 5.64) & -1.73 & (-9.26, 5.81)\\
\cmidrule{2-6}
\multirow[t]{-2}{*}[1\dimexpr\aboverulesep+\belowrulesep+\cmidrulewidth]{\raggedright\arraybackslash Outdoor} & Seasonal & 3.14 & (-3.1, 9.38) & 0.36 & (-6.27, 6.99)\\
\cmidrule{1-6}
\addlinespace[0.3em]
\multicolumn{6}{l}{\textbf{Indoor temperature (°C)}}\\
\hspace{1em}Point & Mean & 1.96 & (0.96, 2.96) & 1.96 & (0.96, 2.96)\\
\cmidrule{1-6}
\hspace{1em} & Mean (all) & 0.64 & (0, 1.29) & 0.64 & (0, 1.29)\\
\cmidrule{2-6}
\hspace{1em} & Mean (daytime) & 0.82 & (-0.08, 1.72) & 0.82 & (-0.08, 1.72)\\
\cmidrule{2-6}
\hspace{1em} & Mean (heating season) & 1.80 & (0.96, 2.64) & 1.80 & (0.96, 2.64)\\
\cmidrule{2-6}
\hspace{1em} & Mean (daytime heating season) & 1.85 & (0.97, 2.73) & 1.85 & (0.97, 2.73)\\
\cmidrule{2-6}
\hspace{1em} & Min. (all) & 3.83 & (2.26, 5.39) & 3.83 & (2.26, 5.39)\\
\cmidrule{2-6}
\multirow[t]{-6}{*}[5\dimexpr\aboverulesep+\belowrulesep+\cmidrulewidth]{\raggedright\arraybackslash Seasonal} & Min. (heating season) & 3.72 & (2.19, 5.25) & 3.72 & (2.19, 5.25)\\
\bottomrule
\end{tabular}
\begin{tablenotes}
\item \small{Note: ATT = Average Treatment Effect on the Treated, DiD = Difference-in-Differences, ETWFE = Extended Two-Way Fixed Effects.}
\item[a] \small{ETWFE models for air pollution outcomes were adjusted for household size, smoking, outdoor temperature, and outdoor humidity. Temperature models not additionally adjusted.}
\end{tablenotes}
\end{threeparttable}
\end{table}

Treatment was associated with similar reductions in both seasonal and
24-h indoor PM\textsubscript{2.5} means (Table~\ref{tbl-did-med}). The
average marginal effect (ATT) from the basic ETWFE model shows that
exposure to the CBHP policy reduced 24h indoor PM\textsubscript{2.5} by
-19 µg/m\textsuperscript{3} (95\%CI: -23, 61). After adjusting outdoor
temperature, dewpoint, household smoking status, and the number of
residents in each household, the ATT decreased to -14
µg/m\textsuperscript{3} (95\%CI: -54, 26). The impact was stronger on
seasonal indoor PM\textsubscript{2.5}, with an average ATT of -36
µg/m\textsuperscript{3} (95\%CI: -61, -12) that was robust to covariate
adjustment. This finding likely reflects the direct benefit of the
policy in replacing coal stoves and air quality improvement.

Overall we found little evidence of an impact of the CBHP policy on 24-h
and seasonal outdoor (local community-level) PM\textsubscript{2.5} or
personal exposures to PM\textsubscript{2.5} and BC. Treatment was
associated with lower, but statistically imprecise, personal 24-h BC
exposures. This finding would be consistent with the expectation that
the policy contributed to reducing air pollutant emissions from solid
fuel burning, as BC serves as a potential indicator of such combustion,
particularly in our rural and peri-urban study villages.

\hypertarget{impact-of-policy-on-health-outcomes}{%
\subsubsection{Impact of policy on health
outcomes}\label{impact-of-policy-on-health-outcomes}}

\hypertarget{tbl-did-health}{}
\begin{table}
\caption{\label{tbl-did-health}Overall impacts of the `coal-to-clean energy' policy on blood pressure,
respiratory outcomes, and inflammatory markers }\tabularnewline

\centering
\begin{threeparttable}
\begin{tabular}{llcccc}
\toprule
\multicolumn{2}{c}{ } & \multicolumn{2}{c}{DiD} & \multicolumn{2}{c}{Adjusted DiD\textsuperscript{a}} \\
\cmidrule(l{3pt}r{3pt}){3-4} \cmidrule(l{3pt}r{3pt}){5-6}
  &   & ATT & (95\% CI) & ATT & (95\% CI)\\
\midrule
\addlinespace[0.3em]
\multicolumn{6}{l}{\textbf{Blood pressure (mmHg)}}\\
\hspace{1em} & Brachial & -0.79 & (-2.63, 1.04) & -1.40 & (-3.31, 0.51)\\
\cmidrule{2-6}
\multirow[t]{-2}{*}[1\dimexpr\aboverulesep+\belowrulesep+\cmidrulewidth]{\raggedright\arraybackslash Systolic BP} & Central & -1.04 & (-2.82, 0.73) & -1.56 & (-3.40, 0.28)\\
\cmidrule{1-6}
\hspace{1em} & Brachial & -1.29 & (-2.62, 0.04) & -1.60 & (-2.96, -0.25)\\
\cmidrule{2-6}
\multirow[t]{-2}{*}[1\dimexpr\aboverulesep+\belowrulesep+\cmidrulewidth]{\raggedright\arraybackslash Diastolic BP} & Central & -1.35 & (-2.66, 0.04) & -1.66 & (-2.97, -0.34)\\
\cmidrule{1-6}
\hspace{1em} & Brachial & 0.50 & (-0.71, 1.70) & 0.21 & (-1.00, 1.41)\\
\cmidrule{2-6}
\multirow[t]{-2}{*}[1\dimexpr\aboverulesep+\belowrulesep+\cmidrulewidth]{\raggedright\arraybackslash Pulse Pressure} & Central & 0.31 & (-0.85, 1.46) & 0.10 & (-1.01, 1.20)\\
\cmidrule{1-6}
\hspace{1em} & Pulse pressure & 0.10 & (-0.12, 1.40) & 0.00 & (-1.20, 1.20)\\
\cmidrule{2-6}
\multirow[t]{-2}{*}[1\dimexpr\aboverulesep+\belowrulesep+\cmidrulewidth]{\raggedright\arraybackslash BP Amplification x10} & Systolic BP & 0.20 & (-0.20, 0.50) & 0.10 & (-0.20, 0.40)\\
\cmidrule{1-6}
\addlinespace[0.3em]
\multicolumn{6}{l}{\textbf{\makecell[l]{Respiratory\\outcomes}}}\\
\hspace{1em} & Any symptom & -7.38 & (-13.98, -0.77) & -7.86 & (-14.63, -1.09)\\
\cmidrule{2-6}
\hspace{1em} & Coughing & -1.59 & (-6.41, 3.23) & -1.98 & (-6.8, 2.84)\\
\cmidrule{2-6}
\hspace{1em} & Phlegm & -1.22 & (-5.58, 3.15) & -1.82 & (-6.34, 2.69)\\
\cmidrule{2-6}
\hspace{1em} & Wheezing attacks & -0.22 & (-3.97, 3.52) & -0.14 & (-3.85, 3.57)\\
\cmidrule{2-6}
\hspace{1em} & Trouble breathing & -4.98 & (-11.81, 1.84) & -4.62 & (-11.59, 2.35)\\
\cmidrule{2-6}
\multirow[t]{-6}{*}[5\dimexpr\aboverulesep+\belowrulesep+\cmidrulewidth]{\raggedright\arraybackslash Self-reported (pp)} & Chest trouble & -6.63 & (-12.51, -0.76) & -6.36 & (-12.14, -0.59)\\
\cmidrule{1-6}
\hspace{1em}Measured & FeNO (ppb) & 0.17 & (-2.24, 2.58) & 0.55 & (-2.13, 3.13)\\
\cmidrule{1-6}
\addlinespace[0.3em]
\multicolumn{6}{l}{\textbf{\makecell[l]{Inflammatory\\markers (\%)}}}\\
\hspace{1em} & IL6 & 6.80 & (-12.2, 30.0) & 5.90 & (-13.8, 30.2)\\
\cmidrule{2-6}
\hspace{1em} & TNF-alpha & 24.30 & (-1.3, 56.4) & 24.70 & (-0.9, 54.2)\\
\cmidrule{2-6}
\hspace{1em} & CRP & 2.70 & (-19.8, 31.6) & 3.80 & (-19.4, 33.6)\\
\cmidrule{2-6}
\multirow[t]{-4}{*}[3\dimexpr\aboverulesep+\belowrulesep+\cmidrulewidth]{\raggedright\arraybackslash } & MDA & 7.60 & (-8.7, 26.9) & 6.50 & (-9.7, 25.5)\\
\bottomrule
\end{tabular}
\begin{tablenotes}
\item \small{Note: ATT = Average Treatment Effect on the Treated, DiD = Difference-in-Differences, pp = percentage points, ppb = parts per billion.}
\item[a] \small{Blood pressure models adjusted for age, sex, waist circumference, smoking, alcohol consumption, and use of blood pressure medication.}
\end{tablenotes}
\end{threeparttable}
\end{table}

Table~\ref{tbl-did-health} shows the impacts of the policy on blood
pressure in basic ETWFE models and models further adjusted for age, sex,
waist circumference, smoking, alcohol consumption, and use of blood
pressure medication. Overall exposure to the CBHP policy demonstrated
reductions in blood pressure of approximately 1.5 mmHg for both systolic
and diastolic BP, but we found little evidence of a meaningful impact on
pulse pressure or BP amplification. The effects on brachial and central
blood pressures were similar.

Table~\ref{tbl-did-health} shows the impacts on self-reported chronic
respiratory symptoms categorized as any symptoms and separately for each
individual symptom type. In both basic and covariate-adjusted ETWFE
models, exposure to the CBHP policy reduced self-report of any poor
respiratory symptoms by around 7 percentage points. This was largely
through reductions in reports of having chest trouble or difficulty
breathing on several or most days of the week.

Table~\ref{tbl-did-health} also shows the impacts of the CBHP on
measured airway inflammation (FeNO), which was conducted in a sub-sample
of 511 participants, including 274 participants with one measurement,
142 with two measurements, 95 participants with 3 measurements. We did
not find evidence that exposure to the policy affected changes in FeNO
in the covariate-adjusted ETWFE model (0.5 ppb, 95\%CI: -2.1, 3.1).
There was some evidence of heterogeneity in the FeNO effects of the
policy by treatment cohort, though the confidence intervals for each of
the cohort-specific effects were large and overlapping. Our results did
not change with sensitivity analyses that included a log-transformed
FeNO outcome and limiting the analysis to participants with at least two
repeated measurements and to those who participated in all three
campaigns (SI Table X)

\hypertarget{mediated-impact-on-blood-pressure}{%
\subsubsection{Mediated impact on blood
pressure}\label{mediated-impact-on-blood-pressure}}

As noted above, we aimed to assess whether any health impacts of the
CBHP policy may work specifically through pathways involving changes in
PM\textsubscript{2.5} and indoor temperature. Below we show results from
several mediation models. We evaluated potential mediation for each
mediator (indoor temperature and personal exposure to
PM\textsubscript{2.5}) separately and in a single model accounting for
multiple mediators, and we set the values of both mediators to the WHO
mean annual interim PM\textsubscript{2.5} and indoor temperature
guidelines. For mediation analysis, we focused on BP outcomes for which
we observed an effect of the policy. In Table~\ref{tbl-bp-med} we show
that conditioning on indoor PM and indoor temperature largely explains
the entire total effect of the CBHP policy on blood pressure for
systolic BP, and roughly half of the total effect for diastolic BP.

\hypertarget{tbl-bp-med}{}
\begin{table}[H]
\caption{\label{tbl-bp-med}Controlled direct effects for the CBHP policy }\tabularnewline

\centering\begingroup\fontsize{10}{12}\selectfont

\begin{threeparttable}
\begin{tabular}{lcccccclc}
\toprule
\multicolumn{1}{c}{ } & \multicolumn{2}{c}{Adjusted Total Effect\textsuperscript{a}} & \multicolumn{6}{c}{CDE Mediated By:\textsuperscript{b}} \\
\cmidrule(l{3pt}r{3pt}){2-3} \cmidrule(l{3pt}r{3pt}){4-9}
\multicolumn{3}{c}{ } & \multicolumn{2}{c}{Indoor PM} & \multicolumn{2}{c}{Indoor Temp} & \multicolumn{2}{c}{PM + Temp} \\
\cmidrule(l{3pt}r{3pt}){4-5} \cmidrule(l{3pt}r{3pt}){6-7} \cmidrule(l{3pt}r{3pt}){8-9}
 & ATT & (95\%CI) & ATT & (95\%CI) & ATT & (95\%CI) & ATT & (95\%CI)\\
\midrule
Brachial SBP & -1.40 & (-3.31, 0.51) & -1.05 & (-3.12, 1.02) & -0.46 & (-2.29, 1.36) & -0.03 & (-2.04, 1.97)\\
Central SBP & -1.56 & (-3.40, 0.28) & -1.15 & (-3.20, 0.89) & -0.68 & (-2.36, 1.00) & -0.20 & (-2.11, 1.70)\\
Brachial DBP & -1.60 & (-2.96, -0.25) & -1.40 & (-2.97, 0.16) & -1.14 & (-2.33, 0.06) & -0.88 & (-2.30, 0.54)\\
Central DBP & -1.66 & (-2.97, -0.34) & -1.40 & (-2.96, 0.16) & -1.32 & (-2.50, -0.14) & -1.02 & (-2.45, 0.41)\\
\bottomrule
\end{tabular}
\begin{tablenotes}
\item \small{Note: Results combined across 30 multiply-imputed datasets. ATT = Average Treatment Effect on the Treated, CDE = Controlled Direct Effect, DBP = Diastolic blood pressure, SBP = Systolic blood pressure.}
\item[a] \small{Adjusted for age, sex, waist circumference, smoking, alcohol consumption, and use of blood pressure medication.}
\item[b] \small{Mediators were set to the mean value for untreated participants at baseline.}
\end{tablenotes}
\end{threeparttable}
\endgroup{}
\end{table}

\hypertarget{aim-2-source-contributions}{%
\subsection{Aim 2: Source
contributions}\label{aim-2-source-contributions}}

Source analysis for this study was conducted using data from all
eligible outdoor PM and personal PM samples. Eligible samples were those
for which PM\textsubscript{2.5} mass and chemical components were
quantified. We evaluated factors contributing to community-outdoor and
personal exposure PM\textsubscript{2.5} using the U.S. EPA's source
apportionment model PMF (positive matrix factorization) 5.0, which has
been widely used for similar analyses in China (Gao et al.~2018; Liu et
al.~2017; Tao et al.~2017). As an optimum PMF result depends on the
appropriate number of input factors, sensitivity analysis using a range
of factors (e.g., range of 3 to 7 factors, based on a combination of the
species that we have and our field-based observations and sources that
have been identified previously in our study region) were conducted to
examine the impact of a different number of factors on the model
results. Detailed information on the procedures of PMF analysis can be
found elsewhere (Wang et al.~2016; Zíková et al.~2016). Briefly, the
scree plot from our principal component analysis indicated that
solutions of between 3 and 5 factors (+/- 1) would be most appropriate,
further supporting our evaluation of 3 to 6 factor solutions from PMF.
As there was no indication that even moving from five factors to six
factors would improve our solution; therefore, a seven factor solution
would not make sense to investigate further
(Figure~\ref{fig-source-figure}).

The chemical analysis data used as the input for the PMF model were
dispersion normalized prior to inclusion in the model. PMF works by
using covariance of compositional variables to separate sources of
ambient PM. However, atmospheric dilution also induces covariance.
Dilution can be quantified in terms of a ventilation coefficient (VC)
and used to normalize the input chemical concentrations and
uncertainties in the original data matrix on a sample by sample basis.
The dispersion normalized concentrations and uncertainties are used as
the input to PMF analysis. Dispersion normalization, as conducted in
this study, is a relatively new application of this conceptual framework
(Dai et al. 2020), developed to adjust for wind speed (dispersion in the
x-y plane) and boundary layer height (dispersion in the z-axis). This
process involves first calculating the sample specific ventilation
coefficient by multiplying the average wind speed by the average
boundary layer height over the sampling duration. The average
ventilation coefficient is also calculated for the village by averaging
all the ventilation coefficients. The dispersion normalized
concentration for any species in any sample is equal to the species
concentration in that sample multiplied by the ventilation coefficient
for that sample and divided by the average ventilation coefficient for
that village. Dividing by the average ventilation coefficient for that
village helps curtail any extreme concentrations driven by an outlier in
the sample ventilation coefficient.

The meteorological data included hourly boundary layer height, 2-m
temperature, 2-m dew point temperature, and 2-m horizontal wind speed
components (u, v), which were obtained from the European Center for
Midrange Weather Forecasting ERA5 reanalysis dataset (0.25 x 0.25
resolution). Values of these meteorological variables were determined at
the village-level by identifying the four surrounding grid points with
values available from the ERA5 reanalysis, and then applying inverse
distance weighted interpolation from those four grid points to the
village. Percent relative humidity was calculated from the 2-m dew point
temperature using the ``weathermetrics'' package (version 1.2.2) in R.
Total hourly wind speed and wind direction were calculated from the
horizontal wind speed components.

The model diagnostics for the three- to six-factor PMF solutions are
given in Table~\ref{tbl-pmf}. Model fit was assessed using Q/Qexp (how
our model fit divided by the expected fit). As the change in Q/Qexp
decreases as more factors are added, the model may be fitting additional
sources that do not improve the overall fit. The largest change in
Q/Qexp was from three to four sources (6.24 to 5.37) while the changes
moving from four to five and five to six were similar, which suggests
that four factors is sufficient and parsimonious to explain the
variation in our data. We assessed the random error in our model by
randomly sampling blocks of data, fitting new models with the blocks,
and comparing how the source profiles compared to that of the original
model (bootstrap (BS) mapping). The three- and four-factor solutions had
high BS mapping (all factors found in \textgreater{} 96.5\% of bootstrap
runs). The additional sources identified in the five-factor (lead) and
six-factor (chloride) solutions have low BS mapping (\textgreater{}
72\%), which means those solutions are not as consistent as the three-
and four-factor solutions. The possibility that multiple, different,
solutions could result in the same Q value was assessed using
displacement. The displacement approach takes the original factor
profiles and modifies the values for each species up or down to maintain
a small change in Q, reruns the solution with the new species values,
and then compares the profiles of the new model to the original. Any
swaps indicate that small changes in the species values could result in
factor profiles that look different from the original solution, and that
the original solution is unstable. None of the factors in any of the
solutions discussed were swapped during displacement, which indicates
that all of the potential solutions are stable. Based on the Q/Qexp, BS
mapping, and interpretability of the factors, the four-factor solution
was selected as the most appropriate source solution for the data.

\hypertarget{tbl-pmf}{}
\begin{table}
\caption{\label{tbl-pmf}PMF error estimation diagnostics }\tabularnewline

\centering\begingroup\fontsize{9}{11}\selectfont

\begin{tabular}{l>{\raggedright\arraybackslash}p{10em}>{\raggedright\arraybackslash}p{10em}>{\raggedright\arraybackslash}p{10em}>{\raggedright\arraybackslash}p{10em}}
\toprule
\multicolumn{1}{c}{ } & \multicolumn{4}{c}{Potential Factor Solution} \\
\cmidrule(l{3pt}r{3pt}){2-5}
Diagnostic & 3 & 4 & 5 & 6\\
\midrule
Qexp & 27936 & 26052 & 24168 & 22284\\
Qtrue & 187681 & 147796 & 123236 & 100316\\
Qrobust & 174407 & 139910 & 117082 & 95932.5\\
Qr/Qexp & 6.24 & 5.37 & 4.84 & 4.3\\
Q/Qexp > 6 & wi-Ca, ns-S, ws-Na, ws-Ca, Al, Cl, Pb & ns-S, Na, Al, Cl, Pb, Nitrate & Nitrate, ws-Na, Al, Chloride & Nitrate, ws-Na, Al\\
DISP \% dQ & <0.1\% & <0.1\% & <0.1\% & <0.1\%\\
DISP swaps & 0 & 0 & 0 & 0\\
BS\_mapping & Dust- 98.5\% & Transported dust- 95\%, Dust- 96.5\%, Sulfur secondary- 97.5\%, Mixed combustion- 96.5\% & Transported dust- 86\%, Mixed combustion- 87\%, Dust- 86\%, Lead- 55\% & Transported dust- 84\%, Mixed combustion- 87.5\%, Dust- 81.5\%, Lead- 72\%
Chloride- 61.5\%
Sulfur secondary- 98.5\%\\
\bottomrule
\end{tabular}
\endgroup{}
\end{table}

The source profiles for the four-factor solution are presented in
Figure~\ref{fig-source-figure}. The first source was identified as dust
based on high percentages of crustal elements like wi-Ca, Si, and wi-Mg.
The second source consisted of non-sulfate sulfur as well as secondary
inorganic ions (ammonium, nitrate, and sulfate). Non-sulfate sulfur is a
tracer for primary coal combustion, while secondary inorganic ions
indicate a secondary source. Since industrial coal burning is a source
of power generation in our study area, it is likely that the second
source is a mixture of primary and secondary emissions that originate
from coal and other sulfurous fuel combustion. Additionally, the mean
source contribution of the second source is higher in outdoor than
personal exposure measurements. Secondary formation occurs outdoors in
the presence of sunlight, so higher outdoor concentrations compared to
personal exposure further support our naming the second source `sulfur
secondary'. The third source had high percentages of ws-Ca nd Al, which
in our study region, has been found to be indicative of transported dust
from dust storms that can occur in the spring. While our samples were
collected during winter months only, it is possible that transported
dust from previous years still remained. The fourth source was
characterized by high percentages of tracers for both coal (OC, wi-K,
chloride, Pb) and biomass combustion (EC, ws-K). Coal and biomass
combustion are anticipated sources of PM pollution in our study setting,
particularly from domestic cooking and heating activities, so this
source is likely a mixture of PM emitted from these two household
combustion sources.

\begin{figure}[H]

{\centering \includegraphics[width=0.8\textwidth,height=\textheight]{images/source-figure.png}

}

\caption{\label{fig-source-figure}Source profiles for the 4-factor PMF
solution to the sum of elements, ions, elemental carbon, and organic
carbon for outdoor and personal PM\textsubscript{2.5} exposure
measurements. The lines separate the major contributing species to each
source}

\end{figure}

We extend the source profiles across the different treatment cohorts in
Figure~\ref{fig-source-season}.

\begin{figure}[H]

{\centering \includegraphics[width=0.8\textwidth,height=\textheight]{images/source-season.png}

}

\caption{\label{fig-source-season}Arithmetic mean dispersion normalized
source contributions found from the 4-factor PMF solution for A outdoor
and B personal PM\textsubscript{2.5} exposure samples by year the group
received treatment.}

\end{figure}

Overall, Table~\ref{tbl-source-did} shows the average treatment effect
of the CBHP policy on outdoor (community-level) and personal exposure to
the mixed combustion source was statistically indistinguishable from the
null. Treatment was associated with lower, but statistically imprecise,
personal exposures to the mixed combustion source. As with BC, this
finding is consistent with the expectation that the policy contributed
to reducing air pollutant emissions from solid fuel burning, as this
`mixed combustion' source most likely reflects solid fuel combustion,
particularly in our study settings. The results were consistent in the
unadjusted and adjusted models.

\hypertarget{tbl-source-did}{}
\begin{table}
\caption{\label{tbl-source-did}Average treatment effect (µg/m\textsuperscript{3}) for outdoor and
personal exposure to the mixed combustion source. }\tabularnewline

\centering
\begin{tabular}{lcccc}
\toprule
\multicolumn{1}{c}{ } & \multicolumn{2}{c}{DiD} & \multicolumn{2}{c}{Adjusted DiD\textsuperscript{a}} \\
\cmidrule(l{3pt}r{3pt}){2-3} \cmidrule(l{3pt}r{3pt}){4-5}
  & ATT & (95\% CI) & ATT & (95\% CI)\\
\midrule
Outdoor & 1.07 & (-4.90, 7.04) & 1.53 & (-4.19, 7.26)\\
Personal exposure & -5.60 & (-13.70, 2.54) & -5.39 & (-13.1, 2.35)\\
\bottomrule
\multicolumn{5}{l}{\rule{0pt}{1em}\textsuperscript{a} \small{Note: Models adjusted for ???}}\\
\end{tabular}
\end{table}

When the average treatment effect of the CBHP policy on outdoor
(community-level) and personal exposure to the mixed combustion source
was were allowed to vary by treatment year and time, the treatment
effect for households most recently treated (i.e., treated in the final
season, Season 4) was associated with lower personal exposures to the
mixed combustion source (Appendix Figure~\ref{fig-afig-mixed-ct}). In
each season, treatment by the CBHP policy was associated with a
reduction in the source contribution to personal PM\textsubscript{2.5}
mass from the mixed combustion source; however, for villages treated in
Seasons 2 and 3, the effect was statistically imprecise. Treatment was
not associated with a reduction or an increase in the source
contribution to community outdoor PM\textsubscript{2.5} mass from the
mixed combustion source. That personal exposure measures of this
specific air pollution source were more indicative of treatment effect
than community outdoor measures of the same air pollution source is
consistent with the expectation that this source, which we determined to
be indicative of a mixture of coal and biomass combustion, is most
characteristic of household use of solid fuels, including coal and
biomass, which would produce emissions that are likely to be nearer to
the people using those fuels than near to the centrally located
community outdoor air samplers.

\hypertarget{aim-3-mediation-by-source-contribution}{%
\subsection{Aim 3: Mediation by source
contribution}\label{aim-3-mediation-by-source-contribution}}

Table~\ref{tbl-med-source} shows results from the mediation analysis by
personal exposure to the mixed combustion source (coal and biomass),
estimated for the subset of participants with personal exposure
measurements. The CDE in this model estimates the impact of exposure to
the CBHP policy on central and systolic blood pressure while holding
constant values of mixed combustion source at the mean baseline values
for untreated population. The marginal policy effects (ATTs) from the
adjusted ETWFE models for this subset of participants were largely
similar to those from the full sample for central SBP (around a 1.6 mmHg
decrease), but slightly smaller for central DBP (-1.7 mmHg in the full
sample vs.~-1.3 in the subset with personal exposure measurements) and
were estimated with greater imprecision. We found little evidence that
these treatment effects were meaningfully mediated by exposure to the
mixed combustion source, as the controlled direct effects were generally
of similar magnitude as the adjusted total effects.

\hypertarget{tbl-med-source}{}
\begin{table}
\caption{\label{tbl-med-source}Average treatment effects and controlled direct effect (mm/Hg) of the
CBHP policy on central systolic and diastolic blood pressure with mixed
combustion source as the potential mediator. }\tabularnewline

\centering
\begin{threeparttable}
\begin{tabular}{lcccccc}
\toprule
\multicolumn{1}{c}{ } & \multicolumn{2}{c}{DiD} & \multicolumn{2}{c}{Adjusted DiD\textsuperscript{a}} & \multicolumn{2}{c}{Adjusted CDE\textsuperscript{b}} \\
\cmidrule(l{3pt}r{3pt}){2-3} \cmidrule(l{3pt}r{3pt}){4-5} \cmidrule(l{3pt}r{3pt}){6-7}
 & ATT & (95\%CI) & ATT & (95\%CI) & ATT & (95\%CI)\\
\midrule
Central SBP & -0.43 & (-3.48, 2.62) & -1.61 & (-5.03, 1.82) & -1.5 & (-4.94, 1.94)\\
Central DBP & -0.58 & (-2.59, 1.43) & -1.26 & (-3.38, 0.87) & -1.37 & (-3.69, 0.96)\\
\bottomrule
\end{tabular}
\begin{tablenotes}
\item \small{Note: ATT = Average Treatment Effect on the Treated, DiD = Difference-in-Differences, CDE = Controlled Direct Effect, DBP = Diastolic Blood Pressure, SBP = Systolic Blood Pressure.}
\item[a] \small{Adjusted for age, sex, waist circumference, smoking, alcohol consumption, and use of blood pressure medication.}
\item[b] \small{Further adjusted for mediation by mixed combustion source (coal and biomass)}
\end{tablenotes}
\end{threeparttable}
\end{table}

\hypertarget{discussion-and-conclusions}{%
\section{Discussion and Conclusions}\label{discussion-and-conclusions}}

Air pollution emitted from residential space heating with coal has
historically been a major contributor to cardio-respiratory disease
burden in northern China (Archer-Nicholls et al. 2016; Yun et al. 2020).
Since the introduction of its 13th 5-Year-Plan (2016-2020), China has
successfully implemented numerous large-scale measures to improve air
quality including programs that incentivize rural household transition
from solid fuels to clean energy sources (Young et al. 2015). The CBHP
policy is among the largest and most ambitious household energy policies
implemented anywhere in the world in recent decades, and its staggered
roll-out provided us with a unique opportunity to prospectively evaluate
this real-world experiment and its effects on air quality and health.

\hypertarget{adoption-of-the-heat-pump-technology-and-adherence-to-the-policy}{%
\subsection{Adoption of the heat pump technology and adherence to the
policy}\label{adoption-of-the-heat-pump-technology-and-adherence-to-the-policy}}

The CBHP policy was successful in driving a rapid household heating
energy transition from coal to electric heating in our treated villages.
There was high uptake and consistent use of the new heat pump technology
and large reductions in coal use in treated villages starting in the
first year post-treatment and continuing into the third year for
villages treated in 2019. We enrolled rural and peri-urban villages
across a wide geographic area and socioeconomic spectrum in Beijing and
observed near universal adoption of the heat pump technologies and
suspension of coal stove use across the different treatment groups and
campaigns. This contrasts with many previous household energy
intervention studies, including several randomized trials, where low
fidelity and compliance with the intervention were considered a major
limitation to achieving their intended air quality or health benefits
(Ezzati and Baumgartner 2017; Harrison et al. Approved February 2024;
Rosenthal et al. 2018).

A number of factors contribute to the successful uptake of the new
technology and adherence to the policy. The initial uptake of the heat
pump was influenced by broad support and perceived benefits of village
and household participation in the policy. At baseline assessment, 49 of
50 village committee interviewees indicated a desire to participate in
the policy by the committee members and their constituents, for reasons
including ease of use of the heat pump and the convenience of no longer
having to add coal throughout the day and night, the desire for a
cleaner local environment, and perceived lower risk of carbon monoxide
poisoning compared with coal stoves. While the availability and cost of
clean fuels were barriers to clean fuel adoption in previous studies
(Rehfuess et al. 2014), in our study, both the upfront costs of the heat
pump and electricity were heavily subsidized, which limited the
financial burden of transition for households. Post-policy
implementation, treated villages no longer had access to
government-subsidized coal, and coal burning was further discouraged
with possible punitive measures (e.g., potential loss of electricity
subsidies).

\hypertarget{impacts-of-the-policy-on-health}{%
\subsection{Impacts of the policy on
health}\label{impacts-of-the-policy-on-health}}

One of the key findings from our comprehensive evaluation of the CBHP
policy was that exposure to the policy reduced systolic and diastolic
blood pressure by \textasciitilde1.5 mmHg, and that most of the observed
BP effects were mediated by improvements in the indoor environment,
specifically reductions in indoor PM\textsubscript{2.5} and increases in
indoor temperature. The total effects of the policy are supported by a
small number of randomized trials of cooking gas or more efficient
biomass cookstoves showing similar or larger reductions in blood
pressure (SBP: -2.1 to -1.3 mmHg; DBP: -0.1 to -3.0 mmHg) (Kumar et al.
2021). In contrast, a recent multi-country randomized trial of an LPG
stove observed a small (\textasciitilde0.6 mmHg) increase in gestational
blood pressure (Ye et al. 2022) despite decreases in exposure to
PM\textsubscript{2.5} that were much larger than in our study. Though,
the study participants were much younger than our study (mean age of 25
versus 61y) and gas stoves can still emit health-damaging air pollutants
like benzine and volatile organic compounds (Kashtan et al. 2023),
especially in contrast with the zero-emission electric heaters
introduced in our study villages. Further, our findings of temperature-
and air quality-mediated impacts of the policy on BP are supported by
observational studies conducted by us and others showing that increased
exposure to household air pollution (Baumgartner et al. 2018, 2011; Dong
et al. 2013; Kanagasabai et al. 2022) and to colder indoor temperatures
(Lv et al. 2022; Sternbach et al. 2022) are associated with higher blood
pressure in rural and peri-urban areas of China, with exposure-response
estimates that align with our mediator estimates.

We did not observe effects of the policy on blood pressure measures of
PP or cPP/SBP amplification. Pulse pressure is measured as the
difference between SBP and DBP, and represents the pulsatile component
of blood flow (Dart and Kingwell 2001). Thus, increases in PP can result
from increases in SBP, decreases in DBP, or both. The lack of effect on
PP in our study is likely attributed to the near identical reductions in
SBP and DBP from the policy. Similarly, PP/SBP amplification is measured
as a ratio of peripheral to central pressures, and the decreases in
central and brachial pressures with the policy were also nearly
identical in our study. Although the duration of our study was nearly
twice as long as most previous household stove intervention studies, it
is still possible that longer-term reductions in BP are required to
observe any structural changes in the caliber or elasticity of arterial
walls that would be reflected in differences in measures PP or SBP/PP
amplification (Dart and Kingwell 2001).

Exposure to the CBHP policy also reduced self-report of any poor
respiratory symptoms (\textasciitilde7 percentage points) with most of
these effects driven by reductions in self-reported chest trouble or
difficulty breathing on several or most days of the week. In Guatemala
exposure to carbon monoxide (used as a surrogate for exposure to
PM\textsubscript{2.5}) and prevalence of chronic respiratory symptoms,
especially wheeze, was reduced among women who received a biomass
chimney stove (i.e., plancha) (Smith-Sivertsen et al. 2009).
{[}{[}{[}Note to Jill: still need to review and discuss results from
trials with respiratory symptoms: Fandino-Del-Rio et al., 2022, Romieu
et al., 2009; Schilmann et al., 2015, Add in null findings from Burwen
et al., 2012, Beltramo et al.,2013 {]}{]}{]}

We found some evidence of heterogeneity in the health benefits of the
policy by treatment cohort, specifically a small increase in BP (2.4
mmHg; 95\%CI: -0.5, 5.3) and self-reported wheeze events (+9 percentage
points; 95\%CI: 0, 18) in the three villages treated in 2021. Notably
this is also the treatment cohort with the smallest decline in point
temperature at the time of BP measurement and both an increase in
self-reported biomass use and several households that continued using
coal. Rather paradoxically, we observed a larger decrease in
PM\textsubscript{2.5} and mixed solid fuel use in this group. It is
possible that the composition of PM and mixed solid fuel was different
in this cohort, with a greater contribution of biomass smoke, however we
are unable to differentiate between biomass and coal in our `mixed solid
fuel' category. This group was also treated during the pandemic, which
could have impacted how the policy was introduced or resulted in changes
to other BP risk factors that we did not evaluate in our study, e.g.,
changes in diet or level of social capital.

We also did not observe impacts of the policy on blood biomarkers of
inflammation and oxidative stress in the sub-sample of participants with
blood collection in S1 and S2. Our results contrast with a natural
experiment in urban Beijing that showed large regional and local air
quality reductions during the 2008 Beijing Olympics and also observed
benefits to airway inflammation (Huang et al. 2012) and blood markers of
inflammation and oxidative stress in healthy urban Beijing residents
during the Olympics compared with before and after (Rich et al. 2012).
Our mediation analysis indicated that the blood pressure effects of the
policy were mediated more through indoor temperature than air pollution.
Although observational studies from rural northern China do show impacts
of exposure to temperature on inflammation and oxidative stress (Wang et
al. 2020; Xu et al. 2019), it's possible that the relatively small
increases in mean indoor temperature in treated households were not
sufficiently large to capture measurable changes in these biomarkers.

\hypertarget{impacts-of-the-policy-on-air-pollution-and-is-sources}{%
\subsection{Impacts of the policy on air pollution and is
sources}\label{impacts-of-the-policy-on-air-pollution-and-is-sources}}

China has a long history of launching ambitious, large-scale policies
and programs to promote clean household energy transition and support
rural energy infrastructure development (Zhang and Smith 2007). The
country was a relatively early initiator of rural electrification
projects in the 1950s and achieved complete (100\%) electrification of
households by 2016 (Yang 2021), which undoubtedly facilitated the policy
choice to replace coal stoves with electric-powered heat pump heaters.
Several decades earlier, China achieved what is likely the largest
improvement in energy efficiency in history in terms of the population
affected by just one program. The National Improved Stove Program (NISP)
and its provincial counterparts were initiated in the early 1980s and
are credited with introducing nearly 200 million improved cooking and
heating stoves by the late 1990s. The NISP implemented mostly chimney
stoves with the primary goal of increased fuel efficiency to reduce
pressure on local forests and a secondary goal of reducing indoor
pollution. NISP biomass stoves showed some success in reducing indoor
PM, though levels were still much higher than health-motivated
guidelines, however the program's so called `improved' coal stoves
provided no measurable air quality benefit (Sinton et al. 2004).

In contrast, our evaluation of the CBHP policy revealed a quantifiable
improvement in indoor air quality, evidenced by a reduction of 36
µg/m\textsuperscript{3} in seasonal measures of indoor
PM\textsubscript{2.5}. Compared to the current World Health Organization
(WHO) guidelines, which state that annual (and 24-h) average
concentrations of PM\textsubscript{2.5} should not exceed 5 (and 15)
µg/m\textsuperscript{3} (Organization 2021), homes at baseline in this
study had indoor PM\textsubscript{2.5} in the range of interim targets
(ITs) 1 and 2. Interim targets have been set by WHO to support the
planning of incremental milestones toward cleaner air, particularly for
cities, regions, and countries with higher air pollution levels. For
PM\textsubscript{2.5}, the four interim (IT) targets for annual and 24-h
means are: IT-1: 35 and 75 µg/m\textsuperscript{3}; IT-2: 25 and 50
µg/m\textsuperscript{3}; IT-3: 15 and 37.5 µg/m\textsuperscript{3}; and
IT-4: 10 and 25 µg/m\textsuperscript{3} (Organization 2021). Indoor
PM\textsubscript{2.5} concentrations were reduced in treated homes,
bringing these homes into the range of IT-4{[}b{]} and realizing some of
the considerable potential for air quality improvement from the
intervention.

The still elevated levels of indoor and personal exposures in treated
homes, despite high compliance with the policy, is most likely
attributable to two main factors: the elevated levels of ambient air
pollution in our study setting (range: 26-38 μg/m\textsuperscript{3} in
treated villages) and the continued use of inefficient and
highly-polluting biomass-burning kangs. Kangs are a relatively simple
and culturally entrenched space heating technique that has been used for
over two thousand years in China (Zhuang et al. 2009). Kangs are usually
fuelled by wood or other biomass that is freely and readily available in
our rural and peri-urban study villages. The CBHP policy did not ban
biomass burning, and we did observe persistent use of kangs based on
both household surveys and the PM\textsubscript{2.5} source analysis and
apportionment results. The continued use of solid fuel stoves (i.e.,
stove stacking) alongside cleaner stoves and fuels has long been a
barrier to achieving the intended air quality benefits of household
energy interventions (Shankar et al. 2020). A notable exception is a
recently completed multi-country randomized trial of LPG cookstoves
which attained near exclusive use of LPG stoves and dramatic reductions
in personal exposures to PM\textsubscript{2.5} (lowered by 66\% compared
with controls) (Johnson et al. 2022), but rather unexpectedly did not
observe health benefits across a range of neonatal, child, and maternal
outcomes (Harrison et al. Approved February 2024).

In addition to the Household Air Pollution Intervention Network (HAPIN)
trial mentioned above, several other recent trials (Alexander et al.
2017; Checkley et al. 2021; Chillrud et al. 2021; Katz et al. 2020) of
interventions have evaluated the impacts of replacing solid fuels for
cooking with cleaner fuels such as liquefied petroleum gas (LPG) or
ethanol. These trials have reported reductions in in-home measures of
air pollutants indicative of combustion (i.e., PM, carbon monoxide).
Yet, baseline pollutant concentrations tended to be higher than what we
observed in our study, in some cases exceeding several hundreds of
micrograms of PM\textsubscript{2.5} per cubic meter of air (Rosa et al.
2014). Thus, even higher percentage reductions in air pollution levels
than what we observed still left WHO interim target levels out of reach.
It may also be worth noting that differences in summary metrics makes
comparisons between our study and others challenging; compared to other
household air pollution intervention studies, our study yielded a larger
sample size of longer-duration, repeated measures of indoor air
pollution (Thomas et al. 2015).

Despite the significant improvements in indoor air quality, the
intervention's effects on reducing personal exposure to
PM\textsubscript{2.5}, BC, and source contributions from the `mixed
combustion' source identified through our study's source apportionment,
as well as ambient outdoor levels of the same air pollution measures,
were more limited. The limited effect of the intervention on personal
and outdoor PM\textsubscript{2.5} and BC levels suggests the persistent
influence of pollution sources that are likely external to the home
environment (e.g., vehicular emissions, industrial activities, power
generation). The source analysis and apportionment identified a source
of mixed combustion as a significant contributor to
PM\textsubscript{2.5}, implying that, in addition to space heating solid
fuel combustion, non-heating related combustion activities may have also
persisted in their contribution to personal exposure and outdoor
pollution.

It may be valuable to compare our results to several recent studies that
were not focused on household air pollution interventions, but still
measured indoor, personal, and / or outdoor air pollution in settings
that might be influenced by similar sources to ours\ldots{}

In this study we were best able to quantify air pollution impacts from
the CBHP policy using long-term measurements at locations nearest to
indoor stoves, which required hundreds of air monitors and thousands of
hours of measurements. The scale and duration of air pollution
measurement achieved in this study would not have been possible without
low-cost air pollution sensors that have proliferated in the past
decade. {[}{[}{[}Ellison to add one or two relevant references from
China-based sensor studies, and maybe one or two reviews of the low-cost
sensor transformation of air pollution measurement{]}{]}{]}. Evaluation
methods for future intervention studies might also include longitudinal
measures of air pollution that track changes over longer periods to
capture delayed effects. An examination of how we evaluate the
effectiveness of interventions like the CBHP policy raises fundamental
questions about the adequacy of our current approaches. Traditional air
pollution metrics may not fully capture the broader, systemic changes
that such policies aim to achieve.

Determining whether the CBHP policy worked required a multifaceted
approach to evaluation that went beyond measures of air quality alone.
By incorporating a broader array of metrics and considering the systemic
nature of air pollution and its health impacts, through this study, we
sought to provide a more nuanced understanding of an intervention's
effectiveness and the ways in which it may need to be augmented or
restructured to achieve desired health outcomes.

\hypertarget{assumptions-strengths-and-limitations}{%
\subsection{Assumptions, strengths and
limitations}\label{assumptions-strengths-and-limitations}}

The validity of our DiD approach is subject to several key assumptions.
First, we assume that anticipation of the CBHP policy did not differ
between treated and untreated villages. We selected villages that were
eligible for the policy but not currently treated. It was generally
understood that the policy would first be implemented in the plains
areas with more updated electric grids and then gradually expand into
more remote and mountainous areas of Beijing, though most of our study
villages were far from Beijing's urban core. In addition to these
geographical parameters, some of our study villages were assigned to the
policy whereas others applied to the local government, but they were
generally unaware if or when they would be treated at the time of
enrollment. Second, our analysis assumes that, in the absence of the
policy, the trends in air quality and health in treated and untreated
villages would have remained the same over time. While we cannot fully
verify this assumption, we observed similar trends in health outcomes
between S1 and S2 in never treated villages and those treated later in
S3 or S4 (ref{[}c{]} SI figures that Talia created a while back for BP
and resp outcomes) and we adjusted for relevant time-varying
confounders, all of which improve the credibility of this assumption.
Fourth, we cannot entirely rule out the possibility that other programs
or policies differentially affected air quality or health in treated and
untreated villages, which could lead to over- or under-estimation of its
effects. Though, we surveyed village leaders about other rural
development or health policies and programs in their villages throughout
our four-year study period and did not identify any co-implemented
programs that would differentially impact treatment groups. Finally, our
mediation analysis assumes no residual confounding between our mediators
(air pollution and temperature) and our health outcomes. {[}{[}{[}need
to add text{]}{]}{]}

Strengths{[}d{]} of this comprehensive, field-based assessment of the
CBHP policy include our quasi-experimental design to evaluate a
real-world clean energy intervention that would be near impossible to
experimentally manipulate at the scale of our study. Our study design
controlled for secular changes in health and we additionally collected
data on and adjusted for important time-varying covariates. Our numerous
sensitivity analyses showed the robustness of our findings to various
analytic decisions. Most previous field-based household energy
intervention studies were less than two-years in duration (Harrison et
al. Approved February 2024; Quansah et al. 2017), and our four-year
study enabled longer-term evaluation of compliance with the coal ban and
heat pump adoption/use and their impacts on air pollution and health. We
retained all 50 villages in this assessment of a village-level
intervention, and were able to successfully obtain measurements from
over 1000 participants into each campaign with individual measurements
despite half of our study occurring during the covid-19 pandemic. By
comparison, previous field-based assessments of household energy
interventions (trials and pre-post designs with controls) and blood
pressure ranged in size from 44 to 324 participants (Harrison et al.
Approved February 2024; Kumar et al. 2021; Onakomaiya et al. 2019), with
exception of the recent multi-country trial that enrolled
\textasciitilde3000 pregnant women (Ye et al. 2022).

This study also has several limitations to consider when interpreting
our results.

First, the covid pandemic started in the middle of our study and roughly
half of our treated villages went into the policy during the pandemic,
which likely had some influence on the roll-out of the policy in those
villages. We observed the largest benefits in BP and several respiratory
outcomes in villages treated before the pandemic. However, we cannot
differentiate between treatment cohort effects attributable the covid
pandemic versus other factors that different between treatment cohorts
(i.e., geographic location, access to biomass, fuel prices, etc)

Second, the policy roll-out began in 2016 though we did not begin
enrolling villages into our study until 2018. Thus, our study villages
are farther from the urban core and generally of lower SES than many
villages treated in the first three years of the policy. Previous
studies of the CBHP policy suggest that treated villages of all SES
levels benefited from less-polluted and warmer indoor environments, but
that the benefits were smaller in lower SES villages compared with
higher SES villages (Barrington-Leigh et al. 2019; Meng et al. 2023).
Further, our more remote and rural study villages also have easier
access to biomass fuel. Thus our results may not be generalizable to all
of Beijing, especially the more urbanized, higher SES villages treated
between 2016 and 2018, and may underestimate the impacts of the policy
on indoor environmental factors that were important cardio-respiratory
health mediators in our study.

Third, like any field-based study, we had a number of constraints with
data collection. We were unable to measure indoor air quality or stove
use in S1 due to logistical and budget constraints, and thus cannot
directly estimate the effects of the policy on indoor PM2.5 for the 10
villages treated in 2019. Similarly, we were unable to take blood in the
last campaign because measurements were in homes rather than clinics to
avoid group contact during the pandemic. In addition, our study
logistics required visiting 50 villages over a period of just several
months. Thus, we were unable to return to villages if a previously
enrolled participant was not at home at the time that staff visited the
village. In such instances, we either randomly selected either another
eligible participant in the same home or we randomly selected another
household with eligible participants from the village roster and our
study participants different slightly across campaigns. Our
village-level study and analysis is robust to participation of a random
sample of participants in each campaign, and we also found no notable
differences in key demographic characteristics or health behaviors
between participants who contributed to a different number of campaigns
or between participants across each of the three campaigns.

Fourth, our respiratory symptoms are self reported and thus our
estimated effects of the policy must be interpreted with caution given
that participants are not blinded to the household energy intervention.
In previous studies of household water filters, for example, (Peel et
al. 2015)

\hypertarget{implications-of-findings}{%
\section{Implications of Findings}\label{implications-of-findings}}

In this comprehensive field-based assessment of the CBHP policy in
Beijing, we observed high fidelity and compliance with the policy in our
study villages and households where nearly all households in treated
villages stopped using coal and shifted to electric-powered heaters.
Exposure to the policy reduced blood pressure and self-reported chronic
respiratory symptoms, and the effects were mediated by reductions in
indoor PM\textsubscript{2.5} and improvements in home temperature. We
did not observe the same benefits of the policy on outdoor air quality
or personal exposures, likely because the relatively high contribution
of other regional and local air pollution sources to outdoor and
personal exposures may have masked the benefits from a single source
reduction. We also did not observe benefits of the policy on different
measures of inflammation and oxidative stress in the sub-sample of
participants with biomarker assessment, even though we observed
respiratory symptoms and BP benefits of the policy in a sensitivity
analysis limited to the same participants. {[}e{]}

Our results showing a home environment and health benefit of a
large-scale and successfully implemented clean energy policy are timely,
as they are synchronous with ongoing and planned clean energy policies
in China and other countries in a global effort to ``ensure access to
affordable, reliable, sustainable, and modern energy for all''
(Sustainable Development Goal-7) and also directly respond to a recent
call-to-action from global cardiovascular societies that emphasized the
urgent need for interventional studies that inform targeted
pollution-reducing strategies to reduce cardiovascular disease (Brauer
et al. 2021).

\hypertarget{data-availability-statement}{%
\section{Data Availability
Statement}\label{data-availability-statement}}

\begin{itemize}
\tightlist
\item
  Description of datasets and code available on our project page at the
  Open Science Foundation
\end{itemize}

\hypertarget{acknowledgements}{%
\section{Acknowledgements}\label{acknowledgements}}

To come\ldots{}

\hypertarget{references}{%
\section{References}\label{references}}

\hypertarget{refs}{}
\begin{CSLReferences}{1}{0}
\leavevmode\vadjust pre{\hypertarget{ref-ahmed2009}{}}%
Ahmed T, Dutkiewicz VA, Shareef A, Tuncel G, Tuncel S, Husain L. 2009.
Measurement of black carbon ({BC}) by an optical method and a
thermal-optical method: {Intercomparison} for four sites. Atmospheric
Environment 43:6305--6311;
doi:\href{https://doi.org/10.1016/j.atmosenv.2009.09.031}{10.1016/j.atmosenv.2009.09.031}.

\leavevmode\vadjust pre{\hypertarget{ref-alexander2018}{}}%
Alexander DA, Northcross A, Karrison T, Morhasson-Bello O, Wilson N,
Atalabi OM, et al. 2018. Pregnancy outcomes and ethanol cook stove
intervention: {A} randomized-controlled trial in {Ibadan}, {Nigeria}.
Environment International 111:152--163;
doi:\href{https://doi.org/10.1016/j.envint.2017.11.021}{10.1016/j.envint.2017.11.021}.

\leavevmode\vadjust pre{\hypertarget{ref-alexander2017}{}}%
Alexander D, Northcross A, Wilson N, Dutta A, Pandya R, Ibigbami T, et
al. 2017. Randomized {Controlled Ethanol Cookstove Intervention} and
{Blood Pressure} in {Pregnant Nigerian Women}. American Journal of
Respiratory and Critical Care Medicine 195:1629--1639;
doi:\href{https://doi.org/10.1164/rccm.201606-1177OC}{10.1164/rccm.201606-1177OC}.

\leavevmode\vadjust pre{\hypertarget{ref-an2021}{}}%
An L, Hong B, Cui X, Geng Y, Ma X. 2021. Outdoor thermal comfort during
winter in {China}'s cold regions: {A} comparative study. Science of The
Total Environment 768:144464;
doi:\href{https://doi.org/10.1016/j.scitotenv.2020.144464}{10.1016/j.scitotenv.2020.144464}.

\leavevmode\vadjust pre{\hypertarget{ref-archer-nicholls2016}{}}%
Archer-Nicholls S, Carter E, Kumar R, Xiao Q, Liu Y, Frostad J, et al.
2016. The {Regional Impacts} of {Cooking} and {Heating Emissions} on
{Ambient Air Quality} and {Disease Burden} in {China}. Environmental
Science \& Technology 50:9416--9423;
doi:\href{https://doi.org/10.1021/acs.est.6b02533}{10.1021/acs.est.6b02533}.

\leavevmode\vadjust pre{\hypertarget{ref-barrington-leigh2019}{}}%
Barrington-Leigh C, Baumgartner J, Carter E, Robinson BE, Tao S, Zhang
Y. 2019. An evaluation of air quality, home heating and well-being under
{Beijing}'s programme to eliminate household coal use. Nature Energy
4:416--423;
doi:\href{https://doi.org/10.1038/s41560-019-0386-2}{10.1038/s41560-019-0386-2}.

\leavevmode\vadjust pre{\hypertarget{ref-baumgartner2018}{}}%
Baumgartner J, Carter E, Schauer JJ, Ezzati M, Daskalopoulou SS, Valois
M-F, et al. 2018. Household air pollution and measures of blood
pressure, arterial stiffness and central haemodynamics. Heart
104:1515--1521;
doi:\href{https://doi.org/10.1136/heartjnl-2017-312595}{10.1136/heartjnl-2017-312595}.

\leavevmode\vadjust pre{\hypertarget{ref-baumgartner2011}{}}%
Baumgartner J, Schauer JJ, Ezzati M, Lu L, Cheng C, Patz JA, et al.
2011. Indoor {Air Pollution} and {Blood Pressure} in {Adult Women
Living} in {Rural China}. Environmental Health Perspectives
119:1390--1395;
doi:\href{https://doi.org/10.1289/ehp.1003371}{10.1289/ehp.1003371}.

\leavevmode\vadjust pre{\hypertarget{ref-brauer2021}{}}%
Brauer M, Casadei B, Harrington RA, Kovacs R, Sliwa K, the WHF Air
Pollution Expert Group. 2021. Taking a {Stand Against Air
Pollution}---{The Impact} on {Cardiovascular Disease}: {A Joint Opinion
From} the {World Heart Federation}, {American College} of {Cardiology},
{American Heart Association}, and the {European Society} of
{Cardiology}. Circulation 143;
doi:\href{https://doi.org/10.1161/CIRCULATIONAHA.120.052666}{10.1161/CIRCULATIONAHA.120.052666}.

\leavevmode\vadjust pre{\hypertarget{ref-callaway2020}{}}%
Callaway B. 2020.
\href{https://doi.org/10.1007/978-3-319-57365-6_352-1}{Difference-in-{Differences}
for {Policy Evaluation}}. In: \emph{Handbook of {Labor}, {Human
Resources} and {Population Economics}} (K.F. Zimmermann, ed). Springer
International Publishing:Cham. 1--61.

\leavevmode\vadjust pre{\hypertarget{ref-callaway2021}{}}%
Callaway B, Sant'Anna PHC. 2021. Difference-in-{Differences} with
multiple time periods. Journal of Econometrics 225:200--230;
doi:\href{https://doi.org/10.1016/j.jeconom.2020.12.001}{10.1016/j.jeconom.2020.12.001}.

\leavevmode\vadjust pre{\hypertarget{ref-card1994}{}}%
Card D, Krueger AB. 1994. Minimum {Wages} and {Employment}: {A Case
Study} of the {Fast-Food Industry} in {New Jersey} and {Pennsylvania}.
American Economic Review 84: 772--93.

\leavevmode\vadjust pre{\hypertarget{ref-checkley2021}{}}%
Checkley W, Williams KN, Kephart JL, Fandiño-Del-Rio M, Steenland NK,
Gonzales GF, et al. 2021. Effects of a {Household Air Pollution
Intervention} with {Liquefied Petroleum Gas} on {Cardiopulmonary
Outcomes} in {Peru}. {A Randomized Controlled Trial}. American Journal
of Respiratory and Critical Care Medicine 203:1386--1397;
doi:\href{https://doi.org/10.1164/rccm.202006-2319OC}{10.1164/rccm.202006-2319OC}.

\leavevmode\vadjust pre{\hypertarget{ref-chillrud2021}{}}%
Chillrud SN, Ae-Ngibise KA, Gould CF, Owusu-Agyei S, Mujtaba M, Manu G,
et al. 2021. The effect of clean cooking interventions on mother and
child personal exposure to air pollution: Results from the {Ghana
Randomized Air Pollution} and {Health Study} ({GRAPHS}). Journal of
Exposure Science \& Environmental Epidemiology 31:683--698;
doi:\href{https://doi.org/10.1038/s41370-021-00309-5}{10.1038/s41370-021-00309-5}.

\leavevmode\vadjust pre{\hypertarget{ref-clark2017}{}}%
Clark S, Carter E, Shan M, Ni K, Niu H, Tseng JTW, et al. 2017. Adoption
and use of a semi-gasifier cooking and water heating stove and fuel
intervention in the {Tibetan Plateau}, {China}. Environmental Research
Letters 12:075004;
doi:\href{https://doi.org/10.1088/1748-9326/aa751e}{10.1088/1748-9326/aa751e}.

\leavevmode\vadjust pre{\hypertarget{ref-costello2015}{}}%
Costello BT, Schultz MG, Black JA, Sharman JE. 2015. Evaluation of a
{Brachial Cuff} and {Suprasystolic Waveform Algorithm Method} to
{Noninvasively Derive Central Blood Pressure}. American Journal of
Hypertension 28:480--486;
doi:\href{https://doi.org/10.1093/ajh/hpu163}{10.1093/ajh/hpu163}.

\leavevmode\vadjust pre{\hypertarget{ref-dai2020}{}}%
Dai Q, Liu B, Bi X, Wu J, Liang D, Zhang Y, et al. 2020. Dispersion
{Normalized PMF Provides Insights} into the {Significant Changes} in
{Source Contributions} to {PM} {\textsubscript{2.5}} after the {COVID-19
Outbreak}. Environmental Science \& Technology 54:9917--9927;
doi:\href{https://doi.org/10.1021/acs.est.0c02776}{10.1021/acs.est.0c02776}.

\leavevmode\vadjust pre{\hypertarget{ref-dart2001}{}}%
Dart AM, Kingwell BA. 2001. Pulse pressure---a review of mechanisms and
clinical relevance. Journal of the American College of Cardiology
37:975--984;
doi:\href{https://doi.org/10.1016/S0735-1097(01)01108-1}{10.1016/S0735-1097(01)01108-1}.

\leavevmode\vadjust pre{\hypertarget{ref-cdcgr2023}{}}%
Dispersed Coal Management Research Group
北京大学能源研究院气候变化与能源转型项目. 2023. 中国散煤综合治理研究报告
{China Dispersed Coal Governance Report}.

\leavevmode\vadjust pre{\hypertarget{ref-dockery2013}{}}%
Dockery DW, Rich DQ, Goodman PG, Clancy L, Ohman-Strickland P, George P,
et al. 2013. \href{https://www.ncbi.nlm.nih.gov/pubmed/24024358}{Effect
of air pollution control on mortality and hospital admissions in
{Ireland}}. Research Report (Health Effects Institute) 3--109.

\leavevmode\vadjust pre{\hypertarget{ref-dominici2014}{}}%
Dominici F, Greenstone M, Sunstein CR. 2014. Science and regulation.
{Particulate} matter matters. Science (New York, NY) 344:257--9;
doi:\href{https://doi.org/10.1126/science.1247348}{10.1126/science.1247348}.

\leavevmode\vadjust pre{\hypertarget{ref-dong2013}{}}%
Dong G-H, Qian Z(Min), Xaverius PK, Trevathan E, Maalouf S, Parker J, et
al. 2013. Association {Between Long-Term Air Pollution} and {Increased
Blood Pressure} and {Hypertension} in {China}. Hypertension 61:578--584;
doi:\href{https://doi.org/10.1161/HYPERTENSIONAHA.111.00003}{10.1161/HYPERTENSIONAHA.111.00003}.

\leavevmode\vadjust pre{\hypertarget{ref-duan2014}{}}%
Duan X, Jiang Y, Wang B, Zhao X, Shen G, Cao S, et al. 2014. Household
fuel use for cooking and heating in {China}: {Results} from the first
{Chinese Environmental Exposure-Related Human Activity Patterns Survey}
({CEERHAPS}). Applied Energy 136:692--703;
doi:\href{https://doi.org/10.1016/j.apenergy.2014.09.066}{10.1016/j.apenergy.2014.09.066}.

\leavevmode\vadjust pre{\hypertarget{ref-ezzati2017}{}}%
Ezzati M, Baumgartner JC. 2017. Household energy and health: Where next
for research and practice? Lancet (London, England) 389:130--132;
doi:\href{https://doi.org/10.1016/S0140-6736(16)32506-5}{10.1016/S0140-6736(16)32506-5}.

\leavevmode\vadjust pre{\hypertarget{ref-fda2018}{}}%
Food and Drug Administration. 2018. Bioanalytical {Method Validation
Guidance} for {Industry}.

\leavevmode\vadjust pre{\hypertarget{ref-goin2023}{}}%
Goin DE, Riddell CA. 2023. Comparing {Two-way Fixed Effects} and {New
Estimators} for {Difference-in-Differences}: {A Simulation Study} and
{Empirical Example}. Epidemiology 34:535;
doi:\href{https://doi.org/10.1097/EDE.0000000000001611}{10.1097/EDE.0000000000001611}.

\leavevmode\vadjust pre{\hypertarget{ref-goodman-bacon2021}{}}%
Goodman-Bacon A. 2021. Difference-in-differences with variation in
treatment timing. Journal of Econometrics 225:254--277;
doi:\href{https://doi.org/10.1016/j.jeconom.2021.03.014}{10.1016/j.jeconom.2021.03.014}.

\leavevmode\vadjust pre{\hypertarget{ref-gould2023}{}}%
Gould CF, Bejarano ML, Kioumourtzoglou M-A, Lee AG, Pillarisetti A,
Schlesinger SB, et al. 2023. Widespread {Clean Cooking Fuel Scale-Up}
and under-5 {Lower Respiratory Infection Mortality}: {An Ecological
Analysis} in {Ecuador}, 1990--2019. Environmental Health Perspectives
131:037017;
doi:\href{https://doi.org/10.1289/EHP11016}{10.1289/EHP11016}.

\leavevmode\vadjust pre{\hypertarget{ref-gbdmaps2016}{}}%
Group GMW. 2016. Burden of disease attributable to coal-burning and
other air pollution sources in {China}.

\leavevmode\vadjust pre{\hypertarget{ref-harrison2024}{}}%
Harrison K, Hartinger S, Jack D, Kaali S, Lydston M, Mortimer K, et al.
Approved February 2024. Household {Air Pollution Interventions} to
{Improve Health} in {Low-} and {Middle-Income Countries}: {An Official
American Thoracic Society Research Statement}.

\leavevmode\vadjust pre{\hypertarget{ref-huang2012}{}}%
Huang W, Wang G, Lu S-E, Kipen H, Wang Y, Hu M, et al. 2012.
Inflammatory and {Oxidative Stress Responses} of {Healthy Young Adults}
to {Changes} in {Air Quality} during the {Beijing Olympics}. American
Journal of Respiratory and Critical Care Medicine 186:1150--1159;
doi:\href{https://doi.org/10.1164/rccm.201205-0850OC}{10.1164/rccm.201205-0850OC}.

\leavevmode\vadjust pre{\hypertarget{ref-rtiinternational2009}{}}%
International R. 2009. Standard {Operating Procedure} for the {X-Ray
Fluorescence Analysis} of {Particulate Matter Deposits} on {Teflon
Filters}: {PM Xrf Analysis}.

\leavevmode\vadjust pre{\hypertarget{ref-johnson2022}{}}%
Johnson M, Pillarisetti A, Piedrahita R, Balakrishnan K, Peel JL,
Steenland K, et al. 2022. Exposure {Contrasts} of {Pregnant Women}
during the {Household Air Pollution Intervention Network Randomized
Controlled Trial}. Environmental Health Perspectives 130:097005;
doi:\href{https://doi.org/10.1289/EHP10295}{10.1289/EHP10295}.

\leavevmode\vadjust pre{\hypertarget{ref-johnston2013}{}}%
Johnston FH, Hanigan IC, Henderson SB, Morgan GG. 2013. Evaluation of
interventions to reduce air pollution from biomass smoke on mortality in
{Launceston}, {Australia}: Retrospective analysis of daily mortality,
1994-2007. BMJ 346:e8446--e8446;
doi:\href{https://doi.org/10.1136/bmj.e8446}{10.1136/bmj.e8446}.

\leavevmode\vadjust pre{\hypertarget{ref-kanagasabai2022}{}}%
Kanagasabai T, Xie W, Yan L, Zhao L, Carter E, Guo D, et al. 2022.
Household {Air Pollution} and {Blood Pressure}, {Vascular Damage}, and
{Subclinical Indicators} of {Cardiovascular Disease} in {Older Chinese
Adults}. American Journal of Hypertension 35:121--131;
doi:\href{https://doi.org/10.1093/ajh/hpab141}{10.1093/ajh/hpab141}.

\leavevmode\vadjust pre{\hypertarget{ref-kashtan2023}{}}%
Kashtan YS, Nicholson M, Finnegan C, Ouyang Z, Lebel ED, Michanowicz DR,
et al. 2023. Gas and {Propane Combustion} from {Stoves Emits Benzene}
and {Increases Indoor Air Pollution}. Environmental Science \&
Technology 57:9653--9663;
doi:\href{https://doi.org/10.1021/acs.est.2c09289}{10.1021/acs.est.2c09289}.

\leavevmode\vadjust pre{\hypertarget{ref-katz2020}{}}%
Katz J, Tielsch JM, Khatry SK, Shrestha L, Breysse P, Zeger SL, et al.
2020. Impact of {Improved Biomass} and {Liquid Petroleum Gas Stoves} on
{Birth Outcomes} in {Rural Nepal}: {Results} of 2 {Randomized Trials}.
Global Health: Science and Practice 8:372--382;
doi:\href{https://doi.org/10.9745/GHSP-D-20-00011}{10.9745/GHSP-D-20-00011}.

\leavevmode\vadjust pre{\hypertarget{ref-keele2015}{}}%
Keele L, Tingley D, Yamamoto T. 2015. Identifying mechanisms behind
policy interventions via causal mediation analysis. Journal of Policy
Analysis and Management 34: 937--963.

\leavevmode\vadjust pre{\hypertarget{ref-khuzestani2017}{}}%
Khuzestani RB, Schauer JJ, Wei Y, Zhang Y, Zhang Y. 2017. A
non-destructive optical color space sensing system to quantify elemental
and organic carbon in atmospheric particulate matter on {Teflon} and
quartz filters. Atmospheric Environment 149:84--94;
doi:\href{https://doi.org/10.1016/j.atmosenv.2016.11.002}{10.1016/j.atmosenv.2016.11.002}.

\leavevmode\vadjust pre{\hypertarget{ref-kumar2021}{}}%
Kumar N, Phillip E, Cooper H, Davis M, Langevin J, Clifford M, et al.
2021. Do improved biomass cookstove interventions improve indoor air
quality and blood pressure? {A} systematic review and meta-analysis.
Environmental Pollution 290:117997;
doi:\href{https://doi.org/10.1016/j.envpol.2021.117997}{10.1016/j.envpol.2021.117997}.

\leavevmode\vadjust pre{\hypertarget{ref-lai2019}{}}%
Lai. 2019. Relative contributions of household solid fuel use and
outdoor air pollution to chemical components of personal {PM2}.5
exposures. Indoor Air-international Journal of Indoor Air Quality and
Climate.

\leavevmode\vadjust pre{\hypertarget{ref-lewington2012}{}}%
Lewington S, LiMing L, Sherliker P, Yu G, Millwood I, Zheng B, et al.
2012. Seasonal variation in blood pressure and its relationship with
outdoor temperature in 10 diverse regions of {China}: The {China
Kadoorie Biobank}. Journal of hypertension 30: 1383.

\leavevmode\vadjust pre{\hypertarget{ref-lindemann2017}{}}%
Lindemann U, Stotz A, Beyer N, Oksa J, Skelton DA, Becker C, et al.
2017. Effect of indoor temperature on physical performance in older
adults during days with normal temperature and heat waves. International
journal of environmental research and public health 14;
doi:\href{https://doi.org/10.3390/ijerph14020186}{10.3390/ijerph14020186}.

\leavevmode\vadjust pre{\hypertarget{ref-lowe2009}{}}%
Lowe A, Harrison W, El-Aklouk E, Ruygrok P, Al-Jumaily AM. 2009.
Non-invasive model-based estimation of aortic pulse pressure using
suprasystolic brachial pressure waveforms. Journal of Biomechanics
42:2111--2115;
doi:\href{https://doi.org/10.1016/j.jbiomech.2009.05.029}{10.1016/j.jbiomech.2009.05.029}.

\leavevmode\vadjust pre{\hypertarget{ref-lv2022}{}}%
Lv Y, Zhu R, Xie J, Yoshino H. 2022. Indoor environment and the blood
pressure of elderly in the cold region of {China}. Indoor and Built
Environment 31:2482--2498;
doi:\href{https://doi.org/10.1177/1420326X221109510}{10.1177/1420326X221109510}.

\leavevmode\vadjust pre{\hypertarget{ref-mccracken2007}{}}%
McCracken JP, Smith KR, Díaz A, Mittleman MA, Schwartz J. 2007. Chimney
{Stove Intervention} to {Reduce Long-term Wood Smoke Exposure Lowers
Blood Pressure} among {Guatemalan Women}. Environmental Health
Perspectives 115:996--1001;
doi:\href{https://doi.org/10.1289/ehp.9888}{10.1289/ehp.9888}.

\leavevmode\vadjust pre{\hypertarget{ref-mccracken2011}{}}%
McCracken J, Smith KR, Stone P, Díaz A, Arana B, Schwartz J. 2011.
Intervention to {Lower Household Wood Smoke Exposure} in {Guatemala
Reduces ST-Segment Depression} on {Electrocardiograms}. Environmental
Health Perspectives 119:1562--1568;
doi:\href{https://doi.org/10.1289/ehp.1002834}{10.1289/ehp.1002834}.

\leavevmode\vadjust pre{\hypertarget{ref-meng2023}{}}%
Meng W, Zhu L, Liang Z, Xu H, Zhang W, Li J, et al. 2023. Significant
but {Inequitable Cost-Effective Benefits} of a {Clean Heating Campaign}
in {Northern China}. Environmental Science \& Technology 57:8467--8475;
doi:\href{https://doi.org/10.1021/acs.est.2c07492}{10.1021/acs.est.2c07492}.

\leavevmode\vadjust pre{\hypertarget{ref-naimi2014}{}}%
Naimi AI, Kaufman JS, MacLehose RF. 2014. Mediation misgivings:
Ambiguous clinical and public health interpretations of natural direct
and indirect effects. International journal of epidemiology 43:1656--61;
doi:\href{https://doi.org/10.1093/ije/dyu107}{10.1093/ije/dyu107}.

\leavevmode\vadjust pre{\hypertarget{ref-niu2024}{}}%
Niu J, Chen X, Sun S. 2024. China's {Coal Ban} policy: {Clearing} skies,
challenging growth. Journal of Environmental Management 349:119420;
doi:\href{https://doi.org/10.1016/j.jenvman.2023.119420}{10.1016/j.jenvman.2023.119420}.

\leavevmode\vadjust pre{\hypertarget{ref-olson2016}{}}%
Olson MR, Graham E, Hamad S, Uchupalanun P, Ramanathan N, Schauer JJ.
2016. Quantification of elemental and organic carbon in atmospheric
particulate matter using color space sensing---hue, saturation, and
value ({HSV}) coordinates. Science of The Total Environment
548--549:252--259;
doi:\href{https://doi.org/10.1016/j.scitotenv.2016.01.032}{10.1016/j.scitotenv.2016.01.032}.

\leavevmode\vadjust pre{\hypertarget{ref-onakomaiya2019}{}}%
Onakomaiya D, Gyamfi J, Iwelunmor J, Opeyemi J, Oluwasanmi M,
Obiezu-Umeh C, et al. 2019. Implementation of clean cookstove
interventions and its effects on blood pressure in low-income and
middle-income countries: Systematic review. BMJ Open 9:e026517;
doi:\href{https://doi.org/10.1136/bmjopen-2018-026517}{10.1136/bmjopen-2018-026517}.

\leavevmode\vadjust pre{\hypertarget{ref-worldhealthorganization2021}{}}%
Organization WH. 2021. {WHO Global Air Quality Guidelines}: {Particulate
Matter PM2}.5 and {PM10}), {Ozone}, {Nitrogen Dioxide}, {Sulfur Dioxide}
and {Carbon Monoxide}.

\leavevmode\vadjust pre{\hypertarget{ref-peel2015}{}}%
Peel JL, Baumgartner J, Wellenius GA, Clark ML, Smith KR. 2015. Are
{Randomized Trials Necessary} to {Advance Epidemiologic Research} on
{Household Air Pollution}? Current Epidemiology Reports 2:263--270;
doi:\href{https://doi.org/10.1007/s40471-015-0054-4}{10.1007/s40471-015-0054-4}.

\leavevmode\vadjust pre{\hypertarget{ref-quansah2017}{}}%
Quansah R, Semple S, Ochieng CA, Juvekar S, Armah FA, Luginaah I, et al.
2017. Effectiveness of interventions to reduce household air pollution
and/or improve health in homes using solid fuel in low-and-middle income
countries: {A} systematic review and meta-analysis. Environment
International 103:73--90;
doi:\href{https://doi.org/10.1016/j.envint.2017.03.010}{10.1016/j.envint.2017.03.010}.

\leavevmode\vadjust pre{\hypertarget{ref-rehfuess2014}{}}%
Rehfuess EA, Puzzolo E, Stanistreet D, Pope D, Bruce NG. 2014. Enablers
and {Barriers} to {Large-Scale Uptake} of {Improved Solid Fuel Stoves}:
{A Systematic Review}. Environmental Health Perspectives 122:120--130;
doi:\href{https://doi.org/10.1289/ehp.1306639}{10.1289/ehp.1306639}.

\leavevmode\vadjust pre{\hypertarget{ref-rich2012}{}}%
Rich DQ, Kipen HM, Huang W, Wang G, Wang Y, Zhu P, et al. 2012.
Association {Between Changes} in {Air Pollution Levels During} the
{Beijing Olympics} and {Biomarkers} of {Inflammation} and {Thrombosis}
in {Healthy Young Adults}. JAMA 307;
doi:\href{https://doi.org/10.1001/jama.2012.3488}{10.1001/jama.2012.3488}.

\leavevmode\vadjust pre{\hypertarget{ref-rosa2014}{}}%
Rosa G, Majorin F, Boisson S, Barstow C, Johnson M, Kirby M, et al.
2014. Assessing the {Impact} of {Water Filters} and {Improved Cook
Stoves} on {Drinking Water Quality} and {Household Air Pollution}: {A
Randomised Controlled Trial} in {Rwanda}. R.K. Hills, ed PLoS ONE
9:e91011;
doi:\href{https://doi.org/10.1371/journal.pone.0091011}{10.1371/journal.pone.0091011}.

\leavevmode\vadjust pre{\hypertarget{ref-rosenthal2018}{}}%
Rosenthal J, Quinn A, Grieshop AP, Pillarisetti A, Glass RI. 2018. Clean
cooking and the {SDGs}: {Integrated} analytical approaches to guide
energy interventions for health and environment goals. Energy for
sustainable development : the journal of the International Energy
Initiative 42:152--159;
doi:\href{https://doi.org/10.1016/j.esd.2017.11.003}{10.1016/j.esd.2017.11.003}.

\leavevmode\vadjust pre{\hypertarget{ref-rubin1987}{}}%
Rubin DB. 1987.
\emph{\href{https://doi.org/10.1002/9780470316696}{Multiple {Imputation}
for {Nonresponse} in {Surveys}}}. 1st ed. Wiley.

\leavevmode\vadjust pre{\hypertarget{ref-ruiz-mercado2013}{}}%
Ruiz-Mercado I, Canuz E, Walker JL, Smith KR. 2013. Quantitative metrics
of stove adoption using {Stove Use Monitors} ({SUMs}). Biomass and
Bioenergy 57:136--148;
doi:\href{https://doi.org/10.1016/j.biombioe.2013.07.002}{10.1016/j.biombioe.2013.07.002}.

\leavevmode\vadjust pre{\hypertarget{ref-scott2011}{}}%
Scott AJ, Scarrott C. 2011. Impacts of residential heating intervention
measures on air quality and progress towards targets in {Christchurch}
and {Timaru}, {New Zealand}. Atmospheric Environment 45:2972--2980;
doi:\href{https://doi.org/10.1016/j.atmosenv.2010.09.008}{10.1016/j.atmosenv.2010.09.008}.

\leavevmode\vadjust pre{\hypertarget{ref-shang2020}{}}%
Shang J, Zhang Y, Schauer JJ, Tian J, Hua J, Han T, et al. 2020.
Associations between source-resolved {PM2}.5 and airway inflammation at
urban and rural locations in {Beijing}. Environment International
139:105635;
doi:\href{https://doi.org/10.1016/j.envint.2020.105635}{10.1016/j.envint.2020.105635}.

\leavevmode\vadjust pre{\hypertarget{ref-shankar2020}{}}%
Shankar AV, Quinn AK, Dickinson KL, Williams KN, Masera O, Charron D, et
al. 2020. Everybody stacks: {Lessons} from household energy case studies
to inform design principles for clean energy transitions. Energy Policy
141:111468;
doi:\href{https://doi.org/10.1016/j.enpol.2020.111468}{10.1016/j.enpol.2020.111468}.

\leavevmode\vadjust pre{\hypertarget{ref-sinton2004}{}}%
Sinton JE, Smith KR, Peabody JW, Yaping L, Xiliang Z, Edwards R, et al.
2004. An assessment of programs to promote improved household stoves in
{China}. Energy for Sustainable Development 8:33--52;
doi:\href{https://doi.org/10.1016/S0973-0826(08)60465-2}{10.1016/S0973-0826(08)60465-2}.

\leavevmode\vadjust pre{\hypertarget{ref-smith-sivertsen2009}{}}%
Smith-Sivertsen T, Díaz E, Pope D, Lie RT, Díaz A, McCracken J, et al.
2009. Effect of {Reducing Indoor Air Pollution} on {Women}'s
{Respiratory Symptoms} and {Lung Function}: {The RESPIRE Randomized
Trial}, {Guatemala}. American Journal of Epidemiology 170:211--220;
doi:\href{https://doi.org/10.1093/aje/kwp100}{10.1093/aje/kwp100}.

\leavevmode\vadjust pre{\hypertarget{ref-snider2018}{}}%
Snider G, Carter E, Clark S, Tseng J(TzuW, Yang X, Ezzati M, et al.
2018. Impacts of stove use patterns and outdoor air quality on household
air pollution and cardiovascular mortality in southwestern {China}.
Environment International 117:116--124;
doi:\href{https://doi.org/10.1016/j.envint.2018.04.048}{10.1016/j.envint.2018.04.048}.

\leavevmode\vadjust pre{\hypertarget{ref-song2023}{}}%
Song C, Liu B, Cheng K, Cole MA, Dai Q, Elliott RJR, et al. 2023.
Attribution of {Air Quality Benefits} to {Clean Winter Heating Policies}
in {China}: {Combining Machine Learning} with {Causal Inference}.
Environmental Science \& Technology 57:17707--17717;
doi:\href{https://doi.org/10.1021/acs.est.2c06800}{10.1021/acs.est.2c06800}.

\leavevmode\vadjust pre{\hypertarget{ref-sternbach2022}{}}%
Sternbach TJ, Harper S, Li X, Zhang X, Carter E, Zhang Y, et al. 2022.
Effects of indoor and outdoor temperatures on blood pressure and central
hemodynamics in a wintertime longitudinal study of {Chinese} adults.
Journal of Hypertension 40:1950--1959;
doi:\href{https://doi.org/10.1097/HJH.0000000000003198}{10.1097/HJH.0000000000003198}.

\leavevmode\vadjust pre{\hypertarget{ref-tan2023}{}}%
Tan X, Chen G, Chen K. 2023. Clean heating and air pollution: {Evidence}
from {Northern China}. Energy Reports 9:303--313;
doi:\href{https://doi.org/10.1016/j.egyr.2022.11.166}{10.1016/j.egyr.2022.11.166}.

\leavevmode\vadjust pre{\hypertarget{ref-thomas2015}{}}%
Thomas E, Wickramasinghe K, Mendis S, Roberts N, Foster C. 2015.
Improved stove interventions to reduce household air pollution in low
and middle income countries: A descriptive systematic review. BMC Public
Health 15:650;
doi:\href{https://doi.org/10.1186/s12889-015-2024-7}{10.1186/s12889-015-2024-7}.

\leavevmode\vadjust pre{\hypertarget{ref-thompson2019}{}}%
Thompson RJ, Li J, Weyant CL, Edwards R, Lan Q, Rothman N, et al. 2019.
Field {Emission Measurements} of {Solid Fuel Stoves} in {Yunnan}, {China
Demonstrate Dominant Causes} of {Uncertainty} in {Household Emission
Inventories}. Environmental Science \& Technology 53:3323--3330;
doi:\href{https://doi.org/10.1021/acs.est.8b07040}{10.1021/acs.est.8b07040}.

\leavevmode\vadjust pre{\hypertarget{ref-tuck2009}{}}%
Tuck MK, Chan DW, Chia D, Godwin AK, Grizzle WE, Krueger KE, et al.
2009. Standard {Operating Procedures} for {Serum} and {Plasma
Collection}: {Early Detection Research Network Consensus Statement}
{\emph{Standard Operating Procedure Integration Working Group}}. Journal
of Proteome Research 8:113--117;
doi:\href{https://doi.org/10.1021/pr800545q}{10.1021/pr800545q}.

\leavevmode\vadjust pre{\hypertarget{ref-vanbuuren2011}{}}%
van Buuren S, Groothuis-Oudshoorn K. 2011. {\textbf{Mice}} :
{Multivariate Imputation} by {Chained Equations} in {\emph{R}}. Journal
of Statistical Software 45;
doi:\href{https://doi.org/10.18637/jss.v045.i03}{10.18637/jss.v045.i03}.

\leavevmode\vadjust pre{\hypertarget{ref-vanderweele2015}{}}%
VanderWeele TJ. 2015. \emph{Explanation in causal inference: Methods for
mediation and interaction}. Oxford University Press:New York.

\leavevmode\vadjust pre{\hypertarget{ref-volckens2017}{}}%
Volckens J, Quinn C, Leith D, Mehaffy J, Henry CS, Miller-Lionberg D.
2017. Development and evaluation of an ultrasonic personal aerosol
sampler. Indoor air 27:409--416;
doi:\href{https://doi.org/10.1111/ina.12318}{10.1111/ina.12318}.

\leavevmode\vadjust pre{\hypertarget{ref-wang2020}{}}%
Wang Q, Zhao Q, Wang G, Wang B, Zhang Y, Zhang J, et al. 2020. The
association between ambient temperature and clinical visits for
inflammation-related diseases in rural areas in {China}. Environmental
Pollution 261:114128;
doi:\href{https://doi.org/10.1016/j.envpol.2020.114128}{10.1016/j.envpol.2020.114128}.

\leavevmode\vadjust pre{\hypertarget{ref-wen2023}{}}%
Wen H, Nie P, Liu M, Peng R, Guo T, Wang C, et al. 2023. Multi-health
effects of clean residential heating: {Evidences} from rural {China}'s
coal-to-gas/electricity project. Energy for Sustainable Development
73:66--75;
doi:\href{https://doi.org/10.1016/j.esd.2023.01.013}{10.1016/j.esd.2023.01.013}.

\leavevmode\vadjust pre{\hypertarget{ref-wooldridge2021}{}}%
Wooldridge JM. 2021. Two-{Way Fixed Effects}, the {Two-Way Mundlak
Regression}, and {Difference-in-Differences Estimators}.;
doi:\href{https://doi.org/10.2139/ssrn.3906345}{10.2139/ssrn.3906345}.

\leavevmode\vadjust pre{\hypertarget{ref-xu2019}{}}%
Xu H, Brook RD, Wang T, Song X, Feng B, Yi T, et al. 2019. Short-term
effects of ambient air pollution and outdoor temperature on biomarkers
of myocardial damage, inflammation and oxidative stress in healthy
adults. Environmental Epidemiology 3:e078;
doi:\href{https://doi.org/10.1097/EE9.0000000000000078}{10.1097/EE9.0000000000000078}.

\leavevmode\vadjust pre{\hypertarget{ref-yan2020}{}}%
Yan L, Carter E, Fu Y, Guo D, Huang P, Xie G, et al. 2020. Study
protocol: {The INTERMAP China Prospective} ({ICP}) study. Wellcome Open
Research 4:154;
doi:\href{https://doi.org/10.12688/wellcomeopenres.15470.2}{10.12688/wellcomeopenres.15470.2}.

\leavevmode\vadjust pre{\hypertarget{ref-yangk2021}{}}%
Yang K. 2021. Power industry provides inexhaustible power for national
rejuvenation; 杨昆:电力工业为民族复兴提供不竭动力.

\leavevmode\vadjust pre{\hypertarget{ref-yap2015}{}}%
Yap P-S, Garcia C. 2015. Effectiveness of {Residential Wood-Burning
Regulation} on {Decreasing Particulate Matter Levels} and
{Hospitalizations} in the {San Joaquin Valley Air Basin}. American
Journal of Public Health 105:772--778;
doi:\href{https://doi.org/10.2105/AJPH.2014.302360}{10.2105/AJPH.2014.302360}.

\leavevmode\vadjust pre{\hypertarget{ref-ye2022}{}}%
Ye W, Steenland K, Quinn A, Liao J, Balakrishnan K, Rosa G, et al. 2022.
Effects of a {Liquefied Petroleum Gas Stove Intervention} on
{Gestational Blood Pressure}: {Intention-to-Treat} and
{Exposure-Response Findings From} the {HAPIN Trial}. Hypertension
79:1887--1898;
doi:\href{https://doi.org/10.1161/HYPERTENSIONAHA.122.19362}{10.1161/HYPERTENSIONAHA.122.19362}.

\leavevmode\vadjust pre{\hypertarget{ref-young2015}{}}%
Young OR, Guttman D, Qi Y, Bachus K, Belis D, Cheng H, et al. 2015.
Institutionalized governance processes: {Comparing} environmental
problem solving in {China} and the {United States}. Global Environmental
Change 31:163--173;
doi:\href{https://doi.org/10.1016/j.gloenvcha.2015.01.010}{10.1016/j.gloenvcha.2015.01.010}.

\leavevmode\vadjust pre{\hypertarget{ref-yu2021}{}}%
Yu C, Kang J, Teng J, Long H, Fu Y. 2021. Does coal-to-gas policy reduce
air pollution? {Evidence} from a quasi-natural experiment in {China}.
Science of The Total Environment 773:144645;
doi:\href{https://doi.org/10.1016/j.scitotenv.2020.144645}{10.1016/j.scitotenv.2020.144645}.

\leavevmode\vadjust pre{\hypertarget{ref-yun2020}{}}%
Yun X, Shen G, Shen H, Meng W, Chen Y, Xu H, et al. 2020. Residential
solid fuel emissions contribute significantly to air pollution and
associated health impacts in {China}. Science Advances 6:eaba7621;
doi:\href{https://doi.org/10.1126/sciadv.aba7621}{10.1126/sciadv.aba7621}.

\leavevmode\vadjust pre{\hypertarget{ref-zhang2007}{}}%
Zhang J(Jim), Smith KR. 2007. Household {Air Pollution} from {Coal} and
{Biomass Fuels} in {China}: {Measurements}, {Health Impacts}, and
{Interventions}. Environmental Health Perspectives 115:848--855;
doi:\href{https://doi.org/10.1289/ehp.9479}{10.1289/ehp.9479}.

\leavevmode\vadjust pre{\hypertarget{ref-zhuang2009}{}}%
Zhuang Z, Li Y, Chen B, Guo J. 2009. Chinese kang as a domestic heating
system in rural northern {China}---{A} review. Energy and Buildings
41:111--119;
doi:\href{https://doi.org/10.1016/j.enbuild.2008.07.013}{10.1016/j.enbuild.2008.07.013}.

\leavevmode\vadjust pre{\hypertarget{ref-zigler2016}{}}%
Zigler CM, Kim C, Choirat C, Hansen JB, Wang Y, Hund L, et al. 2016.
\emph{Causal inference methods for estimating long-term health effects
of air quality regulations. {Research} report 187.} Health Effects
Institute / Health Effects Institute:Boston, MA.

\end{CSLReferences}

\newpage
\appendix
\renewcommand{\thefigure}{A\arabic{figure}}
\renewcommand{\thetable}{A\arabic{table}}
\setcounter{figure}{0}
\setcounter{table}{0}

\hypertarget{appendices}{%
\section{Appendices}\label{appendices}}

\hypertarget{participant-flow-diagram}{%
\subsection{Participant flow diagram}\label{participant-flow-diagram}}

\begin{figure}[H]

{\centering \includegraphics[width=0.8\textwidth,height=\textheight]{images/Participation-flowchart-Apr12.png}

}

\caption{\label{fig-flowchart}Flow chart of BHET study participation at
the participant, household, and village levels across study years.}

\end{figure}

\hypertarget{policy-uptake-1}{%
\subsection{Policy uptake}\label{policy-uptake-1}}

and Table~\ref{tbl-fuel-did} shows results from applying our extended
two-way fixed effects\ldots{}

\begin{figure}[H]

{\centering \includegraphics[width=1\textwidth,height=\textheight]{images/coal-plot.png}

}

\caption{\label{fig-afig-coal}Trends in self-reported coal and biomass,
by treatment season}

\end{figure}

\hypertarget{tbl-fuel-did}{}
\begin{table}[H]
\caption{\label{tbl-fuel-did}Policy impacts on self-reported fuel use (kg) }\tabularnewline

\centering
\begin{tabular}{>{\centering\arraybackslash}p{1.5cm}>{\centering\arraybackslash}p{1.5cm}cccc}
\toprule
\multicolumn{2}{c}{ } & \multicolumn{2}{c}{Coal\textsuperscript{a}} & \multicolumn{2}{c}{Biomass\textsuperscript{b}} \\
\cmidrule(l{3pt}r{3pt}){3-4} \cmidrule(l{3pt}r{3pt}){5-6}
Cohort & Time & ATT & (95\%CI) & ATT & (95\%CI)\\
\midrule
\addlinespace[0.3em]
\multicolumn{6}{l}{\textbf{Average ATT}}\\
All & All & -2361 & (-2677, -2044) & -487 & (-805, -168)\\
\addlinespace[0.3em]
\multicolumn{6}{l}{\textbf{Cohort-Time ATTs}}\\
2019 & 2019 & -2631 & (-2913, -2348) & -653 & (-991, -315)\\
2019 & 2021 & -2416 & (-2847, -1984) & -633 & (-1201, -64)\\
2020 & 2021 & -2018 & (-2474, -1562) & -350 & (-701, 0)\\
2021 & 2021 & -1961 & (-2895, -1027) & 338 & (-30, 705)\\
\bottomrule
\multicolumn{6}{l}{\rule{0pt}{1em}\textsuperscript{a} Joint test that all ATTs are equal: F(3, 2886)= 1.856, p= 0.135}\\
\multicolumn{6}{l}{\rule{0pt}{1em}\textsuperscript{b} Joint test that all ATTs are equal: F(3, 2886)= 5.545, p= 0.001}\\
\end{tabular}
\end{table}

\newpage

\hypertarget{heterogeneity-in-treament-effects}{%
\subsection{Heterogeneity in treament
effects}\label{heterogeneity-in-treament-effects}}

\hypertarget{personal-exposure}{%
\subsubsection{Personal exposure}\label{personal-exposure}}

As noted in the methods section\ldots Table
Table~\ref{tbl-a-het-personal} shows limited evidence that the ATTs
across cohorts and time demonstrate meaningful heterogeneity.

\hypertarget{tbl-a-het-personal}{}
\begin{table}[H]
\caption{\label{tbl-a-het-personal}Heterogenous treatment effects: Personal exposures }\tabularnewline

\centering
\begin{tabular}{>{\centering\arraybackslash}p{1.5cm}>{\centering\arraybackslash}p{1.5cm}cccc}
\toprule
\multicolumn{2}{c}{ } & \multicolumn{2}{c}{PM2.5\textsuperscript{a}} & \multicolumn{2}{c}{Black carbon\textsuperscript{b}} \\
\cmidrule(l{3pt}r{3pt}){3-4} \cmidrule(l{3pt}r{3pt}){5-6}
Cohort & Time & ATT & (95\%CI) & ATT & (95\%CI)\\
\midrule
\addlinespace[0.3em]
\multicolumn{6}{l}{\textbf{Average ATT}}\\
All & All & 1.95 & (-23.34, 27.23) & -0.43 & (-1.67, 0.81)\\
\addlinespace[0.3em]
\multicolumn{6}{l}{\textbf{Cohort-Time ATTs}}\\
2019 & 2019 & -0.05 & (-28.97, 28.87) & -0.69 & (-1.84, 0.45)\\
2019 & 2021 & -4.31 & (-41.92, 33.3) & -0.25 & (-2.11, 1.62)\\
2020 & 2021 & 23.61 & (-19.88, 67.11) & -0.27 & (-2.04, 1.5)\\
2021 & 2021 & -19.06 & (-43.19, 5.07) & -0.56 & (-2.46, 1.34)\\
\bottomrule
\multicolumn{6}{l}{\rule{0pt}{1em}\textsuperscript{a} Joint test that all ATTs are equal: F(3, 1271)= 0.431, p= 0.731}\\
\multicolumn{6}{l}{\rule{0pt}{1em}\textsuperscript{b} Joint test that all ATTs are equal: F(3, 1253)= 0.613, p= 0.607}\\
\end{tabular}
\end{table}

\hypertarget{indoor-pm2.5-1}{%
\subsubsection{\texorpdfstring{Indoor
PM\textsubscript{2.5}}{Indoor PM2.5}}\label{indoor-pm2.5-1}}

Table Table~\ref{tbl-a-het-indoor} shows estimates for cohort-time ATTs
for daily and seasonal indoor PM\textsubscript{2.5}.

\hypertarget{tbl-a-het-indoor}{}
\begin{table}[H]
\caption{\label{tbl-a-het-indoor}Heterogenous treatment effects: Indoor }\tabularnewline

\centering
\begin{tabular}{>{\centering\arraybackslash}p{1.5cm}>{\centering\arraybackslash}p{1.5cm}cccc}
\toprule
\multicolumn{2}{c}{ } & \multicolumn{2}{c}{Daily\textsuperscript{a}} & \multicolumn{2}{c}{Seasonal\textsuperscript{b}} \\
\cmidrule(l{3pt}r{3pt}){3-4} \cmidrule(l{3pt}r{3pt}){5-6}
Cohort & Time & ATT & (95\%CI) & ATT & (95\%CI)\\
\midrule
\addlinespace[0.3em]
\multicolumn{6}{l}{\textbf{Average ATT}}\\
All & All & -14.20 & (-53.94, 25.54) & -36.19 & (-60.74, -11.65)\\
\addlinespace[0.3em]
\multicolumn{6}{l}{\textbf{Cohort-Time ATTs}}\\
2020 & 2021 & -4.71 & (-56.93, 47.5) & -25.44 & (-58.02, 7.13)\\
2021 & 2021 & -37.24 & (-74.15, -0.33) & -59.23 & (-79.61, -38.85)\\
\bottomrule
\multicolumn{6}{l}{\rule{0pt}{1em}\textsuperscript{a} Joint test that all ATTs are equal: F(1, 405)= 0.064, p= 0.8}\\
\multicolumn{6}{l}{\rule{0pt}{1em}\textsuperscript{b} Joint test that all ATTs are equal: F(1, 368)= 0.756, p= 0.385}\\
\end{tabular}
\end{table}

\hypertarget{indoor-temperature}{%
\subsubsection{Indoor temperature}\label{indoor-temperature}}

\hypertarget{tbl-a-het-temp}{}
\begin{table}[H]
\caption{\label{tbl-a-het-temp}Heterogenous treatment effects: Indoor temperature }\tabularnewline

\centering\begingroup\fontsize{9}{11}\selectfont

\begin{tabular}{rrrllrllrll}
\toprule
\multicolumn{2}{c}{ } & \multicolumn{3}{c}{Point temp (°C)} & \multicolumn{3}{c}{Mean temp (°C)} & \multicolumn{3}{c}{Min temp (°C)} \\
\cmidrule(l{3pt}r{3pt}){3-5} \cmidrule(l{3pt}r{3pt}){6-8} \cmidrule(l{3pt}r{3pt}){9-11}
Cohort & Time & ATT & (95\%CI) & p-value & ATT & (95\%CI) & p-value & ATT & (95\%CI) & p-value\\
\midrule
\addlinespace[0.3em]
\multicolumn{11}{l}{\textbf{All times}}\\
 & 2019 & 1.77 & (0.66, 2.88) &  & 0.43 & (-0.71, 1.57) &  & 1.96 & (0.43, 3.48) & \\
\cmidrule{2-11}
 & 2020 &  &  &  & 0.52 & (-0.22, 1.26) &  & 2.42 & (0.54, 4.3) & \\
\cmidrule{2-11}
\multirow[t]{-3}{*}{\raggedleft\arraybackslash \hspace{1em}2019} & 2021 & 2.29 & (0.51, 4.07) &  & 0.79 & (0, 1.57) &  & 4.93 & (2.28, 7.58) & \\
\cmidrule{1-11}
 & 2020 &  &  &  & 0.87 & (-0.2, 1.93) &  & 5.00 & (3.22, 6.79) & \\
\cmidrule{2-11}
\multirow[t]{-2}{*}{\raggedleft\arraybackslash \hspace{1em}2020} & 2021 & 2.36 & (0.54, 4.17) &  & 0.58 & (-0.66, 1.82) &  & 6.87 & (4.35, 9.39) & \\
\cmidrule{1-11}
\hspace{1em}2021 & 2021 & 0.64 & (-1.08, 2.35) & 0.440 & 1.06 & (0.32, 1.79) & 0.320 & 2.04 & (0.08, 4) & 0.000\\
\cmidrule{1-11}
\addlinespace[0.3em]
\multicolumn{11}{l}{\textbf{Daytime}}\\
 & 2019 &  &  &  & 0.44 & (-0.96, 1.83) &  &  &  & \\
\cmidrule{2-11}
 & 2020 &  &  &  & 1.26 & (0.36, 2.17) &  &  &  & \\
\cmidrule{2-11}
\multirow[t]{-3}{*}{\raggedleft\arraybackslash \hspace{1em}2019} & 2021 &  &  &  & 1.50 & (0.55, 2.46) &  &  &  & \\
\cmidrule{1-11}
 & 2020 &  &  &  & 0.28 & (-1.45, 2.02) &  &  &  & \\
\cmidrule{2-11}
\multirow[t]{-2}{*}{\raggedleft\arraybackslash \hspace{1em}2020} & 2021 &  &  &  & 0.13 & (-1.7, 1.97) &  &  &  & \\
\cmidrule{1-11}
\hspace{1em}2021 & 2021 &  &  &  & 1.44 & (0.64, 2.25) & 0.260 &  &  & \\
\cmidrule{1-11}
\addlinespace[0.3em]
\multicolumn{11}{l}{\textbf{Daytime heating}}\\
 & 2019 &  &  &  & 0.80 & (-0.48, 2.09) &  &  &  & \\
\cmidrule{2-11}
 & 2020 &  &  &  & 1.43 & (0.04, 2.83) &  &  &  & \\
\cmidrule{2-11}
\multirow[t]{-3}{*}{\raggedleft\arraybackslash \hspace{1em}2019} & 2021 &  &  &  & 2.33 & (1.03, 3.62) &  &  &  & \\
\cmidrule{1-11}
 & 2020 &  &  &  & 2.63 & (1.87, 3.39) &  &  &  & \\
\cmidrule{2-11}
\multirow[t]{-2}{*}{\raggedleft\arraybackslash \hspace{1em}2020} & 2021 &  &  &  & 2.46 & (1.46, 3.46) &  &  &  & \\
\cmidrule{1-11}
\hspace{1em}2021 & 2021 &  &  &  & 2.13 & (0.67, 3.59) & 0.000 &  &  & \\
\cmidrule{1-11}
\addlinespace[0.3em]
\multicolumn{11}{l}{\textbf{Heating season}}\\
 & 2019 &  &  &  & 1.05 & (-0.1, 2.2) &  & 1.94 & (0.42, 3.47) & \\
\cmidrule{2-11}
 & 2020 &  &  &  & 1.23 & (-0.11, 2.58) &  & 2.41 & (0.53, 4.3) & \\
\cmidrule{2-11}
\multirow[t]{-3}{*}{\raggedleft\arraybackslash \hspace{1em}2019} & 2021 &  &  &  & 2.07 & (0.88, 3.27) &  & 5.34 & (2.66, 8.02) & \\
\cmidrule{1-11}
 & 2020 &  &  &  & 2.71 & (2.04, 3.37) &  & 4.35 & (3.17, 5.53) & \\
\cmidrule{2-11}
\multirow[t]{-2}{*}{\raggedleft\arraybackslash \hspace{1em}2020} & 2021 &  &  &  & 2.48 & (1.33, 3.62) &  & 6.27 & (3.73, 8.81) & \\
\cmidrule{1-11}
\hspace{1em}2021 & 2021 &  &  &  & 1.97 & (0.53, 3.41) & 0.000 & 2.23 & (0.26, 4.21) & 0.000\\
\bottomrule
\end{tabular}
\endgroup{}
\end{table}

\newpage

\hypertarget{blood-pressure-outcomes}{%
\subsubsection{Blood pressure outcomes}\label{blood-pressure-outcomes}}

Table~\ref{tbl-bp-het} shows ATTs by treatment cohort and time, as well
as the results of joint tests of heterogeneity across ATTs.

\hypertarget{tbl-bp-het}{}
\begin{table}[H]
\caption{\label{tbl-bp-het}Heterogenous treatment effects for the total effect of the CBHP policy
on blood pressure. }\tabularnewline

\centering
\begin{threeparttable}
\begin{tabular}{>{\raggedright\arraybackslash}p{2cm}>{\raggedright\arraybackslash}p{2cm}cccc}
\toprule
\multicolumn{2}{c}{ } & \multicolumn{2}{c}{Adjusted DiD\textsuperscript{a}} & \multicolumn{2}{c}{Heterogeneity tests\textsuperscript{b}} \\
\cmidrule(l{3pt}r{3pt}){3-4} \cmidrule(l{3pt}r{3pt}){5-6}
Cohort & Time & ATT & (95\%CI) & F-Statistic & p-value\\
\midrule
\addlinespace[0.3em]
\multicolumn{6}{l}{\textbf{Brachial SBP}}\\
\hspace{1em}2019 & 2019 & -2.36 & (-5.23, 0.5) &  & \\
\hspace{1em}2019 & 2021 & -1.51 & (-4.01, 0.98) &  & \\
\hspace{1em}2020 & 2021 & -1.26 & (-4.97, 2.45) &  & \\
\hspace{1em}2021 & 2021 & 2.39 & (-0.49, 5.28) & 2.3 & 0.080\\
\addlinespace[0.3em]
\multicolumn{6}{l}{\textbf{Central SBP}}\\
\hspace{1em}2019 & 2019 & -2.03 & (-4.69, 0.63) &  & \\
\hspace{1em}2019 & 2021 & -1.96 & (-4.45, 0.52) &  & \\
\hspace{1em}2020 & 2021 & -1.78 & (-5.07, 1.52) &  & \\
\hspace{1em}2021 & 2021 & 2.11 & (-1.09, 5.31) & 1.9 & 0.140\\
\addlinespace[0.3em]
\multicolumn{6}{l}{\textbf{Brachial DBP}}\\
\hspace{1em}2019 & 2019 & -2.66 & (-4.67, -0.65) &  & \\
\hspace{1em}2019 & 2021 & -2.37 & (-4.01, -0.72) &  & \\
\hspace{1em}2020 & 2021 & 0.2 & (-1.54, 1.94) &  & \\
\hspace{1em}2021 & 2021 & 0.78 & (-0.48, 2.05) & 6.8 & 0.000\\
\addlinespace[0.3em]
\multicolumn{6}{l}{\textbf{Central DBP}}\\
\hspace{1em}2019 & 2019 & -2.67 & (-4.57, -0.78) &  & \\
\hspace{1em}2019 & 2021 & -2.55 & (-4.15, -0.94) &  & \\
\hspace{1em}2020 & 2021 & 0.11 & (-1.67, 1.9) &  & \\
\hspace{1em}2021 & 2021 & 1.09 & (-0.06, 2.23) & 10.0 & 0.000\\
\bottomrule
\end{tabular}
\begin{tablenotes}
\item \small{Note: ATT = Average Treatment Effect on the Treated, DiD = Difference-in-Differences, CDE = Controlled Direct Effect.}
\item[a] \small{Adjusted for age, sex, waist circumference, smoking, alcohol consumption, and use of blood pressure medication.}
\item[b] \small{F-statistics and p-values for joint tests of equality across cohort and time ATTs}
\end{tablenotes}
\end{threeparttable}
\end{table}

\newpage

\hypertarget{mediation-analyses-for-blood-pressure}{%
\subsubsection{Mediation analyses for blood
pressure}\label{mediation-analyses-for-blood-pressure}}

Table~\ref{tbl-a-bp-med-het} shows the cohort-time treatment effects for
the mediation model for blood pressure.

\hypertarget{tbl-a-bp-med-het}{}
\begin{table}[H]
\caption{\label{tbl-a-bp-med-het}Heterogenous treatment effects for blood pressure mediation model }\tabularnewline

\centering\begingroup\fontsize{10}{12}\selectfont

\begin{threeparttable}
\begin{tabular}{llcccccccc}
\toprule
\multicolumn{2}{c}{ } & \multicolumn{2}{c}{Adjusted Total Effect\textsuperscript{a}} & \multicolumn{6}{c}{CDE Mediated By:\textsuperscript{b}} \\
\cmidrule(l{3pt}r{3pt}){3-4} \cmidrule(l{3pt}r{3pt}){5-10}
\multicolumn{4}{c}{ } & \multicolumn{2}{c}{Indoor PM} & \multicolumn{2}{c}{Indoor Temp} & \multicolumn{2}{c}{PM + Temp} \\
\cmidrule(l{3pt}r{3pt}){5-6} \cmidrule(l{3pt}r{3pt}){7-8} \cmidrule(l{3pt}r{3pt}){9-10}
Cohort & Time & ATT & (95\%CI) & ATT & (95\%CI) & ATT & (95\%CI) & ATT & (95\%CI)\\
\midrule
\addlinespace[0.3em]
\multicolumn{10}{l}{\textbf{Brachial SBP}}\\
\hspace{1em}2019 & 2019 & -2.36 & (-5.23, 0.50) & -2.15 & (-5.14, 0.84) & -1.69 & (-4.54, 1.15) & -1.24 & (-4.20, 1.72)\\
\hspace{1em}2019 & 2021 & -1.51 & (-4.01, 0.98) & -1.27 & (-4.01, 1.47) & -0.41 & (-2.92, 2.10) & 0.01 & (-2.71, 2.74)\\
\hspace{1em}2020 & 2021 & -1.26 & (-4.97, 2.45) & -0.54 & (-4.25, 3.17) & 0.43 & (-2.86, 3.73) & 1.04 & (-2.59, 4.67)\\
\hspace{1em}2021 & 2021 & 2.39 & (-0.49, 5.28) & 2.68 & (-0.42, 5.79) & 1.95 & (-1.74, 5.64) & 1.88 & (-1.92, 5.67)\\
\addlinespace[0.3em]
\multicolumn{10}{l}{\textbf{Central SBP}}\\
\hspace{1em}2019 & 2019 & -2.03 & (-4.69, 0.63) & -1.75 & (-4.61, 1.11) & -1.40 & (-4.06, 1.27) & -0.89 & (-3.73, 1.95)\\
\hspace{1em}2019 & 2021 & -1.96 & (-4.45, 0.52) & -1.65 & (-4.40, 1.11) & -0.93 & (-3.18, 1.32) & -0.44 & (-2.95, 2.07)\\
\hspace{1em}2020 & 2021 & -1.78 & (-5.07, 1.52) & -1.00 & (-4.36, 2.36) & -0.15 & (-3.18, 2.88) & 0.47 & (-2.95, 3.89)\\
\hspace{1em}2021 & 2021 & 2.11 & (-1.09, 5.31) & 2.45 & (-0.83, 5.73) & 1.66 & (-1.73, 5.05) & 1.63 & (-1.82, 5.08)\\
\addlinespace[0.3em]
\multicolumn{10}{l}{\textbf{Brachial DBP}}\\
\hspace{1em}2019 & 2019 & -2.66 & (-4.67, -0.65) & -2.47 & (-4.70, -0.25) & -2.29 & (-4.18, -0.40) & -1.94 & (-4.03, 0.14)\\
\hspace{1em}2019 & 2021 & -2.37 & (-4.01, -0.72) & -2.10 & (-4.09, -0.11) & -1.81 & (-3.21, -0.41) & -1.50 & (-3.28, 0.27)\\
\hspace{1em}2020 & 2021 & 0.20 & (-1.54, 1.94) & 0.31 & (-1.43, 2.04) & 1.14 & (-0.65, 2.94) & 1.23 & (-0.70, 3.15)\\
\hspace{1em}2021 & 2021 & 0.78 & (-0.48, 2.05) & 1.05 & (-0.59, 2.69) & 0.20 & (-1.21, 1.62) & 0.36 & (-1.34, 2.06)\\
\addlinespace[0.3em]
\multicolumn{10}{l}{\textbf{Central DBP}}\\
\hspace{1em}2019 & 2019 & -2.67 & (-4.57, -0.78) & -2.43 & (-4.58, -0.28) & -2.52 & (-4.34, -0.70) & -2.13 & (-4.18, -0.08)\\
\hspace{1em}2019 & 2021 & -2.55 & (-4.15, -0.94) & -2.20 & (-4.18, -0.22) & -2.18 & (-3.60, -0.76) & -1.80 & (-3.58, -0.03)\\
\hspace{1em}2020 & 2021 & 0.11 & (-1.67, 1.90) & 0.22 & (-1.58, 2.01) & 1.07 & (-0.74, 2.87) & 1.16 & (-0.80, 3.13)\\
\hspace{1em}2021 & 2021 & 1.09 & (-0.06, 2.23) & 1.39 & (-0.16, 2.94) & 0.51 & (-0.80, 1.82) & 0.70 & (-0.94, 2.34)\\
\bottomrule
\end{tabular}
\begin{tablenotes}
\item \small{Note: Results combined across 30 multiply-imputed datasets. ATT = Average Treatment Effect on the Treated, CDE = Controlled Direct Effect, DBP = Diastolic blood pressure, SBP = Systolic blood pressure.}
\item[a] \small{Adjusted for age, sex, waist circumference, smoking, alcohol consumption, and use of blood pressure medication.}
\item[b] \small{Mediators were set to the mean value for untreated participants at baseline.}
\end{tablenotes}
\end{threeparttable}
\endgroup{}
\end{table}

\hypertarget{tbl-a-bp-med-het-tests}{}
\begin{table}
\caption{\label{tbl-a-bp-med-het-tests}Heterogenous treatment effects and tests for cohort-time heterogeneity
across CDEs for multiple mediation blood pressure mediation model. }\tabularnewline

\centering
\begin{threeparttable}
\begin{tabular}{>{\raggedright\arraybackslash}p{2cm}>{\raggedright\arraybackslash}p{2cm}cccc}
\toprule
\multicolumn{2}{c}{ } & \multicolumn{2}{c}{Adjusted CDE\textsuperscript{a}} & \multicolumn{2}{c}{Heterogeneity tests\textsuperscript{b}} \\
\cmidrule(l{3pt}r{3pt}){3-4} \cmidrule(l{3pt}r{3pt}){5-6}
Cohort & Time & ATT & (95\%CI) & F-Statistic & p-value\\
\midrule
\addlinespace[0.3em]
\multicolumn{6}{l}{\textbf{Brachial SBP}}\\
\hspace{1em}2019 & 2019 & -1.24 & (-4.20, 1.72) &  & \\
\hspace{1em}2019 & 2021 & 0.01 & (-2.71, 2.74) &  & \\
\hspace{1em}2020 & 2021 & 1.04 & (-2.59, 4.67) &  & \\
\hspace{1em}2021 & 2021 & 1.88 & (-1.92, 5.67) & 0.8 & 0.513\\
\addlinespace[0.3em]
\multicolumn{6}{l}{\textbf{Central SBP}}\\
\hspace{1em}2019 & 2019 & -0.89 & (-3.73, 1.95) &  & \\
\hspace{1em}2019 & 2021 & -0.44 & (-2.95, 2.07) &  & \\
\hspace{1em}2020 & 2021 & 0.47 & (-2.95, 3.89) &  & \\
\hspace{1em}2021 & 2021 & 1.63 & (-1.82, 5.08) & 0.6 & 0.608\\
\addlinespace[0.3em]
\multicolumn{6}{l}{\textbf{Brachial DBP}}\\
\hspace{1em}2019 & 2019 & -1.94 & (-4.03, 0.14) &  & \\
\hspace{1em}2019 & 2021 & -1.50 & (-3.28, 0.27) &  & \\
\hspace{1em}2020 & 2021 & 1.23 & (-0.70, 3.15) &  & \\
\hspace{1em}2021 & 2021 & 0.36 & (-1.34, 2.06) & 3.9 & 0.008\\
\addlinespace[0.3em]
\multicolumn{6}{l}{\textbf{Central DBP}}\\
\hspace{1em}2019 & 2019 & -2.13 & (-4.18, -0.08) &  & \\
\hspace{1em}2019 & 2021 & -1.80 & (-3.58, -0.03) &  & \\
\hspace{1em}2020 & 2021 & 1.16 & (-0.80, 3.13) &  & \\
\hspace{1em}2021 & 2021 & 0.70 & (-0.94, 2.34) & 4.8 & 0.003\\
\bottomrule
\end{tabular}
\begin{tablenotes}
\item \small{Note: ATT = Average Treatment Effect on the Treated, DiD = Difference-in-Differences, CDE = Controlled Direct Effect.}
\item[a] \small{Adjusted for age, sex, waist circumference, smoking, alcohol consumption, and use of blood pressure medication.}
\item[b] \small{F-statistics and p-values for joint tests of equality across cohort and time ATTs}
\end{tablenotes}
\end{threeparttable}
\end{table}

\newpage

\newpage

\hypertarget{respiratory-outcomes}{%
\subsubsection{Respiratory outcomes}\label{respiratory-outcomes}}

Appendix tables \ref{tbl-a-het-resp}, \ref{tbl-a-het-cough},
\ref{tbl-a-het-phlegm}, \ref{tbl-a-het-wheeze}, \ref{tbl-a-het-breath},
\ref{tbl-a-het-nochest} below show Average Treatment Effect on the
Treated (ATTs) by treatment cohort and time. ATTs are derived from
estimating marginal effects from extended two-way fixed effects models
with additional adjustment for age, sex, and smoking status.

\hypertarget{tbl-a-het-resp}{}
\begin{table}[H]
\caption{\label{tbl-a-het-resp}Heterogenous treatment effects for self-reported respiratory outcomes:
Any respiratory symptom }\tabularnewline

\centering
\begin{tabular}{>{\centering\arraybackslash}p{2cm}>{\centering\arraybackslash}p{2cm}cc}
\toprule
Cohort & Time & ATT & (95\%CI)\\
\midrule
\addlinespace[0.3em]
\multicolumn{4}{l}{\textbf{Average ATT}}\\
All & All & -0.08 & (-0.15, -0.01)\\
\addlinespace[0.3em]
\multicolumn{4}{l}{\textbf{Cohort-Time ATTs}}\\
2019 & 2019 & -0.11 & (-0.20, -0.02)\\
2019 & 2021 & -0.10 & (-0.21, 0.00)\\
2020 & 2021 & 0.01 & (-0.10, 0.13)\\
2021 & 2021 & -0.12 & (-0.22, -0.01)\\
\bottomrule
\multicolumn{4}{l}{\rule{0pt}{1em}\small{Note: Joint test that all ATTs are equal: F(3, 2579)= 1.283, p= 0.278.}}\\
\end{tabular}
\end{table}

\hypertarget{tbl-a-het-cough}{}
\begin{table}[H]
\caption{\label{tbl-a-het-cough}Heterogenous treatment effects for self-reported respiratory outcomes:
Coughing }\tabularnewline

\centering
\begin{tabular}{>{\centering\arraybackslash}p{2cm}>{\centering\arraybackslash}p{2cm}cc}
\toprule
Cohort & Time & ATT & (95\%CI)\\
\midrule
\addlinespace[0.3em]
\multicolumn{4}{l}{\textbf{Average ATT}}\\
All & All & -0.02 & (-0.07, 0.03)\\
\addlinespace[0.3em]
\multicolumn{4}{l}{\textbf{Cohort-Time ATTs}}\\
2019 & 2019 & -0.04 & (-0.11, 0.03)\\
2019 & 2021 & 0.01 & (-0.07, 0.08)\\
2020 & 2021 & -0.03 & (-0.10, 0.05)\\
2021 & 2021 & -0.04 & (-0.09, 0.02)\\
\bottomrule
\multicolumn{4}{l}{\rule{0pt}{1em}\small{Note: Joint test that all ATTs are equal: F(3, 2579)= 0.732, p= 0.533.}}\\
\end{tabular}
\end{table}

\hypertarget{tbl-a-het-phlegm}{}
\begin{table}[H]
\caption{\label{tbl-a-het-phlegm}Heterogenous treatment effects for self-reported respiratory outcomes:
Phlegm }\tabularnewline

\centering
\begin{tabular}{>{\centering\arraybackslash}p{2cm}>{\centering\arraybackslash}p{2cm}cc}
\toprule
Cohort & Time & ATT & (95\%CI)\\
\midrule
\addlinespace[0.3em]
\multicolumn{4}{l}{\textbf{Average ATT}}\\
All & All & -0.02 & (-0.06, 0.03)\\
\addlinespace[0.3em]
\multicolumn{4}{l}{\textbf{Cohort-Time ATTs}}\\
2019 & 2019 & -0.06 & (-0.16, 0.03)\\
2019 & 2021 & -0.03 & (-0.10, 0.04)\\
2020 & 2021 & 0.04 & (-0.02, 0.09)\\
2021 & 2021 & 0.03 & (-0.04, 0.09)\\
\bottomrule
\multicolumn{4}{l}{\rule{0pt}{1em}\small{Note: Joint test that all ATTs are equal: F(3, 2579)= 1.735, p= 0.158.}}\\
\end{tabular}
\end{table}

\hypertarget{tbl-a-het-wheeze}{}
\begin{table}[H]
\caption{\label{tbl-a-het-wheeze}Heterogenous treatment effects for self-reported respiratory outcomes:
Wheezing attacks }\tabularnewline

\centering
\begin{tabular}{>{\centering\arraybackslash}p{2cm}>{\centering\arraybackslash}p{2cm}cc}
\toprule
Cohort & Time & ATT & (95\%CI)\\
\midrule
\addlinespace[0.3em]
\multicolumn{4}{l}{\textbf{Average ATT}}\\
All & All & 0.00 & (-0.04, 0.04)\\
\addlinespace[0.3em]
\multicolumn{4}{l}{\textbf{Cohort-Time ATTs}}\\
2019 & 2019 & -0.02 & (-0.06, 0.01)\\
2019 & 2021 & 0.01 & (-0.04, 0.06)\\
2020 & 2021 & -0.03 & (-0.11, 0.05)\\
2021 & 2021 & 0.09 & (-0.00, 0.18)\\
\bottomrule
\multicolumn{4}{l}{\rule{0pt}{1em}\small{Note: Joint test that all ATTs are equal: F(3, 2579)= 2.923, p= 0.033.}}\\
\end{tabular}
\end{table}

\hypertarget{tbl-a-het-breath}{}
\begin{table}[H]
\caption{\label{tbl-a-het-breath}Heterogenous treatment effects for self-reported respiratory outcomes:
Trouble breathing }\tabularnewline

\centering
\begin{tabular}{>{\centering\arraybackslash}p{2cm}>{\centering\arraybackslash}p{2cm}cc}
\toprule
Cohort & Time & ATT & (95\%CI)\\
\midrule
\addlinespace[0.3em]
\multicolumn{4}{l}{\textbf{Average ATT}}\\
All & All & -0.05 & (-0.12, 0.02)\\
\addlinespace[0.3em]
\multicolumn{4}{l}{\textbf{Cohort-Time ATTs}}\\
2019 & 2019 & -0.06 & (-0.16, 0.04)\\
2019 & 2021 & -0.07 & (-0.16, 0.03)\\
2020 & 2021 & 0.01 & (-0.09, 0.11)\\
2021 & 2021 & -0.07 & (-0.20, 0.06)\\
\bottomrule
\multicolumn{4}{l}{\rule{0pt}{1em}\small{Note: Joint test that all ATTs are equal: F(3, 2579)= 0.718, p= 0.541.}}\\
\end{tabular}
\end{table}

\hypertarget{tbl-a-het-nochest}{}
\begin{table}[H]
\caption{\label{tbl-a-het-nochest}Heterogenous treatment effects for self-reported respiratory outcomes:
Chest trouble }\tabularnewline

\centering
\begin{tabular}{>{\centering\arraybackslash}p{2cm}>{\centering\arraybackslash}p{2cm}cc}
\toprule
Cohort & Time & ATT & (95\%CI)\\
\midrule
\addlinespace[0.3em]
\multicolumn{4}{l}{\textbf{Average ATT}}\\
All & All & -0.06 & (-0.12, -0.01)\\
\addlinespace[0.3em]
\multicolumn{4}{l}{\textbf{Cohort-Time ATTs}}\\
2019 & 2019 & -0.06 & (-0.13, 0.01)\\
2019 & 2021 & -0.06 & (-0.15, 0.03)\\
2020 & 2021 & -0.05 & (-0.16, 0.05)\\
2021 & 2021 & -0.14 & (-0.22, -0.05)\\
\bottomrule
\multicolumn{4}{l}{\rule{0pt}{1em}\small{Note: Joint test that all ATTs are equal: F(3, 2579)= 1.046, p= 0.371.}}\\
\end{tabular}
\end{table}

\hypertarget{outdoor-and-personal-mixed-combustion}{%
\subsubsection{Outdoor and personal mixed
combustion}\label{outdoor-and-personal-mixed-combustion}}

\begin{figure}[H]

{\centering \includegraphics[width=1\textwidth,height=\textheight]{images/did-mixed-ct.png}

}

\caption{\label{fig-afig-mixed-ct}Adjusted and unadjusted treatment
effect for outdoor and personal exposure (µg/m\textsuperscript{3}) to
the mixed combustion source by treatment year.}

\end{figure}

\newpage

\hypertarget{impact-of-sample-composition-on-feno-results}{%
\subsection{Impact of sample composition on FeNO
results}\label{impact-of-sample-composition-on-feno-results}}

Table~\ref{tbl-a-feno} shows differences in the ATTs for the impact of
the CBHP policy on FeNO depending on whether the estimation sample
includes all individuals or is limited to those with repeated measures
across campaigns.

\hypertarget{tbl-a-feno}{}
\begin{table}[H]
\caption{\label{tbl-a-feno}Effects of the CBHP policy on FeNO (ppb) based on the number of
individuals with repeated measurements. }\tabularnewline

\centering
\begin{tabular}{lcccccc}
\toprule
\multicolumn{1}{c}{ } & \multicolumn{2}{c}{All participants} & \multicolumn{2}{c}{Participants with >1 measure} & \multicolumn{2}{c}{Participants with 3 measures} \\
\cmidrule(l{3pt}r{3pt}){2-3} \cmidrule(l{3pt}r{3pt}){4-5} \cmidrule(l{3pt}r{3pt}){6-7}
 & ATT & (95\%CI) & ATT & (95\%CI) & ATT & (95\%CI)\\
\midrule
DiD & 0.17 & (-2.24, 2.58) & 0.13 & (-3.09, 3.35) & -0.57 & (-3.08, 1.94)\\
Adjusted DiD & 0.55 & (-2.03, 3.13) & 0.24 & (-3.19, 3.67) & 0.27 & (-2.39, 2.92)\\
Observations & 794 &  & 526 &  & 252 & \\
\bottomrule
\end{tabular}
\end{table}

\newpage

\hypertarget{impact-of-including-season-3-data}{%
\subsection{Impact of including Season 3
data}\label{impact-of-including-season-3-data}}

Table~\ref{tbl-a-ind-s3} shows differences in the ATTs for the impact of
seasonal indoor PM\textsubscript{2.5} when season 3 data (collected in
41 villages during COVID-19) are included versus excluded.

\hypertarget{tbl-a-ind-s3}{}
\begin{table}[H]
\caption{\label{tbl-a-ind-s3}Effects of the CBHP policy on indoor seasonal PM\textsubscript{2.5}
based on whether Season 3 data are included vs.~excluded. }\tabularnewline

\centering
\begin{tabular}{>{\centering\arraybackslash}p{1.5cm}>{\centering\arraybackslash}p{1.5cm}cccc}
\toprule
\multicolumn{2}{c}{ } & \multicolumn{2}{c}{With Season 3 data} & \multicolumn{2}{c}{Without Season 3 data} \\
\cmidrule(l{3pt}r{3pt}){3-4} \cmidrule(l{3pt}r{3pt}){5-6}
Cohort & Time & ATT & (95\%CI) & ATT & (95\%CI)\\
\midrule
\addlinespace[0.3em]
\multicolumn{6}{l}{\textbf{Average ATT}}\\
All & All & -37.49 & (-60.11, -14.88) & -35.11 & (-59.36, -10.85)\\
\addlinespace[0.3em]
\multicolumn{6}{l}{\textbf{Cohort-Time ATTs}}\\
2020 & 2020 & -36.94 & (-61.39, -12.49) & 0.00 & (NA, NA)\\
2020 & 2021 & -33.51 & (-66.84, -0.18) & -30.22 & (-63.77, 3.32)\\
2021 & 2021 & -46.82 & (-58.57, -35.07) & -44.88 & (-60.41, -29.34)\\
\bottomrule
\multicolumn{6}{l}{\rule{0pt}{1em}\textit{Note: }}\\
\multicolumn{6}{l}{\rule{0pt}{1em}Sample sizes for}\\
\end{tabular}
\end{table}

\hypertarget{about-the-authors}{%
\section*{About the authors}\label{about-the-authors}}
\addcontentsline{toc}{section}{About the authors}

\hypertarget{other-publications}{%
\section*{Other publications}\label{other-publications}}
\addcontentsline{toc}{section}{Other publications}

Li X, Baumgartner J, Barrington-Leigh C, Harper S, Robinson B, Shen G,
et al.~2022a. Socioeconomic and Demographic Associations with Wintertime
Air Pollution Exposures at Household, Community, and District Scales in
Rural Beijing, China. Environ Sci Technol 56:8308--8318;
doi:10.1021/acs.est.1c07402.

Li X, Baumgartner J, Harper S, Zhang X, Sternbach T, Barrington-Leigh C,
et al.~2022b. Field measurements of indoor and community air quality in
rural Beijing before, during, and after the COVID-19 lockdown. Indoor
Air 32:e13095; doi:10.1111/ina.13095.

Sternbach TJ, Harper S, Li X, Zhang X, Carter E, Zhang Y, et al.~2022.
Effects of indoor and outdoor temperatures on blood pressure and central
hemodynamics in a wintertime longitudinal study of Chinese adults. J
Hypertension 40:1950--1959; doi:10.1097/HJH.0000000000003198.

{[}a{]}(\textbf{ellison.carter?})(\textbf{gmail.com?}) Do these numbers
show air pollution levels in S2? While, these numbers do not match with
S2 in Table 5. \emph{Assigned to ellison.carter@gmail.com}
{[}b{]}(\textbf{xiaoyingcsu?})(\textbf{gmail.com?}) Can you help me
verify that this is true? Based on the reductions observed, I think
indoor PM2.5 mass concentrations in treated homes should have reached
IT-4, but I'm not sure. \emph{Assigned to xiaoyingcsu@gmail.com}
{[}c{]}(\textbf{talia.sternbach?})(\textbf{gmail.com?}). DO NOT READ
THIS UNTIL AFTER THE WEEKEND!! But if you can drop in the figures you
created awhile back, that would be great. {[}d{]}This needs organization
and punchier writing but I'm losing steam.
{[}e{]}(\textbf{talia.sternbach?})(\textbf{gmail.com?}). To discuss in
the meeting on Monday.



\end{document}

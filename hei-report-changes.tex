% Options for packages loaded elsewhere
%DIF LATEXDIFF DIFFERENCE FILE
%DIF DEL hei-report.tex           Mon Aug 19 16:11:39 2024
%DIF ADD hei-report-revised.tex   Wed Oct 16 12:02:18 2024
\PassOptionsToPackage{unicode}{hyperref}
\PassOptionsToPackage{hyphens}{url}
\PassOptionsToPackage{dvipsnames,svgnames,x11names}{xcolor}
%
\documentclass[
  letterpaper,
  DIV=11,
  numbers=noendperiod]{scrartcl}

\usepackage{amsmath,amssymb}
\usepackage{iftex}
\ifPDFTeX
  \usepackage[T1]{fontenc}
  \usepackage[utf8]{inputenc}
  \usepackage{textcomp} % provide euro and other symbols
\else % if luatex or xetex
  \usepackage{unicode-math}
  \defaultfontfeatures{Scale=MatchLowercase}
  \defaultfontfeatures[\rmfamily]{Ligatures=TeX,Scale=1}
\fi
\usepackage{lmodern}
\ifPDFTeX\else  
    % xetex/luatex font selection
\fi
% Use upquote if available, for straight quotes in verbatim environments
\IfFileExists{upquote.sty}{\usepackage{upquote}}{}
\IfFileExists{microtype.sty}{% use microtype if available
  \usepackage[]{microtype}
  \UseMicrotypeSet[protrusion]{basicmath} % disable protrusion for tt fonts
}{}
\makeatletter
\@ifundefined{KOMAClassName}{% if non-KOMA class
  \IfFileExists{parskip.sty}{%
    \usepackage{parskip}
  }{% else
    \setlength{\parindent}{0pt}
    \setlength{\parskip}{6pt plus 2pt minus 1pt}}
}{% if KOMA class
  \KOMAoptions{parskip=half}}
\makeatother
\usepackage{xcolor}
\usepackage[right=1in,left=1in]{geometry}
\setlength{\emergencystretch}{3em} % prevent overfull lines
\setcounter{secnumdepth}{3}
% Make \paragraph and \subparagraph free-standing
\makeatletter
\ifx\paragraph\undefined\else
  \let\oldparagraph\paragraph
  \renewcommand{\paragraph}{
    \@ifstar
      \xxxParagraphStar
      \xxxParagraphNoStar
  }
  \newcommand{\xxxParagraphStar}[1]{\oldparagraph*{#1}\mbox{}}
  \newcommand{\xxxParagraphNoStar}[1]{\oldparagraph{#1}\mbox{}}
\fi
\ifx\subparagraph\undefined\else
  \let\oldsubparagraph\subparagraph
  \renewcommand{\subparagraph}{
    \@ifstar
      \xxxSubParagraphStar
      \xxxSubParagraphNoStar
  }
  \newcommand{\xxxSubParagraphStar}[1]{\oldsubparagraph*{#1}\mbox{}}
  \newcommand{\xxxSubParagraphNoStar}[1]{\oldsubparagraph{#1}\mbox{}}
\fi
\makeatother


\providecommand{\tightlist}{%
  \setlength{\itemsep}{0pt}\setlength{\parskip}{0pt}}\usepackage{longtable,booktabs,array}
\usepackage{calc} % for calculating minipage widths
% Correct order of tables after \paragraph or \subparagraph
\usepackage{etoolbox}
\makeatletter
\patchcmd\longtable{\par}{\if@noskipsec\mbox{}\fi\par}{}{}
\makeatother
% Allow footnotes in longtable head/foot
\IfFileExists{footnotehyper.sty}{\usepackage{footnotehyper}}{\usepackage{footnote}}
\makesavenoteenv{longtable}
\usepackage{graphicx}
\makeatletter
\def\maxwidth{\ifdim\Gin@nat@width>\linewidth\linewidth\else\Gin@nat@width\fi}
\def\maxheight{\ifdim\Gin@nat@height>\textheight\textheight\else\Gin@nat@height\fi}
\makeatother
% Scale images if necessary, so that they will not overflow the page
% margins by default, and it is still possible to overwrite the defaults
% using explicit options in \includegraphics[width, height, ...]{}
\setkeys{Gin}{width=\maxwidth,height=\maxheight,keepaspectratio}
% Set default figure placement to htbp
\makeatletter
\def\fps@figure{htbp}
\makeatother
% definitions for citeproc citations
\NewDocumentCommand\citeproctext{}{}
\NewDocumentCommand\citeproc{mm}{%
  \begingroup\def\citeproctext{#2}\cite{#1}\endgroup}
\makeatletter
 % allow citations to break across lines
 \let\@cite@ofmt\@firstofone
 % avoid brackets around text for \cite:
 \def\@biblabel#1{}
 \def\@cite#1#2{{#1\if@tempswa , #2\fi}}
\makeatother
\newlength{\cslhangindent}
\setlength{\cslhangindent}{1.5em}
\newlength{\csllabelwidth}
\setlength{\csllabelwidth}{3em}
\newenvironment{CSLReferences}[2] % #1 hanging-indent, #2 entry-spacing
 {\begin{list}{}{%
  \setlength{\itemindent}{0pt}
  \setlength{\leftmargin}{0pt}
  \setlength{\parsep}{0pt}
  % turn on hanging indent if param 1 is 1
  \ifodd #1
   \setlength{\leftmargin}{\cslhangindent}
   \setlength{\itemindent}{-1\cslhangindent}
  \fi
  % set entry spacing
  \setlength{\itemsep}{#2\baselineskip}}}
 {\end{list}}
\usepackage{calc}
\newcommand{\CSLBlock}[1]{\hfill\break\parbox[t]{\linewidth}{\strut\ignorespaces#1\strut}}
\newcommand{\CSLLeftMargin}[1]{\parbox[t]{\csllabelwidth}{\strut#1\strut}}
\newcommand{\CSLRightInline}[1]{\parbox[t]{\linewidth - \csllabelwidth}{\strut#1\strut}}
\newcommand{\CSLIndent}[1]{\hspace{\cslhangindent}#1}

\usepackage{booktabs}
\usepackage{longtable}
\usepackage{array}
\usepackage{multirow}
\usepackage{wrapfig}
\usepackage{float}
\usepackage{colortbl}
\usepackage{pdflscape}
\usepackage{tabu}
\usepackage{threeparttable}
\usepackage{threeparttablex}
\usepackage[normalem]{ulem}
\usepackage{makecell}
\usepackage{xcolor}
\usepackage{tabularray}
\usepackage[normalem]{ulem}
\usepackage{graphicx}
\UseTblrLibrary{booktabs}
%DIF 147a147
\UseTblrLibrary{rotating} %DIF > 
%DIF -------
\UseTblrLibrary{siunitx}
\NewTableCommand{\tinytableDefineColor}[3]{\definecolor{#1}{#2}{#3}}
\newcommand{\tinytableTabularrayUnderline}[1]{\underline{#1}}
\newcommand{\tinytableTabularrayStrikeout}[1]{\sout{#1}}
\usepackage{wrapfig}
\usepackage{colortbl}
\makeatletter
 \renewenvironment{table}%
   {\renewcommand\familydefault\sfdefault
    \@float{table}}
   {\end@float}
 \renewenvironment{figure}%
   {\renewcommand\familydefault\sfdefault
    \@float{figure}}
   {\end@float}
 \makeatother
%DIF 163a164-165
\usepackage{changes} %DIF > 
\AfterTOCHead[toc]{\sffamily} %DIF > 
%DIF -------
\KOMAoption{captions}{tableheading}
\makeatletter
\@ifpackageloaded{caption}{}{\usepackage{caption}}
\AtBeginDocument{%
\ifdefined\contentsname
  \renewcommand*\contentsname{Table of contents}
\else
  \newcommand\contentsname{Table of contents}
\fi
\ifdefined\listfigurename
  \renewcommand*\listfigurename{List of Figures}
\else
  \newcommand\listfigurename{List of Figures}
\fi
\ifdefined\listtablename
  \renewcommand*\listtablename{List of Tables}
\else
  \newcommand\listtablename{List of Tables}
\fi
\ifdefined\figurename
  \renewcommand*\figurename{Figure}
\else
  \newcommand\figurename{Figure}
\fi
\ifdefined\tablename
  \renewcommand*\tablename{Table}
\else
  \newcommand\tablename{Table}
\fi
}
\@ifpackageloaded{float}{}{\usepackage{float}}
\floatstyle{ruled}
\@ifundefined{c@chapter}{\newfloat{codelisting}{h}{lop}}{\newfloat{codelisting}{h}{lop}[chapter]}
\floatname{codelisting}{Listing}
\newcommand*\listoflistings{\listof{codelisting}{List of Listings}}
\makeatother
\makeatletter
\makeatother
\makeatletter
\@ifpackageloaded{caption}{}{\usepackage{caption}}
\@ifpackageloaded{subcaption}{}{\usepackage{subcaption}}
\makeatother

\ifLuaTeX
  \usepackage{selnolig}  % disable illegal ligatures
\fi
\usepackage{bookmark}

\IfFileExists{xurl.sty}{\usepackage{xurl}}{} % add URL line breaks if available
\urlstyle{same} % disable monospaced font for URLs
\hypersetup{
  pdftitle={How Do Household Energy Transitions Work?},
%DIF 215c218
%DIF <   pdfauthor={Jill Baumgartner (Co-PI); Sam Harper (Co-PI); On behalf of the Beijing Household Energy Transitions Team},
%DIF -------
  pdfauthor={Jill Baumgartner (Co-PI)1; Sam Harper (Co-PI)1; Chris Barrington-Leigh1; Collin Brehmer2; Ellison M. Carter2; Xiaoying Li2; Brian E. Robinson1; Guofeng Shen3; Talia J. Sternbach1; Shu Tao3; Kaibing Xue4; Wenlu Yuan1; Xiang Zhang1; Yuanxun Zhang4}, %DIF > 
%DIF -------
  colorlinks=true,
  linkcolor={blue},
  filecolor={Maroon},
  citecolor={Blue},
  urlcolor={Blue},
  pdfcreator={LaTeX via pandoc}}


\title{How Do Household Energy Transitions Work?\DIFaddbegin \thanks{Affiliations
{[}1{]} McGill University; {[}2{]} Colorado State University; {[}3{]}
Peking University; {[}4{]} University of the Chinese Academy of
Sciences}\DIFaddend }
\author{Jill Baumgartner (Co-PI)\DIFaddbegin \DIFadd{\textsuperscript{1} }\DIFaddend \and Sam Harper
(Co-PI)\DIFaddbegin \DIFadd{\textsuperscript{1} }\DIFaddend \and \DIFdelbegin \DIFdel{On behalf
of the Beijing Household Energy Transitions Team}\DIFdelend \DIFaddbegin \DIFadd{Chris
Barrington-Leigh\textsuperscript{1} }\and \DIFadd{Collin
Brehmer\textsuperscript{2} }\and \DIFadd{Ellison M.
Carter\textsuperscript{2} }\and \DIFadd{Xiaoying Li\textsuperscript{2} }\and \DIFadd{Brian
E. Robinson\textsuperscript{1} }\and \DIFadd{Guofeng
Shen\textsuperscript{3} }\and \DIFadd{Talia J.
Sternbach\textsuperscript{1} }\and \DIFadd{Shu
Tao\textsuperscript{3} }\and \DIFadd{Kaibing Xue\textsuperscript{4} }\and \DIFadd{Wenlu
Yuan\textsuperscript{1} }\and \DIFadd{Xiang Zhang\textsuperscript{1} }\and \DIFadd{Yuanxun
Zhang\textsuperscript{4}}\DIFaddend }
\date{\DIFdelbegin \DIFdel{2024-08-19}\DIFdelend \DIFaddbegin \DIFadd{2024-10-16}\DIFaddend }
%DIF PREAMBLE EXTENSION ADDED BY LATEXDIFF
%DIF UNDERLINE PREAMBLE %DIF PREAMBLE
\RequirePackage[normalem]{ulem} %DIF PREAMBLE
\RequirePackage{color}\definecolor{RED}{rgb}{1,0,0}\definecolor{BLUE}{rgb}{0,0,1} %DIF PREAMBLE
\providecommand{\DIFadd}[1]{{\protect\color{blue}\uwave{#1}}} %DIF PREAMBLE
\providecommand{\DIFdel}[1]{{\protect\color{red}\sout{#1}}} %DIF PREAMBLE
%DIF SAFE PREAMBLE %DIF PREAMBLE
\providecommand{\DIFaddbegin}{} %DIF PREAMBLE
\providecommand{\DIFaddend}{} %DIF PREAMBLE
\providecommand{\DIFdelbegin}{} %DIF PREAMBLE
\providecommand{\DIFdelend}{} %DIF PREAMBLE
\providecommand{\DIFmodbegin}{} %DIF PREAMBLE
\providecommand{\DIFmodend}{} %DIF PREAMBLE
%DIF FLOATSAFE PREAMBLE %DIF PREAMBLE
\providecommand{\DIFaddFL}[1]{\DIFadd{#1}} %DIF PREAMBLE
\providecommand{\DIFdelFL}[1]{\DIFdel{#1}} %DIF PREAMBLE
\providecommand{\DIFaddbeginFL}{} %DIF PREAMBLE
\providecommand{\DIFaddendFL}{} %DIF PREAMBLE
\providecommand{\DIFdelbeginFL}{} %DIF PREAMBLE
\providecommand{\DIFdelendFL}{} %DIF PREAMBLE
\newcommand{\DIFscaledelfig}{0.5}
%DIF HIGHLIGHTGRAPHICS PREAMBLE %DIF PREAMBLE
\RequirePackage{settobox} %DIF PREAMBLE
\RequirePackage{letltxmacro} %DIF PREAMBLE
\newsavebox{\DIFdelgraphicsbox} %DIF PREAMBLE
\newlength{\DIFdelgraphicswidth} %DIF PREAMBLE
\newlength{\DIFdelgraphicsheight} %DIF PREAMBLE
% store original definition of \includegraphics %DIF PREAMBLE
\LetLtxMacro{\DIFOincludegraphics}{\includegraphics} %DIF PREAMBLE
\newcommand{\DIFaddincludegraphics}[2][]{{\color{blue}\fbox{\DIFOincludegraphics[#1]{#2}}}} %DIF PREAMBLE
\newcommand{\DIFdelincludegraphics}[2][]{% %DIF PREAMBLE
\sbox{\DIFdelgraphicsbox}{\DIFOincludegraphics[#1]{#2}}% %DIF PREAMBLE
\settoboxwidth{\DIFdelgraphicswidth}{\DIFdelgraphicsbox} %DIF PREAMBLE
\settoboxtotalheight{\DIFdelgraphicsheight}{\DIFdelgraphicsbox} %DIF PREAMBLE
\scalebox{\DIFscaledelfig}{% %DIF PREAMBLE
\parbox[b]{\DIFdelgraphicswidth}{\usebox{\DIFdelgraphicsbox}\\[-\baselineskip] \rule{\DIFdelgraphicswidth}{0em}}\llap{\resizebox{\DIFdelgraphicswidth}{\DIFdelgraphicsheight}{% %DIF PREAMBLE
\setlength{\unitlength}{\DIFdelgraphicswidth}% %DIF PREAMBLE
\begin{picture}(1,1)% %DIF PREAMBLE
\thicklines\linethickness{2pt} %DIF PREAMBLE
{\color[rgb]{1,0,0}\put(0,0){\framebox(1,1){}}}% %DIF PREAMBLE
{\color[rgb]{1,0,0}\put(0,0){\line( 1,1){1}}}% %DIF PREAMBLE
{\color[rgb]{1,0,0}\put(0,1){\line(1,-1){1}}}% %DIF PREAMBLE
\end{picture}% %DIF PREAMBLE
}\hspace*{3pt}}} %DIF PREAMBLE
} %DIF PREAMBLE
\LetLtxMacro{\DIFOaddbegin}{\DIFaddbegin} %DIF PREAMBLE
\LetLtxMacro{\DIFOaddend}{\DIFaddend} %DIF PREAMBLE
\LetLtxMacro{\DIFOdelbegin}{\DIFdelbegin} %DIF PREAMBLE
\LetLtxMacro{\DIFOdelend}{\DIFdelend} %DIF PREAMBLE
\DeclareRobustCommand{\DIFaddbegin}{\DIFOaddbegin \let\includegraphics\DIFaddincludegraphics} %DIF PREAMBLE
\DeclareRobustCommand{\DIFaddend}{\DIFOaddend \let\includegraphics\DIFOincludegraphics} %DIF PREAMBLE
\DeclareRobustCommand{\DIFdelbegin}{\DIFOdelbegin \let\includegraphics\DIFdelincludegraphics} %DIF PREAMBLE
\DeclareRobustCommand{\DIFdelend}{\DIFOaddend \let\includegraphics\DIFOincludegraphics} %DIF PREAMBLE
\LetLtxMacro{\DIFOaddbeginFL}{\DIFaddbeginFL} %DIF PREAMBLE
\LetLtxMacro{\DIFOaddendFL}{\DIFaddendFL} %DIF PREAMBLE
\LetLtxMacro{\DIFOdelbeginFL}{\DIFdelbeginFL} %DIF PREAMBLE
\LetLtxMacro{\DIFOdelendFL}{\DIFdelendFL} %DIF PREAMBLE
\DeclareRobustCommand{\DIFaddbeginFL}{\DIFOaddbeginFL \let\includegraphics\DIFaddincludegraphics} %DIF PREAMBLE
\DeclareRobustCommand{\DIFaddendFL}{\DIFOaddendFL \let\includegraphics\DIFOincludegraphics} %DIF PREAMBLE
\DeclareRobustCommand{\DIFdelbeginFL}{\DIFOdelbeginFL \let\includegraphics\DIFdelincludegraphics} %DIF PREAMBLE
\DeclareRobustCommand{\DIFdelendFL}{\DIFOaddendFL \let\includegraphics\DIFOincludegraphics} %DIF PREAMBLE
%DIF AMSMATHULEM PREAMBLE %DIF PREAMBLE
\makeatletter %DIF PREAMBLE
\let\sout@orig\sout %DIF PREAMBLE
\renewcommand{\sout}[1]{\ifmmode\text{\sout@orig{\ensuremath{#1}}}\else\sout@orig{#1}\fi} %DIF PREAMBLE
\makeatother %DIF PREAMBLE
%DIF COLORLISTINGS PREAMBLE %DIF PREAMBLE
\RequirePackage{listings} %DIF PREAMBLE
\RequirePackage{color} %DIF PREAMBLE
\lstdefinelanguage{DIFcode}{ %DIF PREAMBLE
%DIF DIFCODE_UNDERLINE %DIF PREAMBLE
  moredelim=[il][\color{red}\sout]{\%DIF\ <\ }, %DIF PREAMBLE
  moredelim=[il][\color{blue}\uwave]{\%DIF\ >\ } %DIF PREAMBLE
} %DIF PREAMBLE
\lstdefinestyle{DIFverbatimstyle}{ %DIF PREAMBLE
	language=DIFcode, %DIF PREAMBLE
	basicstyle=\ttfamily, %DIF PREAMBLE
	columns=fullflexible, %DIF PREAMBLE
	keepspaces=true %DIF PREAMBLE
} %DIF PREAMBLE
\lstnewenvironment{DIFverbatim}{\lstset{style=DIFverbatimstyle}}{} %DIF PREAMBLE
\lstnewenvironment{DIFverbatim*}{\lstset{style=DIFverbatimstyle,showspaces=true}}{} %DIF PREAMBLE
\lstset{extendedchars=\true,inputencoding=utf8}

%DIF END PREAMBLE EXTENSION ADDED BY LATEXDIFF

\begin{document}
\maketitle

\renewcommand*\contentsname{Table of contents}
{
\hypersetup{linkcolor=}
\setcounter{tocdepth}{3}
\tableofcontents
}

\subsection*{Abstract}\label{abstract}

\subsubsection*{Introduction}\label{introduction}

Since 2015, thousands of \DIFaddbegin \DIFadd{rural and peri-urban }\DIFaddend villages across Beijing
and northern China have been treated by a \DIFdelbegin \DIFdel{Coal Ban and Heat Pump (CBHP) subsidy policy }\DIFdelend \DIFaddbegin \DIFadd{household Clean Heating Policy
(CHP) }\DIFaddend that banned household coal burning and subsidized the \DIFdelbegin \DIFdel{cost of
replacement
with }\DIFdelend \DIFaddbegin \DIFadd{costs of
}\DIFaddend electric heaters and electricity. Whether this large-scale policy was
successful in improving air quality and health remains an important and
unresolved question. We estimated the effects of the \DIFdelbegin \DIFdel{CBHP }\DIFdelend \DIFaddbegin \DIFadd{CHP }\DIFaddend policy on air
quality and cardiopulmonary health in Beijing villages, and quantified
how much of the policy's effects on health were mediated by changes in
air pollution and indoor temperature.

\subsubsection*{Methods}\label{methods}

In winter 2018-19 we enrolled 1003 participants in 50 Beijing villages
that were eligible for, but not currently treated by, the \DIFdelbegin \DIFdel{CBHP policy
}\DIFdelend \DIFaddbegin \DIFadd{CHP }\DIFaddend and
followed them over four consecutive winter data collection waves. In
waves 1, 2 and 4, we administered questionnaires and measured
participants' anthropometrics, blood pressure (BP), airway inflammation
(FeNO), and 24-h personal exposure to fine particulate matter
(PM\textsubscript{2.5}). Fasting whole blood samples were obtained at
clinic visits in waves 1 and 2 for analysis of glucose, lipid profile,
and markers of inflammation and oxidative stress. Wintertime outdoor
PM\textsubscript{2.5} was measured in all 4 waves, and wintertime indoor
temperature and \DIFaddbegin \DIFadd{indoor }\DIFaddend PM\textsubscript{2.5} were measured in waves 2, 3
and 4. The PM\textsubscript{2.5} filters were analyzed for \DIFdelbegin \DIFdel{mass }\DIFdelend \DIFaddbegin \DIFadd{their mass,
black carbon, }\DIFaddend and chemical composition, which were used for source
apportionment. To estimate the impacts of the policy we used a
difference-in-differences design that accommodated the staggered
\DIFdelbegin \DIFdel{rollout of the CBHP policy}\DIFdelend \DIFaddbegin \DIFadd{roll-out of the CHP}\DIFaddend . We used `extended' two-way fixed effects models and
marginal effects to quantify the effect of the policy on air pollution
and health \DIFaddbegin \DIFadd{outcomes}\DIFaddend . We further evaluated whether villages treated by
the policy in different years respond differently to the policy, and
\DIFdelbegin \DIFdel{if any of }\DIFdelend \DIFaddbegin \DIFadd{whether }\DIFaddend the observed health impacts of the policy were mediated through
changes in air pollution or home (indoor) temperature.

\subsubsection*{Results}\label{results}

At baseline (wave 1), mean participant age was 60 y (SD=9.2), 60\% were
female, and most (63\%) worked in agriculture. Geometric mean personal
exposures to PM\textsubscript{2.5} were twice as high as outdoor
PM\textsubscript{2.5} (72 versus 36 µg/m\textsuperscript{3}), and the
main source contributors were local and transported dust, regional and
domestic coal and biomass burning, and \DIFdelbegin \DIFdel{aerosols that form through
secondary formation}\DIFdelend \DIFaddbegin \DIFadd{secondary pollutants}\DIFaddend . By waves 2,
3, and 4 there were a cumulative total of 10, 17, and 20 \DIFaddbegin \DIFadd{villages }\DIFaddend (out
of 50 total) \DIFdelbegin \DIFdel{villages }\DIFdelend exposed to the \DIFdelbegin \DIFdel{CBHP policy}\DIFdelend \DIFaddbegin \DIFadd{CHP}\DIFaddend . Uptake and adherence to the policy was
high: among villages treated in wave 2, the proportion of households
using heat pumps and coal heaters, respectively, changed from 3\% and
97\% in wave 1 to 94\% and 3\% in wave 4, with similar \DIFaddbegin \DIFadd{clean energy
}\DIFaddend transitions in villages exposed to the policy in later waves. Marginal
effects derived from multivariable extended two-way fixed effects models
showed that exposure to the policy increased indoor temperature by 1-2°C
and reduced indoor seasonal PM\textsubscript{2.5} by approximately \DIFdelbegin \DIFdel{36 }\DIFdelend \DIFaddbegin \DIFadd{20
}\DIFaddend µg/m\textsuperscript{3}. \DIFdelbegin \DIFdel{Exposure to }\DIFdelend \DIFaddbegin \DIFadd{Treatment by }\DIFaddend the policy also reduced
contributions to PM\textsubscript{2.5} from solid fuel sources,
including household coal burning, and improved blood pressure
(\textasciitilde1.5 mmHg lower systolic and diastolic) and self-reported
respiratory symptoms (\DIFdelbegin \DIFdel{\textasciitilde7 }\DIFdelend \DIFaddbegin \DIFadd{\textasciitilde8 }\DIFaddend percentage point reduction in any
symptoms). There was notable heterogeneity in effects across treatment
cohorts, with larger benefits to indoor PM\textsubscript{2.5} and health
in villages treated in earlier relative to later years. In the mediation
analysis, indoor PM\textsubscript{2.5} and indoor temperature explained
most of the total effect of the policy on systolic BP and roughly half
of the total effect on diastolic BP, but did not explain improvements in
self-reported respiratory symptoms. \DIFdelbegin \DIFdel{The policy did not show }\DIFdelend \DIFaddbegin \DIFadd{We did not find }\DIFaddend evidence of
meaningful effects \DIFaddbegin \DIFadd{of the policy }\DIFaddend on outdoor or personal exposure to
PM\textsubscript{2.5} , or on biomarkers of inflammation and oxidative
stress.

\subsubsection*{Conclusions}\label{conclusions}

In this comprehensive field-based assessment of a \DIFdelbegin \DIFdel{real-world }\DIFdelend \DIFaddbegin \DIFadd{large-scale }\DIFaddend household
energy policy in Beijing, we observed high fidelity and compliance with
the \DIFdelbegin \DIFdel{CBHP policy}\DIFdelend \DIFaddbegin \DIFadd{CHP}\DIFaddend . Exposure to the policy reduced blood pressure and self-reported
chronic respiratory symptoms, and the effects for blood pressure were
mediated by reductions in indoor PM\textsubscript{2.5} and improvements
in home temperature, providing empirical evidence that clean \DIFaddbegin \DIFadd{household
}\DIFaddend energy policies can provide population health benefits.

\newpage

\section{Introduction}\label{introduction-1}

China is deploying an ambitious \DIFaddbegin \DIFadd{clean energy }\DIFaddend policy to transition up to
70\% of households in \DIFdelbegin \DIFdel{northern China }\DIFdelend \DIFaddbegin \DIFadd{its northern provinces }\DIFaddend from residential coal
\DIFdelbegin \DIFdel{heating }\DIFdelend \DIFaddbegin \DIFadd{heaters }\DIFaddend to electric or gas ``clean'' space heating, including a
large-scale roll out across rural and peri-urban Beijing \DIFaddbegin \DIFadd{villages}\DIFaddend ,
referred to in this document as \DIFdelbegin \DIFdel{China's
Coal Ban and Heat Pump (CBHP)subsidy policy}\DIFdelend \DIFaddbegin \DIFadd{the Clean Heating Policy (CHP)}\DIFaddend . To meet
this target the Beijing municipal government announced a two-pronged
program that designates coal-restricted areas and simultaneously offers
subsidies to night-time electricity rates and for the purchase and
installation of electric-powered heat pumps to replace traditional
coal-heating stoves. The policy was piloted in 2015 and, starting in
2016, was rolled out on a village-by-village basis. The variability in
when the policy was applied to each village allowed us to treat the
roll-out of the program as a quasi-randomized intervention and evaluate
its impacts on air quality and health. Household air pollution is a
well-established risk factor for adverse health outcomes over the entire
lifecourse, yet there is no consensus that clean energy interventions
can improve these health outcomes based on evidence from randomized
trials (Lai et al. 2024). Households may be differentially affected by
the \DIFdelbegin \DIFdel{CBHP }\DIFdelend \DIFaddbegin \DIFadd{CHP }\DIFaddend due to factors such as financial constraints and user
preferences, and there is uncertainty about whether and how the policy
may affect indoor and outdoor air pollution, as well as heating
behaviors and health outcomes.

\section{Background}\label{background}

\subsection{Context for the policy}\label{context-for-the-policy}

\DIFdelbegin \DIFdel{Beijing }\DIFdelend \DIFaddbegin \DIFadd{The CHP }\comment{OC 4c}\DIFadd{builds on China's long history of launching
ambitious, large-scale policies and programs to promote clean household
energy transition and support rural energy infrastructure development
(Zhang and Smith 2007). China was a relatively early initiator of rural
electrification projects in the 1950s and achieved complete (100\%)
electrification of households by 2016 (Yang 2021), which undoubtedly
facilitated the current policy option to replace coal stoves with
electric-powered heat pumps. Several decades later, China achieved what
is likely still the largest improvement in household energy efficiency
in history with regards to the population affected by a single program.
The National Improved Stove Program (NISP) and its provincial- and
county-level counterparts were initiated in the early 1980s and are
credited with introducing 180 million improved cooking and heating
stoves by the late 1990s. All NISP stoves had chimneys and some had
manual or electric blowers to promote more efficient combustion (Zhang
and Smith 2007), with the primary goal of increased biomass fuel
efficiency to promote rural welfare and reduce pressure on local forests
and a secondary goal of improving indoor air quality (Sinton et al.
2004). Because NISP focused mainly on biomass cookstoves, it had limited
impacts on the rapid increase in coal heating stove installation during
that same period, most of which were implemented without chimneys and
with rudimentary designs (Zhang and Smith 2007). Though NISP was a
significant achievement in early clean energy transition, especially for
biomass cookstoves, the rural energy demands and air pollution
challenges of 21st century China required a renewed effort to promote
transition to cleaner rural energy, particularly for rural heating where
progress significantly lagged behind energy transition for cooking.
}

\DIFadd{Most of northern China }\DIFaddend has a temperate continental monsoon climate that
is characterized by cold, dry winters and hot, humid summers. Access to
central heating is limited to urban areas and thus most peri-urban and
rural households have historically heated their homes using coal heaters
and biomass \emph{kangs} (a traditional Chinese energy technology that
integrates at least four different home functions including cooking, a
bed for sleeping, space heating, and home ventilation). Household coal
burning was a major contributor to indoor and outdoor air pollution in
northern China, especially in winter. Prior to the \DIFdelbegin \DIFdel{CBHP policy}\DIFdelend \DIFaddbegin \DIFadd{CHP}\DIFaddend , over 100 million
rural households consumed \textasciitilde200 million tons of coal to
meet more than 80\% of northern China's residential space heating demand
(Dispersed Coal Management Research Group 2023), which contributed to
roughly 30\% of wintertime air pollution (GBD MAPS Working Group 2016).
In 2013, \DIFaddbegin \DIFadd{emissions inventories indicated that }\DIFaddend coal combustion from
industrial, electricity, and residential heating sources was the single
largest estimated contributor to population \DIFdelbegin \DIFdel{exposure }\DIFdelend \DIFaddbegin \DIFadd{exposures }\DIFaddend to
PM\textsubscript{2.5} in China and responsible for an estimated 366,000
annual premature deaths (GBD MAPS Working Group 2016).

Banning residential coal burning and providing homes with clean heating
alternatives through the \DIFdelbegin \DIFdel{CBHP policy }\DIFdelend \DIFaddbegin \DIFadd{CHP }\DIFaddend was considered a potentially important
intervention to improve rural development, reduce local and regional
\DIFdelbegin \DIFdel{PM\textsubscript{2.5}}\DIFdelend \DIFaddbegin \DIFadd{fine particulate matter (PM\textsubscript{2.5})}\DIFaddend , and mitigate air
pollution-related health impacts. A number of clean heating options,
including electric heat pumps, gas heaters, and electric resistance
heaters with thermal storage, were \DIFdelbegin \DIFdel{widely }\DIFdelend promoted by the Chinese government
(Dispersed Coal Management Research Group 2023). By 2021, over 36
million households in northern China were treated by the \DIFdelbegin \DIFdel{CBHP policy }\DIFdelend \DIFaddbegin \DIFadd{CHP }\DIFaddend and an
estimated 21 million additional households are expected to be treated by
2025. Whether this large-scale energy policy yielded air quality and
health benefits remains a critical and unresolved question.

\subsection{Prior evidence on household energy interventions and air
pollution}\label{prior-evidence-on-household-energy-interventions-and-air-pollution}

Household energy interventions, mostly cooking-related, that replace
traditional solid fuel stoves with more efficient and less-polluting
alternatives have been implemented and studied extensively in countries
including China over the past several decades. While the introduction of
\DIFdelbegin \DIFdel{cleaner household stoves }\DIFdelend \DIFaddbegin \DIFadd{more efficient household stoves and fuels }\DIFaddend is expected to reduce \DIFdelbegin \DIFdel{air pollution, }\DIFdelend \DIFaddbegin \DIFadd{indoor
air pollution and exposures, evidence of }\DIFaddend their real-world effectiveness
in achieving health-relevant air pollution reductions \DIFdelbegin \DIFdel{is unclear }\DIFdelend \DIFaddbegin \DIFadd{has been mixed,
with some studies actually finding worse air quality in homes that
received the intervention }\DIFaddend (Quansah et al. 2017). \DIFdelbegin \DIFdel{In particular, }\DIFdelend \DIFaddbegin \DIFadd{Further, most previous
studies evaluated smaller-scale interventions implemented by civil
society organizations or investigators themselves, and }\DIFaddend the indoor and
local air quality benefits of large-scale household energy \DIFdelbegin \DIFdel{programs
like
the CBHP subsidy policy }\DIFdelend \DIFaddbegin \DIFadd{policies like
the CHP }\DIFaddend have been rarely empirically investigated, especially at a
sub-city spatial resolution \DIFaddbegin \DIFadd{or in countries in the Global South}\DIFaddend . In
Ireland, county-level residential coal bans in the 1990s were associated
with 40-70\% decreases in black smoke concentrations in ban-affected
areas (Dockery et al. 2013). In Australia, a wood-burning stove exchange
lowered daily wintertime PM\textsubscript{10} from 44 to 27
µg/m\textsuperscript{3} (Johnston et al. 2013), and clean energy
policies in New Zealand were associated with 11-36\% reductions in
winter PM\textsubscript{10} (Scott and Scarrott 2011). The few \DIFaddbegin \DIFadd{previous
}\DIFaddend evaluations of the \DIFdelbegin \DIFdel{CBHP policy }\DIFdelend \DIFaddbegin \DIFadd{CHP }\DIFaddend reported small decreases in outdoor
PM\textsubscript{2.5} (-7 to -2.4 µg/m\textsuperscript{3}) in
municipalities or prefectures treated by the policy compared with
untreated neighboring regions (Niu et al. 2024; Song et al. 2023; Tan et
al. 2023; Yu et al. 2021), and a recent modeling study estimated 36\%
lower personal exposure to PM\textsubscript{2.5} based on
household-reported changes in fuel use (Meng et al. 2023). \DIFdelbegin \DIFdel{However, none of these studies }\DIFdelend \DIFaddbegin \DIFadd{These studies
captured wide geographic areas, but none }\DIFaddend included field-based
measurements of air pollution or personal exposures, which \DIFdelbegin \DIFdel{are known to
}\DIFdelend \DIFaddbegin \DIFadd{can }\DIFaddend differ
considerably from \DIFdelbegin \DIFdel{from modelled estimates based on assumptions of
emissions reductions }\DIFdelend \DIFaddbegin \DIFadd{modeled estimates }\DIFaddend (Thompson et al. 2019), and few
accounted for secular changes in air quality over time, limiting
\DIFdelbegin \DIFdel{any }\DIFdelend conclusions about the causal effect of the policy on air quality.

\subsection{Prior evidence on clean energy interventions and
cardiovascular
outcomes}\label{prior-evidence-on-clean-energy-interventions-and-cardiovascular-outcomes}

Most previous health assessments of household energy interventions have
focused on cookstoves rather than heating technologies, though in many
settings cookstoves are also used for space heating. Randomized trials
of less polluting cookstoves generally indicate a cardiovascular
benefit. In older Guatemalan women, a chimney stove intervention lowered
exposure to air pollution and reduced the occurrence of nonspecific
ST-segment depression (McCracken et al. 2011). Randomized trials in
Guatemala, Nigeria, and Ghana also showed reductions in blood pressure
(systolic range: −3.7 to −1.3 mmHg) in women assigned to gas, ethanol,
or improved combustion biomass stoves. In contrast, recent single
country (Peru) and large multi-country (Household Air Pollution
Intervention Network, HAPIN) \DIFaddbegin \DIFadd{randomized }\DIFaddend trials found no benefit of LPG
stoves on \DIFaddbegin \DIFadd{gestational }\DIFaddend blood pressure (Checkley et al. 2021; Ye et al.
2022) despite much \DIFdelbegin \DIFdel{larger }\DIFdelend \DIFaddbegin \DIFadd{large }\DIFaddend reductions (\textasciitilde66\% lower) in
exposure to PM\textsubscript{2.5} and black carbon than what was
observed in trials showing a BP benefit of intervention (Johnson et al.
2022).

The few population-based evaluations of \DIFaddbegin \DIFadd{large-scale }\DIFaddend residential energy
policies also suggest a cardio-respiratory benefit of clean energy
transition. Residential wood-burning bans were associated with
reductions in cardiovascular hospitalizations (-7\%) in California (Yap
and Garcia 2015) and with reduced cardiovascular (-17.9\%) and
respiratory (−22.8\%) mortality in Australia (Johnston et al. 2013),
though neither study fully controlled for secular changes in health that
were unrelated to the policy. Most relevant to our study are two
quasi-experimental assessments of coal replacement policies. In Ireland,
reductions in respiratory not but cardiovascular mortality were observed
following \DIFdelbegin \DIFdel{a
}\DIFdelend \DIFaddbegin \DIFadd{their }\DIFaddend coal ban (Dockery et al. 2013). A multi-city study of
Chinese adults in cities where the \DIFdelbegin \DIFdel{CBHP policy }\DIFdelend \DIFaddbegin \DIFadd{CHP }\DIFaddend was piloted compared with adults
in cities not in the pilot observed small decreases in chronic lung
diseases (-3.0 to -1.1\%) but no change in physician-diagnosed
cardiovascular diseases, potentially due to the short (one-year)
post-policy evaluation period or confounding by other unmeasured
\DIFdelbegin \DIFdel{city-wide }\DIFdelend \DIFaddbegin \DIFadd{municipality-wide }\DIFaddend air quality or health-related policies (Wen et al.
2023).

Though household air pollution is \DIFaddbegin \DIFadd{considered }\DIFaddend a well-established health
risk factor, which energy interventions can \DIFaddbegin \DIFadd{most effectively }\DIFaddend reduce air
pollution exposures \DIFdelbegin \DIFdel{, improve health , and are }\DIFdelend \DIFaddbegin \DIFadd{and improve health and are also }\DIFaddend scalable and
sustainable \DIFdelbegin \DIFdel{remains a }\DIFdelend \DIFaddbegin \DIFadd{remain }\DIFaddend critical and unanswered \DIFdelbegin \DIFdel{question}\DIFdelend \DIFaddbegin \DIFadd{questions}\DIFaddend . In a recent
Official American Thoracic Society Statement, for example, the committee
did not reach a consensus that household energy interventions (including
gas, ethanol, solar, and improved biomass cookstoves) improved health
outcomes (including respiratory symptoms and blood pressure), with 55\%
\DIFaddbegin \DIFadd{of the committee }\DIFaddend saying no and 45\% saying yes (Lai et al. 2024).

\subsection{Evaluating the mechanisms through which policies may affect
health
outcomes.}\label{evaluating-the-mechanisms-through-which-policies-may-affect-health-outcomes.}

With several \DIFaddbegin \DIFadd{notable }\DIFaddend exceptions (Alexander et al. 2018; Gould et al.
2023; McCracken et al. 2007; McCracken et al. 2011), decades of
household energy intervention studies have \DIFdelbegin \DIFdel{shown }\DIFdelend \DIFaddbegin \DIFadd{found }\DIFaddend limited or no health
benefit, which demonstrates the complexity of \DIFaddbegin \DIFadd{both implementing and
}\DIFaddend evaluating interventions on \DIFdelbegin \DIFdel{exposures like }\DIFdelend cooking or space heating that are central to
daily life (Ezzati and Baumgartner 2017; Lai et al. 2024). Energy
interventions and policies, particularly those implemented at the
household- or village-scales, can produce multiple behavioral,
environmental, and health-related changes, making it important to
investigate the mechanisms through which such policies exert their
health impacts or lack of \DIFdelbegin \DIFdel{impacts }\DIFdelend \DIFaddbegin \DIFadd{impact }\DIFaddend (Dominici et al. 2014). The health
benefits achievable with transition from traditional coal stoves to a
new electric home heating system, for example, may be influenced by
factors including outdoor air quality (Lai 2019), the desirability and
usage patterns of new and traditional stoves (Ezzati and Baumgartner
2017), \DIFaddbegin \DIFadd{average or variability in }\DIFaddend indoor temperature (Lewington et al.
2012), \DIFdelbegin \DIFdel{or }\DIFdelend \DIFaddbegin \DIFadd{and }\DIFaddend behaviors including physical activity \DIFaddbegin \DIFadd{or time spent in the
home }\DIFaddend (Lindemann et al. 2017). Only recently were \DIFdelbegin \DIFdel{such }\DIFdelend \DIFaddbegin \DIFadd{these }\DIFaddend mediating factors
considered in health assessments of household energy interventions, and
rarely in a comprehensive or formalized way (Rosenthal et al. 2018).
Understanding such mechanisms can provide valuable insights into the
success (or failure) of clean energy programs or policies like the \DIFdelbegin \DIFdel{CBHP }\DIFdelend \DIFaddbegin \DIFadd{CHP
}\DIFaddend in meeting their air quality and health targets, and may answer
questions that can inform the design of more effective future energy
interventions (Lai et al. 2024). For example, is there successful uptake
of the \DIFdelbegin \DIFdel{intervention or policy? }\DIFdelend \DIFaddbegin \DIFadd{policy? Are there cardiovascular-enhancing effects of improved
air quality in homes that are treated by the policy? }\DIFaddend Does the policy
lead to heating behavior changes that result in colder homes \DIFdelbegin \DIFdel{which may
}\DIFdelend \DIFaddbegin \DIFadd{and thus
}\DIFaddend offset any cardiovascular-enhancing effects of improved air quality?
Answers to these questions are facilitated by the analysis of mediating
pathways, a key aim of this study.

\section{Specific Aims and Overarching
Approach}\label{specific-aims-and-overarching-approach}

We used three data collection waves in winter 2018/19, winter 2019/20,
and winter 2021/22, as well as a partial wave in winter 2020/21\DIFaddbegin \DIFadd{, }\DIFaddend to
advance the following aims:

\begin{enumerate}
\def\labelenumi{\arabic{enumi}.}
\item
  Estimate how much of the \DIFdelbegin \DIFdel{CBHP policy}\DIFdelend \DIFaddbegin \DIFadd{CHP}\DIFaddend 's overall effect on health, including
  respiratory symptoms and cardiovascular outcomes (blood pressure,
  blood inflammatory and oxidative stress markers), can be attributed to
  its impact on changes in PM\textsubscript{2.5};
\item
  Quantify the impact of the policy on outdoor air quality and personal
  air pollution exposures, and specifically the source contribution from
  household coal burning;
\item
  Quantify the contribution of changes in the chemical composition of
  PM\textsubscript{2.5} from different sources to the overall effect on
  health outcomes.
\end{enumerate}

\section{Study Design and Methods}\label{study-design-and-methods}

\subsection{Study area}\label{study-area}

Beijing is the capital of China (\DIFdelbegin \DIFdel{pop. }\DIFdelend \DIFaddbegin \DIFadd{population of }\DIFaddend 21.9 million in 2020) and
covers a large geographic area (\textasciitilde16,000
km\textsuperscript{2}) that includes a highly developed and
densely-populated urban core that is surrounded by several satellite
towns and thousands of peri-urban and rural villages in the periphery.
Beijing winters \DIFaddbegin \DIFadd{typically }\DIFaddend begin in early November and tend to be cold,
dry, and windy, with the lowest temperatures \DIFdelbegin \DIFdel{most
}\DIFdelend \DIFaddbegin \DIFadd{mostly }\DIFaddend often occurring in
January (-3°C, on average), thus requiring space heating (An et al.
2021). Most urban areas of Beijing are connected to a central heating
grid that supplies home heating from central locations, whereas rural
and many peri-urban areas have historically relied on individual space
heating units that, prior to 2015, were largely fueled by unprocessed
coal (Duan et al. 2014).

\subsection{Location and participant recruitment and
enrolment}\label{location-and-participant-recruitment-and-enrolment}

Between December 2018 and January 2019 we recruited 50 villages across 4
administrative districts (Fangshan, Huairou, Mentougou, and Miyun) in
the Beijing municipality in northern China. The villages predominately
used coal for heating at the time of enrollment and were eligible for
but not currently participating in the \DIFdelbegin \DIFdel{CBHP policy}\DIFdelend \DIFaddbegin \DIFadd{CHP}\DIFaddend . Roughly half of the villages
were expected to enter into the policy during our study
\DIFaddbegin \comment{EC 10c}\DIFaddend (Figure~\DIFdelbegin \DIFdel{\ref{fig-cbhp-map}}\DIFdelend \DIFaddbegin \DIFadd{\ref{fig-chp-map}}\DIFaddend ). We used local guides in each
village to help determine a roster of households that were not vacant
during the winter months, from which we randomly selected households to
recruit for participation.

\begin{figure}[H]

\DIFdelbeginFL %DIFDELCMD < \centering{
%DIFDELCMD < 

%DIFDELCMD < \includegraphics[width=0.7\textwidth,height=\textheight]{images/policy-implementation-map.png}
%DIFDELCMD < 

%DIFDELCMD < }
%DIFDELCMD < %%%
\DIFdelendFL \DIFaddbeginFL \centering{

\includegraphics[width=0.9\textwidth,height=\textheight]{images/village-map.png}

}
\DIFaddendFL 

\caption{\DIFdelbeginFL %DIFDELCMD < \label{fig-cbhp-map}%%%
\DIFdelendFL \DIFaddbeginFL \label{fig-chp-map}\DIFaddendFL Map of village implementation of \DIFdelbeginFL \DIFdelFL{CBHP
}\DIFdelendFL \DIFaddbeginFL \DIFaddFL{CHP. Each
circle represents one recruited village. The colors of the circles
indicate the year the villages were exposed to the household energy
transition }\DIFaddendFL policy\DIFaddbeginFL \DIFaddFL{.}\DIFaddendFL }

\end{figure}%

We recruited approximately 20 households in each village and\DIFdelbegin \DIFdel{randomly
selected one eligible person from each
household to participate}\DIFdelend \DIFaddbegin \DIFadd{, in each
household, obtained a household roster. Our }\comment{EC 2c}\DIFadd{tablet-based
survey incorporated a randomization tool than randomly ordered household
occupants listed on the roster. We recruited a participant in each
household by starting at the top of the randomly ordered list until an
eligible participant was identified}\DIFaddend . Household members were eligible to
participate if they were over 40 years old, lived in the study villages,
were not planning to move out of the village in the next year, and were
not on current immunotherapy or treatment with corticosteroids.
\DIFaddbegin 

\DIFaddend Research staff introduced the study and its measurements to an eligible
adult in each household and answered any questions related to the study.
In follow-up visits to the study villages, staff first approached
households with participants from an earlier wave. \DIFdelbegin \DIFdel{If previous participants were not }\DIFdelend \DIFaddbegin \DIFadd{Due }\comment{EC 2d}\DIFadd{to
study logistics, we were limited to one day for study measurements in
each village and wave, such that participants who were outside of the
village on the measurement day for work or shopping were not able to
participate in that wave. If a previous participant was not }\DIFaddend at home or
refused to participate, staff first tried to randomly recruit \DIFdelbegin \DIFdel{an eligible participant from the same household }\DIFdelend \DIFaddbegin \DIFadd{the next
eligible participant listed on the randomized household roster}\DIFaddend . If there
was not another eligible or willing participant in the \DIFaddbegin \DIFadd{same }\DIFaddend household,
we \DIFdelbegin \DIFdel{randomly selected and
}\DIFdelend recruited a participant from a new household using the \DIFdelbegin \DIFdel{village roster.
}\DIFdelend \DIFaddbegin \DIFadd{same process
for household and participant selection described above. In Wave 2, we
recruited 81 new participants from a previously enrolled household and
189 new households. In Wave 4, we recruited 91 new participants from a
previously enrolled household and 68 new households. Our village level
study utilizes individual-level data such that each participant is
considered independently.
}

\DIFaddend All participants provided written informed consent prior to joining the
study. The study protocols were approved by research ethics boards at
Peking University (IRB00001052-18090), Peking Union Medical College
Hospital (HS-3184) and McGill University (A08-E53-18B).

\subsection{Data Collection Overview}\label{data-collection-overview}

We conducted study measurements over four consecutive waves of data
collection in winter 2018-19, 2019-20, 2020-21, and 2021-22 (referred to
hereafter as Wave 1 {[}\DIFdelbegin \DIFdel{w1}\DIFdelend \DIFaddbegin \DIFadd{W1}\DIFaddend {]}, \DIFdelbegin \DIFdel{w2, w3 and w4}\DIFdelend \DIFaddbegin \DIFadd{W2, W3 and W4}\DIFaddend , respectively). Field data
collection was conducted by \textasciitilde20 trained staff members who
traveled to participants' homes to conduct tablet-based household and
individual questionnaires, measure participant blood pressure, and
distribute temperature sensors (for measurement of indoor temperature
and stove use) and air pollution monitors in all 50 study villages in
\DIFdelbegin \DIFdel{w1, w2, and w4}\DIFdelend \DIFaddbegin \DIFadd{W1, W2, and W4}\DIFaddend . Anthropometrics (height, weight, and waist
circumference), measurement of airway inflammation, and whole blood
samples were obtained no more than a month later at a village clinic in
\DIFdelbegin \DIFdel{w1 and w2. In w3}\DIFdelend \DIFaddbegin \DIFadd{W1 and W2. In W3 }\comment{EC 3b}\DIFaddend , which was during the height of the
COVID-19 pandemic, we limited household measurements to indoor air
quality and sensor-based measurement of indoor temperature and stove use
in 41 villages, including all 17 treated villages and 24 untreated
villages, prior to \DIFaddbegin \DIFadd{Beijing-wide }\DIFaddend COVID-19-related travel restrictions
that halted field data collection. In \DIFdelbegin \DIFdel{w4}\DIFdelend \DIFaddbegin \DIFadd{W4}\DIFaddend , which also occurred during the
COVID-19 pandemic, we returned to conducting individual-level
assessments. However, unlike in \DIFdelbegin \DIFdel{w1 and w2}\DIFdelend \DIFaddbegin \DIFadd{W1 and W2}\DIFaddend , anthropometric measurements
and airway inflammation were assessed in participant homes rather than
\DIFaddbegin \DIFadd{in }\DIFaddend clinics to avoid group contact, and blood samples were not collected.
Outdoor (community) air pollution was measured \DIFdelbegin \DIFdel{throughout the study period}\DIFdelend \DIFaddbegin \DIFadd{in all waves}\DIFaddend .

\subsubsection{Air Pollution}\label{air-pollution}

\paragraph{Outdoor air pollution}\label{outdoor-air-pollution}

\DIFdelbegin \DIFdel{In each village, two sensors for particulate matter air pollution were
set up to measure }\DIFdelend \DIFaddbegin \DIFadd{For }\DIFaddend outdoor (community) PM\textsubscript{2.5} \DIFaddbegin \DIFadd{monitoring, we deployed
between one to three (typically, two) real-time sensors (PMS7003
}\comment{OC 1a}\DIFadd{Plantower, Zefan, Inc.) }\DIFaddend at different locations in each
village. \DIFaddbegin \DIFadd{The sensors were assembled with a data logger, electronic
screen, and a USB hub into a small metal box that was placed inside an
environmental enclosure. }\DIFaddend One sensor was \DIFaddbegin \DIFadd{always }\DIFaddend placed near the center of
the village, and the other \DIFdelbegin \DIFdel{was }\DIFdelend \DIFaddbegin \DIFadd{one or two sensors were }\DIFaddend placed no less than
500m away from the centrally-located sensor. Sensors were \DIFdelbegin \DIFdel{placed }\DIFdelend \DIFaddbegin \DIFadd{positioned }\DIFaddend at
least 1.5m above the ground and \DIFdelbegin \DIFdel{not in a location within sight of a visible point source }\DIFdelend \DIFaddbegin \DIFadd{away from visible point sources }\DIFaddend of
PM\textsubscript{2.5}.

We \DIFdelbegin \DIFdel{collected }\DIFdelend \DIFaddbegin \comment{EC 5b}\DIFadd{co-located the real-time sensors with a gravimetric
(}\DIFaddend filter-based\DIFdelbegin \DIFdel{community PM\textsubscript{2.5} samples to
calibrate the sensor-based PM\textsubscript{2.5} measurements as well as
to conduct }\DIFdelend \DIFaddbegin \DIFadd{) monitor for sensor calibration and }\DIFaddend analysis of chemical
composition for source apportionment. Ultrasonic Personal Aerosol
Samplers (UPAS, Access Sensor Technologies, Fort Collins, CO, USA) were
used to collect filter-based PM\textsubscript{2.5} samples with a flow
rate of 1.0 L/min (Volckens et al. 2017). \DIFaddbegin \DIFadd{The filter-based
PM\textsubscript{2.5} samples were collected and replaced approximately
every seven days throughout the winter, and we rotated the UPAS monitors
between villages. Each UPAS was placed inside a custom-built
environmental enclosure with a tight fit to prevent any resampling of
filtered air. }\DIFaddend Samplers housed 37mm PTFE filters (VWR, 2.0µm pore size)
and were equipped with a cyclone inlet with a 2.5µm cut point designed
to perform under the sampling flow rate.
\DIFdelbegin \DIFdel{For community measurements, a UPAS
was }\DIFdelend \DIFaddbegin 

\DIFadd{In W1, we deployed }\DIFaddend co-located \DIFdelbegin \DIFdel{with each PM\textsubscript{2.5} sensor in each village in
rotation. Every week, the used filters were removed and replaced with a
new filter. }\DIFdelend \DIFaddbegin \DIFadd{outdoor PM\textasciitilde2.5 sensors and
samplers in 44 of the 50 study villages due to logistical constraints,
and obtained sensor data for 40 villages due to instrument failure in 4
villages. Outdoor data for all 50 villages were obtained in waves 2, 3
and 4. }\DIFaddend In total, we \DIFdelbegin \DIFdel{successfully collected 126, 371, }\DIFdelend \DIFaddbegin \DIFadd{collected 138, 374, 279, }\DIFaddend and \DIFdelbegin \DIFdel{289
filter-based, community }\DIFdelend \DIFaddbegin \DIFadd{295 }\DIFaddend outdoor
PM\textsubscript{2.5} \DIFdelbegin \DIFdel{samples in w1, w2, and w4}\DIFdelend \DIFaddbegin \DIFadd{filter samples in W1, W2, W3, and W4}\DIFaddend ,
respectively. Field blank \DIFaddbegin \DIFadd{PTFE }\DIFaddend filters were collected at a rate of
\textasciitilde10\%, subject to the same field conditions as samples.
\DIFaddbegin 

\DIFaddend To support post-sampling determination of organic carbon (OC) and
elemental carbon (EC) fractions of PM\textsubscript{2.5} mass, quartz
filters were co-located with a subset of Teflon filter samples collected
outdoors. Quartz filter-based PM\textsubscript{2.5} samples were
collected using UPAS operating with a flow rate of 1.0 L/min. \DIFdelbegin \DIFdel{UPASs }\DIFdelend \DIFaddbegin \DIFadd{UPAS
monitors }\DIFaddend housed 37 mm quartz filters (VWR, 2.0-µm pore size) and were
equipped with a cyclone inlet with a 2.5µm cut point designed to perform
under the corresponding sampling flow rate. All quartz fiber filters
were baked at 550 °C for a minimum of 8 h to remove organic impurities
prior to sample collection. \DIFaddbegin \DIFadd{The }\DIFaddend PM\textsubscript{2.5} samples collected
on quartz filters were analyzed using established thermo-optical methods
for quantifying elemental carbon (EC) and organic carbon (OC) to, then,
calibrate the colorimetric analysis of EC and OC on Teflon filters
\DIFdelbegin \DIFdel{. In w2, 23 quartz-based outdoor
PM}\DIFdelend \DIFaddbegin \DIFadd{(details of this analysis and subsequent calibration are provided in
``Optical properties and chemical analysis of PM}\DIFaddend \textsubscript{2.5}
\DIFdelbegin \DIFdel{samples and 3 field blanks were collected. In w4,
}\DIFdelend \DIFaddbegin \DIFadd{mass''). We co-located quartz filters with Teflon filter samples for 23
measurements in W2 and }\DIFaddend 11 \DIFdelbegin \DIFdel{quartz-based outdoor PM\textsubscript{2.5} samples and }\DIFdelend \DIFaddbegin \DIFadd{measurements in W4, along with }\DIFaddend 3 \DIFdelbegin \DIFdel{field blanks
were collected.
}%DIFDELCMD < 

%DIFDELCMD < %%%
\DIFdel{For PM\textsubscript{2.5} sensor calibration and quality control, all PMsensors were co-located with a reference-grade PM\textsubscript{2.5}
instrument (Model 5030 Synchronized Hybrid Ambient Realtime Particulate
(SHARP)Monitor, Thermo Fisher Scientific, United States) on the rooftop
of a building at Peking University campus and/or the Tapered Element
Oscillating Microbalance (TEOM, Thermo Scientific™ 1405 TEOM™) at the
Chinese Academy of Sciences University campus for 7 to 10 days before
and after each field wave (Figure 2). Sensor-measured
PM\textsubscript{2.5} concentrations were highly correlated with those
measured by the reference instruments (Spearman correlation coefficients
(rho) \textgreater0.75 in each pre- and post-calibration)}\DIFdelend \DIFaddbegin \DIFadd{quartz field
blanks in both seasons}\DIFaddend .

\begin{figure}[H]

\centering{

\includegraphics[width=0.8\textwidth,height=\textheight]{images/sensor-calibration.png}

}

\caption{\label{fig-calibration}Calibration of real-time sensors against
a reference monitor at University of the Chinese Academy of Sciences.}

\end{figure}%

\paragraph{\texorpdfstring{Indoor
PM\textsubscript{2.5}}{Indoor PM2.5}}\label{indoor-pm2.5}

In \DIFdelbegin \DIFdel{the second, third, and fourth data collection waves }\DIFdelend \DIFaddbegin \DIFadd{study waves 2, 3 and 4, }\DIFaddend we randomly selected six households from the
\DIFdelbegin \DIFdel{20 }\DIFdelend \DIFaddbegin \DIFadd{\textasciitilde20 }\DIFaddend recruited in each village \DIFdelbegin \DIFdel{to measure
indoor concentrations of }\DIFdelend \DIFaddbegin \DIFadd{for measurement of indoor
}\DIFaddend PM\textsubscript{2.5}. In \DIFdelbegin \DIFdel{w4}\DIFdelend \DIFaddbegin \DIFadd{W3 and W4}\DIFaddend , we aimed to monitor indoor
PM\textsubscript{2.5} in the same households \DIFdelbegin \DIFdel{where we
measured indoor PM\textsubscript{2.5} in w2. If a household
dropped out
of the project or }\DIFdelend \DIFaddbegin \DIFadd{sampled in W2. If household
occupants were not at home or if participants }\DIFaddend declined indoor
PM\textsubscript{2.5} monitoring, we \DIFdelbegin \DIFdel{then }\DIFdelend \DIFaddbegin \DIFadd{randomly }\DIFaddend recruited another
household already enrolled in \DIFdelbegin \DIFdel{this studyto
measure indoor PM\textsubscript{2.5}}\DIFdelend \DIFaddbegin \DIFadd{the study}\DIFaddend . In total, indoor
\DIFdelbegin \DIFdel{measurements were
conducted in 300 households in both w2 and w4 and 246 households in w3
(Table1}\DIFdelend \DIFaddbegin \DIFadd{PM\textsubscript{2.5} was measured in 264 households in W2, 346
households in W3, and 244 households in W4 (Table~\ref{tbl-pm-sample}}\DIFaddend ).

\DIFdelbegin %DIFDELCMD < \begin{table}
%DIFDELCMD < 

%DIFDELCMD < %%%
%DIFDELCMD < \caption{%
{%DIFAUXCMD
%DIFDELCMD < \label{tbl-pm-sample}%%%
\DIFdelFL{Household recruitment for overall and
indoor air quality measurements.}}
%DIFAUXCMD
%DIFDELCMD < 

%DIFDELCMD < \centering{
%DIFDELCMD < 

%DIFDELCMD < \centering
%DIFDELCMD < \begin{tblr}[         %% tabularray outer open
%DIFDELCMD < ]                     %% tabularray outer close
%DIFDELCMD < {                     %% tabularray inner open
%DIFDELCMD < colspec={Q[]Q[]Q[]Q[]Q[]Q[]Q[]Q[]},
%DIFDELCMD < cell{1}{1}={c=1,}{halign=c,},
%DIFDELCMD < cell{1}{2}={c=3,}{halign=c,},
%DIFDELCMD < cell{1}{5}={c=4,}{halign=c,},
%DIFDELCMD < }                     %% tabularray inner close
%DIFDELCMD < \toprule
%DIFDELCMD < & Overall &  &  & Indoor &  &  &  \\ \cmidrule[lr]{1-1}\cmidrule[lr]{2-4}\cmidrule[lr]{5-8}
%DIFDELCMD < Sample & Wave 1 & Wave 2 & Wave 4 & Wave 1 & Wave 2 & Wave 3 & Wave 4 \\ \midrule %% TinyTableHeader
%DIFDELCMD < New recruitment & 977 &  196 &   68 &  0 & 300 &   0 &  52 \\
%DIFDELCMD < Wave 1 households & - &  866 &  780 & - &   0 &   0 &   0 \\
%DIFDELCMD < Wave 2 households & - & - &  162 & - & - & 246 & 248 \\
%DIFDELCMD < Total recruitment & 977 & 1062 & 1010 &  0 & 300 & 246 & 300 \\
%DIFDELCMD < \bottomrule
%DIFDELCMD < \end{tblr}
%DIFDELCMD < 

%DIFDELCMD < }
%DIFDELCMD < 

%DIFDELCMD < \end{table}%%%
%DIF < 
%DIFDELCMD < 

%DIFDELCMD < %%%
\DIFdelend Time-resolved indoor PM\textsubscript{2.5} was measured \DIFaddbegin \DIFadd{in all
households }\DIFaddend using the same commercially available sensor (PMS7003
Plantower, Zefan, Inc.) used for outdoor sensor-based
PM\textsubscript{2.5} \DIFdelbegin \DIFdel{and recorded }\DIFdelend \DIFaddbegin \DIFadd{measurements and recorded PM\textsubscript{2.5}
concentrations }\DIFaddend every 1 min. The sensor was placed on a table in a room
where participants reported spending most of their time\DIFdelbegin \DIFdel{when awake}\DIFdelend \DIFaddbegin \DIFadd{, e.g., a living
room or bedroom}\DIFaddend . Indoor PM\textsubscript{2.5} sensors were deployed
between late November and mid January \DIFdelbegin \DIFdel{within field
waves, }\DIFdelend \DIFaddbegin \DIFadd{in each wave, with the start time
}\DIFaddend depending on the village \DIFdelbegin \DIFdel{and household visit schedule. Measurement }\DIFdelend \DIFaddbegin \DIFadd{visit date. Measurements }\DIFaddend continued from the
time of deployment until sensors were \DIFdelbegin \DIFdel{recollected }\DIFdelend \DIFaddbegin \DIFadd{collected }\DIFaddend from homes in late
April\DIFdelbegin \DIFdel{to capture the full heating season}\DIFdelend .

\DIFaddbegin \begin{table}[H]

\caption{\label{tbl-pm-sample}\DIFaddFL{Household recruitment for overall and
indoor air quality measurements.}}

\centering{

\centering
\begin{tblr}[         %% tabularray outer open
]                     %% tabularray outer close
{                     %% tabularray inner open
colspec={Q[]Q[]Q[]Q[]Q[]Q[]Q[]Q[]},
cell{1}{1}={c=1,}{halign=c,},
cell{1}{2}={c=3,}{halign=c,},
cell{1}{5}={c=4,}{halign=c,},
}                     %% tabularray inner close
\toprule
& Overall &  &  & Indoor &  &  &  \\ \cmidrule[lr]{1-1}\cmidrule[lr]{2-4}\cmidrule[lr]{5-8}
Sample & Wave 1 & Wave 2 & Wave 4 & Wave 1 & Wave 2 & Wave 3 & Wave 4 \\ \midrule %% TinyTableHeader
New recruitment & 977 &  196 &   68 &  0 & 300 &   0 &  52 \\
Wave 1 households & - &  866 &  780 & - &   0 &   0 &   0 \\
Wave 2 households & - & - &  162 & - & - & 246 & 248 \\
Total recruitment & 977 & 1062 & 1010 &  0 & 300 & 246 & 300 \\
\bottomrule
\end{tblr}

}

\end{table}%DIF > 

\DIFaddend We randomly selected three households \DIFdelbegin \DIFdel{with PM\textsubscript{2.5} sensors
}\DIFdelend \DIFaddbegin \DIFadd{from the six with indoor
PM\textsubscript{2.5} measurement }\DIFaddend to co-locate a filter-based
PM\textsubscript{2.5} sampler. We collected a 24-h PM\textsubscript{2.5}
filter sample during the first 24-h of \DIFdelbegin \DIFdel{indoor PM\textsubscript{2.5} sensor measurements}\DIFdelend \DIFaddbegin \DIFadd{sensor-based measurement}\DIFaddend .
Filter-based PM\textsubscript{2.5} samples were collected using \DIFdelbegin \DIFdel{Ultrasonic Personal
Aerosol Samplers (UPAS , Access Sensor Technologies) }\DIFdelend \DIFaddbegin \DIFadd{UPAS }\DIFaddend or
Personal Exposure Monitors (PEMs, Apex Pro) operating with flow rates of
1.0 and 1.8 L/min, respectively. Both samplers housed 37 mm PTFE filters
(VWR, 2.0-μm pore size) and were equipped with a cyclone inlet with a
2.5 μm cut point designed to perform under the corresponding sampling
flow rate. In total, we collected \DIFdelbegin \DIFdel{149 and 148 }\DIFdelend \DIFaddbegin \DIFadd{150 and 151 }\DIFaddend indoor
PM\textsubscript{2.5} filter samples in \DIFdelbegin \DIFdel{w2 and w4}\DIFdelend \DIFaddbegin \DIFadd{W2 and W4}\DIFaddend , respectively. \DIFaddbegin \DIFadd{We did
not measure filter-based indoor PM\textsubscript{2.5} in S3 to avoid
contact with household occupants during the COVID-19 pandemic. Field
blanks were collected at a rate of approximately 10\%.
}\DIFaddend 

As with the \DIFdelbegin \DIFdel{community }\DIFdelend outdoor air sampling, to support post-sampling determination
of organic carbon (OC) and elemental carbon (EC) fractions of
PM\textsubscript{2.5} mass, quartz filters were co-located with a subset
of Teflon filter samples collected in homes. Filter-based
PM\textsubscript{2.5} samples were collected using Personal Exposure
Monitors (PEMs, Apex Pro) operating with flow rates of 1.8 L/min. PEMs
housed 37 mm quartz filters (VWR, 2.0µm pore size) and were equipped
with a cyclone inlet with a 2.5µm cut point designed to perform under
the corresponding sampling flow rate. All quartz fiber filters were
baked at 550°C for a minimum of 8 h to remove organic impurities prior
to sample collection. PM\textsubscript{2.5} samples collected on quartz
filters were analyzed using established thermo-optical methods for
quantifying \DIFdelbegin \DIFdel{elemental carbon (EC ) and organic carbon (OC ) }\DIFdelend \DIFaddbegin \DIFadd{EC and OC }\DIFaddend to, then, calibrate the colorimetric analysis of
EC and OC on Teflon filters. In \DIFdelbegin \DIFdel{w2}\DIFdelend \DIFaddbegin \DIFadd{W2}\DIFaddend , 71 quartz-based indoor
PM\textsubscript{2.5} samples and 14 field blanks were successfully
collected. In \DIFdelbegin \DIFdel{w4}\DIFdelend \DIFaddbegin \DIFadd{W4}\DIFaddend , indoor PM\textsubscript{2.5} samples for gravimetric
analysis had to be collected on two types of PTFE sample media (Zefluor
and Teflo filters), due to \DIFdelbegin \DIFdel{discontinuation of
manufacturing of the Zefluor filter media}\DIFdelend \DIFaddbegin \DIFadd{the discontinuation of Zefluor filters}\DIFaddend . To
ensure that quartz filters were deployed with both types of Teflon-based
filter media, 73 quartz-based indoor PM\textsubscript{2.5} samples were
collected concurrently with Zefluor samples, and 47 quartz indoor
PM\textsubscript{2.5} samples were collected alongside Teflo samples.
For indoor quartz PM\textsubscript{2.5} mass sampling in \DIFdelbegin \DIFdel{w4}\DIFdelend \DIFaddbegin \DIFadd{W4}\DIFaddend , 18 field
blanks were collected.

\paragraph{\texorpdfstring{Personal exposure to PM\textsubscript{2.5}
and black
carbon}{Personal exposure to PM2.5 and black carbon}}\label{personal-exposure-to-pm2.5-and-black-carbon}

\DIFdelbegin \DIFdel{To measure personal exposure we used }\DIFdelend \DIFaddbegin \DIFadd{In study waves 1, 2 and 4, we randomly selected approximately ten study
participants in each village for 24-h personal exposure measurement
using }\DIFaddend two types of \DIFdelbegin \DIFdel{samplers: Personal
Exposure Monitors (PEMs , Apex Pro; Casella, UK) and Ultrasonic Personal
Aerosol Samplers (UPAS, Access Sensor Technologies, Fort Collins, CO,
USA). }\DIFdelend \DIFaddbegin \DIFadd{PM\textsubscript{2.5} samplers: }\DIFaddend PEMs \DIFaddbegin \DIFadd{and UPAS. The
PEMs }\DIFaddend actively sampled air at a flow rate of 1.8 L/min, and UPAS sampled
air at 1.0 L/min (Volckens et al. 2017). Both samplers housed 37 mm PTFE
filters (VWR, 2.0µm pore size) and were equipped with a cyclone inlet
with a 2.5µm cutpoint. Sampler flow rates were calibrated the night
before deployment and measured immediately after the sampling period.
Only 2\% of the post-sampling measurements deviated from the target flow
rate by greater than +/-10\%. Participants were instructed to wear \DIFaddbegin \DIFadd{the
sampler in either }\DIFaddend a small waistpack (for the PEM and sampling pump)\DIFdelbegin \DIFdel{or }\DIFdelend \DIFaddbegin \DIFadd{, }\DIFaddend an
arm band\DIFdelbegin \DIFdel{or }\DIFdelend \DIFaddbegin \DIFadd{, or a }\DIFaddend cross-body sling (for the UPAS) for 24-h, which they
could remove from their body and place within 2 meters while sleeping,
sitting, or bathing. Field blanks for personal air pollution exposure
measurements were collected at a rate of \textasciitilde10\% in each
village. \DIFaddbegin \DIFadd{Across the study waves 1, 2 and 4, study participants
contributed 494, 498, and 499 personal PM\textsubscript{2.5}
measurements, respectively.
}\DIFaddend 

\DIFaddbegin \begin{table}[H]

\caption{\label{tbl-filters}\DIFaddFL{Count of total outdoor and personal exposure
PM\textsubscript{2.5} samples (filters) collected over the course of the
project and number included for analysis.}}

\centering{

\centering
\begin{talltblr}[         %% tabularray outer open
entry=none,label=none,
note{a}={Number of samples that met inclusion criteria for analysis (see text).},
]                     %% tabularray outer close
{                     %% tabularray inner open
colspec={Q[]Q[]Q[]Q[]Q[]Q[]Q[]Q[]Q[]},
cell{1}{2}={c=2,}{halign=c,},
cell{1}{4}={c=2,}{halign=c,},
cell{1}{6}={c=2,}{halign=c,},
cell{1}{8}={c=2,}{halign=c,},
}                     %% tabularray inner close
\toprule
& Wave 1 &  & Wave 2 &  & Wave 3 &  & Wave 4 &  \\ \cmidrule[lr]{2-3}\cmidrule[lr]{4-5}\cmidrule[lr]{6-7}\cmidrule[lr]{8-9}
PM2.5 sample type & Total & Included\textsuperscript{a} & Total & Included\textsuperscript{a} & Total & Included\textsuperscript{a} & Total & Included\textsuperscript{a} \\ \midrule %% TinyTableHeader
Outdoor & 138 & 126 & 374 & 363 & 279 & 213 & 295 & 266 \\
Indoor &  &  & 150 & 150 &  &  & 151 & 138 \\
Personal & 494 & 448 & 498 & 429 &  &  & 499 & 418 \\
Blank &  52 &  52 &  56 &  56 &  27 &  24 & 101 &  95 \\
\bottomrule
\end{talltblr}

}

\end{table}%DIF > 

\DIFaddend \paragraph{\texorpdfstring{Gravimetric analyses of PTFE filter-based
PM\textsubscript{2.5}
samples}{Gravimetric analyses of PTFE filter-based PM2.5 samples}}\label{gravimetric-analyses-of-ptfe-filter-based-pm2.5-samples}

All filters were \DIFdelbegin \DIFdel{placed in individually labeled cases, sealed in plastic
bags, and then transported to a field laboratory and immediately stored
in a -20°C freezer. Following completion of the field sampling campaign,
the samples and blanks were transported to }\DIFdelend \DIFaddbegin \DIFadd{gravimetrically analyzed (weighed) for their mass
before and after deployment at a laboratory at }\DIFaddend Colorado State
University\DIFdelbegin \DIFdel{,
where they were stored in a -20°C freezer prior to gravimetric and
chemical analysis. }%DIFDELCMD < 

%DIFDELCMD < %%%
\DIFdel{All }\DIFdelend \DIFaddbegin \DIFadd{. Briefly, the }\DIFaddend filters were placed in an
environmentally-controlled equilibration chamber (21-22°C, 30-34\%
relative humidity) for at least 24-h before tare and gross weighing\DIFdelbegin \DIFdel{(L'Orange et al.
2021). Before weighing }\DIFdelend \DIFaddbegin \DIFadd{.
Before each weight was taken, }\DIFaddend we neutralized static charges by passing
the filters over a polonium-210 strip. Filters were weighed on a
microbalance (Mettler Toledo Inc., XS3DU, USA) with \DIFdelbegin \DIFdel{1µg }\DIFdelend \DIFaddbegin \DIFadd{1 µg }\DIFaddend resolution in
triplicate or more, until the differences among the last three weights
were less than 3 μg. The \DIFdelbegin \DIFdel{average of three readings }\DIFdelend \DIFaddbegin \DIFadd{filters were stored in individually labeled
cases and sealed in plastic bags to avoid contamination during
transportation and storage. After deployment, filter samples and blanks
were immediately stored in a -20°C freezer and, at the end of each field
campaign, were transported to Colorado State University, where they were
stored in a -20°C freezer prior to gravimetric and chemical analysis.
The difference in the average filter weights from before versus after
deployment }\DIFaddend was used to determine \DIFdelbegin \DIFdel{filter }\DIFdelend \DIFaddbegin \DIFadd{PM\textsubscript{2.5} }\DIFaddend mass, which was
then blank-corrected using the median value of blank filters (\DIFdelbegin \DIFdel{3µg }\DIFdelend \DIFaddbegin \DIFadd{3 µg }\DIFaddend for
UPAS-collected filters {[}53\% of \DIFaddbegin \DIFadd{filter }\DIFaddend samples{]}; 33µg for
PEM-collected filters {[}47\% of filter samples{]}), and
PM\textsubscript{2.5} concentrations were calculated by dividing the
mass by the sampled air volume.

\DIFaddbegin \DIFadd{We }\comment{EC 5b}\DIFadd{excluded gravimetric (filter) samples meeting any of
the following four criteria from the statistical analysis: (1) run time
of less than 80\% of a 24-h target for personal exposure measurement, as
this is a commonly used cutoff for establishing whether a sample is
considered representative of a typical day; (2) negative mass or
extremely high mass values (e.g., \textgreater{} 2000
µg/m\textsuperscript{3}) that indicate a potential error in data
collection or data entry; (3) missing information on sampling volume of
air; (4) filters were damaged including punctures, tears, or holes; or
(5) filters were missing during gravimetric analysis and therefore had
no mass data. The final counts of gravimetric PM\textsubscript{2.5}
samples and blanks that met our criteria for inclusion in statistical
analysis are given in Table~\ref{tbl-filters}.
}

\DIFaddend \paragraph{\texorpdfstring{Adjusting sensor-based PM\textsubscript{2.5}
using filter-based gravimetric
measurements}{Adjusting sensor-based PM2.5 using filter-based gravimetric measurements}}\label{adjusting-sensor-based-pm2.5-using-filter-based-gravimetric-measurements}

\DIFaddbegin \DIFadd{Pre- and post- campaign sensor calibration }\comment{OC 1b}\DIFadd{was conducted
to assess whether low-cost sensors responded linearly to
PM\textsubscript{2.5} concentrations measured by co-located federal
equivalent method (FEM) instruments. Sensors were deployed alongside the
FEM instruments for 7-10 days before and after each field campaign. In
waves 1, 2 and 3, we co-located the sensors with a rooftop Thermo
Electron Synchronized Hybrid Ambient Real-Time Particulate (SHARP)
Monitor (model 5030) at Peking University, which is a typical urban site
(urban site). In waves 2, 3 and 4, we additionally co-located the
sensors with a rooftop Tapered Element Oscillating Microbalance Method
(TEOM, Thermo Scientific™ 1405 TEOM™) at the University of the Chinese
Academy of Sciences, which is located in a peri-urban area of Beijing
(peri-urban site) where we also have study villages. The FEM instrument
at Peking University was not functioning after wave 1 data collection,
so we instead calibrated the sensors using data from the nearest China
National Environmental Monitoring Centre (CNEMC) monitor (publicly
available }\href{https://quotsoft.net/air}{\DIFadd{here}}\DIFadd{). The closest distance
from the government monitoring stations to Peking University and Chinese
Academy of Sciences University campuses are 1.7 and 9.9 km,
respectively.
}

\DIFadd{We }\comment{OC 1b}\DIFadd{evaluated the performance of all PM\textsubscript{2.5}
sensors for the 7-10 day deployments described above that occurred
before and after each study wave (Figure~\ref{fig-calibration}).
Sensor-measured PM\textsubscript{2.5} values were highly correlated with
the FEM instruments (Spearman correlation (rho) \textgreater0.75 in all
pre- and post-calibration campaigns). Daily collections of 24-hour
Zeflour (Teflon) and quartz filter samples accompanied the sensors'
measurements. For pre- and post-campaign calibration periods, as
described above, the filter-based data were supplemental to the FEM
data; whereas sensor calibration during field campaigns, as described
below, was achieved using calibration with concurrently collected
filter-based data (since FEM references were not available in the
field). We also monitored the sensor-collected data throughout each
study wave to identify sensors in need of repair or replacement (e.g.,
logging data with the wrong time stamps or only ``0'' values) and
exclude them from deployment. This approach aimed to maintain consistent
and accurate measurements from the PM sensors throughout the study.
}

\DIFadd{Sensor calibration during each wave was conducted by deploying
filter-based measurements concurrently with sensor-based measurements,
to establish the linear regression between the low-cost sensor monitored
data and the reference data, and then apply the slope of the linear
regression to adjust the low-cost sensor monitored data. To calibrate
the outdoor and indoor measurements of the low-cost sensors in the
field, we collocated an Ultrasonic Personal Aerosol Samplers (UPAS,
Access Sensor Technologies) (Volckens et al. 2017) or Personal Exposure
Monitors (PEMs, Apex Pro) with low-cost sensors to collect
filter-derived PM\textsubscript{2.5} samples during each field season
(wintertime) (Li et al. 2022). The UPAS and PEM were equipped with a
cyclone inlet with a 2.5 μm cut point designed to perform under the
sampling flow rate of 1 and 1.8 L/min, respectively, and housed a 37 mm
PTFE filter (VWR, 2.0-μm pore size). The filter samples were transported
to Colorado State University, where they were stored in a −20 °C freezer
prior to PM\textsubscript{2.5} mass measurement.
}

\DIFaddend We established linear regression models between the filter-based
PM\textsubscript{2.5} mass \DIFdelbegin \DIFdel{concentrations }\DIFdelend (i.e., the `gold standard' reference) and the
sensor-based PM\textsubscript{2.5} \DIFdelbegin \DIFdel{concentrations
}\DIFdelend averaged over the same sampling
\DIFdelbegin \DIFdel{period as the filter-based samples. The
slopes of the models were used as the adjustment factors for the
sensor-based PM\textsubscript{2.5} concentrations. }\DIFdelend \DIFaddbegin \DIFadd{periods. }\DIFaddend Separate regression models were conducted for indoor and
outdoor sensors and for each \DIFdelbegin \DIFdel{data
collection }\DIFdelend \DIFaddbegin \DIFadd{study }\DIFaddend wave given the sensitivity of the
sensors to relative humidity, temperature, and particle sources, which
may differ for indoor versus outdoor conditions and across \DIFdelbegin \DIFdel{waves In w3}\DIFdelend \DIFaddbegin \DIFadd{years. The
model slopes were used as the adjustment factors for the sensor-based
PM\textsubscript{2.5} concentrations for that wave. In W3}\DIFaddend , where only
sensor-based measurements were conducted for indoor
PM\textsubscript{2.5}, we applied an adjustment factor \DIFdelbegin \DIFdel{developed from a
linear regression model that incorporated data from both w2 and w4}\DIFdelend \DIFaddbegin \DIFadd{that was
developed from paired indoor filter-sensor data from W2 and W4}\DIFaddend .

\DIFdelbegin \DIFdel{The PM sensors were also evaluated before and after each data collection
wave to identify any sensors that needed further repair or replacement.
The PM\textsubscript{2.5} sensors underwent a calibration process that
began with synchronization to real-time }\DIFdelend \DIFaddbegin \DIFadd{We }\comment{OC 1c}\DIFadd{identified larger than normal biases and root mean
square errors (RMSE) between the sensors and FEM instruments during the
post-W3 calibration, however further scrutiny of these data indicate
that the differences can be attributed to atypically low air pollution
and high humidity during co-location rather than a malfunction of the
sensors themselves. Sensor calibration }\comment{OC 1d}\DIFadd{results directly
informed the data correction processes applied to account for biases.
Generally, the higher correlations between the }\DIFaddend PM\textsubscript{2.5}
\DIFdelbegin \DIFdel{monitors
at Peking University (PKU) campus. This pre- and post-wave calibration
included a week-long session using the
Beta Attenuation Monitor (BAM)
alongside daily 24-hour filter samples. During this time, approximately
240 sensors were placed on the rooftop of the College of Urban and Environmental Sciences building, each recording data every minute. A
similar approach was taken at the University of Chinese Academy of
Sciences (UCAS) campus, where around 400 PM sensors were installed on
the rooftop of the Environmental Monitoring Site of the College of
Resources and Environment, with datalogging at one-minute intervals.
Daily collections of 24-hour PTFE and quartz filter samples accompanied
the sensors' measurements to ensure accuracy. The calibration process
was repeated post-fieldwork to account for any potential shifts or
discrepancies in sensor performance. This approach aimed to maintain
consistent }\DIFdelend \DIFaddbegin \DIFadd{sensor response and FEM data allowed for the application of linear
regression models to adjust sensor measurements, }\DIFaddend and \DIFdelbegin \DIFdel{accurate measurements from the }\DIFdelend \DIFaddbegin \DIFadd{filter-based
gravimetric samples were used to validate and correct the sensor data.
}

\DIFadd{We used }\comment{EC 5, OC 1d}\DIFadd{the adjusted sensor-based
}\DIFaddend PM\DIFdelbegin \DIFdel{sensors throughout the
study}\DIFdelend \DIFaddbegin \DIFadd{\textsubscript{2.5} measurements to calculate a wintertime seasonal
mean for indoor and outdoor PM\textsubscript{2.5} for the period of
January 15 to March 15 in each wave to facilitate consistent comparisons
across villages in each wave and over time. Additionally, we captured a
24-hour indoor PM\textsubscript{2.5} concentration that was temporarily
matched with the timing of personal exposure assessments in the same
household to facilitate a comparison between indoor
PM\textsubscript{2.5} and personal exposure results taken during the
same period}\DIFaddend .

\paragraph{\DIFdelbegin \DIFdel{Chemical analysis of PM
mass}\DIFdelend \DIFaddbegin \texorpdfstring{Optical properties and chemical analysis of
PM\textsubscript{2.5}
mass}{Optical properties and chemical analysis of PM2.5 mass}\DIFaddend }\DIFdelbegin %DIFDELCMD < \label{chemical-analysis-of-pm-mass}
%DIFDELCMD < %%%
\DIFdelend \DIFaddbegin \label{optical-properties-and-chemical-analysis-of-pm2.5-mass}
\DIFaddend 

We analyzed the \DIFaddbegin \DIFadd{optical properties and }\DIFaddend chemical composition of \DIFdelbegin \DIFdel{community }\DIFdelend \DIFaddbegin \DIFadd{outdoor
}\DIFaddend and personal exposure \DIFaddbegin \DIFadd{gravimetric }\DIFaddend PM\textsubscript{2.5} samples to
quantify the individual components and species. \DIFdelbegin \DIFdel{PM\textsubscript{2.5} component concentrations }\DIFdelend \DIFaddbegin \DIFadd{For each sample, the
components }\DIFaddend were determined by dividing the quantified component mass by
the sampled air volume, after correcting for field blanks collected in
the corresponding \DIFaddbegin \DIFadd{study }\DIFaddend wave.

\DIFdelbegin \DIFdel{Elemental analysisof PM\textsubscript{2.5} mass was performed using a
Thermo Scientific Quant'X Evo energy-dispersive X-ray fluorescence
(EDXRF) spectrometer with Wintrace software version 10.3 using standard
methods (RTI International 2009). Quantitative mass concentrations of 22
individual elements (Mg, Al, Si, S, K, Ca, Ti, Cr, Mn, Fe, Ni, Cu, Zn,
Ga, As, Se, Cd, In, Sn, Sb, Te, I)were determined empirically using
linear standard curves. Standard curves were generated from commercial,
single and dual element, thin film standards from MicroMatter
Technologies Inc.~(Montreal, Canada) in addition to blank films. The
quality of the
analysis method was evaluated by analyzing a National
Institute of Standards and Technology (NIST) standard reference material
(SRM) 2783 Air particulate on filter media (Gaithersburg, MD, USA). Elements for which at least 80\% of PM\textsubscript{2.5} masssamples
yielded quantifiable element mass were included for positive matrix
factorization and source analysis and apportionment. Those elements
were: Si, Mg, Fe, S, Ca, Al, K, Pb.
}%DIFDELCMD < 

%DIFDELCMD < %%%
\DIFdel{For analysis of water-soluble ions, a portion of each PTFE filter was extracted in 15 mL deionized water (DI Water) in a Nalgene Amber HDPE
bottle using sonication without heat for 40 min. The extracts were
filtered to ensure that insoluble particles were removed using a 0.2 μm
PTFE syringe filter. Water-soluble ions were measured using a dual
channel Dionex ICS-3000 ion chromatography system. Specifically, a
Dionex IonPac CS12A analytical (3 × 150 mm)column with eluent of 20 mM
methanesulfonic acid at a flow rate of 0.5 mL/min was used to measure
cations (Ca2+}\DIFdelend \DIFaddbegin \DIFadd{Following gravimetric analysis, all PTFE filters were analyzed
non-destructively for black carbon (BC) using an optical transmissometer
data acquisition system (SootScan\textsuperscript{TM} OT21 Optical
Transmissometer; Magee Scientific, Berkeley, CA, USA). Light attenuation
through each filter was measured before and after deployment in the
field campaign. To calculate BC mass}\DIFaddend , \DIFdelbegin \DIFdel{Mg2+, Na+, NH4+, K+), while a Dionex IonPac AS14A
analytical (4 × 250 mm) column with an eluent of
1 mM sodium
bicarbonate}\DIFdelend \DIFaddbegin \DIFadd{the difference between the pre-
and post- light attenuation was converted to a mass surface loading
using the classical Magee mass absorption cross-sections of 16.6
m\textsuperscript{2}}\DIFaddend /\DIFdelbegin \DIFdel{8 mM sodium carbonate at a flow rate of 1 mL/min was used to
measure anions (SO42−}\DIFdelend \DIFaddbegin \DIFadd{g for the 880 nm channel optical BC (Ahmed et al.
2009). BC concentrations were calculated by multiplying surface loadings
by the sampled surface area of the filters (8.6 cm\textsuperscript{2}
for UPAS-collected filters; 7.1 cm\textsuperscript{2} for PEM-collected
filters), correcting for the field blank mass using the median value of
blanks (0.31 μg for UPAS-collected filters; 0.01 μg for PEM-collected
filters)}\DIFaddend , \DIFdelbegin \DIFdel{NO3−, Cl−)(Sullivan et al., 2008)}\DIFdelend \DIFaddbegin \DIFadd{and finally dividing by the sampled air volume}\DIFaddend .

Organic (OC) and elemental carbon (EC) on PTFE filters were
\DIFaddbegin \DIFadd{non-destructively }\DIFaddend measured using an optical color space sensing system.
The CIE-Lab color space optical sensing system measures the optical
properties of the PM\textsubscript{2.5} samples \DIFdelbegin \DIFdel{, and these properties }\DIFdelend \DIFaddbegin \DIFadd{which }\DIFaddend are used to
develop \DIFdelbegin \DIFdel{the }\DIFdelend EC and OC predictive models. The CIE-Lab color system is a
color-opponent space that includes all of the color models, with
dimension L* for lightness and a* and b* for the color-opponent
dimensions. More information about the CIE Lab color space system, its
formulation, and its specific application to the analysis of OC and EC
fractions of fine particulate matter pollution is provided in Khuzestani
et al. (Khuzestani et al. 2017). Briefly, all \DIFdelbegin \DIFdel{the Teflon (PTFE ) }\DIFdelend \DIFaddbegin \DIFadd{PTFE }\DIFaddend and quartz filters
\DIFdelbegin \DIFdel{collected }\DIFdelend \DIFaddbegin \DIFadd{samples and blanks }\DIFaddend were analyzed using the i1Pro Colorimeter (X-Rite,
INC. Grand Rapids, MI). The colorimeter sensor was placed directly over
the filters, and the color components were measured under the D65
instrument internal illumination light source. Each \DIFaddbegin \DIFadd{filter }\DIFaddend sample was
analyzed in triplicate, and the average value of each color coordinate
was applied as the optical property of the sample (Olson et al. 2016).
CIE Standard Illuminant D65 simulates average midday light and is a
commonly used standard illuminant, as defined by the International
Commission on Illumination (CIE). The CIE-Lab color space response
variables were used in separate random forest models for EC and OC.

The reference measurements for the random forest model development were
EC and OC \DIFdelbegin \DIFdel{determined from }\DIFdelend \DIFaddbegin \DIFadd{measured on }\DIFaddend quartz filters collected \DIFdelbegin \DIFdel{indoors }\DIFdelend \DIFaddbegin \DIFadd{both in study homes }\DIFaddend and
outdoors (as described above). \DIFdelbegin \DIFdel{PM\textsubscript{2.5} samples collected on }\DIFdelend \DIFaddbegin \DIFadd{The }\DIFaddend quartz filters were analyzed for OC
and EC \DIFdelbegin \DIFdel{using }\DIFdelend \DIFaddbegin \DIFadd{with }\DIFaddend a Sunset Laboratory OC/EC Lab instrument (Sunset
Laboratories, Inc., MODEL, USA) \DIFdelbegin \DIFdel{according to }\DIFdelend \DIFaddbegin \DIFadd{using }\DIFaddend the default Sunset Analyzer
protocol. A section of each quartz filter underwent a combined thermal
desorption-optical transmittance measurement based on NIOSH methods 5040
to differentiate and quantify the EC and OC components in
\DIFaddbegin \DIFadd{PM\textsubscript{2.5} }\DIFaddend mass. For the thermal desorption component, the
\DIFdelbegin \DIFdel{sample }\DIFdelend \DIFaddbegin \DIFadd{filter }\DIFaddend is oxidized twice \DIFdelbegin \DIFdel{, according to }\DIFdelend \DIFaddbegin \DIFadd{using }\DIFaddend a strict temperature regime. The first
oxidation stage thermally removes OC in a mobile phase of pure helium
gas \DIFdelbegin \DIFdel{to be }\DIFdelend \DIFaddbegin \DIFadd{that is }\DIFaddend converted from carbon dioxide (\DIFdelbegin \DIFdel{CO2}\DIFdelend \DIFaddbegin \DIFadd{CO\textsubscript{2}}\DIFaddend ) to
methane (\DIFdelbegin \DIFdel{CH4}\DIFdelend \DIFaddbegin \DIFadd{CH\textsubscript{4}}\DIFaddend ) gas and measured by a flame ionization
detector (FID). The second oxidation stage proceeds in a mixture of
helium and oxygen to oxidize EC, which is also quantified by the FID.
The FID is internally calibrated with methane, and external quality
control checks are made with sucrose standards. To correct for the
potential production of EC \DIFdelbegin \DIFdel{by
OC pyrolysis during the first heating }\DIFdelend \DIFaddbegin \DIFadd{during OC pyrolysis in the first oxidation
}\DIFaddend stage, light transmission from a laser through the filter section was
monitored throughout analysis. Reduced light transmittance corresponds
to EC generated by the laboratory analysis.

\DIFdelbegin \DIFdel{Following gravimetric analysis , all PTFE filters were also analyzed for
black carbon (BC) using an optical transmissometer data acquisition
system (SootScan\^{}TM OT21 Optical Transmissometer; Magee Scientific, Berkeley, CA, USA) . Light attenuation through each filter was measured
before and after sampling in the field.
To calculate BC mass , the
difference between the pre- and post- light attenuation was
converted to a mass surface loading using the classical Magee mass absorption
cross-sections of 16.6 m\textsuperscript{2}}\DIFdelend \DIFaddbegin \DIFadd{Elemental analysis of PM\textsubscript{2.5} mass was performed using a
Thermo Scientific Quant'X Evo energy-dispersive X-ray fluorescence
(EDXRF) spectrometer with Wintrace software version 10.3 using standard
methods (RTI International 2009). Quantitative mass concentrations of 22
individual elements (Mg, Al, Si, S, K, Ca, Ti, Cr, Mn, Fe, Ni, Cu, Zn,
Ga, As, Se, Cd, In, Sn, Sb, Te, I) were determined empirically using
linear standard curves. Standard curves were generated from commercial,
single and dual element, thin film standards from MicroMatter
Technologies Inc.~(Montreal, Canada) in addition to blank films. The
quality of the analysis method was evaluated by analyzing a National
Institute of Standards and Technology (NIST) standard reference material
(SRM) 2783 Air particulate on filter media (Gaithersburg, MD, USA).
Elements for which at least 80\% of PM\textsubscript{2.5} mass samples
yielded quantifiable element mass were included for positive matrix
factorization and for source analysis and apportionment. Those elements
were: Si, Mg, Fe, S, Ca, Al, K, Pb.
}

\DIFadd{For analysis of water-soluble ions, a portion of each PTFE filter was
extracted in 15 mL deionized water (DI Water) in a Nalgene Amber HDPE
bottle using sonication without heat for 40 min. The extracts were
filtered to ensure that insoluble particles were removed using a 0.2 μm
PTFE syringe filter. Water-soluble ions were measured using a dual
channel Dionex ICS-3000 ion chromatography system. Specifically, a
Dionex IonPac CS12A analytical (3 × 150 mm) column with eluent of 20 mM
methanesulfonic acid at a flow rate of 0.5 mL}\DIFaddend /\DIFdelbegin \DIFdel{g for the 880 nm channel
optical BC (Ahmed }\DIFdelend \DIFaddbegin \DIFadd{min was used to measure
cations (Ca\textsuperscript{2+}, Mg\textsuperscript{2+},
Na\textsuperscript{+}, NH\textsuperscript{4+}, K\textsuperscript{+}),
while a Dionex IonPac AS14A analytical (4 × 250 mm) column with an
eluent of 1 mM sodium bicarbonate/8 mM sodium carbonate at a flow rate
of 1 mL/min was used to measure anions
(SO\textsubscript{4}\textsuperscript{2-}, NO\textsuperscript{3-},
Cl\textsuperscript{-}) (Sullivan }\DIFaddend et al. \DIFdelbegin \DIFdel{2009).
BC concentrations were calculated by
multiplying surface loadings by the sampled surface area of the filters
(8.6 cm\textsuperscript{2} for UPAS-collected filters; 7.1
cm\textsuperscript{2} for PEM-collected filters) , correcting for the
field blank mass using the median value of blanks (0.31 μg for
UPAS-collected filters; 0.01 μg for PEM-collected filters}\DIFdelend \DIFaddbegin \DIFadd{2008).
}

\begin{itemize}
\item
  \emph{\DIFadd{wi (Water Insoluble Species)}}\DIFadd{: `wi' }\comment{PMF 5a}\DIFadd{refers to
  the fraction of particulate matter (PM) that does not dissolve in
  water. These species typically include elements such as potassium (K),
  calcium (Ca}\DIFaddend ), and \DIFdelbegin \DIFdel{finally
dividing by }\DIFdelend \DIFaddbegin \DIFadd{magnesium (Mg), among others, that remain as
  particulate matter after water extraction. In this study, due to
  budget constraints, we did not analyze water-soluble organic carbon.
  Instead, we determined the water insoluble fraction by subtracting the
  amount of the elemental species measured using ion chromatography (IC)
  from the total elemental amount determined by X-ray fluorescence
  (XRF). For instance, }\DIFaddend the \DIFdelbegin \DIFdel{sampled air volume. }%DIFDELCMD < 

%DIFDELCMD < %%%
\DIFdel{For statistical analysis, we estimated the effect of the policy on
personal exposures to PM\textsubscript{2.5} and BC using the results
from filter-based measurements collected over 24-h periods. We measured
indoor and outdoor PM\textsubscript{2.5} for up to 6 months in our study
households,
  and thus we calculated both the 24-h mean values (to
coincide with the same 24-h period that personal exposure samples were
collected) and the wintertime seasonal mean values (with winter `season'
defined as January 15 to March 15)of PM \textsubscript{2.5}.
}\DIFdelend \DIFaddbegin \DIFadd{total amount of potassium (K) was measured
  using XRF, and the water-soluble potassium fraction was quantified
  using IC. The difference between these two values was taken as the
  water insoluble fraction of potassium. This approach is consistent
  with practices in air quality research that use these analyses.
}\item
  \emph{\DIFadd{ws (Water Soluble Species)}}\DIFadd{: `ws' }\comment{PMF 5a}\DIFadd{refers to the
  fraction of particulate matter that dissolves in water, typically
  including major ions such as sulfate, nitrate, ammonium, and certain
  soluble forms of metals. The water soluble fraction was extracted from
  particulate samples using deionized water, and the extract was
  analyzed using ion chromatography (IC) to determine the concentrations
  of individual water soluble ions, such as sulfate
  (SO\textsubscript{4}\textsuperscript{2-}), nitrate
  (NO\textsuperscript{3-}), and soluble metal ions. This approach is
  well-documented in the scientific literature and follows established
  protocols for the determination of anions in PM.
}\item
  \emph{\DIFadd{ns-S (Non-Sulfate Sulfur)}}\DIFadd{: `ns-S' refers to the sulfur present
  in particulate matter that is not in the form of sulfate -- i.e.,
  non-sulfate (ns) sulfur (S). This includes species such as elemental
  sulfur, organosulfur compounds, and other non-sulfate
  sulfur-containing compounds. Total sulfur content in particulate
  matter was determined using X-ray fluorescence (XRF), consistent with
  the method we employed for this project (RTI International 2009). The
  sulfate (SO\textsubscript{4}\textsuperscript{2-}) content was
  quantified using ion chromatography. The non-sulfate sulfur (ns-S) was
  then calculated by subtracting the sulfate sulfur from the total
  sulfur determined using XRF analysis. This approach is supported by
  studies such as those by Shakya and Peltier (2015) and Secrest et al.
  (2016), which have utilized similar methodologies to distinguish
  sulfur sources in PM studies.
}\end{itemize}
\DIFaddend 

\subsubsection{Outdoor and indoor (household) air
temperature}\label{outdoor-and-indoor-household-air-temperature}

Hourly outdoor temperature and relative humidity data were obtained from
the extensive network of meteorological
\href{http://beijingair.sinaapp.com}{stations} in Beijing. We used
digital thermometers (Tianjianhuayi Inc., Beijing, China) to measure
indoor `point' temperature in the five minutes prior to BP measurement.
Staff measured temperature in a centrally located room, away from
heating sources and direct sunlight, by placing the probe in mid-air at
a height that approximated the participant's shoulder height. In a
random 75\% subsample of households in each wave, we also conducted
long-term measurements of indoor temperature by placing a real-time
temperature sensor (iButton DS1921G-F5; Thermochron, Maxim Inc., USA) in
the room where participants reported spending most of their daytime
hours when indoors. Sensors were wall-mounted at a standardized height
(\textasciitilde1.5 to 2 meters), away from major heating sources,
windows, and doors, and were programmed to log a temperature reading
every 125 minutes for up to 4 months to capture the full winter period
and early spring weeks when heating may still intermittently occur.
Prior to the start of each wave, we co-located all of the sensors and
measured temperature over two days and compared the readings. Sensors
recording values \textgreater1°C from the group median value were \DIFdelbegin \DIFdel{excluded from }\DIFdelend \DIFaddbegin \DIFadd{not
deployed for }\DIFaddend data collection.

\subsubsection{Objective measurement of household stove use using
sensors}\label{objective-measurement-of-household-stove-use-using-sensors}

Following methods used in a previous intervention evaluation study in
rural China (Clark et al. 2017), we objectively measured household
heating stove use in a random sample of households selected, also at
random, for either short- or long-term measurement. We measured
short-term (24-h) stove use for all household heating stoves in 315 and
227 households in \DIFdelbegin \DIFdel{w2 and w3}\DIFdelend \DIFaddbegin \DIFadd{W2 and W3}\DIFaddend , respectively. Long-term stove use was
assessed in 324, 273, and 585 homes in \DIFdelbegin \DIFdel{w2, w3, and w4}\DIFdelend \DIFaddbegin \DIFadd{W2, W3, and W4}\DIFaddend , respectively, for
a period of \textasciitilde6 months. We measured stove use using the
same real-time temperature data loggers used to measure seasonal indoor
temperature (iButton DS1921G-F5; Thermochron, Maxim Inc., USA). Field
staff placed the sensors on stoves and programmed them to record surface
temperature every 125 minutes, a timing decision based on pilot
assessments showing that shorter time intervals did not affect the
number of heating events detected or heating time recorded. Sensors were
placed on the surfaces of biomass and \DIFdelbegin \DIFdel{coal-fuelled }\DIFdelend \DIFaddbegin \DIFadd{coal-fueled }\DIFaddend stoves and radiators.
For heat pumps, sensors were placed on the heat exchanger coil on
air-to-air units and on the radiator of air-to-water units.

The number and duration of stove combustion events were identified from
the temperature data using criteria defined based on the observed
changes in the peak shape of the time series temperature curves (i.e.,
changes in the slope or in absolute temperature compared with the indoor
ambient temperature). This approach was specific to heating stoves but
developed based on stove use identification for cookstoves in previous
studies by us and others (Clark et al. 2017; Ruiz-Mercado et al. 2013;
Snider et al. 2018). We developed separate criteria for each stove type
given the observed stove-specific differences in heating patterns. These
criteria were coded into stove-specific algorithms to systematically
identify the number and duration of heating events across households. A
stratified random sample of stove use temperature files (15\% for each
stove type and measurement duration - short-term/\DIFdelbegin \DIFdel{24 h }\DIFdelend \DIFaddbegin \DIFadd{24-h }\DIFaddend or
long-term/\textasciitilde6 \DIFdelbegin \DIFdel{mo }\DIFdelend \DIFaddbegin \DIFadd{month }\DIFaddend - combination) were manually coded to
develop the test criteria. The number and duration of heating events
were identified by the algorithms in the remaining 85\% of files. We
compared heating periods identified manually with those identified by
the algorithm to check for systematic differences and possible
overfitting.

\subsubsection{Questionnaires}\label{questionnaires}

Field staff administered household and individual-level questionnaires
to assess household demographic information and educational attainment,
household assets, house structure, stove and fuel use patterns
(including a complete roster of heating methods and their contributions
in each room), and individual health behaviors including exercise
frequency, smoking, alcohol consumption, medication use, and
clinician-diagnosed health conditions. We used Surveybe
computer-assisted personal interview (CAPI) software to collect survey
data via handheld electronic tablets. Questions were read to
participants in Mandarin-Chinese, and their responses were recorded into
tablets.

Prior to the start of data collection, all questions were translated
from English into Chinese and then back-translated to English for
quality assurance. Many questions were adapted from previous field
studies of household energy and blood pressure conducted in rural
Beijing or other rural sites in China (Baumgartner et al. 2018; Yan et
al. 2020), and all questions were iteratively tested with \DIFaddbegin \DIFadd{study }\DIFaddend staff
and adapted \DIFdelbegin \DIFdel{prior to }\DIFdelend \DIFaddbegin \DIFadd{before }\DIFaddend implementation. Prior to each wave in this study, the
questionnaire and \DIFaddbegin \DIFadd{all }\DIFaddend other study measurements were tested in 12
households located in a Beijing village that was eligible for our study
but was instead selected for testing. We used the test village to \DIFaddbegin \DIFadd{train
study staff, }\DIFaddend assess whether the questions were understandable and \DIFaddbegin \DIFadd{being
}\DIFaddend interpreted as intended\DIFaddbegin \DIFadd{, }\DIFaddend and to identify any problems with the study
measurements or their implementation. Study protocols were subsequently
adapted prior to the start of data collection.

In addition to household and individual participant questionnaires, we
\DIFaddbegin \DIFadd{also }\DIFaddend conducted village surveys with one representative from each village
committee to understand how the policy was implemented in that village
and to inquire about any other rural development or health programs
being implemented in the village. Committee members answered questions
about committee and villager interest in the policy and, for treated
villages, assignment versus application to the policy, any home or
village renovations required by the upper-level government prior to heat
pump installation, decision-making for the type and brand of heating
technology, level of subsidies provided for heaters and electricity, and
technical and \DIFdelbegin \DIFdel{logistic }\DIFdelend \DIFaddbegin \DIFadd{logistical }\DIFaddend guidance to villagers.

\subsubsection{Blood pressure}\label{blood-pressure}

Following 5 min of quiet rest, at least three brachial and central
systolic (bSBP/cSBP) and diastolic (bDBP/cDBP) blood pressures (BPs)
were taken by trained staff at 1 min apart on the participant's
supported right arm. We used an automated oscillometric device (BP+;
Uscom Ltd, New Zealand) that estimates central pressures from the
brachial cuff pressure fluctuations. Central pressures were \DIFaddbegin \DIFadd{previously
}\DIFaddend validated against invasive cBP measurements in \DIFdelbegin \DIFdel{previous }\DIFdelend \DIFaddbegin \DIFadd{earlier }\DIFaddend studies (Costello
et al. 2015; Lowe et al. 2009). The BP devices were factory calibrated
by the manufacturer prior to the start of the first and fourth waves. Up
to five measurements were taken if the difference between the last two
was \textgreater5 mmHg or staff were unable to obtain a reading. The BP
measurements were conducted in the participant's home and staff were
trained to follow strict quality control procedures, including use of an
appropriately sized cuff, correct positioning of the arm, both feet on
the ground, and ensuring 5 min of quiet rest before measurement. Details
are described in the standard operating procedures
\href{https://osf.io/gmka5}{(SOP)}. The average of the final two
measurements was used for statistical analysis unless only one BP
measurement was obtained (n = 13 observations), in which case a single
measurement was used. The time of day, day of the week, and indoor
temperature prior to BP measurement were also recorded.

\subsubsection{Self-reported respiratory symptoms and airway
inflammation}\label{self-reported-respiratory-symptoms-and-airway-inflammation}

During questionnaire assessment, participants were asked about chronic
airway symptoms including cough, phlegm, wheeze, and tightness in the
chest using questions validated for use in Mandarin-Chinese and
developed from the standard St.~George's Respiratory Questionnaire
\DIFdelbegin \DIFdel{. }\DIFdelend \DIFaddbegin \comment{OC 1e}\DIFadd{(Xu et al. 2009). }\DIFaddend The Mandarin-Chinese questions were
extensively piloted with rural and peri-urban Beijing residents to
ensure that the health terminology and symptom time patterns were
adequate and understandable to the local population.

In a \textasciitilde25\% random subsample of participants, we also
measured the fractional concentration of exhaled nitric oxide (FeNO), a
non-invasive and established marker of airway inflammation, using a
portable handheld device (Aerocrine, Solna, Sweden) fit with a NIOX
VERO® sensor, following ATS recommendations and guidelines (ATS/ERS
2005). Briefly, FeNO measurement was performed with participants in a
standing position. They inhaled NO-free air through a mouthpiece with an
NO-scrubber attached, followed by controlled expiration for 10 s through
the mouthpiece at 50±5 mL/s. A nose clip was used to avoid nasal
inhalation, and accurate flow rate was achieved using visual and
auditory cues generated by the device. Detailed methods are provided in
our previous study of air pollution and FeNO in Beijing adults (Shang et
al. 2020). At least two measurements were obtained for each participant.

\subsubsection{Blood inflammatory and oxidative stress
markers}\label{blood-inflammatory-and-oxidative-stress-markers}

Trained nurses collected 20 ml of whole blood in a labeled vacutainer
via venipuncture using standard techniques (Tuck et al. 2009). Details
are described in our published \href{https://osf.io/zwpfg}{SOP}.
Briefly, fasting blood samples were collected by experienced
phlebotomists (nurses) in the morning and stored at 4-10°C prior to
centrifugation. Two serum aliquots from each participant were then
placed in a -30°C freezer for temporary storage. Collection-to-storage
time was \textless4 hrs for all samples in both waves where blood
samples were collected. Within 3-5 days of collection, the samples were
transported in styrofoam containers with dry ice to a -80°C freezer with
a backup generator and alarm system at Peking University.

The first aliquot was analyzed for glucose and a complete lipid profile
within two months of collection, and results were communicated to
participants. The second aliquot was stored in the -80°C freezer for
analysis of biomarkers of systemic inflammation (C-reactive protein
{[}CRP{]}, interleukin-6 {[}IL-6{]}, tumour necrosis factor alpha
{[}TNF-\(\alpha\){]} and malondialdehyde {[}MDA{]}) at the University of
the Chinese Academy of Sciences between July and September of 2023.
These biomarkers were selected because they are associated with the
development of cardiovascular disease and events (e.g., Danesh et al.
2008; Emerging Risk Factors Collaboration 2012; Pearson et al. 2003;
Ridker 2001; Ridker et al. 2000), and both acute and longer-term
exposures to air pollution have been associated with changes in
inflammatory and oxidative stress markers (e.g., Huang et al. 2012;
Kipen et al. 2010; Pope III et al. 2004; Rich et al. 2012; Rückerl et
al. 2007).

We followed standard methods for analysis (Food and Drug Administration
2018). For inflammatory markers (IL-6, TNF-\(\alpha\), CRP), the optic
densities (OD) of all samples were measured using an automated ELISA
reader. Every plate had 8 standard samples used to generate a standard
curve that related OD and standard inflammatory marker concentration. A
standard curve for each microplate was generated by a computer software
program based on a 4-parameter method. Each plate included at least 3
control samples to ensure the stability of standard curves. All samples
\DIFaddbegin \comment{OC 1f}\DIFaddend , standards, and controls were measured in duplicate, and
the average was used for statistical analysis. For oxidative stress
biomarkers (MDA), the chromatographic peak areas of all samples were
measured using HPLC with UV detector and HPLC-MS/MS. Every plate had 7
standard samples used to generate a standard curve that related peak
area and concentration of \DIFdelbegin \DIFdel{each standard }\DIFdelend \DIFaddbegin \DIFadd{the }\DIFaddend oxidative stress marker. A standard curve
for each plate was generated using a computer software program based on
a linear method. Each plate included at least 3 control samples to
ensure the stability of standard curves. Standards and controls \DIFaddbegin \DIFadd{for MDA
}\DIFaddend were measured in duplicate and samples were measured once due to high
precision in \DIFdelbegin \DIFdel{a
pilot study }\DIFdelend \DIFaddbegin \DIFadd{our }\comment{OC 1f}\DIFadd{pre-analysis pilot study with duplicate
testing and evidence from many previous studies showing high stability
in measurement }\DIFaddend (Food and Drug Administration 2018). \DIFaddbegin \DIFadd{Boxplots showing
distributions of the inflammatory and oxidative stress markers are
provided in Appendix Figure~\ref{fig-afig-biomarkers}.
}\DIFaddend 

\subsubsection{Anthropometric
measurements.}\label{anthropometric-measurements.}

Body weight, height, and waist circumference were measured at the clinic
visit in \DIFdelbegin \DIFdel{the first two waves and }\DIFdelend \DIFaddbegin \DIFadd{W1 and W2 and }\DIFaddend in participant homes in \DIFdelbegin \DIFdel{the last wave}\DIFdelend \DIFaddbegin \DIFadd{W4 to avoid unnecessary
contact during the COVID-19 pandemic}\DIFaddend . Weight was measured in light
indoor clothing without shoes in kilograms to one decimal place, using
standing scales supported on a steady surface. The scales were
calibrated prior to the start of each wave, and the same staff member
\DIFdelbegin \DIFdel{stepped }\DIFdelend \DIFaddbegin \DIFadd{weighed themself }\DIFaddend on the scale each morning to ensure that it was
functioning properly. Height was measured without shoes in centimeters
to one decimal place with a stadiometer. Waist circumference was
measured without clothing obstruction at one centimeter above the
participant's navel at minimal respiration in centimeters to one decimal
place. The measuring \DIFdelbegin \DIFdel{tape was }\DIFdelend \DIFaddbegin \DIFadd{tapes were }\DIFaddend replaced at the start of each wave to
avoid stretching.

\subsection{Measuring policy impacts}\label{measuring-policy-impacts}

To understand how Beijing's policy works we used a
difference-in-differences (DiD) design (Callaway 2020), leveraging the
staggered \DIFdelbegin \DIFdel{rollout }\DIFdelend \DIFaddbegin \DIFadd{roll-out }\DIFaddend of the policy across multiple villages to estimate
its impact on health outcomes and understand the mechanisms through
which it works. Simple comparisons of treated and untreated (i.e.,
control) villages after the \DIFdelbegin \DIFdel{CBHP policy }\DIFdelend \DIFaddbegin \DIFadd{CHP }\DIFaddend has been implemented are likely to be
biased by unmeasured village-level characteristics (e.g., migration,
average winter temperature, wealth) that are associated with health
outcomes. Similarly, comparisons of only treated villages before and
after exposure to the program are susceptible to bias by other factors
associated with changes in outcomes over time (i.e., secular trends,
\DIFaddbegin \DIFadd{potential health }\DIFaddend impacts of the COVID-19 pandemic). By comparing
\DIFdelbegin \DIFdel{the }\DIFdelend \emph{changes} in outcomes among treated villages to \DIFdelbegin \DIFdel{the }\DIFdelend \emph{changes} in
outcomes among untreated villages, the DiD approach controls for any
unmeasured time-invariant characteristics of villages as well as for any
general secular trends affecting outcomes in all villages that are
unrelated to the policy.

\begin{figure}[H]

\DIFdelbeginFL %DIFDELCMD < \centering{
%DIFDELCMD < 

%DIFDELCMD < \includegraphics[width=0.75\textwidth,height=\textheight]{images/ddfig.png}
%DIFDELCMD < 

%DIFDELCMD < }
%DIFDELCMD < %%%
\DIFdelendFL \DIFaddbeginFL \centering{

\includegraphics[width=0.75\textwidth,height=\textheight]{images/fig-didfig-1.pdf}

}
\DIFaddendFL 

\caption{\label{fig-didfig}Stylized example of
difference-in-differences}

\end{figure}%

The DiD design compares outcomes before and after an intervention in a
treated group relative to the same outcomes measured in a control group.
The control group trend provides the crucial ``counterfactual'' estimate
of what \DIFdelbegin \DIFdel{would have happened }\DIFdelend \DIFaddbegin \emph{\DIFadd{would have happened}} \DIFaddend in the treated group had it not been
treated. By comparing each group to itself, this approach helps to
control for both measured and unmeasured fixed differences between the
treated and control groups. By measuring changes over time in outcomes
in the control group unaffected by the treatment, this approach also
controls for any unmeasured factors affecting outcome trends in both
treated and control groups. This is important since there are often many
potential factors affecting outcome trends that cannot be disentangled
from the policy if one only studies the treated group (as in a
traditional pre-post design).

The canonical DiD design (Card and Krueger 1994) compares two groups
(treated and control) at two different time periods (pre- and
post-intervention, Figure~\ref{fig-didfig}). In the first time period
both groups are untreated, and in the second time period one group is
exposed to the intervention. If we assume that the differences between
the groups would have remained constant in the absence of the
intervention (the parallel trends assumption), then an unbiased estimate
of the impact of the intervention in the post-treatment period can be
calculated by subtracting the pre-post difference in the untreated group
from the pre-post difference in the treated group. The estimand of
interest in a typical DiD analysis is the average treatment effect on
the treated (i.e, the \(ATT\)), which is a contrast of the
post-intervention outcomes in the treated group with the counterfactual
estimate of outcomes in the same population in the absence of treatment.

When multiple groups are treated at different time periods, \DIFdelbegin \DIFdel{the }\DIFdelend \DIFaddbegin \DIFadd{as with the
CHP, the }\DIFaddend most common approach has been to use a two-way fixed effects
model to estimate the impact of the intervention which controls for
secular trends and differences between villages. However, recent
evidence suggests that traditional two-way fixed effects estimation of
the treatment effect may be biased in the context of heterogeneous
treatment effects, i.e., where the effects of treatment vary for
different groups treated at different time periods (Callaway and
Sant'Anna 2021; Goodman-Bacon 2021). The bias is due to the fact that
\DIFaddbegin \DIFadd{when there are multiple groups treated at different times }\DIFaddend the two-way
fixed effects estimate is a weighted average of several `2 x 2' DiD
estimates, some of which involve using already treated units as controls
for later treated units, which can lead to bias (Baker et al. 2022). We
take advantage of new developments in the econometrics literature
(Callaway and Sant'Anna 2021; Sun and Abraham 2021; Wooldridge 2021)
that relax the assumption of homogeneity in the context of staggered
policy \DIFdelbegin \DIFdel{rollouts }\DIFdelend \DIFaddbegin \DIFadd{roll-outs }\DIFaddend but also allow straightforward interpretation of
\(ATT\)s for assessing policy impacts. This decision was motivated by
the many behavioral, social, or economic factors that might affect both
new heat pump use and coal stove suspension (e.g., energy prices and
availability, wintertime temperature, COVID-19 pandemic, user
preferences) over time in our study, and thus the possibility that the
effect of the policy on air pollution and health may be dynamic over
time and/or heterogeneous across treatment cohorts.

\subsection{Measuring pathways and
mechanisms}\label{measuring-pathways-and-mechanisms}

To estimate how much of the \DIFdelbegin \DIFdel{CBHP intervention }\DIFdelend \DIFaddbegin \DIFadd{CHP }\DIFaddend may work through different mechanisms,
we used causal mediation analysis. Causal approaches to mediation
attempt to discern between, and clarify the necessary assumptions for
identifying, different kinds of mediated effects. \DIFdelbegin \DIFdel{Taking
as an example the directed acyclic graph (DAG) in Figure~\ref{fig-dag1}}\DIFdelend \DIFaddbegin \DIFadd{Figure~\ref{fig-dag1}
shows directed acyclic graphs (DAGs) to illustrate the total effect (a)
and the potential direct and indirect effects of the CHP (b)}\DIFaddend , with \(T\)
as the policy, \(X\) as a set of pre-treatment covariates, \DIFaddbegin \DIFadd{and \(Y\) as
systolic blood pressure as an example outcome. The `total effect' of the
policy }\comment{OC 2c}\DIFadd{is an estimate of how much the overall outcomes
(\(Y\)) would change for a change in exposure (versus no exposure) to
the CHP (\(T\)). Part (b) of Figure~\ref{fig-dag1} adds }\DIFaddend \(M_{1}\) as
PM\textsubscript{2.5} \DIFdelbegin \DIFdel{, }\DIFdelend \DIFaddbegin \DIFadd{and }\DIFaddend \(M_{2}\) as indoor temperature \DIFdelbegin \DIFdel{, and
\(Y\) as systolic blood pressure, }\DIFdelend \DIFaddbegin \DIFadd{as potential
mediators that are affected by the policy, and }\DIFaddend we can define the
controlled direct effect (\(CDE\)) as the effect of the \DIFdelbegin \DIFdel{CBHP policy }\DIFdelend \DIFaddbegin \DIFadd{CHP }\DIFaddend on systolic
blood pressure if we fix the values of PM\textsubscript{2.5} and indoor
temperature to a fixed reference level for the entire population. For
example, we can estimate the impact of the policy on health outcomes
while holding PM\textsubscript{2.5} and indoor temperature at uniform
levels of average background exposure, or some other hypothetical level
\DIFaddbegin \DIFadd{(VanderWeele 2015)}\DIFaddend .

\begin{figure}[H]

\DIFaddbeginFL \caption{\label{fig-dag1}\DIFaddFL{Hypothetical Directed Acyclic Graphs (DAGs)
showing (a) total effect and (b) direct and indirect effects with
outcome (\(Y\)), pre-treatment covariates (\(X\)), policy (\(T\)),
multiple mediators (\(M_{1},M_{2}\)), as well as covariates for the
mediators (\(W\)).}}

\begin{minipage}{\linewidth}

\centering{

\includegraphics[width=0.6\textwidth,height=\textheight]{images/dag0.png}

}

\subcaption{\label{fig-dag1-1}\DIFaddFL{DAG for Total effect}}

\end{minipage}%DIF > 
\newline
\begin{minipage}{\linewidth}

\DIFaddendFL \centering{

\includegraphics[width=0.6\textwidth,height=\textheight]{images/dag1.png}

}

\DIFdelbeginFL %DIFDELCMD < \caption{%
{%DIFAUXCMD
%DIFDELCMD < \label{fig-dag1}%%%
\DIFdelFL{Hypothetical Directed Acyclic Graph showing
direct and indirect effects with outcome (\(Y\)), pre-treatment
covariates (\(X\)), policy (\(T\)), multiple mediators
(\(M_{1},M_{2}\)), as well as covariates for the mediators (\(W\)).}}
%DIFAUXCMD
\DIFdelendFL \DIFaddbeginFL \subcaption{\label{fig-dag1-2}\DIFaddFL{DAG for direct and indirect effects}}
\DIFaddendFL 

\DIFaddbeginFL \end{minipage}%DIF > 

\DIFaddendFL \end{figure}%

Although other mediated effects such as ``natural'' direct and indirect
effects are theoretically estimable (VanderWeele 2015), they involve
challenging ``cross-world'' assumptions that are difficult to anchor in
policy (Naimi et al. 2014). Other approaches to mechanisms have focused
on principal stratification (e.g., Zigler et al. 2016), although
conceptual difficulties with identifying the (unverifiable) principal
strata make it challenging for questions of mediation. Because
controlled direct effects are considered more directly policy relevant
for public health, we focused on estimating these mediated quantities.

\section{Data Analysis}\label{data-analysis}

To understand how the policy's impact on health may be mediated by
different potential mediators, we need to first estimate the total
effect of the policy on the outcomes \DIFaddbegin \DIFadd{(shown in Figure~\ref{fig-dag1}
part (a)}\DIFaddend , then estimate the \(CDE\)s after adjustment for potential
mediators and any residual mediator-outcome confounding. As discussed
above, in order for the mediators to `explain' the total effects of the
policy on health, the policy should affect the mediators, and the
mediators should also affect the outcomes.

\subsection{Total Effect}\label{total-effect}

To estimate the total effect of the policy we used a DiD analysis that
accommodates staggered treatment \DIFdelbegin \DIFdel{rollout}\DIFdelend \DIFaddbegin \DIFadd{roll-out}\DIFaddend . To allow for heterogeneity in
the context of staggered \DIFdelbegin \DIFdel{rollout }\DIFdelend \DIFaddbegin \DIFadd{roll-out }\DIFaddend we used `extended' two-way fixed
effects (ETWFE) models (Wooldridge 2021) to estimate the total effect of
the \DIFdelbegin \DIFdel{CBHP policy}\DIFdelend \DIFaddbegin \DIFadd{CHP}\DIFaddend . The mean outcome (replaced by a suitable link function
\(g(\cdot)\) for binary or count outcomes) was defined using a set of
linear predictors:

\begin{equation}\phantomsection\DIFdelbegin %DIFDELCMD < \label{eq-etwfe}{Y_{ijt}=g(\mu_{ijt}) = \alpha + \sum_{r=q}^{T} \beta_{r} d_{r} + \sum_{s=r}^{T} \gamma_{s} fs_{t}+ \sum_{r=q}^{T} \sum_{s=r}^{T} \tau_{rt} (d_{r} \times fs_{t}) + \varepsilon_{ijt}}%%%
\DIFdelend \DIFaddbegin \label{eq-etwfe}{Y_{ijt}=g(\mu_{ijt}) = \alpha + \sum_{r=q}^{T} \beta_{r} d_{r} + \sum_{s=r}^{T} \gamma_{s} fs_{t}+ \sum_{r=q}^{T} \sum_{s=r}^{T} \tau_{rs} (d_{r} \times fs_{t}) + \varepsilon_{ijt}}\DIFaddend \end{equation}

where \(Y_{ijt}\) is the outcome for individual \(i\) in village \(j\)
at time \(t\), \(d_{r}\) represent treatment cohort dummies, i.e., fixed
effects for \DIFdelbegin \DIFdel{cohorts of villages }\DIFdelend \DIFaddbegin \DIFadd{each cohort of villages \(r\) }\DIFaddend that were first exposed to the
policy at the same time \(q\) (e.g., in 2019, 2020, or 2021), \(fs_{t}\)
are \DIFdelbegin \DIFdel{time
fixed effects }\DIFdelend \DIFaddbegin \DIFadd{fixed effects for each time period \(s\) }\DIFaddend corresponding to different
winter data collection waves (2018-19, 2019-20, or 2021-22), and
\DIFdelbegin \DIFdel{\(\tau_{rt}\) are the }\DIFdelend \DIFaddbegin \DIFadd{\(\tau_{rs}\) are the treatment }\DIFaddend cohort-time \(ATT\)s in the context of a
linear model. For \DIFdelbegin \DIFdel{binary or count }\DIFdelend \DIFaddbegin \DIFadd{non-linear }\DIFaddend outcomes the cohort-time \(ATT\)s are
derived by estimating \DIFaddbegin \DIFadd{average }\DIFaddend marginal effects from non-linear models
(Arel-Bundock 2024). For \DIFaddbegin \DIFadd{binary and count outcomes we used logit and
Poisson models, respectively, and for skewed outcomes (e.g., black
carbon, PM\textsubscript{2.5}, inflammatory markers) we used generalized
linear models with a Gamma distribution and a log link, based on the
specification tests recommended by Manning and Mullahy (2001). For }\DIFaddend all
models we \DIFdelbegin \DIFdel{cluster
}\DIFdelend \DIFaddbegin \DIFadd{clustered }\DIFaddend standard errors at the village level, consistent
with the unit of treatment assignment (Cameron and Miller 2015). The
ETWFE and other approaches that allow for several (potentially
heterogeneous) treatment effects may also be averaged to provide a
weighted summary \(ATT\). Several potential possibilities are feasible,
including weighting by treatment cohorts or time since policy adoption
(Goin and Riddell 2023). We generally focus on two types of \(ATT\)s for
this report: simple averages across all treatment cohorts and the full
set of cohort-time \(ATT\)s to evaluate heterogeneous treatment effects.
Although we primarily focus on reporting the simple average \(ATT\) for
most outcomes, we also used omnibus joint \emph{F}-tests to assess
whether there was sufficient evidence to reject the assumption of
homogeneity across the \(ATT\)s.

\subsection{Mediation Analysis}\label{mediation-analysis}

As noted above, with respect to the mediation analysis we are chiefly
interested in the \(CDE\), which can be derived by adding relevant
mediators \(M\) to Equation~\ref{eq-etwfe}. If we also allow for
exposure-mediator interaction and potentially allow for adjustment for
confounders \(W\) of the mediator-outcome effect, we can extend equation
Equation~\ref{eq-etwfe} as follows:

\begin{equation}\phantomsection\DIFdelbegin %DIFDELCMD < \label{eq-etwfem}{
%DIFDELCMD < \begin{aligned}
%DIFDELCMD < Y_{ijt}=g(\mu_{ijt}) = \alpha + \sum_{r=q}^{T} \beta_{r} d_{r} + \sum_{s=r}^{T} \gamma_{s} fs_{t}+ \sum_{r=q}^{T} \sum_{s=r}^{T} \tau_{rt} (d_{r} \times fs_{t}) \\ + \delta M_{it} + \sum_{r=q}^{T} \sum_{s=r}^{T} \eta_{rt} (d_{r} \times fs_{t} \times M_{it}) + \zeta \mathbf{W} + \varepsilon_{ijt}
%DIFDELCMD < \end{aligned}
%DIFDELCMD < }%%%
\DIFdelend \DIFaddbegin \label{eq-etwfem}{
\begin{aligned}
Y_{ijt}=g(\mu_{ijt}) = \alpha + \sum_{r=q}^{T} \beta_{r} d_{r} + \sum_{s=r}^{T} \gamma_{s} fs_{t}+ \sum_{r=q}^{T} \sum_{s=r}^{T} \tau_{rs} (d_{r} \times fs_{t}) \\ + \delta M_{it} + \sum_{r=q}^{T} \sum_{s=r}^{T} \eta_{rs} (d_{r} \times fs_{t} \times M_{it}) + \zeta \mathbf{W} + \varepsilon_{ijt}
\end{aligned}
}\DIFaddend \end{equation}

where now \(\delta\) is the conditional effect of the mediator \(M\) at
the reference level of the treatment (again, represented via the series
of group-time interaction terms), and the collection of \DIFdelbegin \DIFdel{\(\eta\) }\DIFdelend \DIFaddbegin \DIFadd{\(\eta_{rs}\)
}\DIFaddend terms are coefficients for the product terms allowing for
mediator-treatment interaction. Finally, \(\zeta\) is a vector of
coefficients for the set of confounders contained within \(\mathbf{W}\).
\DIFdelbegin %DIFDELCMD < 

%DIFDELCMD < %%%
\DIFdelend As noted above, in the staggered DiD framework that allows for
heterogeneity we do not have a single treatment effect but a collection
of group-time treatment effects that may be averaged in different ways.
This extends to the estimation of the \(CDE\), in which case we will
also have several \(CDE\)s that can be averaged to make inferences about
the extent to which the policy's impact is mediated by
PM\textsubscript{2.5} \DIFaddbegin \DIFadd{or temperature}\DIFaddend . Based on the setup in
Equation~\ref{eq-etwfem} the \(CDE\) is estimated as:
\DIFdelbegin \DIFdel{\(\delta + \eta_{rt}MT\)}\DIFdelend \DIFaddbegin \DIFadd{\(\tau_{rs} + \eta_{rs}M\)}\DIFaddend . In the absence of interaction between the
exposure and the mediator (i.e., \DIFdelbegin \DIFdel{\(\eta_{rt}=0\)}\DIFdelend \DIFaddbegin \DIFadd{\(\eta_{rs}=0\)}\DIFaddend ) the \(CDE\) will
simply be the estimated treatment effects
\DIFdelbegin \DIFdel{\(\sum_{r=q}^{T} \sum_{s=r}^{T} \tau_{rt}\)}\DIFdelend \DIFaddbegin \DIFadd{\(\sum_{r=q}^{T} \sum_{s=r}^{T} \tau_{rs}\)}\DIFaddend , i.e., the effect of the
policy holding \(M\) constant. For a valid estimate of the \(CDE\) we
must account for confounding of the mediator-outcome effect, represented
by \(W\) in the equation above. The inclusion of baseline measures of
both the outcome and the proposed mediators inherent in our DiD strategy
help to reduce the potential for unmeasured confounding of the
mediator-outcome effect (Keele et al. 2015). Given the large number of
outcomes of interest in this study, as well as the potential for
heterogeneous treatment effects, we limited the mediation analysis to
health outcomes for which we observed \DIFaddbegin \DIFadd{some evidence of }\DIFaddend a total effect of
the \DIFdelbegin \DIFdel{CBHP policy}\DIFdelend \DIFaddbegin \DIFadd{CHP}\DIFaddend .

\subsection{Identification of potential confounders and model
covariates}\label{identification-of-potential-confounders-and-model-covariates}

In contrast to typical analytic approaches such as regression adjustment
or propensity scores that solely focus on measured covariates, our DiD
approach helps to minimize the risk of some sources of \emph{unmeasured}
confounding. Treatment cohort fixed effects control for measured and
unmeasured time-constant factors that may differ between treatment
cohorts (e.g., genetics, altitude), and time fixed effects control for
secular trends, capturing any unmeasured factors that affect outcomes in
all treatment cohorts (including the untreated) similarly over the study
period (e.g., background improvements in ambient air quality or
household transition to more efficient heating \DIFaddbegin \DIFadd{unrelated to the CHP}\DIFaddend ).
The latter are particularly helpful in the context of the documented
declines in \DIFaddbegin \DIFadd{ambient }\DIFaddend PM\textsubscript{2.5} in \DIFdelbegin \DIFdel{China attributable to sources }\DIFdelend \DIFaddbegin \DIFadd{the Beijing region
attributable to air quality improvement programs and policies }\DIFaddend other than
the \DIFdelbegin \DIFdel{CBHP }\DIFdelend \DIFaddbegin \DIFadd{CHP }\DIFaddend policy (Van Donkelaar et al. 2021; Zhang et al. 2019)\DIFaddbegin \DIFadd{.
}\DIFaddend 

For models estimating the effect of the policy on \DIFaddbegin \DIFadd{indoor temperature and
}\DIFaddend health outcomes, we used DAGs (Pearl 2000) to identify potential
time-varying causes of both treatment by the policy and our study
outcome(s) that could differ between treatment groups, and adjusted for
those potential confounders in the regression models. For the mediation
analysis, we identified potential mediator-outcome confounders using the
same approach. These variables were identified from the relevant
peer-reviewed literature and our team's substantive knowledge about the
\DIFdelbegin \DIFdel{CBHP policy. In the
multivariable models , we also adjusted for strong predictors of the outcome that were not affected by treatment, and thus not confounders,
to improve model precision. The covariates included in each of the
models are provided in the tables.
}%DIFDELCMD < 

%DIFDELCMD < %%%
\DIFdel{For air pollution outcomes}\DIFdelend \DIFaddbegin \DIFadd{CHP. For models estimating the effect of the policy on air pollution
outcomes, the main predictors of personal exposures and indoor air
quality in rural China are inconsistent across studies (e.g., Lee et al.
(2021); Ni et al. (2016)). Thus}\DIFaddend , we considered the following covariates
\DIFaddbegin \DIFadd{as potential determinants of air pollution in our study setting}\DIFaddend : village
population and total number of households in the village; temperature,
relative humidity, \DIFaddbegin \DIFadd{dew point, }\DIFaddend wind direction, wind speed, boundary layer
height; home area and home area heated; home insulation; smoking status
of \DIFdelbegin \DIFdel{participant and whether or not they lived with a smoker; whether }\DIFdelend \DIFaddbegin \DIFadd{the participant; whether }\comment{EC 12c, OC 12f}\DIFaddend or not the household
reported \DIFdelbegin \DIFdel{using }\DIFdelend \DIFaddbegin \DIFadd{residential }\DIFaddend wood (i.e., biomass) \DIFdelbegin \DIFdel{for
household energy activities}\DIFdelend \DIFaddbegin \DIFadd{burning}\DIFaddend , and if so,
self-reported quantity \DIFdelbegin \DIFdel{of wood.
Potential non-linearity between continuous covariates and our study
outcomeswere evaluated using natural cubic splines with different
degrees of freedom.
Ultimately, the following covariates were included
}\DIFdelend \DIFaddbegin \DIFadd{used.
}

\DIFadd{Exposure to tobacco smoke is important for both air pollution and health
outcomes, and we used the participant responses to survey questions
related to their current smoking status (i.e., is the participant a
current smoker, former smoker, or never smoker) and, for never smokers,
history of passive smoking (i.e., has the participant ever lived with a
smoker in the same house for at least 6 months, with possible responses
of never, yes but not currently, and yes at present). The survey
responses were used to create the following four distinct tobacco
smoking categories: }\emph{\DIFadd{current smoker}}\DIFadd{, defined as currently smoking
at the time of survey; }\emph{\DIFadd{former smoker}}\DIFadd{, defined as previously
smoking but no longer smoking at the time of the survey (not accounting
for duration of cessation); }\emph{\DIFadd{never smoker with a history of living
with a smoker}}\DIFadd{, defined as a never smoker who is currently living with a
smoker or has previously lived with a smoker for at least 6 months
(i.e., `passive' smoking exposure); and }\emph{\DIFadd{never smoker}}\DIFadd{, defined as
no history of smoking and no history of living with a smoker for more
than 6 months.
}

\DIFadd{Ultimately, we included the following measured time-varying covariates
}\DIFaddend in the final DiD models for \DIFdelbegin \DIFdel{outdoor, indoor,
and personal exposures to
air pollution, based on whether measurable changes in the covariate over
time were observed}\DIFdelend \DIFaddbegin \DIFadd{each outcome-specific model. Models for air
pollution outcomes were adjusted for household size, tobacco smoking,
outdoor temperature, and outdoor dew point. As a sensitivity analysis,
we additionally adjusted for district of residence given the baseline
district-level differences in energy use, socioeconomic status and
altitude, especially for villages in Fangshan compared with the other
three districts. Temperature models were adjusted for the number of
rooms, wintertime occupants in the household, age of the primary
respondent, and wealth index. Models for blood pressure were adjusted
for age, sex, waist circumference, tobacco smoking, alcohol consumption,
and use of blood pressure medication. For self-reported respiratory
outcomes we adjusted for age, gender, tobacco smoking, occupation,
frequency of drinking, frequency of farming. Measured respiratory
outcome (FeNO) models included adjustment for age, gender, body mass
index, frequency of drinking, tobacco smoking, and frequency of
exercise, occupation, time of measurement. Inflammatory marker outcome
models were adjusted for age, waist circumference, occupation, wealth
index quantile, frequency of drinking, tobacco smoking, and frequency of
farming}\DIFaddend . For the final \DIFdelbegin \DIFdel{adjusted }\DIFdelend \DIFaddbegin \DIFadd{covariate-adjusted }\DIFaddend DiD model for personal
exposure \DIFdelbegin \DIFdel{source contributions due to mixed combustion of solid fuels
(hereafter }\DIFdelend `mixed combustion' \DIFdelbegin \DIFdel{)}\DIFdelend \DIFaddbegin \DIFadd{source contributions}\DIFaddend , we adjusted for
\DIFdelbegin \DIFdel{: }\DIFdelend temperature (represented by a spline with 2 degrees of freedom)\DIFdelbegin \DIFdel{; participant smokingstatus; }\DIFdelend \DIFaddbegin \DIFadd{, tobacco
smoking, }\DIFaddend and whether or not the household reported using biomass fuel.
For the final \DIFdelbegin \DIFdel{adjusted }\DIFdelend \DIFaddbegin \DIFadd{covariate-adjusted }\DIFaddend DiD model for outdoor (community)
`mixed combustion' source contributions, the following covariates were
included: total number of households in the village\DIFdelbegin \DIFdel{; village population;
}\DIFdelend \DIFaddbegin \DIFadd{, village population,
}\DIFaddend and ambient relative humidity (represented by a spline with 2 degrees of
freedom).

\subsection{\texorpdfstring{Multiple imputation for covariates and
indoor PM\textsubscript{2.5} in analyses with BP
outcomes}{Multiple imputation for covariates and indoor PM2.5 in analyses with BP outcomes}}\label{multiple-imputation-for-covariates-and-indoor-pm2.5-in-analyses-with-bp-outcomes}

Blood pressure was measured at household visits but several key
covariates like waist circumference, height, and weight were measured at
the clinic visits in \DIFdelbegin \DIFdel{w1 and w2}\DIFdelend \DIFaddbegin \DIFadd{W1 and W2}\DIFaddend . Thus, we were missing covariate
information for individuals who were unable to attend the clinic visits
(\textasciitilde15-20\% of participants in each wave). Additionally,
since we only measured indoor PM\textsubscript{2.5} in a subsample of
300 homes in \DIFdelbegin \DIFdel{w2 and w4}\DIFdelend \DIFaddbegin \DIFadd{W2 and W4}\DIFaddend , we were missing indoor PM\textsubscript{2.5} for
all participants in \DIFdelbegin \DIFdel{w1 }\DIFdelend \DIFaddbegin \DIFadd{W1 }\DIFaddend with BP measures, as well as for a sub-sample of
participants in \DIFdelbegin \DIFdel{w2 and w4}\DIFdelend \DIFaddbegin \DIFadd{W2 and W4}\DIFaddend . To prepare data for the BP outcomes analysis
we used multiple imputation with chained equations (MICE) to impute
missing \DIFdelbegin \DIFdel{covariate data and missing }\DIFdelend indoor PM\textsubscript{2.5} \DIFaddbegin \DIFadd{and missing covariate data }\DIFaddend values
for individuals who participated in the household visit but not the
clinic visit. This allowed us to retain observations with BP
measurements that would have otherwise been dropped in adjusted and
mediation models using complete-case analysis. Imputation was performed
with the \emph{MICE} package (van Buuren and Groothuis-Oudshoorn 2011)
in \emph{R} (\DIFdelbegin \DIFdel{m }\DIFdelend \DIFaddbegin \DIFadd{\(m\) }\DIFaddend = 30 imputation datasets, with 30 iterations each),
and the difference-in-differences and mediation analyses were conducted
for each of the 30 datasets. We then used Rubin's Rules to combine point
estimates and standard errors while accounting for both within- and
between-dataset variances (Rubin 1987).
\DIFaddbegin 

\DIFadd{Appendix }\comment{EC 2a, 2b}\DIFadd{Table~\ref{tbl-a-mi} shows that most
measures had no or \textless1\% missing data, with the exception of
measured waist circumference, height and weight (all around 15\%
missing). Appendix Table~\ref{tbl-a-mi-cohort} also shows the number and
percent of missing observations by treatment enrollment cohort and
outcome, and we found little evidence that the percent of missing
observations differed substantially between treatment cohorts. }\DIFaddend In
Appendix Figure~\ref{fig-afig-mi} we show kernel density plots for the
distribution of imputed values for BMI, waist circumference, and indoor
PM\textsubscript{2.5}, all of which closely approximated the observed
values.

\section{Results}\label{results-1}

We retained all 50 study villages during this four-year longitudinal
assessment of village treatment by the \DIFdelbegin \DIFdel{CBHP policy}\DIFdelend \DIFaddbegin \DIFadd{CHP}\DIFaddend , though we were only able to
visit 41 villages in winter 2020-21 (\DIFdelbegin \DIFdel{w3) and }\DIFdelend \DIFaddbegin \DIFadd{W3) during which we }\DIFaddend were limited to
village and household-level measurements of air quality, indoor
temperature, and stove use \DIFdelbegin \DIFdel{in that wave due to }\DIFdelend \DIFaddbegin \DIFadd{due to travel }\DIFaddend restrictions during the
\DIFdelbegin \DIFdel{COVID }\DIFdelend \DIFaddbegin \DIFadd{COVID-19 }\DIFaddend pandemic.

By \DIFdelbegin \DIFdel{w2, w3, and w4 }\DIFdelend \DIFaddbegin \DIFadd{W2, W3, and W4 }\DIFaddend there were a cumulative total of 10, 17, and 20 (out
of 50 total) study villages treated by the \DIFdelbegin \DIFdel{CBHP }\DIFdelend \DIFaddbegin \DIFadd{CHP }\DIFaddend policy, respectively. All
of the treated villages \DIFaddbegin \DIFadd{in our study }\DIFaddend selected to install
electric-powered air-source heat pumps with 200 RMB per meter square (up
to 24,000 RMB) in subsidies and were also provided with 80\% night-time
electricity subsidies up to 10,\DIFdelbegin \DIFdel{000kWh }\DIFdelend \DIFaddbegin \DIFadd{000 kWh }\DIFaddend per heating season. To limit coal
use, villages enrolled in the policy were no longer allowed to place
orders for subsidized coal with the district-level governments that
manage the procurement and distribution of coal for residential heating
in Beijing. In addition, village committee leaders in treated villages
reported feeling accountable to the Environmental Protection Department
for limited coal-related air pollution, and were motivated to encourage
residents to not burn coal. Some villages were equipped with government
air pollution monitors and the Environmental Protection Department
conducted village inspections and issued warnings about coal burning.
Households burning coal in treated villages were at risk of losing their
electricity subsidy.

Appendix \DIFaddbegin \comment{EC 3a, 3c}\DIFaddend Figure~\ref{fig-flowchart} \DIFdelbegin \DIFdel{shows }\DIFdelend \DIFaddbegin \DIFadd{and
Table~\ref{tbl-samples} show }\DIFaddend the participation of villages, households,
and participants across the four waves of data collection \DIFdelbegin \DIFdel{.
}\DIFdelend \DIFaddbegin \DIFadd{and the number
of sampled participants, households, and villages by study wave.
Appendix Table~\ref{tbl-dist-stats} also shows }\comment{EC 10a}\DIFadd{selected
demographic characteristics by district. }\DIFaddend We conducted measurements in
over 1000 participants in each of the three measurement waves that
included individual-level measurements. In total, we enrolled
\DIFdelbegin \DIFdel{1432 }\DIFdelend \DIFaddbegin \comment{EC 3a}\DIFadd{1438 }\DIFaddend participants into the study, of which 630 (43\%)
\DIFdelbegin \DIFdel{participated in all three waves}\DIFdelend \DIFaddbegin \DIFadd{individuals contributed 1890 observations across all the three waves in
which health measurements were conducted}\DIFaddend , 443 (31\%) \DIFdelbegin \DIFdel{participated in }\DIFdelend \DIFaddbegin \DIFadd{individuals
contributed 886 observations across }\DIFaddend two waves and 365 (25\%) \DIFaddbegin \DIFadd{individuals
}\DIFaddend participated in a single wave. \DIFdelbegin \DIFdel{We did not observe any
notable differences in demographic characteristics or }\DIFdelend \DIFaddbegin \DIFadd{Table~\ref{tbl-each-campaign} shows
selected demographic characteristics and }\DIFaddend health behaviors between
participants who contributed to \DIFdelbegin \DIFdel{a different number of waves
(}\DIFdelend \DIFaddbegin \DIFadd{each study wave.
}\DIFaddend Table~\ref{tbl-each-campaign} \DIFdelbegin \DIFdel{) or }\DIFdelend \DIFaddbegin \DIFadd{shows selected demographic characteristics
and health behaviors between participants who contributed to each study
wave}\comment{EC 2b}\DIFadd{. We found no differences in gender or smoking across
waves, but overall BMI and waist circumference increased over
time.}\comment{EC 4b} \DIFadd{Table~\ref{tbl-diff-campaign} shows similar
measures according to whether or }\DIFaddend between participants in each of the
three waves with individual measurements\DIFdelbegin \DIFdel{(Table~\ref{tbl-diff-campaign})}\DIFdelend \DIFaddbegin \DIFadd{, and we found some evidence
that individuals contributing more than 1 wave of data had slightly
higher BMI and lower waist circumference}\DIFaddend .

\begin{table}

\caption{\label{tbl-each-campaign}\DIFdelbeginFL \DIFdelFL{Demographic }\DIFdelendFL \DIFaddbeginFL \DIFaddFL{Selected demographic }\DIFaddendFL and health
characteristics of participants in each study wave.}

\DIFdelbeginFL %DIFDELCMD < \centering{
%DIFDELCMD < 

%DIFDELCMD < \centering
%DIFDELCMD < \begin{tblr}[         %% tabularray outer open
%DIFDELCMD < ]                     %% tabularray outer close
%DIFDELCMD < {                     %% tabularray inner open
%DIFDELCMD < width={1\linewidth},
%DIFDELCMD < colspec={X[0.4]X[0.2]X[0.2]X[0.2]},
%DIFDELCMD < row{1}={,cmd=\bfseries,},
%DIFDELCMD < column{1}={halign=l,},
%DIFDELCMD < column{2}={halign=l,},
%DIFDELCMD < column{3}={halign=l,},
%DIFDELCMD < column{4}={halign=l,},
%DIFDELCMD < }                     %% tabularray inner close
%DIFDELCMD < \toprule
%DIFDELCMD < Characteristic & Wave 1 (2018-19) N=1003 & Wave 2 (2019-20) N=1110 & Wave 4 (2021-22) N=1028 \\ \midrule %% TinyTableHeader
%DIFDELCMD < Female, n (\%) & 580 (57.8) & 653 (58.8) & 612 (59.5) \\
%DIFDELCMD < Current smoker, n (\%) & 257 (25.6) & 295 (26.6) & 265 (25.8) \\
%DIFDELCMD < Any smoke exposure, n (\%) & 788 (78.6) & 897 (80.8) & 843 (82) \\
%DIFDELCMD < Age in years, Mean (SD) & 60.7 (9.2) & 61.4 (9.1) & 63.1 (9) \\
%DIFDELCMD < BMI in kg/m2, Mean (SD) & 26.1 (3.7) & 25.7 (3.5) & 26.1 (4) \\
%DIFDELCMD < Waist circumference in cm, Mean (SD) & 86.8 (10.2) & 87.4 (9.4) & 91.4 (10.7) \\
%DIFDELCMD < \bottomrule
%DIFDELCMD < \end{tblr}
%DIFDELCMD < 

%DIFDELCMD < }
%DIFDELCMD < %%%
\DIFdelendFL \DIFaddbeginFL \centering{

\centering
\begin{talltblr}[         %% tabularray outer open
entry=none,label=none,
note{a}={Chi-square test for categorical and F-test for continuous characteristics.},
]                     %% tabularray outer close
{                     %% tabularray inner open
width={1\linewidth},
colspec={X[0.380952380952381]X[0.142857142857143]X[0.142857142857143]X[0.142857142857143]X[0.0952380952380952]X[0.0952380952380952]},
cell{1}{1}={c=1,}{halign=c,},
cell{1}{2}={c=3,}{halign=c,},
cell{1}{5}={c=2,}{halign=c,},
column{1}={halign=l,},
column{2}={halign=l,},
column{3}={halign=l,},
column{4}={halign=l,},
}                     %% tabularray inner close
\toprule
& Estimates &  &  & Test for Equality &  \\ \cmidrule[lr]{1-1}\cmidrule[lr]{2-4}\cmidrule[lr]{5-6}
Characteristic & Wave 1 (2018-19) N=1003 & Wave 2 (2019-20) N=1110 & Wave 4 (2021-22) N=1028 & Statistic\textsuperscript{a} & p-value \\ \midrule %% TinyTableHeader
Female, n (\%) & 597 (59.5) & 654 (58.9) & 617 (60.0) & 0.270 & 0.874 \\
Current smoker, n (\%) & 257 (25.6) & 295 (26.6) & 265 (25.8) & 0.292 & 0.864 \\
Passive smoke exposure, n (\%) & 486 (48.4) & 538 (48.5) & 486 (47.3) & 0.234 & 0.890 \\
Any smoke exposure, n (\%) & 795 (79.3) & 898 (80.9) & 857 (83.4) & 5.616 & 0.060 \\
Age in years, Mean (SD) & 60.1 (9.3) & 61.1 (9.1) & 63.3 (9.0) & 31.980 & 0.000 \\
BMI (kg/m2), Mean (SD) & 26.1 (3.7) & 25.7 (3.5) & 26.2 (3.8) & 4.209 & 0.030 \\
Waist circumference (cm), Mean (SD) & 86.8 (10.2) & 87.4 (9.4) & 91.3 (10.4) & 54.171 & 0.000 \\
\bottomrule
\end{talltblr}

}
\DIFaddendFL 

\end{table}%

\begin{table}

\caption{\label{tbl-diff-campaign}\DIFdelbeginFL \DIFdelFL{Demographic }\DIFdelendFL \DIFaddbeginFL \DIFaddFL{Selected demographic }\DIFaddendFL and health
characteristics of participants who contributed to different numbers of
study waves.}

\DIFdelbeginFL %DIFDELCMD < \centering{
%DIFDELCMD < 

%DIFDELCMD < \centering
%DIFDELCMD < \begin{tblr}[         %% tabularray outer open
%DIFDELCMD < ]                     %% tabularray outer close
%DIFDELCMD < {                     %% tabularray inner open
%DIFDELCMD < width={1\linewidth},
%DIFDELCMD < colspec={X[0.4]X[0.2]X[0.2]X[0.2]},
%DIFDELCMD < row{1}={,cmd=\bfseries,},
%DIFDELCMD < column{1}={halign=l,},
%DIFDELCMD < column{2}={halign=l,},
%DIFDELCMD < column{3}={halign=l,},
%DIFDELCMD < column{4}={halign=l,},
%DIFDELCMD < }                     %% tabularray inner close
%DIFDELCMD < \toprule
%DIFDELCMD < Characteristic & 1 Wave N=365 & 2 Waves N=443 & 3 Waves N=630 \\ \midrule %% TinyTableHeader
%DIFDELCMD < Female, n (\%) & 211 (57.8) & 253 (57.1) & 370 (58.7) \\
%DIFDELCMD < Current smoker, n (\%) & 110 (30.1) & 117 (26.4) & 161 (25.6) \\
%DIFDELCMD < Any smoke exposure, n (\%) & 288 (78.9) & 360 (81.3) & 498 (79) \\
%DIFDELCMD < Age in years, Mean (SD) & 59.9 (9.2) & 60.5 (8.8) & 61.3 (9.1) \\
%DIFDELCMD < BMI in kg/m2, Mean (SD) & 26.3 (3.6) & 25.8 (3.5) & 26.1 (3.7) \\
%DIFDELCMD < Waist circumference in cm, Mean (SD) & 90.3 (9.8) & 86.5 (10) & 86.9 (10.4) \\
%DIFDELCMD < \bottomrule
%DIFDELCMD < \end{tblr}
%DIFDELCMD < 

%DIFDELCMD < }
%DIFDELCMD < %%%
\DIFdelendFL \DIFaddbeginFL \centering{

\centering
\begin{talltblr}[         %% tabularray outer open
entry=none,label=none,
note{a}={Chi-square test for categorical and F-test for continuous characteristics.},
]                     %% tabularray outer close
{                     %% tabularray inner open
width={1\linewidth},
colspec={X[0.380952380952381]X[0.142857142857143]X[0.142857142857143]X[0.142857142857143]X[0.0952380952380952]X[0.0952380952380952]},
cell{1}{1}={c=1,}{halign=c,},
cell{1}{2}={c=3,}{halign=c,},
cell{1}{5}={c=2,}{halign=c,},
column{1}={halign=l,},
column{2}={halign=l,},
column{3}={halign=l,},
column{4}={halign=l,},
}                     %% tabularray inner close
\toprule
& Estimates &  &  & Test for Equality &  \\ \cmidrule[lr]{1-1}\cmidrule[lr]{2-4}\cmidrule[lr]{5-6}
Characteristic & 1 Wave N=365 & 2 Waves N=886 & 3 Waves N=1890 & Statistic\textsuperscript{a} & p-value \\ \midrule %% TinyTableHeader
Female, n (\%) & 211 (57.8) & 532 (60.0) & 1125 (59.5) & 0.542 & 0.763 \\
Current smoker, n (\%) & 110 (30.1) & 230 (26.0) & 477 (25.2) & 3.817 & 0.148 \\
Passive smoke exposure, n (\%) & 172 (47.1) & 425 (48.0) & 913 (48.3) & 0.099 & 0.952 \\
Any smoke exposure, n (\%) & 293 (80.2) & 732 (82.6) & 1526 (80.7) & 1.607 & 0.448 \\
Age in years, Mean (SD) & 26.3 (3.6) & 25.8 (3.6) & 26.0 (3.7) & 1.532 & 0.433 \\
BMI (kg/m2), Mean (SD) & 59.8 (9.3) & 61.0 (8.9) & 62.1 (9.3) & 11.718 & 0.000 \\
Waist circumference (cm), Mean (SD) & 90.3 (9.8) & 88.1 (10.2) & 88.7 (10.2) & 4.438 & 0.024 \\
\bottomrule
\end{talltblr}

}
\DIFaddendFL 

\end{table}%

\subsection{Description of study sample \DIFaddbegin \DIFadd{by
treatment}\DIFaddend }\DIFdelbegin %DIFDELCMD < \label{description-of-study-sample}
%DIFDELCMD < %%%
\DIFdelend \DIFaddbegin \label{description-of-study-sample-by-treatment}
\DIFaddend 

\begin{table}

\caption{\label{tbl-desc}Descriptive statistics for selected
demographic, health, and environmental measures at baseline, by
treatment status.}

\DIFdelbeginFL %DIFDELCMD < \centering{
%DIFDELCMD < 

%DIFDELCMD < \centering
%DIFDELCMD < \begin{talltblr}[         %% tabularray outer open
%DIFDELCMD < entry=none,label=none,
%DIFDELCMD < note{}={},
%DIFDELCMD < ]                     %% tabularray outer close
%DIFDELCMD < {                     %% tabularray inner open
%DIFDELCMD < colspec={Q[]Q[]Q[]Q[]Q[]Q[]Q[]},
%DIFDELCMD < cell{1}{2}={c=2,}{halign=c,},
%DIFDELCMD < cell{1}{4}={c=2,}{halign=c,},
%DIFDELCMD < column{1}={halign=r,},
%DIFDELCMD < column{2}={halign=r,},
%DIFDELCMD < column{3}={halign=r,},
%DIFDELCMD < column{4}={halign=r,},
%DIFDELCMD < column{5}={halign=r,},
%DIFDELCMD < column{6}={halign=r,},
%DIFDELCMD < column{7}={halign=r,},
%DIFDELCMD < row{1}={halign=c,},
%DIFDELCMD < column{1}={font=\fontsize{0.75em}{1.05em}\selectfont,},
%DIFDELCMD < column{2}={font=\fontsize{0.75em}{1.05em}\selectfont,},
%DIFDELCMD < column{3}={font=\fontsize{0.75em}{1.05em}\selectfont,},
%DIFDELCMD < column{4}={font=\fontsize{0.75em}{1.05em}\selectfont,},
%DIFDELCMD < column{5}={font=\fontsize{0.75em}{1.05em}\selectfont,},
%DIFDELCMD < column{6}={font=\fontsize{0.75em}{1.05em}\selectfont,},
%DIFDELCMD < column{7}={font=\fontsize{0.75em}{1.05em}\selectfont,},
%DIFDELCMD < row{3}={,cmd=\bfseries,},
%DIFDELCMD < row{9}={,cmd=\bfseries,},
%DIFDELCMD < row{25}={,cmd=\bfseries,},
%DIFDELCMD < }                     %% tabularray inner close
%DIFDELCMD < \toprule
%DIFDELCMD < & Never treated (N=603) &  & Ever treated (N=400) &  &  &  \\ \cmidrule[lr]{2-3}\cmidrule[lr]{4-5}
%DIFDELCMD < & Mean & Std. Dev. & Mean & Std. Dev. & Diff. in Means & Std. Error \\ \midrule %% TinyTableHeader
%DIFDELCMD < Demographics:                 &              &              &              &              &              &             \\
%DIFDELCMD < Age (years)                   & \num{59.9}  & \num{9.4}   & \num{60.4}  & \num{9.2}   & \num{0.5}   & \num{0.6}  \\
%DIFDELCMD < Female (\%)                  & \num{59.5}  & \num{49.1}  & \num{59.1}  & \num{49.2}  & \num{-0.4}  & \num{3.2}  \\
%DIFDELCMD < No education (\%)            & \num{11.5}  & \num{31.9}  & \num{12.3}  & \num{32.9}  & \num{0.9}   & \num{2.1}  \\
%DIFDELCMD < Primary education (\%)       & \num{75.5}  & \num{43.0}  & \num{77.6}  & \num{41.7}  & \num{2.1}   & \num{2.8}  \\
%DIFDELCMD < Secondary+ education (\%)    & \num{12.6}  & \num{33.2}  & \num{9.8}   & \num{29.7}  & \num{-2.9}  & \num{2.0}  \\
%DIFDELCMD < Health measures:              &              &              &              &              &              &             \\
%DIFDELCMD < Never smoker (\%)            & \num{61.9}  & \num{48.6}  & \num{59.5}  & \num{49.1}  & \num{-2.4}  & \num{3.2}  \\
%DIFDELCMD < Former smoker (\%)           & \num{11.9}  & \num{32.4}  & \num{15.1}  & \num{35.8}  & \num{3.2}   & \num{2.2}  \\
%DIFDELCMD < Current smoker (\%)          & \num{26.2}  & \num{44.0}  & \num{25.4}  & \num{43.6}  & \num{-0.8}  & \num{2.8}  \\
%DIFDELCMD < Never drinker (\%)           & \num{55.9}  & \num{49.7}  & \num{52.5}  & \num{50.0}  & \num{-3.4}  & \num{3.2}  \\
%DIFDELCMD < Occasional drinker (\%)      & \num{26.0}  & \num{43.9}  & \num{25.5}  & \num{43.6}  & \num{-0.5}  & \num{2.8}  \\
%DIFDELCMD < Daily drinker (\%)           & \num{17.8}  & \num{38.3}  & \num{21.9}  & \num{41.4}  & \num{4.1}   & \num{2.6}  \\
%DIFDELCMD < Systolic (mmHg)               & \num{131.4} & \num{16.8}  & \num{128.7} & \num{14.3}  & \num{-2.7}  & \num{1.0}  \\
%DIFDELCMD < Diastolic (mmHg)              & \num{82.7}  & \num{11.6}  & \num{82.1}  & \num{11.3}  & \num{-0.6}  & \num{0.8}  \\
%DIFDELCMD < Waist circumference (cm)      & \num{87.7}  & \num{10.5}  & \num{85.4}  & \num{9.5}   & \num{-2.3}  & \num{0.8}  \\
%DIFDELCMD < Body mass index (kg/m2)       & \num{26.3}  & \num{3.7}   & \num{25.8}  & \num{3.6}   & \num{-0.5}  & \num{0.3}  \\
%DIFDELCMD < Frequency of coughing (\%)   & \num{18.7}  & \num{39.0}  & \num{19.7}  & \num{39.8}  & \num{1.0}   & \num{2.6}  \\
%DIFDELCMD < Frequency of wheezing (\%)   & \num{6.2}   & \num{24.2}  & \num{6.6}   & \num{24.8}  & \num{0.3}   & \num{1.6}  \\
%DIFDELCMD < Shortness of breath (\%)     & \num{29.2}  & \num{45.5}  & \num{34.3}  & \num{47.5}  & \num{5.1}   & \num{3.0}  \\
%DIFDELCMD < Chest trouble (\%)           & \num{11.6}  & \num{32.0}  & \num{14.1}  & \num{34.9}  & \num{2.5}   & \num{2.2}  \\
%DIFDELCMD < Any respiratory problem (\%) & \num{50.6}  & \num{50.0}  & \num{54.3}  & \num{49.9}  & \num{3.7}   & \num{3.2}  \\
%DIFDELCMD < Environmental measures:       &              &              &              &              &              &             \\
%DIFDELCMD < Temperature (°C)              & \num{13.8}  & \num{3.6}   & \num{13.5}  & \num{3.3}   & \num{-0.3}  & \num{0.2}  \\
%DIFDELCMD < Personal PM2.5 (ug/m3)        & \num{150.2} & \num{300.3} & \num{103.8} & \num{107.3} & \num{-46.3} & \num{19.1} \\
%DIFDELCMD < \bottomrule
%DIFDELCMD < \end{talltblr}
%DIFDELCMD < 

%DIFDELCMD < }
%DIFDELCMD < %%%
\DIFdelendFL \DIFaddbeginFL \centering{

\centering
\resizebox{\ifdim\width>\linewidth 0.95\linewidth\else\width\fi}{!}{
\begin{tblr}[         %% tabularray outer open
]                     %% tabularray outer close
{                     %% tabularray inner open
colspec={Q[]Q[]Q[]Q[]Q[]Q[]Q[]},
cell{1}{2}={c=2,}{halign=c,},
cell{1}{4}={c=2,}{halign=c,},
column{1}={font=\fontsize{0.9em}{1.2em}\selectfont,halign=l,},
column{2}={font=\fontsize{0.9em}{1.2em}\selectfont,halign=r,},
column{3}={font=\fontsize{0.9em}{1.2em}\selectfont,halign=r,},
column{4}={font=\fontsize{0.9em}{1.2em}\selectfont,halign=r,},
column{5}={font=\fontsize{0.9em}{1.2em}\selectfont,halign=r,},
column{6}={font=\fontsize{0.9em}{1.2em}\selectfont,halign=r,},
column{7}={font=\fontsize{0.9em}{1.2em}\selectfont,halign=r,},
row{3}={,cmd=\bfseries,},
row{13}={,cmd=\bfseries,},
row{31}={,cmd=\bfseries,},
rowsep={.1em},
}                     %% tabularray inner close
\toprule
& Never enrolled (N=603) &  & Ever enrolled (N=400) &  &  &  \\ \cmidrule[lr]{2-3}\cmidrule[lr]{4-5}
& Mean & Std. Dev. & Mean & Std. Dev. & Diff. in Means & Std. Error \\ \midrule %% TinyTableHeader
Demographics:                 &              &              &              &              &              &             \\
Age (years)                   & \num{59.9}  & \num{9.4}   & \num{60.4}  & \num{9.2}   & \num{0.5}   & \num{0.6}  \\
Female (\%)                  & \num{59.5}  & \num{49.1}  & \num{60.0}  & \num{49.1}  & \num{0.5}   & \num{3.2}  \\
No education (\%)            & \num{11.5}  & \num{31.9}  & \num{12.3}  & \num{32.9}  & \num{0.9}   & \num{2.1}  \\
Primary education (\%)       & \num{75.5}  & \num{43.0}  & \num{77.6}  & \num{41.7}  & \num{2.1}   & \num{2.8}  \\
Secondary+ education (\%)    & \num{12.6}  & \num{33.2}  & \num{9.8}   & \num{29.7}  & \num{-2.9}  & \num{2.0}  \\
Wealth index (bottom 25\%)   & \num{26.9}  & \num{44.4}  & \num{22.3}  & \num{41.7}  & \num{-4.6}  & \num{2.8}  \\
Wealth index (25-50\%)       & \num{23.6}  & \num{42.5}  & \num{27.0}  & \num{44.5}  & \num{3.4}   & \num{2.9}  \\
Wealth index (50-75\%)       & \num{24.7}  & \num{43.1}  & \num{25.5}  & \num{43.6}  & \num{0.8}   & \num{2.9}  \\
Wealth index (top 25\%)      & \num{24.8}  & \num{43.2}  & \num{25.2}  & \num{43.5}  & \num{0.4}   & \num{2.9}  \\
Health measures:              &              &              &              &              &              &             \\
Never smoker (\%)            & \num{21.8}  & \num{41.3}  & \num{19.1}  & \num{39.4}  & \num{-2.7}  & \num{2.6}  \\
Former smoker (\%)           & \num{11.9}  & \num{32.4}  & \num{15.1}  & \num{35.8}  & \num{3.2}   & \num{2.2}  \\
Passive smoker (\%)          & \num{39.6}  & \num{49.0}  & \num{40.2}  & \num{49.1}  & \num{0.6}   & \num{3.2}  \\
Current smoker (\%)          & \num{26.2}  & \num{44.0}  & \num{25.4}  & \num{43.6}  & \num{-0.8}  & \num{2.8}  \\
Never drinker (\%)           & \num{55.9}  & \num{49.7}  & \num{52.5}  & \num{50.0}  & \num{-3.4}  & \num{3.2}  \\
Occasional drinker (\%)      & \num{26.0}  & \num{43.9}  & \num{25.5}  & \num{43.6}  & \num{-0.5}  & \num{2.8}  \\
Daily drinker (\%)           & \num{17.8}  & \num{38.3}  & \num{21.9}  & \num{41.4}  & \num{4.1}   & \num{2.6}  \\
Systolic (mmHg)               & \num{131.4} & \num{16.8}  & \num{128.7} & \num{14.3}  & \num{-2.7}  & \num{1.0}  \\
Diastolic (mmHg)              & \num{82.7}  & \num{11.6}  & \num{82.1}  & \num{11.3}  & \num{-0.6}  & \num{0.8}  \\
Waist circumference (cm)      & \num{87.7}  & \num{10.5}  & \num{85.4}  & \num{9.5}   & \num{-2.3}  & \num{0.8}  \\
Body mass index (kg/m2)       & \num{26.3}  & \num{3.7}   & \num{25.8}  & \num{3.6}   & \num{-0.5}  & \num{0.3}  \\
Frequency of coughing (\%)   & \num{18.7}  & \num{39.0}  & \num{19.7}  & \num{39.8}  & \num{1.0}   & \num{2.6}  \\
Frequency of phlegm (\%)     & \num{27.6}  & \num{44.7}  & \num{23.7}  & \num{42.6}  & \num{-3.8}  & \num{2.8}  \\
Frequency of wheezing (\%)   & \num{6.2}   & \num{24.2}  & \num{6.6}   & \num{24.8}  & \num{0.3}   & \num{1.6}  \\
Shortness of breath (\%)     & \num{29.2}  & \num{45.5}  & \num{34.3}  & \num{47.5}  & \num{5.1}   & \num{3.0}  \\
Chest trouble (\%)           & \num{11.6}  & \num{32.0}  & \num{14.1}  & \num{34.9}  & \num{2.5}   & \num{2.2}  \\
Any respiratory problem (\%) & \num{50.6}  & \num{50.0}  & \num{54.3}  & \num{49.9}  & \num{3.7}   & \num{3.2}  \\
Environmental measures:       &              &              &              &              &              &             \\
Temperature (°C)              & \num{13.8}  & \num{3.6}   & \num{13.5}  & \num{3.3}   & \num{-0.3}  & \num{0.2}  \\
Personal PM2.5 (ug/m3)        & \num{127.1} & \num{145.3} & \num{102.3} & \num{105.5} & \num{-24.7} & \num{11.9} \\
Black carbon (ug/m3)          & \num{4.4}   & \num{5.3}   & \num{3.3}   & \num{3.4}   & \num{-1.1}  & \num{0.4}  \\
\bottomrule
\end{tblr}
}

}
\DIFaddendFL 

\end{table}%

Table\DIFdelbegin \DIFdel{4 }\DIFdelend \DIFaddbegin \DIFadd{~\ref{tbl-desc} }\DIFaddend shows the distribution of selected demographic,
health, and environmental characteristics from the baseline survey,
prior to any villages being enrolled in the \DIFdelbegin \DIFdel{CBHP policy}\DIFdelend \DIFaddbegin \DIFadd{CHP}\DIFaddend . We provide means and
standard deviations separately for villages that eventually enter into
the policy with those that never do so. As noted above, although our DiD
identification strategy allows for fixed differences between treated and
untreated villages, overall the differences at baseline are generally
small and the groups seem well balanced on most measures, with the
exception of personal exposure to PM\textsubscript{2.5}, which was lower
in villages that were eventually treated.

\subsection{Summary of PM and BC
measurements}\label{summary-of-pm-and-bc-measurements}

At baseline \DIFaddbegin \DIFadd{before the policy was rolled out in any study villages}\DIFaddend ,
PM\textsubscript{2.5} and BC concentrations were higher, on average, for
personal exposures compared with outdoor concentrations. From \DIFdelbegin \DIFdel{w2 }\DIFdelend \DIFaddbegin \DIFadd{W2 }\DIFaddend onward,
with the inclusion of indoor air pollution measurements, personal
exposure air pollution concentrations were still higher than indoor or
outdoor concentrations, with indoor levels being higher than outdoors
(Table~\ref{tbl-pm-season}). This trend (personal \textgreater{} indoor
\textgreater{} outdoor) was observed among households in treated and
untreated villages. Personal, indoor, and outdoor geometric mean (95\%
confidence interval) concentrations of PM\textsubscript{2.5} were 72
(65, 80), 45 (39, 53), and \DIFdelbegin \DIFdel{31 (28, 35) }\DIFdelend \DIFaddbegin \DIFadd{33 (29, 36) µg/m\textsuperscript{3}}\DIFaddend ,
respectively, and elevated relative to health-based guidelines. The
current World Health Organization (WHO) guidelines state that annual
average \DIFdelbegin \DIFdel{concentrations of }\DIFdelend \DIFaddbegin \DIFadd{exposures to }\DIFaddend PM\textsubscript{2.5} should not exceed 5
µg/m\textsuperscript{3}, while 24-hour average exposures should not
exceed 15 µg/m\textsuperscript{3} for more than 3 to 4 days per year
(World Health Organization 2021). Interim targets have been set to
support the planning of incremental milestones toward cleaner air,
particularly for cities, regions, and countries with higher air
pollution levels. For PM\textsubscript{2.5}, the four interim (IT)
targets for annual and 24-h means are: IT-1: 35 and 75
µg/m\textsuperscript{3}; IT-2: 25 and 50 µg/m\textsuperscript{3}; IT-3:
15 and 37.5 µg/m\textsuperscript{3}; and IT-4: 10 and 25
µg/m\textsuperscript{3} (World Health Organization 2021). \DIFdelbegin \DIFdel{In our study,
}\DIFdelend \DIFaddbegin \DIFadd{The }\DIFaddend baseline
personal exposures to PM\textsubscript{2.5} \DIFdelbegin \DIFdel{did not even meet
}\DIFdelend \DIFaddbegin \DIFadd{in our study aligned with
}\DIFaddend IT-1, indicating considerable opportunity for air quality exposure
reduction with intervention.

\DIFaddbegin \begin{table}

\caption{\label{tbl-pm-season}\DIFaddFL{Arithmetic and geometric means for air
pollutant concentrations (micrograms per cubic meter) by wave.}}

\centering{

\centering
\begin{talltblr}[         %% tabularray outer open
entry=none,label=none,
note{}={Note: Est. = Estimate, CI = 95 percent confidence interval, GM = Geometric Mean},
]                     %% tabularray outer close
{                     %% tabularray inner open
width={1\linewidth},
colspec={X[0.2]X[0.171428571428571]X[0.0571428571428571]X[0.0285714285714286]X[0.114285714285714]X[0.0285714285714286]X[0.114285714285714]X[0.0285714285714286]X[0.114285714285714]X[0.0285714285714286]X[0.114285714285714]},
cell{1}{4}={c=2,}{halign=c,},
cell{1}{6}={c=2,}{halign=c,},
cell{1}{8}={c=2,}{halign=c,},
cell{1}{10}={c=2,}{halign=c,},
cell{3}{1}={c=11}{},cell{8}{1}={c=11}{},cell{15}{1}={c=11}{},
cell{4}{1}={preto={\hspace{1em}}},
cell{5}{1}={preto={\hspace{1em}}},
cell{6}{1}={preto={\hspace{1em}}},
cell{7}{1}={preto={\hspace{1em}}},
cell{9}{1}={preto={\hspace{1em}}},
cell{10}{1}={preto={\hspace{1em}}},
cell{11}{1}={preto={\hspace{1em}}},
cell{12}{1}={preto={\hspace{1em}}},
cell{13}{1}={preto={\hspace{1em}}},
cell{14}{1}={preto={\hspace{1em}}},
cell{16}{1}={preto={\hspace{1em}}},
cell{17}{1}={preto={\hspace{1em}}},
cell{18}{1}={preto={\hspace{1em}}},
cell{19}{1}={preto={\hspace{1em}}},
cell{20}{1}={preto={\hspace{1em}}},
cell{21}{1}={preto={\hspace{1em}}},
cell{22}{1}={preto={\hspace{1em}}},
row{3}={halign=l,cmd=\bfseries,},
row{8}={halign=l,cmd=\bfseries,},
row{15}={halign=l,cmd=\bfseries,},
cell{4}{2}={r=2,}{valign=h,},
cell{6}{2}={r=2,}{valign=h,},
cell{9}{2}={r=2,}{valign=h,},
cell{11}{2}={r=2,}{valign=h,},
cell{13}{2}={r=2,}{valign=h,},
cell{16}{2}={r=2,}{valign=h,},
cell{18}{2}={r=2,}{valign=h,},
cell{20}{2}={r=2,}{valign=h,},
cell{4}{1}={r=4,}{valign=h,},
cell{11}{1}={r=4,}{valign=h,},
cell{18}{1}={r=4,}{valign=h,},
cell{9}{1}={r=2,}{valign=h,},
cell{16}{1}={r=2,}{valign=h,},
column{1}={font=\fontsize{0.8em}{1.1em}\selectfont,},
column{2}={font=\fontsize{0.8em}{1.1em}\selectfont,},
column{3}={font=\fontsize{0.8em}{1.1em}\selectfont,},
column{4}={font=\fontsize{0.8em}{1.1em}\selectfont,},
column{5}={font=\fontsize{0.8em}{1.1em}\selectfont,},
column{6}={font=\fontsize{0.8em}{1.1em}\selectfont,},
column{7}={font=\fontsize{0.8em}{1.1em}\selectfont,},
column{8}={font=\fontsize{0.8em}{1.1em}\selectfont,},
column{9}={font=\fontsize{0.8em}{1.1em}\selectfont,},
column{10}={font=\fontsize{0.8em}{1.1em}\selectfont,},
column{11}={font=\fontsize{0.8em}{1.1em}\selectfont,},
row{2}={halign=c,},
hline{6,11,13,18,20}={2,3,4,5,6,7,8,9,10,11}{solid, 0.1em, gray},
hline{8,15}={1,2,3,4,5,6,7,8,9,10,11}{solid, 0.1em, gray},
}                     %% tabularray inner close
\toprule
&  &  & Wave 1 &  & Wave 2 &  & Wave 3 &  & Wave 4 &  \\ \cmidrule[lr]{4-5}\cmidrule[lr]{6-7}\cmidrule[lr]{8-9}\cmidrule[lr]{10-11}
&  &  & Est. & CI & Est. & CI & Est. & CI & Est. & CI \\ \midrule %% TinyTableHeader
Personal measurements &&&&&&&&&& \\
Filter-derived & 24h PM2.5 & Mean & 117 & [105, 129] & 97 & [87, 107] &  &  & 84 & [72, 96] \\
Filter-derived & 24h PM2.5 & GM & 72 & [65, 80] & 60 & [54, 66] &  &  & 47 & [42, 52] \\
Filter-derived & 24h BC & Mean & 3.9 & [3.5, 4.4] & 3.6 & [2.9, 4.2] &  &  & 3.7 & [2.9, 4.5] \\
Filter-derived & 24h BC & GM & 2.6 & [2.4, 2.8] & 1.9 & [1.7, 2.1] &  &  & 1.7 & [1.5, 1.9] \\
Indoor measurements &&&&&&&&&& \\
Sensor-derived & Seasonal PM2.5 & Mean &  &  & 94 & [84, 103] & 84 & [75, 94] & 67 & [59, 75] \\
Sensor-derived & Seasonal PM2.5 & GM &  &  & 71 & [65, 77] & 63 & [57, 70] & 47 & [42, 52] \\
Filter-derived & 24h PM2.5 & Mean &  &  & 69 & [59, 78] &  &  & 58 & [48, 68] \\
Filter-derived & 24h PM2.5 & GM &  &  & 45 & [39, 53] &  &  & 33 & [27, 40] \\
Filter-derived & 24h BC & Mean &  &  & 2.7 & [2.1, 3.2] &  &  & 2.9 & [2.2, 3.5] \\
Filter-derived & 24h BC & GM &  &  & 1.6 & [1.3, 2.0] &  &  & 1.6 & [1.3, 1.9] \\
Outdoor measurements &&&&&&&&&& \\
Sensor-derived & Seasonal PM2.5 & Mean & 47 & [45, 48] & 55 & [54, 56] & 33 & [32, 34] & 33 & [32, 34] \\
Sensor-derived & Seasonal PM2.5 & GM & 36 & [35, 37] & 40 & [39, 41] & 23 & [22, 23] & 22 & [22, 23] \\
Filter-derived & Seasonal PM2.5 & Mean & 38 & [34, 42] & 38 & [34, 41] & 25 & [23, 28] & 26 & [24, 28] \\
Filter-derived & Seasonal PM2.5 & GM & 33 & [29, 36] & 30 & [28, 32] & 21 & [19, 23] & 22 & [21, 24] \\
Filter-derived & Seasonal BC & Mean & 1.5 & [1.3, 1.6] & 1.4 & [1.3, 1.5] &  &  & 1.2 & [1.1, 1.2] \\
Filter-derived & Seasonal BC & GM & 1.3 & [1.1, 1.4] & 1.1 & [1.0, 1.2] &  &  & 1.0 & [0.9, 1.1] \\
\bottomrule
\end{talltblr}

}

\end{table}%DIF > 

\DIFaddend We also present the geometric and arithmetic means (and 95\% confidence
intervals) for PM\textsubscript{2.5} and BC in each measurement wave
(Table~\ref{tbl-pm-season}). Wave 3 (2020/2021) was a partial wave that
took place over a time period impacted by the COVID-19 pandemic and did
not involve filter-based air pollution sample collection.

\subsection{Policy uptake}\label{policy-uptake}

Each year of the study, participants reported the types of fuels and
stoves and the amount of fuel used for space heating in winter. Based on
these data, heating energy types were classified into four categories:
exclusive use of a heat pump (``Heat pump exclusively''), use of a heat
pump and a biomass-fueled kang (``\DIFdelbegin \DIFdel{Heat pump }\DIFdelend \DIFaddbegin \DIFadd{Heatpump }\DIFaddend with biomass kang''), use of
solid fuel heater with an electric heating devices other than heat pumps
(``Coal stove and/or biomass kang with electric heater'), and exclusive
use of solid fuel (`Coal stove and/or biomass kang'). In villages
treated by the policy, Figure~\ref{fig-sankey} shows meaningful
transitions from solid fuel to electric-powered heat pumps for all
treatment cohorts. For example, the proportion of households in the
group treated in 2019 (\DIFdelbegin \DIFdel{w2}\DIFdelend \DIFaddbegin \DIFadd{W2}\DIFaddend ) using heat pumps increased from 3\% in \DIFdelbegin \DIFdel{w1 }\DIFdelend \DIFaddbegin \DIFadd{W1 }\DIFaddend to
93\% in \DIFdelbegin \DIFdel{w2 }\DIFdelend \DIFaddbegin \DIFadd{W2 }\DIFaddend and 96\% in \DIFdelbegin \DIFdel{w4}\DIFdelend \DIFaddbegin \DIFadd{W4}\DIFaddend . Conversely, use of coal stoves decreased from
97\% in \DIFdelbegin \DIFdel{w1 }\DIFdelend \DIFaddbegin \DIFadd{W1 }\DIFaddend to 8\% in \DIFdelbegin \DIFdel{w2 }\DIFdelend \DIFaddbegin \DIFadd{W2 }\DIFaddend and 3\% in \DIFdelbegin \DIFdel{w4}\DIFdelend \DIFaddbegin \DIFadd{W4}\DIFaddend . We observed similar stove use
transitions for households in villages treated in 2020 (\DIFdelbegin \DIFdel{w3}\DIFdelend \DIFaddbegin \DIFadd{W3}\DIFaddend ). In the
three villages treated in 2021, we observed overall less exclusive use
of the heat pump and a slightly larger proportion of households
continuing to use coal.

\begin{figure}[H]

\centering{

\includegraphics[width=1\textwidth,height=\textheight]{images/sanky.jpg}

}

\caption{\label{fig-sankey}Transitions to different energy sources
across study waves}

\end{figure}%

We also observed a substantial decline in the amount of self-reported
coal used in villages treated by \DIFdelbegin \DIFdel{CBHP policy (Appendix
Figure~\ref{fig-afig-coal}}\DIFdelend \DIFaddbegin \DIFadd{the CHP (Figure~\ref{fig-coal}}\DIFaddend ), though
the reduction in coal use was smaller with each subsequent treatment
cohort (Appendix Table~\ref{tbl-fuel-did}). Biomass (i.e., wood
logs/twigs or charcoal), usually burned in kangs for both cooking and
space heating, was not expressly targeted by the \DIFdelbegin \DIFdel{CBHP policy}\DIFdelend \DIFaddbegin \DIFadd{CHP}\DIFaddend . We observed
declines in self-reported biomass use in villages treated in 2019 and
2020, but there was a small increase in biomass consumption in the
cohort treated last (2021).

\DIFaddbegin \begin{figure}[H]

\centering{

\includegraphics[width=1\textwidth,height=\textheight]{images/coal-plot.png}

}

\caption{\label{fig-coal}\DIFaddFL{Trends in self-reported coal and biomass, by
treatment season.}}

\end{figure}%DIF > 

\DIFaddend In never treated villages, we also observed a transition \DIFaddbegin \DIFadd{from solid fuel
}\DIFaddend to clean energy over the four year study but it was much slower than in
\DIFdelbegin \DIFdel{treated
villages }\DIFdelend \DIFaddbegin \DIFadd{villages exposed to the CHP}\DIFaddend . The proportion of households that reported
using electric heat pumps increased from 5\% in \DIFdelbegin \DIFdel{w1 }\DIFdelend \DIFaddbegin \DIFadd{W1 }\DIFaddend to 10\% in \DIFdelbegin \DIFdel{w2 }\DIFdelend \DIFaddbegin \DIFadd{W2 }\DIFaddend and
25\% in \DIFdelbegin \DIFdel{w4}\DIFdelend \DIFaddbegin \DIFadd{W4}\DIFaddend , and those who adopted heat pumps tended to use them
exclusively. Commensurately, the reported expenditures on electricity
increased gradually over time in the untreated villages\DIFdelbegin \DIFdel{(data not shown)}\DIFdelend . The percentage
of untreated households using solid fuel with other types of electric
devices remained relatively stable, ranging from 64\% to 70\% across
waves. Self-reported use of biomass also remained stable, at
approximately one ton of fuel each winter, whereas exclusive use of
solid fuel decreased from 30\% in \DIFdelbegin \DIFdel{w1 }\DIFdelend \DIFaddbegin \DIFadd{W1 }\DIFaddend to 7\% in \DIFdelbegin \DIFdel{w4}\DIFdelend \DIFaddbegin \DIFadd{W4}\DIFaddend .

\subsection{Aim 1: Policy impacts and potential
mediation}\label{aim-1-policy-impacts-and-potential-mediation}

\subsubsection{Impact of \DIFdelbegin \DIFdel{the CBHP }\DIFdelend policy on potential mediators \DIFaddbegin \DIFadd{of air pollution
and indoor
temperature}\DIFaddend }\DIFdelbegin %DIFDELCMD < \label{impact-of-the-cbhp-policy-on-potential-mediators}
%DIFDELCMD < %%%
\DIFdelend \DIFaddbegin \label{impact-of-policy-on-potential-mediators-of-air-pollution-and-indoor-temperature}
\DIFaddend 

The average marginal effect (\(ATT\)) from the basic ETWFE model
(Table~\ref{tbl-did-med}) shows that exposure to the \DIFdelbegin \DIFdel{CBHP policy reduced 24h }\DIFdelend \DIFaddbegin \DIFadd{CHP reduced 24-h
}\DIFaddend indoor PM\textsubscript{2.5} by \DIFdelbegin \DIFdel{-19 }\DIFdelend \DIFaddbegin \DIFadd{-18.9 }\DIFaddend µg/m\textsuperscript{3} (95\%CI:
\DIFdelbegin \DIFdel{-61, 22}\DIFdelend \DIFaddbegin \DIFadd{-56.1, 18.4}\DIFaddend ). After adjusting \DIFaddbegin \DIFadd{for }\DIFaddend outdoor temperature, \DIFdelbegin \DIFdel{dewpoint}\DIFdelend \DIFaddbegin \DIFadd{dew point}\DIFaddend ,
household smoking status, and the number of residents in each household,
the \(ATT\) \DIFdelbegin \DIFdel{decreased to -14 }\DIFdelend \DIFaddbegin \DIFadd{was -20.0 }\DIFaddend µg/m\textsuperscript{3} (95\%CI: \DIFdelbegin \DIFdel{-54, 26}\DIFdelend \DIFaddbegin \DIFadd{-45.6, 5.5}\DIFaddend ). The
\DIFaddbegin \DIFadd{basic DiD }\DIFaddend impact was stronger on seasonal indoor PM\textsubscript{2.5},
with an average \(ATT\) of \DIFdelbegin \DIFdel{-36 }\DIFdelend \DIFaddbegin \DIFadd{-30.9 }\DIFaddend µg/m\textsuperscript{3} (95\%CI: \DIFdelbegin \DIFdel{-61,
-12)that
was robust to covariate adjustment }\DIFdelend \DIFaddbegin \DIFadd{-53.2,
-8.7). After adjustment the average \(ATT\) on seasonal indoor
PM\textsubscript{2.5} was -20.3 µg/m\textsuperscript{3} (95\%CI: -37.5,
-3.0)}\DIFaddend . This finding likely reflects the direct benefit of the policy in
replacing coal stoves and air quality improvement. We found
\DIFaddbegin \comment{EC 9a}\DIFaddend little evidence of heterogeneity in \(ATT\)s across
cohort and time (all p-values \textgreater{} \DIFdelbegin \DIFdel{0.385 }\DIFdelend \DIFaddbegin \DIFadd{0.4 }\DIFaddend for tests of
heterogeneity, see Appendix Table~\ref{tbl-a-het-indoor}). \DIFdelbegin %DIFDELCMD < 

%DIFDELCMD < %%%
\DIFdelend Overall we
found little evidence of an impact of the \DIFdelbegin \DIFdel{CBHP policy on 24-h
and seasonal outdoor (community}\DIFdelend \DIFaddbegin \DIFadd{CHP on different measures of
outdoor (local, community-level}\DIFaddend ) PM\textsubscript{2.5} or personal
exposures to PM\textsubscript{2.5} and BC\DIFdelbegin \DIFdel{. Treatment was associated with lower,
but statistically imprecise
, personal 24-h BC exposures. This
finding is consistent with the expectation that the
policy contributed
to reducing air pollutant emissions from solid fuel burning, as BC
serves as a potential indicator of such combustion.
}\DIFdelend \DIFaddbegin \DIFadd{, with adjusted \(ATT\)s of
-1.7 and 0.4 for 24-hr and seasonal outdoor PM\textsubscript{2.5},
respectively. Adjusted estimates for personal PM\textsubscript{2.5} and
personal BC were 0.5 and -0.4, respectively, and generally all of the
estimates for outdoor and personal exposure impacts were imprecise
(Table~\ref{tbl-did-med}), and we provide the full set of cohort-time
\(ATT\)s for personal exposure in Appendix
Table~\ref{tbl-a-het-personal}. Appendix Table~\ref{tbl-a-ind-s3} and
Table~\ref{tbl-a-out-s3} show the impact of including Wave 3 data on the
estimates of the impact of the policy on indoor (seasonal) and outdoor
(24-hr and seasonal) PM\textsubscript{2.5}, respectively. Generally this
improved precision but did not affect the magnitude of our estimates.
Further adjustment }\comment{EC 10b}\DIFadd{for district of residence in the
covariate-adjusted models had little impact on our results (Appendix
Table~\ref{tbl-dist-fe}).
}\DIFaddend 

\DIFdelbegin \DIFdel{With respect }\DIFdelend \DIFaddbegin \begin{table}

\caption{\label{tbl-did-med}\DIFaddFL{Treatment effect on outdoor and indoor
PM\textsubscript{2.5}, personal exposure to PM\textsubscript{2.5} and
black carbon, and measures of indoor temperature. Outdoor and indoor
PM\textsubscript{2.5} were derived from sensor measurements after being
adjusted based on co-located gravimetric PM\textsubscript{2.5}
measurements. 24h indicates the mean PM\textsubscript{2.5}
concentrations during the 24 hours when personal exposure samples were
collected in each village. `Seasonal' indicates the seasonal mean
PM\textsubscript{2.5} concentrations in each village, from Jan.~15th to
Mar.~15th.}}

\centering{

\centering
\begin{talltblr}[         %% tabularray outer open
entry=none,label=none,
note{}={Note: ATT = Average Treatment Effect on the Treated, DiD = Difference-in-Differences, ETWFE = Extended Two-Way Fixed Effects.},
note{a}={ETWFE models for air pollution outcomes were adjusted for household size, smoking, outdoor temperature, and outdoor dewpoint. Temperature models adjusted for the number of rooms and wintertime occupants in the household, age of the primary respondent, and wealth index},
note{b}={The indoor 24-hr PM2.5 concentration was determined over the time period concurrent with when the personal PM2.5 concentration was determined.},
]                     %% tabularray outer close
{                     %% tabularray inner open
colspec={Q[]Q[]Q[]Q[]Q[]Q[]Q[]},
cell{1}{4}={c=2,}{halign=c,},
cell{1}{6}={c=2,}{halign=c,},
cell{3}{1}={c=7}{},cell{10}{1}={c=7}{},
cell{4}{1}={preto={\hspace{1em}}},
cell{5}{1}={preto={\hspace{1em}}},
cell{6}{1}={preto={\hspace{1em}}},
cell{7}{1}={preto={\hspace{1em}}},
cell{8}{1}={preto={\hspace{1em}}},
cell{9}{1}={preto={\hspace{1em}}},
cell{11}{1}={preto={\hspace{1em}}},
cell{12}{1}={preto={\hspace{1em}}},
cell{13}{1}={preto={\hspace{1em}}},
cell{14}{1}={preto={\hspace{1em}}},
cell{15}{1}={preto={\hspace{1em}}},
cell{16}{1}={preto={\hspace{1em}}},
cell{17}{1}={preto={\hspace{1em}}},
cell{18}{1}={preto={\hspace{1em}}},
row{3}={halign=l,cmd=\bfseries,},
row{10}={halign=l,cmd=\bfseries,},
cell{4}{1}={r=2,}{valign=h,},
cell{6}{1}={r=2,}{valign=h,},
cell{8}{1}={r=2,}{valign=h,},
cell{12}{1}={r=6,}{valign=h,},
column{1}={halign=l,},
column{2}={halign=l,},
column{3}={halign=c,},
column{4}={halign=c,},
column{5}={halign=c,},
column{6}={halign=c,},
column{7}={halign=c,},
}                     %% tabularray inner close
\toprule
&  &  & DiD &  & Adjusted DiD &  \\ \cmidrule[lr]{4-5}\cmidrule[lr]{6-7}
&   & Obs & ATT & (95\% CI) & ATT\textsuperscript{a} & (95\% CI) \\ \midrule %% TinyTableHeader
Air pollution &&&&&& \\
Personal & PM2.5 &  1270 & -3.0 & (-26.1, 20.1) & 0.2 & (-19.6, 19.9) \\
Personal & Black carbon &  1161 & -0.6 & (-1.7, 0.6) & -0.4 & (-1.5, 0.6) \\
Indoor & 24-hr PM2.5\textsuperscript{b} &   399 & -18.9 & (-56.1, 18.4) & -20.0 & (-45.6, 5.5) \\
Indoor & Seasonal PM2.5 &   366 & -30.9 & (-53.2, -8.7) & -20.3 & (-37.5, -3.0) \\
Outdoor & 24-hr PM2.5 & 11174 & -0.5 & (-5.5, 4.4) & -2.1 & (-10.0, 5.8) \\
Outdoor & Seasonal PM2.5 &   139 & 1.7 & (-3.4, 6.7) & 0.5 & (-4.8, 5.9) \\
Indoor temperature &&&&&& \\
Point & Mean &  2999 & 1.9 & (0.9, 2.9) & 1.9 & (0.9, 2.9) \\
Seasonal & Mean (all) &  1350 & 0.7 & (-0.1, 1.4) & 0.7 & (-0.1, 1.4) \\
Seasonal & Mean (daytime) &  1346 & 0.8 & (0.0, 1.5) & 0.8 & (0.0, 1.5) \\
Seasonal & Mean (heating season) &  1350 & 1.8 & (0.9, 2.7) & 1.8 & (0.9, 2.7) \\
Seasonal & Mean (daytime heating season) &  1346 & 2.0 & (1.0, 2.9) & 1.9 & (1.0, 2.9) \\
Seasonal & Min. (all) &  1350 & 4.2 & (2.3, 6.0) & 4.2 & (2.4, 6.1) \\
Seasonal & Min. (heating season) &  1350 & 4.2 & (2.3, 6.0) & 4.2 & (2.4, 6.0) \\
\bottomrule
\end{talltblr}

}

\end{table}%DIF > 

\DIFadd{With respect }\comment{EC 12a, 12d}\DIFaddend to the other potential mediator of
temperature, Table~\ref{tbl-did-med} shows that exposure to the \DIFdelbegin \DIFdel{CBHP policy }\DIFdelend \DIFaddbegin \DIFadd{CHP
}\DIFaddend increased mean household point temperature by \DIFdelbegin \DIFdel{around 2}\DIFdelend \DIFaddbegin \DIFadd{1.9}\DIFaddend °C (95\%CI: \DIFdelbegin \DIFdel{1, 3}\DIFdelend \DIFaddbegin \DIFadd{0.9, 2.9}\DIFaddend ),
with similar impacts on mean seasonal temperatures during the heating
season. The \DIFdelbegin \DIFdel{CBHP policy }\DIFdelend \DIFaddbegin \DIFadd{CHP }\DIFaddend had considerably stronger impacts on average seasonal
minimum temperatures, which increased by 3.8°C (95\%CI: 2.3, 5.4).
\DIFaddbegin \DIFadd{Additionally adjusting models for the number of rooms and wintertime
occupants in the household, age of the primary respondent, and wealth
index had little impact on \(ATT\)s.
}\DIFaddend 

\subsubsection{Impact of the \DIFdelbegin \DIFdel{CBHP }\DIFdelend policy on health
outcomes}\DIFdelbegin %DIFDELCMD < \label{impact-of-the-cbhp-policy-on-health-outcomes}
%DIFDELCMD < %%%
\DIFdelend \DIFaddbegin \label{impact-of-the-policy-on-health-outcomes}
\DIFaddend 

\DIFaddbegin \begin{table}

\caption{\label{tbl-did-health}\DIFaddFL{Overall impacts of the CHP on blood
pressure, respiratory outcomes, inflammatory markers and MDA.}}

\centering{

\centering
\begin{talltblr}[         %% tabularray outer open
entry=none,label=none,
note{}={Note: ATT = Average Treatment Effect on the Treated, DiD = Difference-in-Differences, ETWFE = Extended Two-Way Fixed Effects, Obs = observations, pp = percentage points, ppb = parts per billion.},
note{a}={ETWFE models for blood pressure models adjusted for age, sex, waist circumference, smoking, alcohol consumption, and use of blood pressure medication. Self-reported respiratory outcomes adjusted for age, gender, smoking, occupation, frequency of drinking, frequency of farming. Measured respiratory outcome (FeNO) adjusted age, gender, body mass index, frequency of drinking, smoking, and frequency of exercise, occupation, time of measurement. Inflammatory marker and MDA outcome models adjusted for age, waist circumference, occupation, wealth index quantile, frequency of drinking, tobacco smoking, and frequency of farming.},
]                     %% tabularray outer close
{                     %% tabularray inner open
colspec={Q[]Q[]Q[]Q[]Q[]Q[]Q[]},
cell{1}{4}={c=2,}{halign=c,},
cell{1}{6}={c=2,}{halign=c,},
cell{3}{1}={c=7}{},cell{12}{1}={c=7}{},cell{20}{1}={c=7}{},
cell{4}{1}={preto={\hspace{1em}}},
cell{5}{1}={preto={\hspace{1em}}},
cell{6}{1}={preto={\hspace{1em}}},
cell{7}{1}={preto={\hspace{1em}}},
cell{8}{1}={preto={\hspace{1em}}},
cell{9}{1}={preto={\hspace{1em}}},
cell{10}{1}={preto={\hspace{1em}}},
cell{11}{1}={preto={\hspace{1em}}},
cell{13}{1}={preto={\hspace{1em}}},
cell{14}{1}={preto={\hspace{1em}}},
cell{15}{1}={preto={\hspace{1em}}},
cell{16}{1}={preto={\hspace{1em}}},
cell{17}{1}={preto={\hspace{1em}}},
cell{18}{1}={preto={\hspace{1em}}},
cell{19}{1}={preto={\hspace{1em}}},
cell{21}{1}={preto={\hspace{1em}}},
cell{22}{1}={preto={\hspace{1em}}},
cell{23}{1}={preto={\hspace{1em}}},
cell{24}{1}={preto={\hspace{1em}}},
cell{25}{1}={preto={\hspace{1em}}},
row{3}={halign=l,cmd=\bfseries,},
row{12}={halign=l,cmd=\bfseries,},
row{20}={halign=l,cmd=\bfseries,},
cell{4}{1}={r=2,}{valign=h,},
cell{6}{1}={r=2,}{valign=h,},
cell{8}{1}={r=2,}{valign=h,},
cell{10}{1}={r=2,}{valign=h,},
cell{13}{1}={r=6,}{valign=h,},
cell{21}{1}={r=4,}{valign=h,},
column{1}={halign=l,},
column{2}={halign=l,},
column{3}={halign=c,},
column{4}={halign=c,},
column{5}={halign=c,},
column{6}={halign=c,},
column{7}={halign=c,},
}                     %% tabularray inner close
\toprule
&  &  & DiD &  & Adjusted DiD &  \\ \cmidrule[lr]{4-5}\cmidrule[lr]{6-7}
&   & Obs & ATT & (95\% CI) & ATT\textsuperscript{a} & (95\% CI) \\ \midrule %% TinyTableHeader
Blood pressure &&&&&& \\
Systolic BP (mmHg) & Brachial & 3082 & -0.8 & (-2.6, 1.0) & -1.4 & (-3.3, 0.5) \\
Systolic BP (mmHg) & Central & 3081 & -1.0 & (-2.8, 0.7) & -1.6 & (-3.4, 0.3) \\
Diastolic BP (mmHg) & Brachial & 3082 & -1.3 & (-2.6, 0.0) & -1.6 & (-3.0, -0.2) \\
Diastolic BP (mmHg) & Central & 3081 & -1.4 & (-2.7, -0.0) & -1.7 & (-3.0, -0.3) \\
Pulse Pressure & Brachial & 3082 & 0.5 & (-0.7, 1.7) & 0.2 & (-1.0, 1.4) \\
Pulse Pressure & Central & 3081 & 0.3 & (-0.8, 1.5) & 0.1 & (-1.0, 1.2) \\
BP Amplification x100 & Pulse pressure & 3081 & 0.1 & (-1.1, 1.4) & -0.0 & (-1.2, 1.2) \\
BP Amplification x100 & Systolic BP & 3081 & 0.2 & (-0.2, 0.5) & 0.1 & (-0.2, 0.4) \\
Respiratory outcomes &&&&&& \\
Self-reported (pp) & Any symptom & 3076 & -7.7 & (-12.8, -2.5) & -7.5 & (-12.7, -2.3) \\
Self-reported (pp) & Coughing & 3076 & -2.6 & (-7.2, 2.0) & -2.7 & (-7.1, 1.7) \\
Self-reported (pp) & Phlegm & 3076 & -1.3 & (-5.5, 2.9) & -1.6 & (-5.6, 2.4) \\
Self-reported (pp) & Wheezing attacks & 3076 & 0.7 & (-2.3, 3.8) & 1.0 & (-1.9, 3.9) \\
Self-reported (pp) & Trouble breathing & 3076 & -4.4 & (-9.9, 1.0) & -3.4 & (-9.2, 2.4) \\
Self-reported (pp) & Chest trouble & 3076 & -4.2 & (-8.8, 0.5) & -3.4 & (-8.1, 1.3) \\
Measured & FeNO (ppb) &  793 & 0.9 & (-1.6, 3.3) & 0.3 & (-2.2, 2.8) \\
Inflammatory markers &&&&&& \\
Measured & IL6 (pg/mL) & 1603 & 0.9 & (-0.2, 1.9) & 0.8 & (-0.3, 2.0) \\
Measured & TNF-alpha (pg/mL) & 1603 & 1.0 & (0.1, 1.8) & 0.8 & (-0.1, 1.7) \\
Measured & CRP (mg/L) & 1603 & 0.1 & (-0.4, 0.6) & 0.1 & (-0.5, 0.6) \\
Measured & MDA (µM) & 1603 & 0.3 & (-0.1, 0.8) & 0.2 & (-0.2, 0.6) \\
\bottomrule
\end{talltblr}

}

\end{table}%DIF > 

\DIFaddend Table~\ref{tbl-did-health} shows the impacts of the policy on blood
pressure in basic ETWFE models and models further adjusted for age, sex,
waist circumference, smoking, alcohol consumption, and use of blood
pressure medication. Overall exposure to the \DIFdelbegin \DIFdel{CBHP policy }\DIFdelend \DIFaddbegin \DIFadd{CHP }\DIFaddend demonstrated reductions
in blood pressure of approximately 1.5 mmHg for both systolic and
diastolic BP, but we found little evidence of a meaningful impact on
pulse pressure or BP amplification. The effects \DIFaddbegin \DIFadd{of the policy }\DIFaddend on
brachial and central blood pressures were similar\DIFaddbegin \DIFadd{, and were consistent
}\comment{EC 2e}\DIFadd{when restricted to only those participants enrolled in W1
(i.e., excluding participants recruited in later waves, see Appendix
Table~\ref{tbl-a-bp-sample})}\DIFaddend . However, the average effects in
Table~\ref{tbl-did-health} conceal a fair amount of heterogeneity in
treatment effects for blood pressure across treatment cohorts and time.
Appendix Table~\ref{tbl-bp-het} shows that treatment impacts were
considerably stronger for the earlier compared to the later-treated
cohorts. For example, the \DIFdelbegin \DIFdel{CBHP policy }\DIFdelend \DIFaddbegin \DIFadd{CHP }\DIFaddend reduced central DBP in the year of
treatment by -2.7 mmHg (95\%CI: \DIFdelbegin \DIFdel{-4.6, -0.8}\DIFdelend \DIFaddbegin \DIFadd{-4.7, -0.7}\DIFaddend ) for the villages first
treated in wave 2, but increased central DBP by 1.1 mmHg (95\%CI: -0.1,
2.2) for the villages treated in wave 4 (\emph{p}-value for
heterogeneity \textless0.0001).

Table~\ref{tbl-did-health} shows the impacts on self-reported chronic
respiratory symptoms categorized as any symptoms and separately for each
individual symptom type. \DIFdelbegin \DIFdel{In both basic and }\DIFdelend \DIFaddbegin \DIFadd{Based on the }\DIFaddend covariate-adjusted ETWFE models,
exposure to the \DIFdelbegin \DIFdel{CBHP policy }\DIFdelend \DIFaddbegin \DIFadd{CHP }\DIFaddend reduced self-report of any poor respiratory symptoms
by \DIFdelbegin \DIFdel{around 7 }\DIFdelend \DIFaddbegin \DIFadd{7.5 }\DIFaddend percentage points (95\%CI: \DIFdelbegin \DIFdel{-14, -1}\DIFdelend \DIFaddbegin \DIFadd{-12.7, -2.3}\DIFaddend ). This \DIFdelbegin \DIFdel{was largely through reductions }\DIFdelend \DIFaddbegin \DIFadd{overall effect was
mostly due to reductions of roughly 3 percentage points }\DIFaddend in reports of
\DIFaddbegin \DIFadd{coughing, }\DIFaddend having chest trouble\DIFaddbegin \DIFadd{, }\DIFaddend or difficulty breathing\DIFaddbegin \DIFadd{, respectively,
}\DIFaddend on several or most days of the week. \DIFaddbegin \DIFadd{We }\comment{EC 9b}\DIFadd{found limited
evidence that the CHP reduced self-reported symptoms of phlegm (-1.6,
95\%CI: -5.6, 2.4) or wheezing (1.0, 95\%CI: -1.9, 3.9). }\DIFaddend Appendix tables
\ref{tbl-a-het-resp}, \ref{tbl-a-het-cough}, \ref{tbl-a-het-phlegm},
\ref{tbl-a-het-wheeze}, \ref{tbl-a-het-breath}, \ref{tbl-a-het-nochest}
show little evidence of any systematic heterogeneity in the cohort-time
treatment effects across the \DIFdelbegin \DIFdel{different
respiratory outcomes }\DIFdelend \DIFaddbegin \DIFadd{outcomes of any symptom, coughing, chest
trouble, or trouble breathing, but the overall small \(ATT\)s for phlegm
and wheezing may be due to heterogeneity as the CHP reduced symptoms in
earlier treated cohorts but increased symptoms in later cohorts}\DIFaddend .

Table~\ref{tbl-did-health} also shows the impacts of the \DIFdelbegin \DIFdel{CBHP policy }\DIFdelend \DIFaddbegin \DIFadd{CHP }\DIFaddend on FeNO,
which was conducted in a sub-sample of 511 participants, including 274
participants with one measurement, 142 with two measurements, 95
participants with 3 measurements. We did not find evidence that the
policy affected changes in FeNO in the covariate-adjusted ETWFE
\DIFdelbegin \DIFdel{model
(0.5 }\DIFdelend \DIFaddbegin \comment{EC 9b}\DIFadd{model (0.3 }\DIFaddend ppb, 95\%CI: \DIFdelbegin \DIFdel{-2.1, 3.1}\DIFdelend \DIFaddbegin \DIFadd{-2.2, 2.8}\DIFaddend ). There was some
evidence of heterogeneity in the FeNO effects of the policy by treatment
cohort \DIFaddbegin \DIFadd{Appendix Figure~\ref{fig-feno-het}}\DIFaddend , though the confidence
intervals for each of the cohort-specific effects were \DIFdelbegin \DIFdel{large
and
overlapping(Appendix }\DIFdelend \DIFaddbegin \DIFadd{wide and
overlapping}\DIFaddend . Our results did not change with sensitivity analyses that
limited the analysis to participants with at least two repeated
measurements and to those who participated in all three waves (Appendix
Table~\ref{tbl-a-feno})

\DIFaddbegin \DIFadd{We also }\comment{EC 9a}\DIFadd{found limited evidence of an impact of the CHP on
markers of inflammation or oxidative stress. The basic DiD analyses
showed an increase of 1.0 (95\% CI: 0.1, 1.8) in TNF-\(\alpha\) but this
estimate was reduced to 0.8 (-0.1, 1.7) after adjustment for waist
circumference, occupation, wealth index quantile, frequency of drinking,
tobacco smoking, and frequency of farming. The adjusted estimates for
the effect of the policy on IL-6, CRP, and MDA were also generally small
and measured with limited precision.
}

\DIFaddend \subsubsection{Mediated impact on health
outcomes}\label{mediated-impact-on-health-outcomes}

As noted above, we aimed to assess whether any health impacts of the \DIFdelbegin \DIFdel{CBHP policy }\DIFdelend \DIFaddbegin \DIFadd{CHP
}\DIFaddend may work specifically through pathways involving changes in
PM\textsubscript{2.5} and indoor temperature. Below we show results from
several mediation models. We \DIFaddbegin \DIFadd{focus the mediation analysis on the BP and
respiratory outcomes for which we observed stronger evidence of total
effects of the policy. We }\DIFaddend evaluated potential mediation for each
mediator (indoor temperature and exposure to indoor
PM\textsubscript{2.5}) separately and in a single model accounting for
multiple mediators, and we set the values of both mediators to the mean
value for untreated participants at baseline (\DIFdelbegin \DIFdel{wave 1).
For mediation
analysis, we focused on BP outcomes for which we observed an effect
}\DIFdelend \DIFaddbegin \DIFadd{W1).
}

\DIFadd{In Table~\ref{tbl-bp-med} we show estimates of the \(CDE\)s for
different sets of potential mediators. }\comment{OC 2d}\DIFadd{The first column
of the first panel shows the covariate-adjusted total \(ATT\) of the CHP
on brachial SBP (i.e., a 1.4 mmHg decrease as seen in
Table~\ref{tbl-did-health} above). The second panel shows the \(CDE\),
i.e., the effect of exposure (vs.~no exposure) to the CHP on brachial
SBP in a counterfactual population where we intervene to fix the value
of indoor PM\textsubscript{2.5} to the average value for untreated
participants at baseline. The \(CDE\) is -0.8 mmHg (95\% CI -2.9, 1.3),
demonstrating that, under the assumptions outlined in Section 5.2,
roughly 40\% of the total effect of the CHP is mediated by the impact of
the policy on indoor PM\textsubscript{2.5}. The third panel shows the
estimated \(CDE\) when holding constant the value of indoor temp
(without simultaneous adjustment for PM\textsubscript{2.5}) to that of
the untreated participants. This \(CDE\) is even smaller (-0.3 mmHg,
95\%CI -2.2, 1.6) suggesting a somewhat stronger role for indoor
temperature in mediating the total effect of the CHP on brachial SBP.
Finally, the last panel shows a similar \(CDE\) }\DIFaddend of \DIFdelbegin \DIFdel{the policy. In }\DIFdelend \DIFaddbegin \DIFadd{0.3 mmHg (95\% CI:
-1.9, 2.5) when setting both indoor PM\textsubscript{2.5} and indoor
temperature to their respective pre-treatment means. Holding the values
of PM\textsubscript{2.5} and indoor temperature at pre-intervention
values effectively eliminates these pathways by which the CHP can effect
BP, so the small value of the \(CDE\) adjusting for both mediators
suggests that the CHP effect on BP would be effectively null were it not
for its impact on the mediators. Overall the results in
}\DIFaddend Table~\ref{tbl-bp-med} \DIFdelbegin \DIFdel{we show }\DIFdelend \DIFaddbegin \DIFadd{indicate }\DIFaddend that conditioning on indoor
PM\textsubscript{2.5} and indoor temperature largely \DIFdelbegin \DIFdel{explains }\DIFdelend \DIFaddbegin \DIFadd{explain }\DIFaddend the entire
total effect of the \DIFdelbegin \DIFdel{CBHP policy }\DIFdelend \DIFaddbegin \DIFadd{CHP }\DIFaddend on blood pressure for systolic BP, as the
\DIFdelbegin \DIFdel{CDE }\DIFdelend \DIFaddbegin \DIFadd{\(CDE\) }\DIFaddend conditional on both mediators was reduced to 0.03 \DIFdelbegin \DIFdel{(95\%CI: -2.0, 2.0) }\DIFdelend for brachial
SBP. The \DIFdelbegin \DIFdel{CDE for diastolic BP was
}\DIFdelend \DIFaddbegin \DIFadd{\(CDE\)s for brachial and central DBP were }\DIFaddend roughly half the
value of the total effect. \DIFaddbegin \DIFadd{Appendix Table~\ref{tbl-a-bp-med-het} shows
heterogeneous treatment effects for the mediation models for SBP and DBP
and are generally consistent with the patterns of mediation for the
overall \(CDE\)s.
}\DIFaddend 

\DIFaddbegin \begin{table}

\caption{\label{tbl-bp-med}\DIFaddFL{Controlled direct effects for the CHP on
blood pressure.}}

\centering{

\centering
\begin{talltblr}[         %% tabularray outer open
entry=none,label=none,
note{}={Note: Results combined across 30 multiply-imputed datasets (average of 3082 observations per dataset). ATT = Average Treatment Effect on the Treated, CDE = Controlled Direct Effect, CI = Confidence Interval, DBP = Diastolic blood pressure, PM = Particulate matter, SBP = Systolic blood pressure.},
note{a}={Adjusted for age, sex, waist circumference, smoking, alcohol consumption, and use of blood pressure medication.},
note{b}={Mediators were set to the mean value for untreated participants at baseline.},
]                     %% tabularray outer close
{                     %% tabularray inner open
colspec={Q[]Q[]Q[]Q[]Q[]Q[]Q[]Q[]Q[]},
cell{2}{4}={c=2,}{halign=c,},
cell{2}{6}={c=2,}{halign=c,},
cell{2}{8}={c=2,}{halign=c,},
cell{1}{2}={c=2,}{halign=c,},
cell{1}{4}={c=6,}{halign=c,},
column{1}={halign=l,},
column{2}={halign=c,},
column{3}={halign=c,},
column{4}={halign=c,},
column{5}={halign=c,},
column{6}={halign=c,},
column{7}={halign=c,},
column{8}={halign=c,},
column{9}={halign=c,},
column{1}={font=\fontsize{0.9em}{1.2em}\selectfont,},
column{2}={font=\fontsize{0.9em}{1.2em}\selectfont,},
column{3}={font=\fontsize{0.9em}{1.2em}\selectfont,},
column{4}={font=\fontsize{0.9em}{1.2em}\selectfont,},
column{5}={font=\fontsize{0.9em}{1.2em}\selectfont,},
column{6}={font=\fontsize{0.9em}{1.2em}\selectfont,},
column{7}={font=\fontsize{0.9em}{1.2em}\selectfont,},
column{8}={font=\fontsize{0.9em}{1.2em}\selectfont,},
column{9}={font=\fontsize{0.9em}{1.2em}\selectfont,},
}                     %% tabularray inner close
\toprule
& Adjusted Total Effect &  & CDE Mediated By: &  &  &  &  &  \\ \cmidrule[lr]{2-3}\cmidrule[lr]{4-9}
&  &  & Indoor PM &  & Indoor Temp &  & PM + Temp &  \\ \cmidrule[lr]{4-5}\cmidrule[lr]{6-7}\cmidrule[lr]{8-9}
& ATT\textsuperscript{a} & (95\%CI) & ATT\textsuperscript{b} & (95\%CI) & ATT\textsuperscript{b} & (95\%CI) & ATT\textsuperscript{b} & (95\%CI) \\ \midrule %% TinyTableHeader
Brachial SBP & -1.4 & (-3.3, 0.5) & -0.8 & (-2.9, 1.3) & -0.3 & (-2.2, 1.6) & 0.3 & (-1.9, 2.5) \\
Central SBP & -1.4 & (-3.3, 0.4) & -0.8 & (-2.9, 1.3) & -0.4 & (-2.2, 1.3) & 0.2 & (-1.9, 2.4) \\
Brachial DBP & -1.6 & (-2.9, -0.3) & -1.1 & (-2.7, 0.5) & -1.1 & (-2.3, 0.1) & -0.6 & (-2.1, 0.9) \\
Central DBP & -1.6 & (-2.9, -0.3) & -1.1 & (-2.7, 0.6) & -1.2 & (-2.4, -0.0) & -0.7 & (-2.2, 0.9) \\
\bottomrule
\end{talltblr}

}

\end{table}%DIF > 

\DIFaddend Table~\ref{tbl-resp-med} shows estimates from similar analyses for the
\DIFdelbegin \DIFdel{CDE }\DIFdelend \DIFaddbegin \DIFadd{\(CDE\) }\DIFaddend of the policy on respiratory outcomes. For respiratory outcomes
we focus on mediation by personal exposure to PM\textsubscript{2.5} and
point temperature and therefore these estimates are derived for the
subset of individuals with measures of personal exposure. Thus the total
adjusted \(ATT\)s in Table~\ref{tbl-resp-med} are not directly
comparable with those in Table~\ref{tbl-did-health}. We estimate the
\DIFdelbegin \DIFdel{CDEs }\DIFdelend \DIFaddbegin \DIFadd{\(CDE\)s }\DIFaddend holding the values of both mediators to the average levels for
never treated households at baseline. Overall we find no evidence that
any of the total effects we observed for self-reported respiratory
outcomes in Table~\ref{tbl-did-health} were mediated by personal
exposure to PM\textsubscript{2.5} or indoor temperature. Generally the
\DIFdelbegin \DIFdel{CDEs }\DIFdelend \DIFaddbegin \DIFadd{\(CDE\)s }\DIFaddend for all of the outcomes are statistically indistinguishable
from the total effects estimated without controlling for mediators.

\DIFaddbegin \begin{table}

\caption{\label{tbl-resp-med}\DIFaddFL{Controlled direct effects of the CHP on
self-reported respiratory outcomes}}

\centering{

\centering
\begin{talltblr}[         %% tabularray outer open
entry=none,label=none,
note{}={Note: Estimated for the subset of individuals with measured personal exposure (n=1270). ATT = Average Treatment Effect on the Treated, CDE = Controlled Direct Effect, CI = Confidence Interval.},
note{a}={Adjusted for age, gender, smoking, occupation, frequency of drinking, frequency of farming.},
note{b}={Mediators were set to the mean value for untreated participants at baseline.},
]                     %% tabularray outer close
{                     %% tabularray inner open
colspec={Q[]Q[]Q[]Q[]Q[]Q[]Q[]Q[]Q[]},
cell{2}{4}={c=2,}{halign=c,},
cell{2}{6}={c=2,}{halign=c,},
cell{2}{8}={c=2,}{halign=c,},
cell{1}{2}={c=2,}{halign=c,},
cell{1}{4}={c=6,}{halign=c,},
column{1}={halign=l,},
column{2}={halign=c,},
column{3}={halign=c,},
column{4}={halign=c,},
column{5}={halign=c,},
column{6}={halign=c,},
column{7}={halign=c,},
column{8}={halign=c,},
column{9}={halign=c,},
column{1}={font=\fontsize{0.8em}{1.1em}\selectfont,},
column{2}={font=\fontsize{0.8em}{1.1em}\selectfont,},
column{3}={font=\fontsize{0.8em}{1.1em}\selectfont,},
column{4}={font=\fontsize{0.8em}{1.1em}\selectfont,},
column{5}={font=\fontsize{0.8em}{1.1em}\selectfont,},
column{6}={font=\fontsize{0.8em}{1.1em}\selectfont,},
column{7}={font=\fontsize{0.8em}{1.1em}\selectfont,},
column{8}={font=\fontsize{0.8em}{1.1em}\selectfont,},
column{9}={font=\fontsize{0.8em}{1.1em}\selectfont,},
}                     %% tabularray inner close
\toprule
& Adjusted Total Effect &  & CDE Mediated By: &  &  &  &  &  \\ \cmidrule[lr]{2-3}\cmidrule[lr]{4-9}
&  &  & Personal PM &  & Indoor Temp &  & PM + Temp &  \\ \cmidrule[lr]{4-5}\cmidrule[lr]{6-7}\cmidrule[lr]{8-9}
& ATT\textsuperscript{a} & (95\%CI) & ATT\textsuperscript{b} & (95\%CI) & ATT\textsuperscript{b} & (95\%CI) & ATT\textsuperscript{b} & (95\%CI) \\ \midrule %% TinyTableHeader
Any symptom & -13.0 & (-20.5, -5.5) & -12.7 & (-20.7, -4.6) & -15.1 & (-23.3, -6.9) & -15.0 & (-23.7, -6.3) \\
Coughing & -10.5 & (-19.2, -1.8) & -11.9 & (-20.7, -3.1) & -14.5 & (-23.1, -5.8) & -14.7 & (-22.9, -6.6) \\
Phlegm & -9.5 & (-16.6, -2.4) & -8.8 & (-16.2, -1.4) & -14.7 & (-24.3, -5.1) & -13.3 & (-21.3, -5.4) \\
Wheezing attacks & -4.2 & (-11.1, 2.6) & -2.8 & (-8.7, 3.1) & -10.2 & (-22.0, 1.6) & -3.8 & (-10.0, 2.4) \\
Trouble breathing & -9.5 & (-19.3, 0.3) & -9.6 & (-19.6, 0.4) & -13.8 & (-26.3, -1.3) & -13.1 & (-25.2, -1.0) \\
Chest trouble & -5.0 & (-12.2, 2.1) & -4.9 & (-11.2, 1.4) & -4.6 & (-11.8, 2.5) & -3.0 & (-8.9, 2.9) \\
\bottomrule
\end{talltblr}

}

\end{table}%DIF > 

\DIFaddend \subsection{Aim 2: Source
contributions}\label{aim-2-source-contributions}

Source analysis for this study was conducted using data from all
eligible outdoor \DIFdelbegin \DIFdel{PM and personal PM }\DIFdelend \DIFaddbegin \DIFadd{and personal exposure PM\textsubscript{2.5} }\DIFaddend samples.
Eligible samples were those for which \DIFaddbegin \DIFadd{both }\DIFaddend PM\textsubscript{2.5} mass
and chemical components were quantified. \DIFaddbegin \DIFadd{Individual chemical species
concentrations (means and 95\% confidence intervals) for outdoor and
personal samples by study wave are provided in Appendix Tables
Table~\ref{tbl-species-outdoor} and Table~\ref{tbl-species-personal},
respectively. }\comment{EC 6}\DIFaddend We evaluated factors contributing to
community-outdoor and personal exposure \DIFdelbegin \DIFdel{to }\DIFdelend PM\textsubscript{2.5} using the
U.S. EPA's source apportionment model PMF (positive matrix
factorization) 5.0, which has been widely used for \DIFdelbegin \DIFdel{similar }\DIFdelend \DIFaddbegin \DIFadd{air pollution
}\DIFaddend analyses in China (Gao et al. \DIFdelbegin \DIFdel{~}\DIFdelend 2018; Liu et al. \DIFdelbegin \DIFdel{~}\DIFdelend 2017; Tao et al. \DIFdelbegin \DIFdel{~}\DIFdelend 2017).
As an optimum PMF result depends on the appropriate number of input
factors, sensitivity analysis using a range of factors (e.g., \DIFdelbegin \DIFdel{range of }\DIFdelend 3 to 7\DIFdelbegin \DIFdel{factors}\DIFdelend ,
based on a combination of the \DIFdelbegin \DIFdel{species that we have and our field-based observationsand sources that
have been identified previously }\DIFdelend \DIFaddbegin \DIFadd{measured chemical species, field
observations, and sources previously identified }\DIFaddend in our study region)
were conducted to examine the impact of a different number of factors on
the model results. Detailed information on the procedures of PMF
analysis can be found elsewhere (Wang et al. \DIFdelbegin \DIFdel{~}\DIFdelend 2016; Zíková et al. \DIFdelbegin \DIFdel{~}\DIFdelend 2016).
Briefly, the scree plot from our principal component analysis indicated
that solutions of between 3 and 5 factors (+/- 1) would be most
appropriate, further supporting our evaluation of 3 to 6 factor
solutions from PMF. As there was no indication that even moving from \DIFdelbegin \DIFdel{five to six }\DIFdelend \DIFaddbegin \DIFadd{5
to 6 }\DIFaddend factors would improve our solution\DIFdelbegin \DIFdel{; therefore, a seven factor solution would not make sense to investigate further }\DIFdelend \DIFaddbegin \DIFadd{, we did not further investigate
7 factors }\DIFaddend (Figure~\ref{fig-source-figure}).

\DIFaddbegin \subsubsection{\DIFadd{Source analysis using positive matrix
factorization}}\label{source-analysis-using-positive-matrix-factorization}

\DIFaddend The chemical analysis data used \DIFdelbegin \DIFdel{as the inputs for the }\DIFdelend \DIFaddbegin \DIFadd{in the }\DIFaddend PMF model were dispersion
normalized prior to \DIFdelbegin \DIFdel{inclusion in the model. PMF works by
using }\DIFdelend \DIFaddbegin \DIFadd{their inclusion. PMF uses the }\DIFaddend covariance of
compositional variables to separate sources of \DIFdelbegin \DIFdel{ambient }\DIFdelend PM. However, atmospheric
dilution also induces covariance. Dilution can be quantified in terms of
a ventilation coefficient (VC) and used to normalize the input chemical
concentrations and uncertainties in the original data matrix on a sample
by sample basis. The dispersion normalized concentrations and
uncertainties are used as the inputs to PMF analysis. Dispersion
normalization, \DIFaddbegin \comment{EC 6d}\DIFaddend as conducted in this study, is a
relatively new application of this conceptual framework (Dai et al.
2020), developed to adjust for wind speed (dispersion in the x-y plane)
and boundary layer height (dispersion in the z-axis). This process
involves first calculating the sample specific ventilation coefficient
by multiplying the average wind speed by the average boundary layer
height over the sampling duration. The average ventilation coefficient
is also calculated for the village by averaging all the ventilation
coefficients. The dispersion normalized concentration for any species in
any sample is equal to the species concentration in that sample
multiplied by the ventilation coefficient for that sample and divided by
the average ventilation coefficient for that village. Dividing by the
average ventilation coefficient for that village helps curtail any
extreme concentrations driven by an outlier in the sample ventilation
coefficient.

The meteorological data included hourly boundary layer height, 2-m
temperature, 2-m dew point temperature, and 2-m horizontal wind speed
components (u, v), which were obtained from the European Center for
Midrange Weather Forecasting ERA5 reanalysis dataset (0.25 x 0.25
resolution). Values of these meteorological variables were determined at
the village-level by identifying the four surrounding grid points with
values available from the ERA5 reanalysis, and then applying inverse
distance weighted interpolation from those four grid points to the
village. Percent relative humidity was calculated from the 2-m dew point
temperature using the ``weathermetrics'' package (version 1.2.2) in R
(Anderson et al. 2016). Total hourly wind speed and wind direction were
calculated from the horizontal wind speed components.

The model diagnostics \DIFdelbegin \DIFdel{for the three- to six-factor }\DIFdelend \DIFaddbegin \comment{EC 6e}\DIFadd{for the 3- to 6-factor }\DIFaddend PMF
solutions are \DIFdelbegin \DIFdel{given }\DIFdelend \DIFaddbegin \DIFadd{shown }\DIFaddend in Table~\ref{tbl-pmf}. Model fit was assessed using
\DIFdelbegin \DIFdel{Q/Qexp (how
}\DIFdelend \DIFaddbegin \DIFadd{a ratio of }\DIFaddend our model fit \DIFaddbegin \DIFadd{(Q) }\DIFaddend divided by the expected fit \DIFaddbegin \DIFadd{(Qexp}\DIFaddend ). As the
change in Q/Qexp \DIFdelbegin \DIFdel{decreases as more factorsare added}\DIFdelend \DIFaddbegin \DIFadd{decreased with more factors}\DIFaddend , the model may be fitting
additional sources that do not improve the overall fit. The largest
change in Q/Qexp was from \DIFdelbegin \DIFdel{three to four }\DIFdelend \DIFaddbegin \DIFadd{3 to 4 }\DIFaddend sources (6.24 to 5.37) while the
changes moving from \DIFdelbegin \DIFdel{four to five and five to six }\DIFdelend \DIFaddbegin \DIFadd{4 to 5 factors and 5 to 6 factors }\DIFaddend were similar,
which suggests that \DIFdelbegin \DIFdel{the four factors solution is sufficient }\DIFdelend \DIFaddbegin \DIFadd{4 factors are sufficient and parsimonious }\DIFaddend to explain
the variation in our data. We assessed the random error in our model by
randomly sampling blocks of data, fitting new models with the blocks,
and comparing how the source profiles compared \DIFdelbegin \DIFdel{to }\DIFdelend \DIFaddbegin \DIFadd{with }\DIFaddend the original model
(bootstrap \DIFdelbegin \DIFdel{(BS)
}\DIFdelend mapping). The \DIFdelbegin \DIFdel{three- and four-factor }\DIFdelend \DIFaddbegin \DIFadd{3- and 4-factor }\DIFaddend solutions had high \DIFdelbegin \DIFdel{BS }\DIFdelend \DIFaddbegin \DIFadd{bootstrap
}\DIFaddend mapping (all factors \DIFdelbegin \DIFdel{found in \textgreater{} 96.5}\DIFdelend \DIFaddbegin \DIFadd{identified in \textgreater96.5}\DIFaddend \% of bootstrap
runs). The additional sources identified in the \DIFdelbegin \DIFdel{five-factor }\DIFdelend \DIFaddbegin \DIFadd{5-factor }\DIFaddend (lead) and
six-factor (chloride) solutions had low \DIFdelbegin \DIFdel{BS }\DIFdelend \DIFaddbegin \DIFadd{bootstrap }\DIFaddend mapping
(\textgreater{} 72\%), \DIFdelbegin \DIFdel{which
means those solutions are not as consistent as the three- and four-factor }\DIFdelend \DIFaddbegin \DIFadd{indicating that these solutions are less
consistent than the 3- and 4-factor }\DIFaddend solutions. The possibility that
multiple \DIFdelbegin \DIFdel{, different,
}\DIFdelend solutions could result in the same Q value was assessed using
displacement. The displacement approach takes the original factor
profiles and modifies \DIFaddbegin \DIFadd{(+/-) }\DIFaddend the values for each species \DIFdelbegin \DIFdel{up or down }\DIFdelend to maintain a
small change in Q, reruns the solution with the new species values, and
then compares the profiles of the new model to the original. Any swaps
indicate that small changes \DIFdelbegin \DIFdel{in }\DIFdelend \DIFaddbegin \DIFadd{to }\DIFaddend the species values could result in factor
profiles that \DIFdelbegin \DIFdel{look }\DIFdelend \DIFaddbegin \DIFadd{are }\DIFaddend different from the original solution, \DIFdelbegin \DIFdel{and }\DIFdelend \DIFaddbegin \DIFadd{suggesting }\DIFaddend that
the original solution is unstable. None of the factors \DIFaddbegin \DIFadd{in any of the
solutions discussed }\DIFaddend were swapped during displacement, which indicates
that all of the potential solutions are stable. Based on the Q/Qexp,
\DIFdelbegin \DIFdel{BS }\DIFdelend \DIFaddbegin \DIFadd{bootstrap }\DIFaddend mapping, and interpretability of the factors, the \DIFdelbegin \DIFdel{four-factor }\DIFdelend \DIFaddbegin \DIFadd{4-factor
}\DIFaddend solution was selected as \DIFdelbegin \DIFdel{the most appropriate source solution }\DIFdelend \DIFaddbegin \DIFadd{most appropriate }\DIFaddend for the data.

\DIFaddbegin \begin{table}

\caption{\label{tbl-pmf}\DIFaddFL{PMF error estimation diagnostics.}}

\centering{

\centering
\begin{tblr}[         %% tabularray outer open
]                     %% tabularray outer close
{                     %% tabularray inner open
width={1\linewidth},
colspec={X[0.142857142857143]X[0.214285714285714]X[0.214285714285714]X[0.214285714285714]X[0.214285714285714]},
cell{1}{2}={c=4,}{halign=c,},
column{1}={halign=c,},
column{2}={halign=c,},
column{3}={halign=c,},
column{4}={halign=c,},
column{5}={halign=c,},
row{2}={halign=c,},
cell{7}{1}={}{halign=l,},
cell{10}{1}={}{halign=l,},
cell{7}{2}={}{halign=l,},
cell{10}{2}={}{halign=l,},
cell{7}{3}={}{halign=l,},
cell{10}{3}={}{halign=l,},
cell{7}{4}={}{halign=l,},
cell{10}{4}={}{halign=l,},
cell{7}{5}={}{halign=l,},
cell{10}{5}={}{halign=l,},
}                     %% tabularray inner close
\toprule
& Potential Factor Solution &  &  &  \\ \cmidrule[lr]{2-5}
Diagnostic & 3 & 4 & 5 & 6 \\ \midrule %% TinyTableHeader
Qexp & 27936 & 26052 & 24168 & 22284 \\
Qtrue & 187681 & 147796 & 123236 & 100316 \\
Qrobust & 174407 & 139910 & 117082 & 95932.5 \\
Qr/Qexp & 6.24 & 5.37 & 4.84 & 4.3 \\
Q/Qexp > 6 & wi-Ca, ns-S, ws-Na, ws-Ca, Al, Cl, Pb & ns-S, Na, Al, Cl, Pb, Nitrate & Nitrate, ws-Na, Al, Chloride & Nitrate, ws-Na, Al \\
DISP \% dQ & <0.1\% & <0.1\% & <0.1\% & <0.1\% \\
DISP swaps & 0 & 0 & 0 & 0 \\
BS\_mapping & Dust- 98.5\% & Transported dust- 95\%, Dust- 96.5\%, Sulfur secondary- 97.5\%, Mixed combustion- 96.5\% & Transported dust- 86\%, Mixed combustion- 87\%, Dust- 86\%, Lead- 55\% & Transported dust- 84\%, Mixed combustion- 87.5\%, Dust- 81.5\%, Lead- 72\%, Chloride- 61.5\%, Sulfur secondary- 98.5\% \\
\bottomrule
\end{tblr}

}

\end{table}%DIF > 

\DIFaddend The source profiles for the four-factor solution are presented in
Figure~\ref{fig-source-figure}. \DIFaddbegin \DIFadd{We sought to develop a defensible source
analysis solution, which we found to have 4 factors and then identify
and name those sources, jointly informed by our field observations,
knowledge of local sources, and relevant previous studies. The pooled
PMF analysis, which combined outdoor and personal exposure samples, led
to a more robust factor solution due to the increased number of samples.
This pooled analysis determined that the optimum solution was a 4-factor
model, which identified the major sources as dust, transported dust,
secondary sulfur, and a combustion mixture of coal and biomass burning,
and possibly some tobacco smoking. The pooled approach was found to be
stable, with high bootstrap mapping and no factor swaps during the
displacement tests, indicating the reliability of the factor solutions.
}

\DIFadd{As sensitivity analyses }\comment{EC 6d}\DIFadd{, we additionally conducted a
disaggregated analysis (Appendix Figure~\ref{fig-pmf-sup}) where
personal and outdoor samples were analyzed separately. We observed some
differences in the number of factors and source attributions, however,
the core findings remained largely consistent with the pooled results.
For the personal exposure samples, the 3-factor solution identified
dust, transported dust, and mixed combustion (a combination of biomass
burning and secondary PM), while the 5-factor solution further split the
mixed combustion factor into distinct coal combustion and sulfur
secondary factors. Similarly, for outdoor samples, the 3-factor solution
identified mixed combustion, secondary PM, and dust, while the 5-factor
solution introduced transported dust and a refined characterization of
secondary species contributions. The primary sources of
pollution---dust, secondary sulfur, and mixed combustion---were
consistent across pooled and disaggregated analyses. Thus, while the
disaggregated analysis does provide additional granularity, it does not
fundamentally change the identification of the key pollution sources
that we observed in the pooled analysis.
}

\DIFadd{We also evaluated PMF results disaggregated }\comment{EC 6c, EC PMF-1}\DIFadd{by
day and by month (Appendix Figure~\ref{fig-pmf-sup-day} and
Figure~\ref{fig-pmf-sup-month}), where the results are further
color-coded by district. Due to yearly field campaign schedules, the
timing of sampling in villages and districts was correlated. Therefore,
this approach to source analysis does not yield results that allow us to
disentangle changes in sources over time, within season, since they also
potentially embed changes in sources across villages and districts in
this study.
}

\DIFadd{Thus, we concluded that the pooled analysis was the most parsimonious
and interpretable approach for explaining the major sources of
PM\textsubscript{2.5} in our study. The pooled results are both
representative and stable, making them the most appropriate for
addressing the study's primary research questions.
}

\subsubsection{\texorpdfstring{Description of PM\textsubscript{2.5}
sources
identified}{Description of PM2.5 sources identified}}\label{description-of-pm2.5-sources-identified}

\DIFaddend The first source was identified as dust\DIFdelbegin \DIFdel{based on }\DIFdelend \DIFaddbegin \DIFadd{, characterized by }\DIFaddend high
percentages of crustal elements like wi-Ca, Si, and wi-Mg. The second
source \DIFdelbegin \DIFdel{consisted of }\DIFdelend \DIFaddbegin \DIFadd{contained }\DIFaddend non-sulfate sulfur \DIFdelbegin \DIFdel{as well as }\DIFdelend \DIFaddbegin \DIFadd{and }\DIFaddend secondary inorganic ions
(ammonium, nitrate, and sulfate). Non-sulfate sulfur is a tracer for
primary coal combustion, while secondary inorganic ions indicate a
secondary source. \DIFdelbegin \DIFdel{Since }\DIFdelend \DIFaddbegin \DIFadd{Given the }\DIFaddend industrial coal burning \DIFdelbegin \DIFdel{is a source
of power generation }\DIFdelend in our study area,
\DIFdelbegin \DIFdel{it is likely that the second
source is a mixture of }\DIFdelend \DIFaddbegin \DIFadd{the secondary source likely combines }\DIFaddend primary and secondary emissions
\DIFdelbegin \DIFdel{that originate
}\DIFdelend from coal and other sulfurous fuel combustion. Additionally, the \DIFdelbegin \DIFdel{mean
source contribution of the second source is higher in outdoor than
personal
exposure measurements. Secondary formationoccurs outdoors in
the presence of sunlight, so higher outdoor concentrations compared to
personal exposure further support our naming the second source `sulfur
secondary'. }\DIFdelend \DIFaddbegin \DIFadd{higher
outdoor concentrations of the secondary source compared with personal
exposures support its identification as `sulfur secondary' due to its
sunlight-driven secondary formation. The factor named ``sulfur
secondary'' was intended to reflect the contribution of sulfur, as
measured by the XRF analysis, that was not associated with sulfate. Had
this contribution been coupled with other species associated with direct
air pollutant emissions from coal, and had those species not clustered
with any other factors, we may have named this source a ``household coal
combustion'' source. However, some species that are typical of direct
air emissions from coal combustion are also clustered with species that
are typical of direct air emissions from biomass burning, leading to
naming the factor as ``mixed combustion''. It is not surprising that
residential coal and biomass emissions were difficult to fully separate,
being unable to analyze samples for organic tracers.
}

\DIFaddend The third source had high percentages of ws-Ca nd Al, which in our study
region, has been found to be indicative of transported dust from dust
storms that can occur in the spring. While our samples were collected
during winter months only, it is possible that transported dust from
previous years still remained. The fourth source was characterized by
high percentages of tracers for both coal (OC, wi-K, chloride, Pb) and
biomass combustion (EC, ws-K). Coal and biomass combustion are
anticipated sources of PM\DIFaddbegin \DIFadd{\textsubscript{2.5} }\DIFaddend pollution in our study
setting, particularly from domestic cooking and heating activities, so
this source is likely a mixture of PM emitted from these two household
combustion sources. We extend the source profiles across the different
treatment cohorts in Figure~\ref{fig-source-season}.

\begin{figure}[H]

\centering{

\includegraphics[width=0.9\textwidth,height=\textheight]{images/source-figure.png}

}

\caption{\label{fig-source-figure}Source profiles for the 4-factor PMF
solution to the sum of elements, ions, elemental carbon, and organic
carbon for outdoor and personal PM\textsubscript{2.5} exposure
measurements. The lines separate the major contributing species to each
source\DIFaddbeginFL \DIFaddFL{.}\DIFaddendFL }

\end{figure}%

\begin{figure}[H]

\centering{

\includegraphics[width=0.8\textwidth,height=\textheight]{images/source-season2.png}

}

\caption{\label{fig-source-season}Arithmetic mean dispersion normalized
source contributions found from the 4-factor PMF solution for \textbf{A}
outdoor and \textbf{B} personal PM\textsubscript{2.5} exposure samples
by year the group received treatment.}

\end{figure}%

\DIFaddbegin \subsubsection{\DIFadd{Impact of policy on outdoor and personal exposure to the
mixed combustion
source}}\label{impact-of-policy-on-outdoor-and-personal-exposure-to-the-mixed-combustion-source}

\DIFadd{Overall, }\DIFaddend Table~\ref{tbl-source-did} shows \DIFaddbegin \DIFadd{that }\DIFaddend the average treatment
effect of the \DIFdelbegin \DIFdel{CBHP policy on community outdoor }\DIFdelend \DIFaddbegin \DIFadd{CHP on outdoor (community) }\DIFaddend levels and personal exposure
levels of the mixed combustion source was statistically
indistinguishable from the null. Treatment was associated with lower,
but statistically imprecise, personal exposures to the mixed combustion
source. As with \DIFdelbegin \DIFdel{BC, }\DIFdelend \DIFaddbegin \DIFadd{personal exposure to BC, an indicator of combustion
pollution, }\DIFaddend this finding is consistent with the expectation that the
policy contributed to \DIFdelbegin \DIFdel{reducing air pollutant emissions from solid fuel burning, as this
}\DIFdelend \DIFaddbegin \DIFadd{reduced solid fuel emissions, as the }\DIFaddend `mixed
combustion' source most likely reflects solid fuel combustion \DIFdelbegin \DIFdel{,
particularly }\DIFdelend in our
study \DIFdelbegin \DIFdel{settings}\DIFdelend \DIFaddbegin \DIFadd{setting}\DIFaddend . The results were consistent across \DIFdelbegin \DIFdel{both the unadjusted and
adjusted }\DIFdelend \DIFaddbegin \DIFadd{the basic and
covariate-adjusted }\DIFaddend models.

\DIFaddbegin \begin{table}

\caption{\label{tbl-source-did}\DIFaddFL{Average treatment effect
(µg/m\textsuperscript{3}) for outdoor and personal exposure to the mixed
combustion source.}}

\centering{

\centering
\begin{talltblr}[         %% tabularray outer open
entry=none,label=none,
note{}={Note: ATT = Average Treatment Effect on the Treated, CI = confidence interval, DiD = Difference-in-Differences.},
note{a}={Personal exposure model adjusted for temperature (represented by a spline with 2 degrees of freedom), participant smoking status, and household reported using biomass fuel. Outdoor model adjusted for the total number of households in the village, total village population, and ambient relative humidity (represented by a spline with 2 degrees of freedom).},
]                     %% tabularray outer close
{                     %% tabularray inner open
colspec={Q[]Q[]Q[]Q[]Q[]Q[]},
cell{1}{3}={c=2,}{halign=c,},
cell{1}{5}={c=2,}{halign=c,},
column{1}={halign=l,},
column{2}={halign=c,},
column{3}={halign=c,},
column{4}={halign=c,},
column{5}={halign=c,},
}                     %% tabularray inner close
\toprule
&  & DiD &  & Adjusted DiD &  \\ \cmidrule[lr]{3-4}\cmidrule[lr]{5-6}
& Obs & ATT & (95\% CI)\textsuperscript{a} & ATT & (95\% CI) \\ \midrule %% TinyTableHeader
Outdoor &  717 & 1.07 & (-4.90, 7.04) & 1.53 & (-4.19, 7.26) \\
Personal exposure & 1158 & -5.60 & (-13.70, 2.54) & -5.39 & (-13.1, 2.35) \\
\bottomrule
\end{talltblr}

}

\end{table}%DIF > 

\DIFaddend When the average treatment effects of the \DIFdelbegin \DIFdel{CBHP }\DIFdelend \DIFaddbegin \DIFadd{CHP }\DIFaddend policy on community
outdoor levels and personal exposure levels of the mixed combustion
source were allowed to vary by treatment year and time, the treatment
effect for households most recently treated (i.e., treated in the final
wave, \DIFdelbegin \DIFdel{w4}\DIFdelend \DIFaddbegin \DIFadd{W4}\DIFaddend ) was associated with lower personal exposures to the mixed
combustion source (Appendix Figure~\ref{fig-afig-mixed-ct}). In each
wave, treatment by the \DIFdelbegin \DIFdel{CBHP policy }\DIFdelend \DIFaddbegin \DIFadd{CHP }\DIFaddend was associated with a reduction in the source
contribution to personal PM\textsubscript{2.5} mass from the mixed
combustion source; however, for villages treated in \DIFdelbegin \DIFdel{w2 and w3}\DIFdelend \DIFaddbegin \DIFadd{W2 and W3}\DIFaddend , the
effect was statistically imprecise. Treatment was not associated with a
reduction or an increase in the source contribution to community outdoor
PM\textsubscript{2.5} mass from the mixed combustion source. Personal
exposure measures of this specific air pollution source were found to be
more indicative of treatment effect than community outdoor measures of
the same source. This finding aligns with the expectation that the \DIFaddbegin \DIFadd{mixed
}\DIFaddend pollution source, identified as a mixture of coal and biomass
combustion, is characteristic of household use of solid fuels. These
fuels, including coal and biomass, produce emissions that are likely to
be closer to the people using them rather than near the centrally
located community outdoor air samplers.

\subsection{Aim 3: Mediation by source
contribution}\label{aim-3-mediation-by-source-contribution}

Table~\ref{tbl-med-source} shows results from the mediation analysis by
personal exposure to the mixed combustion source (coal and biomass),
estimated for the subset of participants with personal exposure
measurements. The \DIFdelbegin \DIFdel{CDE }\DIFdelend \DIFaddbegin \DIFadd{\(CDE\) }\DIFaddend in this model estimates the impact of exposure
to the \DIFdelbegin \DIFdel{CBHP policy }\DIFdelend \DIFaddbegin \DIFadd{CHP }\DIFaddend on central and systolic blood pressure while holding constant
values of mixed combustion source at the mean baseline values for
untreated population. The marginal policy effects (\DIFdelbegin \DIFdel{ATTs}\DIFdelend \DIFaddbegin \DIFadd{\(ATT\)s}\DIFaddend ) from the
adjusted ETWFE models for this subset of participants were largely
similar to those from the full sample for central SBP (around a 1.6 mmHg
decrease), but slightly smaller for central DBP (\DIFdelbegin \DIFdel{-1.7 }\DIFdelend \DIFaddbegin \DIFadd{-1.6 }\DIFaddend mmHg in the full
sample vs.~\DIFdelbegin \DIFdel{-1.3 }\DIFdelend \DIFaddbegin \DIFadd{-0.9 }\DIFaddend mmHg in the subset with personal exposure measurements)
and were estimated with greater imprecision. We found little evidence
that these treatment effects were meaningfully mediated by exposure to
the mixed combustion source, as the controlled direct effects were
generally of similar magnitude as the adjusted total effects.

\DIFaddbegin \begin{table}

\caption{\label{tbl-med-source}\DIFaddFL{Average treatment effects and controlled
direct effect (mm/Hg) of the CHP on central systolic and diastolic blood
pressure with mixed combustion source as the potential mediator.}}

\centering{

\centering
\begin{talltblr}[         %% tabularray outer open
entry=none,label=none,
note{}={Note: ATT = Average Treatment Effect on the Treated, CI = confidence interval, DiD = Difference-in-Differences, CDE = Controlled Direct Effect, DBP = Diastolic Blood Pressure, SBP = Systolic Blood Pressure.},
note{a}={Adjusted for age, sex, waist circumference, smoking, alcohol consumption, and use of blood pressure medication.},
note{b}={Further adjusted for mediation by mixed combustion source (coal and biomass)},
]                     %% tabularray outer close
{                     %% tabularray inner open
colspec={Q[]Q[]Q[]Q[]Q[]Q[]Q[]Q[]},
cell{1}{3}={c=2,}{halign=c,},
cell{1}{5}={c=2,}{halign=c,},
cell{1}{7}={c=2,}{halign=c,},
column{1}={halign=l,},
column{2}={halign=c,},
column{3}={halign=c,},
column{4}={halign=c,},
column{5}={halign=c,},
column{6}={halign=c,},
column{7}={halign=c,},
column{8}={halign=c,},
}                     %% tabularray inner close
\toprule
&  & DiD &  & Adjusted DiD &  & Adjusted CDE &  \\ \cmidrule[lr]{3-4}\cmidrule[lr]{5-6}\cmidrule[lr]{7-8}
& Obs & ATT & (95\% CI)\textsuperscript{a} & ATT & (95\% CI)\textsuperscript{b} & ATT & (95\% CI) \\ \midrule %% TinyTableHeader
Central SBP & 942 & -1.6 & (-5.3, 2.2) & -1.5 & (-5.0, 2.0) & -1.5 & (-5.0, 2.1) \\
Central DBP & 942 & -0.9 & (-3.3, 1.5) & -1.1 & (-3.2, 0.9) & -1.3 & (-3.6, 0.9) \\
\bottomrule
\end{talltblr}

}

\end{table}%DIF > 

\DIFaddend \section{Discussion and Conclusions}\label{discussion-and-conclusions}

Air pollution emitted from residential space heating with coal has
historically been a major contributor to cardio-respiratory disease
burden in northern China (Archer-Nicholls et al. 2016; Yun et al. 2020).
Since the introduction of its 13\textsuperscript{th} 5-Year-Plan
(2016-2020), China has successfully implemented numerous large-scale
measures to improve air quality including programs that incentivize
rural household transition from solid fuels to clean energy sources
(Young et al. 2015). The \DIFdelbegin \DIFdel{CBHP policy }\DIFdelend \DIFaddbegin \DIFadd{CHP }\DIFaddend is among the largest and most ambitious
household energy policies implemented anywhere in the world in recent
decades, and its staggered roll-out provided a unique opportunity to
prospectively evaluate this real-world experiment and its effects on air
quality and health.

\subsection{Adoption of the heat pump technology and adherence to the
policy}\label{adoption-of-the-heat-pump-technology-and-adherence-to-the-policy}

The \DIFdelbegin \DIFdel{CBHP policy }\DIFdelend \DIFaddbegin \DIFadd{CHP }\DIFaddend was successful in driving a rapid household heating energy
transition from coal stoves to electric heat pumps in the treated study
villages, with little difference in coal stove suspension or heat pump
adoption for those treated before versus during the COVID-19 pandemic.
There was high uptake and consistent use of the new heat pump
technology\DIFdelbegin \DIFdel{and }\DIFdelend \DIFaddbegin \DIFadd{, as well as }\DIFaddend large reductions in coal use in treated villages
starting in the first year post-treatment and continuing into the third
year of treatment for the villages first treated in 2019. We enrolled
rural and peri-urban villages across a wide geographic area and
socioeconomic spectrum in Beijing and observed near universal adoption
of the heat pump technologies and suspension of coal stove use across
the different treatment groups and waves. This contrasts with many
previous household energy intervention studies, including several
randomized trials, where low fidelity and compliance with the
intervention stoves were considered major limitations to achieving their
intended air quality or health benefits (Ezzati and Baumgartner 2017;
Lai et al. 2024; Rosenthal et al. 2018).

A number of factors \DIFdelbegin \DIFdel{contribute }\DIFdelend \DIFaddbegin \DIFadd{contributed }\DIFaddend to the successful uptake of the new
technology and adherence to the policy. The initial uptake of the heat
pump technology was influenced by broad support and perceived benefits
of village and household participation in the policy. At baseline
assessment, 49 of 50 village committee interviewees indicated a desire
to participate in the policy by the committee members and their
constituents, for reasons including the ease of use of the heat pump,
the convenience of no longer having to add coal throughout the day and
especially the night, the desire for a cleaner local environment, and a
perceived lower risk of carbon monoxide poisoning without coal stoves
(data not reported). While the availability and cost of clean fuels are
well-established barriers to their adoption and sustained use over time
(Rehfuess et al. 2014), in our study, both the upfront costs of the heat
pump technology and a portion of electricity use were subsidized by the
government, which limited the financial burden of clean energy
transition for households. Further, after policy implementation, treated
villages no longer had access to government-subsidized coal, and
household coal burning was further discouraged with possible punitive
measures (e.g., potential loss of electricity subsidies).

\subsection{Impacts of the policy on
health}\label{impacts-of-the-policy-on-health}

One of the key findings from our comprehensive evaluation of the \DIFdelbegin \DIFdel{CBHP
policy }\DIFdelend \DIFaddbegin \DIFadd{CHP }\DIFaddend was
that exposure to the policy reduced systolic and diastolic blood
pressure by \textasciitilde1.5 mmHg, and that most of the observed BP
effects were mediated by improvements in the indoor environment,
specifically reductions in indoor PM\textsubscript{2.5} and increases in
indoor temperature. The total effects of the policy are consistent with
a small number of randomized trials of gas cookstoves or more efficient
biomass cookstoves showing reductions in blood pressure of similar
magnitudes (Kumar et al. 2021). In contrast, recent randomized trials of
liquefied petroleum gas (LPG) stoves in multiple countries observed no
effect or a small (\textasciitilde0.6 mmHg) increase in blood pressure
(Checkley et al. 2021; Ye et al. 2022) despite large decreases in
personal exposures to PM\textsubscript{2.5} and black carbon. The
inconsistency between our results and the LPG stove trial may stem from
large differences in age (mean ages of 25y and 48y in the trials versus
61y in our sample) and that gas stoves can still emit health-damaging
air pollutants including \DIFdelbegin \DIFdel{benzine }\DIFdelend \DIFaddbegin \DIFadd{benzene }\DIFaddend and volatile organic compounds (Kashtan
et al. 2023), especially in contrast with the zero-emission
electric-powered heat pumps introduced to our study villages. Our
findings of temperature- and air quality-mediated impacts of the policy
on BP are also supported by observational studies showing that increased
exposure to household air pollution (Baumgartner et al. 2018, 2011; Dong
et al. 2013; Kanagasabai et al. 2022) and to colder indoor temperatures
(Lv et al. 2022; Sternbach et al. 2022) are associated with higher blood
pressure in rural and peri-urban areas of China, with exposure-response
estimates that reasonably align with our estimates of the policy impact
on BP after conditioning on temperature and PM\textsubscript{2.5} in the
mediation analysis.

We did not observe effects of the policy on measures of PP or cPP/SBP
amplification. Pulse pressure is measured as the difference between SBP
and DBP, and represents the pulsatile component of blood flow (Dart and
Kingwell 2001). Thus, increases in PP can result from increases in SBP,
decreases in DBP, or both. The lack of effect on PP in our study is
attributed to the similar reductions in SBP and DBP from the policy.
Similarly, PP/SBP amplification is measured as a ratio of peripheral to
central pressures, and the decreases in central and brachial pressures
with the policy were also nearly identical in our study. Although the
duration of our study was nearly twice as long as most previous
household stove intervention studies conducted over two years or less,
it is still possible that even longer-term reductions in BP are required
to observe any structural changes in the caliber or elasticity of
arterial walls that would subsequently be reflected in differences in PP
or SBP/PP amplification (Dart and Kingwell 2001).

Our study also contributes to the limited evidence that transition from
solid fuel to clean energy can reduce the self-report of symptoms
consistent with chronic respiratory tract irritation. Exposure to the
\DIFdelbegin \DIFdel{CBHP policy }\DIFdelend \DIFaddbegin \DIFadd{CHP }\DIFaddend reduced self-report of any chronic respiratory symptoms \DIFdelbegin \DIFdel{(\textasciitilde7 percentage points) }\DIFdelend \DIFaddbegin \DIFadd{by roughly 7
percentage points, }\DIFaddend with most of these effects driven by reductions in
self-reported chest trouble or difficulty breathing on several or most
days of the week. These findings align with previous randomized trials
in Guatemala and Mexico, where biomass chimney stove interventions
lowered indoor carbon monoxide and reduced the self-reported prevalence
of chronic respiratory symptoms, especially \DIFdelbegin \DIFdel{cough and wheeze}\DIFdelend \DIFaddbegin \DIFadd{coughing and wheezing}\DIFaddend , in
younger women after 12 and 18 months of intervention (Romieu et al.
2009; Smith-Sivertsen et al. 2009). In contrast \DIFdelbegin \DIFdel{our our }\DIFdelend \DIFaddbegin \DIFadd{to our }\DIFaddend findings,
introduction of a solar cooker in Senegal provided no benefit to air
pollution or self-reported respiratory symptoms (Beltramo and Levine
2013), and a recent trial of gas stoves in Peru similarly found no
reduction in respiratory symptoms within the year of intervention
despite very large reductions in PM\textsubscript{2.5} (Checkley et al.
2021).
\DIFdelbegin \DIFdel{It is possible that our
longer-term (4-yr) study
}\DIFdelend 

We did not, however, find evidence that the reductions in chronic
respiratory symptoms were mediated by changes in personal exposure to
PM\textsubscript{2.5} or indoor temperature. This is not particularly
unexpected since we did not observe an effect of the policy on personal
exposure to PM\textsubscript{2.5} and any impacts of temperature on
chronic respiratory symptoms would more likely arise from large, rapid
changes in temperature (D'Amato et al. 2018) whereas we observed small,
gradual changes in our study. Future work will consider mediation by
seasonal indoor PM\textsubscript{2.5}, which is a longer-term measure of
`usual' air pollution than 24-h personal exposure and was shown to be
reduced by the policy in our study homes.

We found some evidence \DIFaddbegin \comment{EC 11}\DIFaddend of heterogeneity in the health
benefits of the policy by treatment cohort. Generally the policy showed
strong reductions for the villages treated early and weak or \DIFdelbegin \DIFdel{even small
}\DIFdelend \DIFaddbegin \DIFadd{null
}\DIFaddend increases in BP for the last 3 villages exposed to the policy in 2021.
We found less evidence for heterogeneity for self-reported respiratory
symptoms, but did note potential evidence for increases in self-reported
\DIFaddbegin \DIFadd{phlegm and }\DIFaddend wheezing attacks with the policy in the three villages
treated in 2021. Notably this was also the treatment cohort with the
smallest improvement in point temperature at the time of BP measurement
and an increase in self-reported biomass use. Paradoxically, we observed
a larger decrease in PM\textsubscript{2.5} and mixed solid fuel use in
this group. It is possible that the composition of PM and mixed solid
fuel was different in this cohort, with a greater contribution of
biomass smoke, however we are unable to differentiate between biomass
and coal in our `mixed solid fuel\DIFdelbegin \DIFdel{combustion}\DIFdelend ' category. \DIFdelbegin \DIFdel{This group was }\DIFdelend \DIFaddbegin \DIFadd{Notably, this group of
villages were }\DIFaddend also treated during the \DIFaddbegin \DIFadd{COVID-19 }\DIFaddend pandemic, which could
have impacted how the policy was introduced \DIFdelbegin \DIFdel{or
}\DIFdelend \DIFaddbegin \DIFadd{in unpredictable and
difficult-to-measure ways, which could have }\DIFaddend resulted in changes to other
\DIFdelbegin \DIFdel{BP }\DIFdelend \DIFaddbegin \DIFadd{health }\DIFaddend risk factors that we did not evaluate in our study, e.g., changes
in \DIFdelbegin \DIFdel{diet}\DIFdelend \DIFaddbegin \DIFadd{dietary intake}\DIFaddend .

We also found little evidence of an impact of the policy on blood
biomarkers of inflammation and oxidative stress in the sub-sample of
participants with blood collection in waves 1 and 2, but these were
estimated with imprecision. Our results contrast with a natural
experiment in urban Beijing that showed large regional and local air
quality reductions during the 2008 Beijing Olympics and also observed
benefits to airway inflammation (Huang et al. 2012) and blood markers of
inflammation and oxidative stress in healthy urban Beijing residents
during the Olympics compared with before and after (Rich et al. 2012).
Our mediation analysis \DIFaddbegin \comment{OC 2b}\DIFaddend indicated that the blood pressure
effects of the policy were mediated \DIFaddbegin \DIFadd{through both indoor temperature and
air pollution, and the effects of the policy on SBP were mediated }\DIFaddend more
through indoor temperature than air pollution. Although observational
studies from rural northern China show impacts of exposure to
temperature on inflammation and oxidative stress (Wang et al. 2020; Xu
et al. 2019), it's possible that the relatively small increases in mean
indoor temperature in treated households we observed were not
sufficiently large to capture measurable changes in these biomarkers.

\subsection{Impacts of the policy on air pollution and its
sources}\label{impacts-of-the-policy-on-air-pollution-and-its-sources}

\DIFdelbegin \DIFdel{China has a long history of launching ambitious, large-scale policies
and programs to promote clean household energy transition and support
rural energy infrastructure development (Zhang and Smith 2007). It was a
relatively early initiator of rural electrification projects in the
1950s and achieved complete (100\%) electrification of households by
2016 (Yang 2021), which undoubtedly facilitated the policy option to
replace coal stoves with electric-powered heat pump heaters. Several
decades later, China achieved what is likely still the largest
improvement in household energy efficiency in history with regards to
the population affected by a single program: the National Improved Stove
Program (NISP) and its provincial counterparts were initiated in the
early 1980s and are credited with introducing nearly 200 million
improved cooking and heating stoves by the late 1990s. All NISP stoves
had chimneys and some had manual or electric blazers to promote more
efficient combustion (Zhang and Smith 2007), with the primary goal of
increased biomass fuel efficiency to reduce pressure on local forests
and a secondary goal of improving indoor air quality (Sinton et al.
2004). Because NISP focused mainly on biomass stoves for cooking, it had
limited impacts on the rapid increase in coal heating stove installation
during that period, most of which were implemented without chimneys
(Zhang and Smith 2007).
}%DIFDELCMD < 

%DIFDELCMD < %%%
\DIFdel{In contrast to NISP, the primary }\DIFdelend \DIFaddbegin \DIFadd{The primary }\DIFaddend aim of the \DIFdelbegin \DIFdel{CBHP policy }\DIFdelend \DIFaddbegin \DIFadd{CHP }\DIFaddend was to reduce air pollution emissions and
improve regional air quality, and it specifically targeted coal-burning
stoves in northern China. Our evaluation of the \DIFdelbegin \DIFdel{CBHP policy }\DIFdelend \DIFaddbegin \DIFadd{CHP }\DIFaddend indicates a
substantial improvement in indoor air quality, with a reduction of
roughly \DIFdelbegin \DIFdel{-36
}\DIFdelend \DIFaddbegin \DIFadd{-20 }\DIFaddend µg/m\textsuperscript{3} (95\%CI: \DIFdelbegin \DIFdel{-61, -12}\DIFdelend \DIFaddbegin \DIFadd{-38, -3}\DIFaddend ) in wintertime
indoor PM\textsubscript{2.5} \DIFdelbegin \DIFdel{resulting in geometric mean indoor }\DIFdelend \DIFaddbegin \DIFadd{levels. Still there is considerable room
for indoor air quality improvement given that the indoor
}\DIFaddend PM\textsubscript{2.5} levels \DIFdelbegin \DIFdel{of }\DIFdelend \DIFaddbegin \DIFadd{in treated households in W4 (GM=}\DIFaddend 49
µg/m\textsuperscript{3}\DIFdelbegin \DIFdel{(}\DIFdelend \DIFaddbegin \DIFadd{, }\DIFaddend 95\%CI: \DIFdelbegin \DIFdel{43, 57) in treated households in w4. At baseline, indoor
PM\textsubscript{2.5} levels in treated households were above the WHO's
IT-1 (annual and 24-h means: 35 and 75 }\DIFdelend \DIFaddbegin \DIFadd{42, 52) were still \textasciitilde10
times higher than the annual WHO annual air quality guideline (5
}\DIFaddend µg/m\textsuperscript{3}) \DIFdelbegin \DIFdel{, which
is the first stepping stone in high-pollution areas towards achieving
cleaner air }\DIFdelend (World Health Organization 2021).
\DIFdelbegin \DIFdel{Following the policy,
indoor PM\textsubscript{2.5} levels were approaching and in some cases
even reaching the IT-1 guideline value.
}\DIFdelend 

Similar to our indoor results, several recent randomized trials with
high compliance (exclusive or near exclusive) in the use of LPG stoves
in rural settings with low outdoor pollution in Peru, Ghana, Guatemala,
India, and Rwanda (Checkley et al. 2021; Chillrud et al. 2021; Katz et
al. 2020) (Johnson et al. 2022) found lower exposures to
PM\textsubscript{2.5} (32-69\%) in the intervention group compared with
controls using traditional solid fuel stoves, but even in these
relatively low pollution settings, post-intervention mean exposures
(range: 24 to 52 µg/m\textsuperscript{3} ) were still well-above the
WHO's annual air quality guideline (5 µg/m\textsuperscript{3})\DIFdelbegin \DIFdel{and, like
our study, more aligned with IT-1 and IT-2}\DIFdelend . A trial
in urban and peri-urban Nigeria, a high pollution setting more similar
to our Beijing sites, did not observe an air pollution benefit of
ethanol stoves but did observe improved birth and pregnancy outcomes and
blood pressure (Alexander et al. 2018; Alexander et al. 2017).

Nonetheless, comparisons of the indoor PM\textsubscript{2.5} benefits of
the \DIFdelbegin \DIFdel{CBHP policy }\DIFdelend \DIFaddbegin \DIFadd{CHP }\DIFaddend in our study with previous assessments of household energy
interventions in China suggest that the \DIFdelbegin \DIFdel{CBHP policy }\DIFdelend \DIFaddbegin \DIFadd{CHP }\DIFaddend performed well. Homes with
the NISP biomass chimney stoves had modestly lower indoor
PM\DIFaddbegin \DIFadd{\textsubscript{4} }\DIFaddend than traditional open fire stoves (\DIFdelbegin \DIFdel{293 versus }\DIFdelend 223 \DIFaddbegin \DIFadd{versus 293
}\DIFaddend µg/m\textsuperscript{3}), though post-intervention air pollution levels
were still an order of magnitude higher than the current
health-motivated WHO (24-h) guideline (Sinton et al. 2004). The NISP's
so-called ``improved'' coal heating stoves unexpectedly emitted higher
concentrations of PM\DIFaddbegin \DIFadd{\textsubscript{4} }\DIFaddend and carbon monoxide than the
traditional coal stoves (Edwards et al. 2004). In southwestern China
(Sichuan), a difference-in-differences analysis of an
government-supported household energy package pilot (semi-gasifier
cookstove, water heater, pelletized biomass fuel) observed decreased
indoor PM\textsubscript{2.5} (24--67\%) in women treated by the energy
package, but greater reductions (48--70\%) were observed in untreated
women, a result likely influenced by an unexpectedly large transition in
gas cookstoves in untreated homes during the study period (Baumgartner
et al. 2019).

The relatively high post-policy indoor air pollution levels and the
limited benefit to personal exposures and outdoor PM\textsubscript{2.5}
in treated villages in our study---despite excellent compliance with the
policy---is likely due to three key factors. First, a quarter of our
study households had at least one tobacco smoker, which is a large
contributor to personal exposures to PM\textsubscript{2.5} in our study
settings, especially during wintertime when people tend to spend more
time indoors (Li et al. 2022). Second, although has Beijing rapidly and
impressively reduced outdoor PM\textsubscript{2.5} over the last decade
(annual mean of 89 \DIFdelbegin \DIFdel{μg}\DIFdelend \DIFaddbegin \DIFadd{µg}\DIFaddend /m\textsuperscript{3} in 2013 decreased to 30
\DIFdelbegin \DIFdel{μg}\DIFdelend \DIFaddbegin \DIFadd{µg}\DIFaddend /m\textsuperscript{3} in 2022) (Zhang et al. 2023), the wintertime
outdoor PM\textsubscript{2.5} levels in the treated study villages
remained high enough across the study waves (range of means: 26-38
\DIFdelbegin \DIFdel{μg}\DIFdelend \DIFaddbegin \DIFadd{µg}\DIFaddend /m\textsuperscript{3} in treated villages) to limit the minimum
exposure achievable with an indoor stove. The contribution of outdoor
sources to personal exposures is further supported by our source
apportionment analyses, which showed a clear contribution of regional
sources (secondary sulfur, transported dust) to personal exposures.
Finally, the continued use of biomass-burning kangs likely also
contributed to indoor PM\textsubscript{2.5} and personal exposures.
Kangs are a relatively simple and culturally entrenched combined cooking
and space heating technique that have been used in China for over two
thousand years (Zhuang et al. 2009). Kangs are mostly \DIFdelbegin \DIFdel{fuelled }\DIFdelend \DIFaddbegin \DIFadd{fueled }\DIFaddend by wood or
other biomass that is freely and widely available in our study villages.
The \DIFdelbegin \DIFdel{CBHP policy }\DIFdelend \DIFaddbegin \DIFadd{CHP }\DIFaddend did not ban biomass burning, and we observed persistent
self-reported use of kangs after heat pump installation. Continued use
of traditional solid fuel stoves alongside cleaner stoves and fuels
(i.e., stove stacking) has long been a barrier to achieving large
reductions in indoor and personal exposures after intervention (Shankar
et al. 2020). A notable exception is the HAPIN trial which attained near
exclusive use of LPG stoves and dramatic reductions in personal
exposures to PM\textsubscript{2.5} (lowered by 66\% compared with
controls, 70 versus 24 \DIFdelbegin \DIFdel{μg}\DIFdelend \DIFaddbegin \DIFadd{µg}\DIFaddend /m\textsuperscript{3}) (Johnson et al. 2022),
though the impressive air quality improvements were not accompanied by
any health benefits across a range of neonatal, child, and maternal
outcomes (Lai et al. 2024).

To comprehensively evaluate a large-scale policy like the \DIFdelbegin \DIFdel{CBHP policy}\DIFdelend \DIFaddbegin \DIFadd{CHP}\DIFaddend , our
study's measurement approach required extensive long-term measurements
in \textgreater1000 households in multiple waves using over 500 air
pollution monitors that collected thousands of hours of measurements.
The scale and duration of air pollution measurement achieved in this
study would not have been possible without low-cost air pollution
sensors that have proliferated in the past decade. Our use of low-cost
sensors to capture long-term (5-6 months) indoor air quality data in
rural settings places it at the forefront of applying cutting-edge
technology to understand and mitigate household air pollution. This
approach is somewhat unique for China as most studies, including those
using lower-cost air pollution sensing networks (Chao et al. 2021; Mei
et al. 2020), focus on urban air quality, driven by consideration for
urban population demographic changes and industrial, power generation,
and vehicular emissions (Shen et al. 2017). By focusing on rural indoor
environments, our study addresses a crucial gap, offering insights into
the effectiveness of a specific policy (\DIFdelbegin \DIFdel{CBHP}\DIFdelend \DIFaddbegin \DIFadd{CHP}\DIFaddend ) on a micro-scale. Future
evaluations of household energy interventions might also consider
longitudinal measures of air pollution that track changes over longer
periods to capture delayed effects. Estimating the causal effects of the
\DIFdelbegin \DIFdel{CBHP policy }\DIFdelend \DIFaddbegin \DIFadd{CHP }\DIFaddend required a multifaceted approach to evaluation that incorporated a
study design (difference-in-differences) and analytical methods (ETWFE,
causal mediation) that are less common for evaluating air quality
interventions. By incorporating a broader array of metrics and
considering the systemic nature of air pollution and its health impacts,
through this study, we sought to provide a more nuanced understanding of
an intervention's effectiveness and the ways in which it may need to be
augmented or restructured to achieve desired health outcomes.

\subsection{Assumptions, strengths, and
limitations}\label{assumptions-strengths-and-limitations}

The validity of our DiD approach is subject to two key assumptions
(Callaway and Sant'Anna 2021; Wooldridge 2021). First, no anticipation:
we assume that anticipation of the \DIFdelbegin \DIFdel{CBHP policy }\DIFdelend \DIFaddbegin \DIFadd{CHP }\DIFaddend did not affect outcomes prior to
policy implementation and did not differ between treated and untreated
villages. We selected villages that were eligible for the policy but not
currently treated. It was generally understood that the policy would
first be implemented in the plains areas with more updated electric
grids and then gradually expand into more remote and mountainous areas
of Beijing, though most of our study villages were far from Beijing's
urban core. In addition to these geographical parameters, some of our
study villages were assigned to the policy whereas others applied to the
local government, but they were generally unaware of if or when they
would be treated at the time of enrollment. Second, parallel trends: our
analysis assumes that in the absence of the policy the trends in air
quality and health in treated and untreated villages would have remained
the same over time. Because the parallel trends assumption is based on a
counterfactual it cannot be empirically verified (similar to the
assumption of no unmeasured confounding)\DIFdelbegin \DIFdel{;
however, we observed similar }\DIFdelend \DIFaddbegin \DIFadd{. However, }\comment{EC 4a}\DIFadd{given
that we had two }\DIFaddend pre-intervention \DIFdelbegin \DIFdel{trends in BP outcomes
between waves 1 and 2 in never treated villages and those treated later
in waves 3 and 4 (see Appendix Figure~\ref{fig-afig-pt3} and Figure~\ref{fig-afig-pt4}). We }\DIFdelend \DIFaddbegin \DIFadd{periods prior to the 2020 and 2021
cohorts being treated, we assessed the similarity of pre-intervention
trends between W1 and W2 for the never treated group and the groups
eventually treated in 2020 and 2021. Estimates and 95\% CIs for the
difference in pre-trends are given in the Appendix for BP outcomes
(Figure~\ref{fig-afig-pt-bp}), personal exposure to
PM\textsubscript{2.5} and black carbon (Figure~\ref{fig-afig-pt-pe-bc}),
and self-reported respiratory outcomes (Figure~\ref{fig-afig-pt-resp}).
We did not find strong evidence for systematic differences in the
pre-policy trends for any outcome, but some estimates were imprecise and
tests for pre-trends are generally considered to have low power (Roth
2022). We }\DIFaddend also adjusted for relevant time-varying confounders in
estimating total effects and in mediation analyses, which aims to
improve the credibility of parallel trends assumption.

We also note that, in general, the addition of covariates to our
``basic'' ETWFE models did not \DIFdelbegin \DIFdel{appreciably }\DIFdelend \DIFaddbegin \DIFadd{meaningfully }\DIFaddend change our estimates.
Nevertheless, we cannot entirely rule out the possibility that other
programs or policies differentially affected air quality or health in
treated and untreated villages, which could lead to over- or
under-estimation of its effects. To investigate this possibility we
surveyed village leaders about other rural development or health
policies and programs in their villages throughout our four-year study
period and did not identify any co-implemented programs that would
differentially impact villages by treatment status and affect outcomes
or mediators. \DIFaddbegin \DIFadd{Though }\comment{EC 1}\DIFadd{a number of municipality- and
district-level COVID-19-related preventative measures were implemented
during the pandemic (e.g., lockdowns, travel restrictions), our study
villages were not differentially exposed to any such measures. }\DIFaddend Finally,
our mediation analysis assumes no residual time-varying confounding
between our mediators (air pollution and temperature) and our health
outcomes. Although we measured and evaluated a large number of
time-varying risk factors for BP, we cannot entirely eliminate the
possibility of potential residual confounding which could over-or
under-estimate the mediating effects of indoor environmental factors.

Strengths of this comprehensive, field-based assessment of the \DIFdelbegin \DIFdel{CBHP
policy }\DIFdelend \DIFaddbegin \DIFadd{CHP
}\DIFaddend include our quasi-experimental design to evaluate a real-world clean
energy intervention that would be near impossible to experimentally
manipulate at the scale of our study. Our study design controlled for
secular changes in health and we additionally collected data on and
adjusted for important time-varying covariates. It's perhaps worth
noting that control for secular trends was important in our context. For
personal exposure, supplementary analyses that dropped the time fixed
effects or compared treated vs.~untreated villages with covariate
adjustment would have suggested a substantial impact of the policy
(Appendix Table~\ref{tbl-a-fe}). Our numerous sensitivity analyses
showed the robustness of our findings to various analytic decisions.
Most previous field-based household energy intervention studies were
less than two-years in duration with a single post-treatment wave (Lai
et al. 2024; Quansah et al. 2017), and our four-year study enabled
longer-term evaluation of compliance with the coal ban and heat pump
adoption/use and their impacts on air pollution and health. Despite the
logistical challenges of COVID-19 pandemic shutdowns and related
government restrictions that occurred throughout half of our study
period, we were able to continue the study and successfully retain
participants in all 50 study villages over four years. Our large sample
size of 1000 participants in 50 villages across multiple study waves
enabled us to evaluate both the total effects of the policy and
separately for different treatment cohorts. By comparison, the few
previous field-based assessments of household energy interventions
(trials and pre-post designs with controls) and blood pressure ranged in
size from 44 to 324 participants (Kumar et al. 2021; Lai et al. 2024;
Onakomaiya et al. 2019), with exception of HAPIN trial that enrolled
\textasciitilde3000 pregnant women (Ye et al. 2022).

This study also has several limitations to consider when interpreting
our results. First, the COVID-19 pandemic began in the middle of our
study\DIFdelbegin \DIFdel{and roughly }\DIFdelend \DIFaddbegin \DIFadd{. Although }\comment{EC 1}\DIFadd{there were no recorded COVID-19 infections
in our study villages during the study period, Beijing's
municipality-wide preventative measures impacted all study villages at
the same time (e.g., travel restrictions, closure of public spaces,
lockdowns, and quarantines). Roughly }\DIFaddend half of our treated villages
entered into the policy during the pandemic, which likely had some
influence on \DIFdelbegin \DIFdel{the roll-out of
the policy in those villages}\DIFdelend \DIFaddbegin \DIFadd{its roll out}\DIFaddend . We observed the largest benefits in BP and
several respiratory outcomes in villages treated before the pandemic
\DIFaddbegin \DIFadd{compared with those treated after it started}\DIFaddend . However, we cannot
differentiate between treatment cohort effects attributable treatment
during the COVID-19 pandemic versus other factors that different between
treatment cohorts (i.e., geographic location, access to biomass, fuel
prices).

Second, the \DIFdelbegin \DIFdel{CBHP policy }\DIFdelend \DIFaddbegin \DIFadd{CHP }\DIFaddend roll-out began in 2016 but we did not begin enrolling
villages into our study until 2018. Thus, many of our study villages are
farther from the urban core and generally of lower socioeconomic status
than many villages treated in the first three years of the policy.
Previous studies of the \DIFdelbegin \DIFdel{CBHP policy }\DIFdelend \DIFaddbegin \DIFadd{CHP }\DIFaddend suggest that treated villages of all
socioeconomic levels benefited from less-polluted and warmer indoor
environments, but that the benefits were larger in wealthier villages
that were more likely to use the heat pumps more often and set to a
higher indoor temperature (Barrington-Leigh et al. 2019; Meng et al.
2023). Further, most of our study villages had relatively easy access to
(free) biomass fuel, and may be more likely to use biomass-burning kangs
compared with villages near the urban core where biomass fuel is less
readily available. Thus our results may not be generalizable to all of
rural and peri-urban Beijing, especially to the more urbanized,
wealthier villages treated between 2016 and 2018, and may underestimate
the impacts of the policy on indoor environmental factors that were
important cardio-respiratory health mediators in our study.

Third, like any field-based study, we had a number of constraints with
data collection. We were unable to measure indoor air quality in \DIFdelbegin \DIFdel{w1 }\DIFdelend \DIFaddbegin \DIFadd{W1 }\DIFaddend due
to logistical and budget constraints, and thus cannot directly estimate
the effects of the policy on indoor PM\textsubscript{2.5} for the 10
villages treated in 2019, which is also the treatment cohort that
experienced the greatest health benefits. Similarly, we were unable to
collect blood samples in the last wave because all of our measurements
were conducted in participant homes rather than clinics to avoid group
contact during the pandemic. In addition, our study logistics required
visiting 50 villages over a period of just several months. Thus, we were
unable to return to villages if a previously enrolled participant was
not at home at the time that staff visited the village. In such
instances, we either randomly selected either another eligible
participant in the same home or we randomly selected another household
with eligible participants from the village roster and our study
participants differed slightly across waves. Though this is unlikely to
impact our findings since our village-level study and analysis is robust
to participation of a random sample of participants in each wave, and
there were \DIFdelbegin \DIFdel{no }\DIFdelend \DIFaddbegin \DIFadd{few }\DIFaddend notable differences in key demographic characteristics or
health behaviors between participants who contributed to a different
number of waves or between participants across each of the three waves
\DIFaddbegin \DIFadd{that included individual-level measurements}\DIFaddend .

Fourth, respiratory symptoms were self-reported and thus our estimated
effects on respiratory symptoms must be interpreted with caution.
Participants in our study were aware of their treatment status, and
knowledge of being treated by a policy aimed at reducing local air
pollution may have affected their reporting of the perceived health
benefits of intervention (Peel et al. 2015). Previous trials of improved
biomass stoves, for example, noted a tendency of participants to report
favorable response to the stove regardless of its physiologic efficacy
(Burwen and Levine 2012; Smith-Sivertsen et al. 2009). Such reporting
bias could have inaccurately increased the estimated effect of the
policy, though we did not observe consistent effects across all
respiratory outcomes or treatment cohorts, which one might expect if
treated participants were inclined to give more favorable responses. Our
study also benefited from co-measurement of BP, an objective measure
that is less likely to be biased by participant or staff awareness of
treatment status because our staff consistently followed strict quality
control guidelines for measurement across all study villages and homes,
regardless of treatment status.

\DIFaddbegin \DIFadd{Finally, }\comment{EC 7}\DIFadd{some remarks concerning power and uncertainty. We
do not distinguish our results based on traditional notions of
``statistical significance'' (Wasserstein et al. 2019), but given the
large number of outcomes and wide range of reported estimates and
uncertainties, questions about the power of our design are relevant.
Using observed effect sizes to ask questions about power (sometimes
called `post-hoc power calculations') is illogical given the direct
relationship between power and observed p-values or confidence limits
(Hoenig and Heisey 2001). Instead we aim to put our estimates in context
using a retrospective design analysis (Gelman and Carlin 2014).
Retrospective design analyses use plausible values for hypothetical
effect sizes to ask about the strength of evidence a replicated study
with our design and level of precision would be likely to provide. The
main quantities for a design analysis include the probability that the
replicated estimate would be ``statistically significant'' (power), the
probability that the replicated estimate would have the wrong sign
(``S-bias''), and the ratio of the replicated estimate divided by the
true effect size (the ``exaggeration ratio'' or ``M-bias''). We report
these values for a range of conservative, but plausible, effect sizes
below.
}

\DIFadd{For blood pressure we used a value of a 2.5 mmHg decrease in systolic or
a 2 mmHg decrease in diastolic (Baumgartner et al. 2011; Steenland et
al. 2018). For self-reported respiratory outcomes we assumed
hypothetical effects on the order of a 10\% decline for each outcome
(translated into absolute percentage point declines), and for
inflammatory markers we hypothesized true effects of a roughly 5\%
decline (Pope III et al. 2004; Tang et al. 2020), again translated into
absolute terms.
}

\DIFadd{The full results for all health outcomes are shown in the Appendix
(Table~\ref{tbl-rd}). As an example, consider our estimate of the 7.5
percentage point decrease in any respiratory symptom (95\% CI 2.3,
12.7). Assuming that the true effect of the policy was a reduction of 5
percentage points, a replicated estimate with our design features and
precision would have 47\% power, virtually no chance of reporting the
wrong sign (S-bias = 0), and in expectation the estimated effect will be
only 1.4 times too high. Similarly, if the true effect of the policy on
brachial systolic BP was a small reduction of 2.5 mmHg, a replication of
our study design would have roughly 73\% power, a 2\% chance of having
the wrong sign, and an average exaggeration ratio of just 0.5, meaning
it would be unlikely to exaggerate the true effect.
}

\DIFadd{The greater precision for blood pressure and respiratory outcomes lends
greater confidence to these estimated impacts. On the other hand, our
estimates of the impact on inflammatory markers }\comment{EC 9a}\DIFadd{contain
greater uncertainty (larger SEs), largely due to logistical constraints
that prevented data collection in the last wave. For example, our
adjusted \(ATT\) for the effect of the CHP on IL-6 was an increase of
0.8 pg/mL but our data are compatible with effects as small as a 0.3
pg/mL decrease and as large as a 2.0 pg/mL increase. Table~\ref{tbl-rd}
shows that if the true effect of the CHP on IL-6 is a modest 0.2 pg/mL
decline, a replication using our design and level of precision will have
roughly 6\% power, a 39\% chance of having the wrong sign, and the
estimated effect will on average be nearly 2.5 times too high. Because
we chose generally conservative values for hypothetical effects, design
analyses with larger hypothetical true effects would tend to show
greater power, lower likelihood of reporting the wrong sign, and less
exaggerated effects.
}

\DIFadd{Because of the inherent uncertainty surrounding hypothesized ``true''
effects, it often makes sense to conduct retrospective design analyses
for a range of effect sizes. We show estimates for BP reductions of 0 to
4 mmHg in Appendix Figure~\ref{fig-rd-bp}. The HEI-funded objectives of
this study were designed based on pre-study power calculations for blood
pressure reductions of roughly 0.25 standard deviations (roughly 4 mmHg
for older rural population with an average SD of 16 mmHg). Appendix
Figure~\ref{fig-rd-bp} shows that if the true effect of the policy were
similar to our pre-study estimates of 0.25 SD difference in systolic BP,
a study with our level of precision would have \textgreater90\% power,
virtually no chance of reporting an estimate with the wrong sign, and an
average exaggeration ratio of just 1.04. This suggests that our design
was well-powered to detect clinically meaningful effects (Rahimi et al.
2021) on blood pressure.
}

\DIFaddend \section{Implications of Findings}\label{implications-of-findings}

In this comprehensive field-based assessment of \DIFdelbegin \DIFdel{a real-world CBHP policy
}\DIFdelend \DIFaddbegin \DIFadd{the rural Clean Heating
Policy }\DIFaddend in Beijing, we observed high fidelity and compliance with the
policy in our study villages and households where nearly all households
in treated villages stopped using coal and shifted to electric-powered
heaters. Exposure to the policy reduced blood pressure and self-reported
chronic respiratory symptoms, and \DIFdelbegin \DIFdel{the effects }\DIFdelend \DIFaddbegin \DIFadd{these health benefits }\DIFaddend were mediated by
reductions in indoor PM\textsubscript{2.5} and improvements in home
temperature. We did not observe the same benefits of the policy on
outdoor air quality or personal exposures, likely because the relatively
high contribution of other regional and local air pollution sources to
outdoor and personal exposures may have masked the benefits from a
single source reduction. We also did not observe benefits of the policy
on different measures of inflammation and oxidative stress in the
sub-sample of participants with biomarker assessment, even though we
observed respiratory symptoms and BP benefits of the policy in a
sensitivity analysis limited to the same participants. \DIFaddbegin \DIFadd{Still,
}\comment{EC 9d}\DIFadd{our overall findings indicate that this ambitious policy
achieved its goals in dramatically reducing residential coal burning and
improving indoor environmental quality, which provided modest benefits
to health.
}\DIFaddend 

Our results showing an indoor \DIFdelbegin \DIFdel{air quality }\DIFdelend \DIFaddbegin \DIFadd{environment }\DIFaddend and cardio-respiratory health
benefit of a real-world, large-scale clean energy policy are timely, as
they are synchronous with ongoing and planned clean energy policies in
China and other countries in a global effort to ``ensure access to
affordable, reliable, sustainable, and modern energy for all''
(Sustainable Development Goal-7) and directly respond to a recent
call-to-action from global cardiovascular societies that emphasized the
urgent need for interventional studies that inform targeted
pollution-reducing strategies to reduce cardiovascular disease (Brauer
et al. 2021).

\section{Data Availability Statement}\label{data-availability-statement}

The de-identified data, code, documentation, and study resources
including the standard operating procedures for all study measurements
are openly available on the Open Science Foundation (OSF)
\href{https://osf.io/8twds/?view_only=c41dd3d6228240d6aad92f81371c5339}{platform}.

\section{Acknowledgements}\label{acknowledgements}

In addition to the HEI funding that supported this work, we also
acknowledge support from the Canadian Institutes for Health Research
(CIHR \#159477) and the Social Sciences and Humanities Research Council
(SSHRC \#430-2017-00998 and \#435-2016-0531). \DIFaddbegin \DIFadd{HEI funding
}\comment{EC 13}\DIFadd{supported the addition of indoor temperature and indoor
air quality measurements starting in wave 2 to wave 4 and also fully
supported the data collection campaigns in waves 3 and 4. }\DIFaddend None of these
funders had any role in study design, data collection and analysis,
decision to publish, or preparation of this report.

We would like to thank and acknowledge the over 1400 study participants
and the over 50 field staff members who assisted with data collection
and laboratory analysis. This work would not have been possible without
the dedicated efforts and contributions of investigators and trainees
who contributed to the development of ideas, data collection, and
results that are reported here and in publications resulting from this
work. Here, we wish to acknowledge Koren Mann, \DIFdelbegin \DIFdel{Brian Robinson, Chris
Barrington-Leigh, }\DIFdelend Arijit Nandi, Robert
Platt, \DIFdelbegin \DIFdel{Talia Sternbach, Xiang
Zhang, Wenlu Yuan, Collin Brehmer, }\DIFdelend Kennedy Hirst, Enkhuun Byambadorj, Kaibing Xue, \DIFdelbegin \DIFdel{Siobhan Carroll, }\DIFdelend and Martha Lee.
We also acknowledge the efforts of project coordinators in Canada and
China: \DIFaddbegin \DIFadd{Xinwei Liu, Jing Shang, Xiaoxia Hu, Jian Ma, }\DIFaddend Leona Siaw, Neha
Ahmed, and Laojie Li.

\section{References}\label{references}

\phantomsection\label{refs}
\begin{CSLReferences}{1}{1}
\bibitem[\citeproctext]{ref-ahmed2009}
Ahmed T, Dutkiewicz VA, Shareef A, Tuncel G, Tuncel S, Husain L. 2009.
Measurement of black carbon ({BC}) by an optical method and a
thermal-optical method: {Intercomparison} for four sites. Atmospheric
Environment 43:6305--6311;
doi:\href{https://doi.org/10.1016/j.atmosenv.2009.09.031}{10.1016/j.atmosenv.2009.09.031}.

\bibitem[\citeproctext]{ref-alexander2018}
Alexander DA, Northcross A, Karrison T, Morhasson-Bello O, Wilson N,
Atalabi OM, et al. 2018. Pregnancy outcomes and ethanol cook stove
intervention: {A} randomized-controlled trial in {Ibadan}, {Nigeria}.
Environment International 111:152--163;
doi:\href{https://doi.org/10.1016/j.envint.2017.11.021}{10.1016/j.envint.2017.11.021}.

\bibitem[\citeproctext]{ref-alexander2017}
Alexander D, Northcross A, Wilson N, Dutta A, Pandya R, Ibigbami T, et
al. 2017. Randomized {Controlled Ethanol Cookstove Intervention} and
{Blood Pressure} in {Pregnant Nigerian Women}. American Journal of
Respiratory and Critical Care Medicine 195:1629--1639;
doi:\href{https://doi.org/10.1164/rccm.201606-1177OC}{10.1164/rccm.201606-1177OC}.

\bibitem[\citeproctext]{ref-an2021}
An L, Hong B, Cui X, Geng Y, Ma X. 2021. Outdoor thermal comfort during
winter in {China}'s cold regions: {A} comparative study. Science of The
Total Environment 768:144464;
doi:\href{https://doi.org/10.1016/j.scitotenv.2020.144464}{10.1016/j.scitotenv.2020.144464}.

\bibitem[\citeproctext]{ref-anderson2016}
Anderson GB, Peng RD, Ferreri JM. 2016. Weathermetrics: {Functions} to
{Convert Between Weather Metrics}.

\bibitem[\citeproctext]{ref-archer-nicholls2016}
Archer-Nicholls S, Carter E, Kumar R, Xiao Q, Liu Y, Frostad J, et al.
2016. The {Regional Impacts} of {Cooking} and {Heating Emissions} on
{Ambient Air Quality} and {Disease Burden} in {China}. Environmental
Science \& Technology 50:9416--9423;
doi:\href{https://doi.org/10.1021/acs.est.6b02533}{10.1021/acs.est.6b02533}.

\bibitem[\citeproctext]{ref-arel-bundock2024}
Arel-Bundock V. 2024. Marginaleffects: {Predictions}, {Comparisons},
{Slopes}, {Marginal Means}, and {Hypothesis Tests}.

\bibitem[\citeproctext]{ref-baker2022}
Baker AC, Larcker DF, Wang CCY. 2022. How much should we trust staggered
difference-in-differences estimates? Journal of Financial Economics
144:370--395;
doi:\href{https://doi.org/10.1016/j.jfineco.2022.01.004}{10.1016/j.jfineco.2022.01.004}.

\bibitem[\citeproctext]{ref-barrington-leigh2019}
Barrington-Leigh C, Baumgartner J, Carter E, Robinson BE, Tao S, Zhang
Y. 2019. An evaluation of air quality, home heating and well-being under
{Beijing}'s programme to eliminate household coal use. Nature Energy
4:416--423;
doi:\href{https://doi.org/10.1038/s41560-019-0386-2}{10.1038/s41560-019-0386-2}.

\bibitem[\citeproctext]{ref-baumgartner2018}
Baumgartner J, Carter E, Schauer JJ, Ezzati M, Daskalopoulou SS, Valois
M-F, et al. 2018. Household air pollution and measures of blood
pressure, arterial stiffness and central haemodynamics. Heart
104:1515--1521;
doi:\href{https://doi.org/10.1136/heartjnl-2017-312595}{10.1136/heartjnl-2017-312595}.

\bibitem[\citeproctext]{ref-baumgartner2019}
Baumgartner J, Clark S, Carter E, Lai A, Zhang Y, Shan M, et al. 2019.
Effectiveness of a {Household Energy Package} in {Improving Indoor Air
Quality} and {Reducing Personal Exposures} in {Rural China}.
Environmental Science \& Technology 53:9306--9316;
doi:\href{https://doi.org/10.1021/acs.est.9b02061}{10.1021/acs.est.9b02061}.

\bibitem[\citeproctext]{ref-baumgartner2011}
Baumgartner J, Schauer JJ, Ezzati M, Lu L, Cheng C, Patz JA, et al.
2011. Indoor {Air Pollution} and {Blood Pressure} in {Adult Women
Living} in {Rural China}. Environmental Health Perspectives
119:1390--1395;
doi:\href{https://doi.org/10.1289/ehp.1003371}{10.1289/ehp.1003371}.

\bibitem[\citeproctext]{ref-beltramo2013}
Beltramo T, Levine DI. 2013. The effect of solar ovens on fuel use,
emissions and health: Results from a randomised controlled trial.
Journal of Development Effectiveness 5:178--207;
doi:\href{https://doi.org/10.1080/19439342.2013.775177}{10.1080/19439342.2013.775177}.

\bibitem[\citeproctext]{ref-brauer2021}
Brauer M, Casadei B, Harrington RA, Kovacs R, Sliwa K, the WHF Air
Pollution Expert Group. 2021. Taking a {Stand Against Air
Pollution}---{The Impact} on {Cardiovascular Disease}: {A Joint Opinion
From} the {World Heart Federation}, {American College} of {Cardiology},
{American Heart Association}, and the {European Society} of
{Cardiology}. Circulation 143;
doi:\href{https://doi.org/10.1161/CIRCULATIONAHA.120.052666}{10.1161/CIRCULATIONAHA.120.052666}.

\bibitem[\citeproctext]{ref-burwen2012}
Burwen J, Levine DI. 2012. A rapid assessment randomized-controlled
trial of improved cookstoves in rural {Ghana}. Energy for Sustainable
Development 16:328--338;
doi:\href{https://doi.org/10.1016/j.esd.2012.04.001}{10.1016/j.esd.2012.04.001}.

\bibitem[\citeproctext]{ref-callaway2020}
Callaway B. 2020.
\href{https://doi.org/10.1007/978-3-319-57365-6_352-1}{Difference-in-{Differences}
for {Policy Evaluation}}. In: \emph{Handbook of {Labor}, {Human
Resources} and {Population Economics}} (K.F. Zimmermann, ed). Springer
International Publishing:Cham. 1--61.

\bibitem[\citeproctext]{ref-callaway2021}
Callaway B, Sant'Anna PHC. 2021. Difference-in-{Differences} with
multiple time periods. Journal of Econometrics 225:200--230;
doi:\href{https://doi.org/10.1016/j.jeconom.2020.12.001}{10.1016/j.jeconom.2020.12.001}.

\bibitem[\citeproctext]{ref-cameron2015}
Cameron AC, Miller DL. 2015. A practitioner's guide to cluster-robust
inference. Journal of Human Resources 50: 317--372.

\bibitem[\citeproctext]{ref-card1994}
Card D, Krueger AB. 1994. Minimum {Wages} and {Employment}: {A Case
Study} of the {Fast-Food Industry} in {New Jersey} and {Pennsylvania}.
American Economic Review 84: 772--93.

\bibitem[\citeproctext]{ref-chao2021}
Chao C-Y, Zhang H, Hammer M, Zhan Y, Kenney D, Martin RV, et al. 2021.
Integrating {Fixed Monitoring Systems} with {Low-Cost Sensors} to
{Create High-Resolution Air Quality Maps} for the {Northern China Plain
Region}. ACS Earth and Space Chemistry 5:3022--3035;
doi:\href{https://doi.org/10.1021/acsearthspacechem.1c00174}{10.1021/acsearthspacechem.1c00174}.

\bibitem[\citeproctext]{ref-checkley2021}
Checkley W, Williams KN, Kephart JL, Fandiño-Del-Rio M, Steenland NK,
Gonzales GF, et al. 2021. Effects of a {Household Air Pollution
Intervention} with {Liquefied Petroleum Gas} on {Cardiopulmonary
Outcomes} in {Peru}. {A Randomized Controlled Trial}. American Journal
of Respiratory and Critical Care Medicine 203:1386--1397;
doi:\href{https://doi.org/10.1164/rccm.202006-2319OC}{10.1164/rccm.202006-2319OC}.

\bibitem[\citeproctext]{ref-chillrud2021}
Chillrud SN, Ae-Ngibise KA, Gould CF, Owusu-Agyei S, Mujtaba M, Manu G,
et al. 2021. The effect of clean cooking interventions on mother and
child personal exposure to air pollution: Results from the {Ghana
Randomized Air Pollution} and {Health Study} ({GRAPHS}). Journal of
Exposure Science \& Environmental Epidemiology 31:683--698;
doi:\href{https://doi.org/10.1038/s41370-021-00309-5}{10.1038/s41370-021-00309-5}.

\bibitem[\citeproctext]{ref-clark2017}
Clark S, Carter E, Shan M, Ni K, Niu H, Tseng JTW, et al. 2017. Adoption
and use of a semi-gasifier cooking and water heating stove and fuel
intervention in the {Tibetan Plateau}, {China}. Environmental Research
Letters 12:075004;
doi:\href{https://doi.org/10.1088/1748-9326/aa751e}{10.1088/1748-9326/aa751e}.

\bibitem[\citeproctext]{ref-costello2015}
Costello BT, Schultz MG, Black JA, Sharman JE. 2015. Evaluation of a
{Brachial Cuff} and {Suprasystolic Waveform Algorithm Method} to
{Noninvasively Derive Central Blood Pressure}. American Journal of
Hypertension 28:480--486;
doi:\href{https://doi.org/10.1093/ajh/hpu163}{10.1093/ajh/hpu163}.

\bibitem[\citeproctext]{ref-damato2018}
D'Amato M, Molino A, Calabrese G, Cecchi L, Annesi-Maesano I, D'Amato G.
2018. The impact of cold on the respiratory tract and its consequences
to respiratory health. Clinical and Translational Allergy 8:20;
doi:\href{https://doi.org/10.1186/s13601-018-0208-9}{10.1186/s13601-018-0208-9}.

\bibitem[\citeproctext]{ref-dai2020}
Dai Q, Liu B, Bi X, Wu J, Liang D, Zhang Y, et al. 2020. Dispersion
{Normalized PMF Provides Insights} into the {Significant Changes} in
{Source Contributions} to {PM} {\textsubscript{2.5}} after the {COVID-19
Outbreak}. Environmental Science \& Technology 54:9917--9927;
doi:\href{https://doi.org/10.1021/acs.est.0c02776}{10.1021/acs.est.0c02776}.

\bibitem[\citeproctext]{ref-danesh2008}
Danesh J, Kaptoge S, Mann AG, Sarwar N, Wood A, Angleman SB, et al.
2008. Long-term interleukin-6 levels and subsequent risk of coronary
heart disease: Two new prospective studies and a systematic review. PLoS
medicine 5:e78;
doi:\href{https://doi.org/10.1371/journal.pmed.0050078}{10.1371/journal.pmed.0050078}.

\bibitem[\citeproctext]{ref-dart2001}
Dart AM, Kingwell BA. 2001. Pulse pressure---a review of mechanisms and
clinical relevance. Journal of the American College of Cardiology
37:975--984;
doi:\href{https://doi.org/10.1016/S0735-1097(01)01108-1}{10.1016/S0735-1097(01)01108-1}.

\bibitem[\citeproctext]{ref-cdcgr2023}
Dispersed Coal Management Research Group
北京大学能源研究院气候变化与能源转型项目. 2023. 中国散煤综合治理研究报告
{China Dispersed Coal Governance Report}.

\bibitem[\citeproctext]{ref-dockery2013}
Dockery DW, Rich DQ, Goodman PG, Clancy L, Ohman-Strickland P, George P,
et al. 2013. \href{https://www.ncbi.nlm.nih.gov/pubmed/24024358}{Effect
of air pollution control on mortality and hospital admissions in
{Ireland}}. Research Report (Health Effects Institute) 3--109.

\bibitem[\citeproctext]{ref-dominici2014}
Dominici F, Greenstone M, Sunstein CR. 2014. Science and regulation.
{Particulate} matter matters. Science (New York, NY) 344:257--9;
doi:\href{https://doi.org/10.1126/science.1247348}{10.1126/science.1247348}.

\bibitem[\citeproctext]{ref-dong2013}
Dong G-H, Qian Z(Min), Xaverius PK, Trevathan E, Maalouf S, Parker J, et
al. 2013. Association {Between Long-Term Air Pollution} and {Increased
Blood Pressure} and {Hypertension} in {China}. Hypertension 61:578--584;
doi:\href{https://doi.org/10.1161/HYPERTENSIONAHA.111.00003}{10.1161/HYPERTENSIONAHA.111.00003}.

\bibitem[\citeproctext]{ref-duan2014}
Duan X, Jiang Y, Wang B, Zhao X, Shen G, Cao S, et al. 2014. Household
fuel use for cooking and heating in {China}: {Results} from the first
{Chinese Environmental Exposure-Related Human Activity Patterns Survey}
({CEERHAPS}). Applied Energy 136:692--703;
doi:\href{https://doi.org/10.1016/j.apenergy.2014.09.066}{10.1016/j.apenergy.2014.09.066}.

\bibitem[\citeproctext]{ref-edwards2004}
Edwards RD, Smith KR, Zhang J, Ma Y. 2004. Implications of changes in
household stoves and fuel use in {China}. Energy Policy 32:395--411;
doi:\href{https://doi.org/10.1016/S0301-4215(02)00309-9}{10.1016/S0301-4215(02)00309-9}.

\bibitem[\citeproctext]{ref-ERF2012}
Emerging Risk Factors Collaboration. 2012. C-reactive protein,
fibrinogen, and cardiovascular disease prediction. New England Journal
of Medicine 367: 1310--1320.

\bibitem[\citeproctext]{ref-ezzati2017}
Ezzati M, Baumgartner JC. 2017. Household energy and health: Where next
for research and practice? Lancet (London, England) 389:130--132;
doi:\href{https://doi.org/10.1016/S0140-6736(16)32506-5}{10.1016/S0140-6736(16)32506-5}.

\bibitem[\citeproctext]{ref-fda2018}
Food and Drug Administration. 2018. Bioanalytical {Method Validation
Guidance} for {Industry}.

\DIFaddbegin \bibitem[\citeproctext]{ref-gao2018}
\DIFadd{Gao J, Wang K, Wang Y, Liu S, Zhu C, Hao J, et al. 2018.
Temporal-spatial characteristics and source apportionment of }{\DIFadd{PM2}}\DIFadd{.5 as
well as its associated chemical species in the }{\DIFadd{Beijing-Tianjin-Hebei}}
\DIFadd{region of }{\DIFadd{China}}\DIFadd{. Environmental Pollution 233:714--724;
doi:}\href{https://doi.org/10.1016/j.envpol.2017.10.123}{\DIFadd{10.1016/j.envpol.2017.10.123}}\DIFadd{.
}

\DIFaddend \bibitem[\citeproctext]{ref-gbdmaps2016}
GBD MAPS Working Group. 2016. Burden of disease attributable to
coal-burning and other air pollution sources in {China}.

\DIFaddbegin \bibitem[\citeproctext]{ref-gelman2014}
\DIFadd{Gelman A, Carlin J. 2014. Beyond }{\DIFadd{Power Calculations}}\DIFadd{: }{\DIFadd{Assessing Type
S}} \DIFadd{(}{\DIFadd{Sign}}\DIFadd{) and }{\DIFadd{Type M}} \DIFadd{(}{\DIFadd{Magnitude}}\DIFadd{) }{\DIFadd{Errors}}\DIFadd{. Perspectives on
Psychological Science 9:641--651;
doi:}\href{https://doi.org/10.1177/1745691614551642}{\DIFadd{10.1177/1745691614551642}}\DIFadd{.
}

\DIFaddend \bibitem[\citeproctext]{ref-goin2023}
Goin DE, Riddell CA. 2023. Comparing {Two-way Fixed Effects} and {New
Estimators} for {Difference-in-Differences}: {A Simulation Study} and
{Empirical Example}. Epidemiology 34:535;
doi:\href{https://doi.org/10.1097/EDE.0000000000001611}{10.1097/EDE.0000000000001611}.

\bibitem[\citeproctext]{ref-goodman-bacon2021}
Goodman-Bacon A. 2021. Difference-in-differences with variation in
treatment timing. Journal of Econometrics 225:254--277;
doi:\href{https://doi.org/10.1016/j.jeconom.2021.03.014}{10.1016/j.jeconom.2021.03.014}.

\bibitem[\citeproctext]{ref-gould2023}
Gould CF, Bejarano ML, Kioumourtzoglou M-A, Lee AG, Pillarisetti A,
Schlesinger SB, et al. 2023. Widespread {Clean Cooking Fuel Scale-Up}
and under-5 {Lower Respiratory Infection Mortality}: {An Ecological
Analysis} in {Ecuador}, 1990--2019. Environmental Health Perspectives
131:037017;
doi:\href{https://doi.org/10.1289/EHP11016}{10.1289/EHP11016}.

\DIFaddbegin \bibitem[\citeproctext]{ref-hoenig2001}
\DIFadd{Hoenig JM, Heisey DM. 2001. The }{\DIFadd{Abuse}} \DIFadd{of }{\DIFadd{Power}}\DIFadd{: }{\DIFadd{The Pervasive
Fallacy}} \DIFadd{of }{\DIFadd{Power Calculations}} \DIFadd{for }{\DIFadd{Data Analysis}}\DIFadd{. The American
Statistician 55:19--24;
doi:}\href{https://doi.org/10.1198/000313001300339897}{\DIFadd{10.1198/000313001300339897}}\DIFadd{.
}

\DIFaddend \bibitem[\citeproctext]{ref-huang2012}
Huang W, Wang G, Lu S-E, Kipen H, Wang Y, Hu M, et al. 2012.
Inflammatory and {Oxidative Stress Responses} of {Healthy Young Adults}
to {Changes} in {Air Quality} during the {Beijing Olympics}. American
Journal of Respiratory and Critical Care Medicine 186:1150--1159;
doi:\href{https://doi.org/10.1164/rccm.201205-0850OC}{10.1164/rccm.201205-0850OC}.

\bibitem[\citeproctext]{ref-johnson2022}
Johnson M, Pillarisetti A, Piedrahita R, Balakrishnan K, Peel JL,
Steenland K, et al. 2022. Exposure {Contrasts} of {Pregnant Women}
during the {Household Air Pollution Intervention Network Randomized
Controlled Trial}. Environmental Health Perspectives 130:097005;
doi:\href{https://doi.org/10.1289/EHP10295}{10.1289/EHP10295}.

\bibitem[\citeproctext]{ref-johnston2013}
Johnston FH, Hanigan IC, Henderson SB, Morgan GG. 2013. Evaluation of
interventions to reduce air pollution from biomass smoke on mortality in
{Launceston}, {Australia}: Retrospective analysis of daily mortality,
1994-2007. BMJ 346:e8446--e8446;
doi:\href{https://doi.org/10.1136/bmj.e8446}{10.1136/bmj.e8446}.

\bibitem[\citeproctext]{ref-kanagasabai2022}
Kanagasabai T, Xie W, Yan L, Zhao L, Carter E, Guo D, et al. 2022.
Household {Air Pollution} and {Blood Pressure}, {Vascular Damage}, and
{Subclinical Indicators} of {Cardiovascular Disease} in {Older Chinese
Adults}. American Journal of Hypertension 35:121--131;
doi:\href{https://doi.org/10.1093/ajh/hpab141}{10.1093/ajh/hpab141}.

\bibitem[\citeproctext]{ref-kashtan2023}
Kashtan YS, Nicholson M, Finnegan C, Ouyang Z, Lebel ED, Michanowicz DR,
et al. 2023. Gas and {Propane Combustion} from {Stoves Emits Benzene}
and {Increases Indoor Air Pollution}. Environmental Science \&
Technology 57:9653--9663;
doi:\href{https://doi.org/10.1021/acs.est.2c09289}{10.1021/acs.est.2c09289}.

\bibitem[\citeproctext]{ref-katz2020}
Katz J, Tielsch JM, Khatry SK, Shrestha L, Breysse P, Zeger SL, et al.
2020. Impact of {Improved Biomass} and {Liquid Petroleum Gas Stoves} on
{Birth Outcomes} in {Rural Nepal}: {Results} of 2 {Randomized Trials}.
Global Health: Science and Practice 8:372--382;
doi:\href{https://doi.org/10.9745/GHSP-D-20-00011}{10.9745/GHSP-D-20-00011}.

\bibitem[\citeproctext]{ref-keele2015}
Keele L, Tingley D, Yamamoto T. 2015. Identifying mechanisms behind
policy interventions via causal mediation analysis. Journal of Policy
Analysis and Management 34: 937--963.

\bibitem[\citeproctext]{ref-khuzestani2017}
Khuzestani RB, Schauer JJ, Wei Y, Zhang Y, Zhang Y. 2017. A
non-destructive optical color space sensing system to quantify elemental
and organic carbon in atmospheric particulate matter on {Teflon} and
quartz filters. Atmospheric Environment 149:84--94;
doi:\href{https://doi.org/10.1016/j.atmosenv.2016.11.002}{10.1016/j.atmosenv.2016.11.002}.

\bibitem[\citeproctext]{ref-kipen2010}
Kipen H, Rich D, Huang W, Zhu T, Wang G, Hu M, et al. 2010. Measurement
of inflammation and oxidative stress following drastic changes in air
pollution during the {Beijing Olympics}: A panel study approach. Annals
of the New York Academy of Sciences 1203:160--167;
doi:\href{https://doi.org/10.1111/j.1749-6632.2010.05638.x}{10.1111/j.1749-6632.2010.05638.x}.

\bibitem[\citeproctext]{ref-kumar2021}
Kumar N, Phillip E, Cooper H, Davis M, Langevin J, Clifford M, et al.
2021. Do improved biomass cookstove interventions improve indoor air
quality and blood pressure? {A} systematic review and meta-analysis.
Environmental Pollution 290:117997;
doi:\href{https://doi.org/10.1016/j.envpol.2021.117997}{10.1016/j.envpol.2021.117997}.

\DIFdelbegin \bibitem[\citeproctext]{ref-lorange2021}
\DIFdel{L'Orange C, Neymark G, Carter E, Volckens J. 2021. A }%DIFDELCMD < {%%%
\DIFdel{High-throughput}%DIFDELCMD < }%%%
\DIFdel{,
}%DIFDELCMD < {%%%
\DIFdel{Robotic System}%DIFDELCMD < } %%%
\DIFdel{for }%DIFDELCMD < {%%%
\DIFdel{Analysis}%DIFDELCMD < } %%%
\DIFdel{of }%DIFDELCMD < {%%%
\DIFdel{Aerosol Sampling Filters}%DIFDELCMD < }%%%
\DIFdel{. Aerosol
and Air Quality Research 21:210037;
doi:}\href{https://doi.org/10.4209/aaqr.210037}{\DIFdel{10.4209/aaqr.210037}}%DIFAUXCMD
\DIFdel{.
}%DIFDELCMD < 

%DIFDELCMD < %%%
\DIFdelend \bibitem[\citeproctext]{ref-lai2019}
Lai. 2019. Relative contributions of household solid fuel use and
outdoor air pollution to chemical components of personal {PM2}.5
exposures. Indoor Air-international Journal of Indoor Air Quality and
Climate.

\bibitem[\citeproctext]{ref-lai2024}
Lai PS, Lam NL, Gallery B, Lee AG, Adair-Rohani H, Alexander D, et al.
2024. Household {Air Pollution Interventions} to {Improve Health} in
{Low-} and {Middle-Income Countries}: {An Official American Thoracic
Society Research Statement}. American Journal of Respiratory and
Critical Care Medicine 209:909--927;
doi:\href{https://doi.org/10.1164/rccm.202402-0398ST}{10.1164/rccm.202402-0398ST}.

\DIFaddbegin \bibitem[\citeproctext]{ref-lee2021}
\DIFadd{Lee M, Carter E, Yan L, Chan Q, Elliott P, Ezzati M, et al. 2021.
Determinants of personal exposure to }{\DIFadd{PM2}}\DIFadd{.5 and black carbon in
}{\DIFadd{Chinese}} \DIFadd{adults: }{\DIFadd{A}} \DIFadd{repeated-measures study in villages using solid
fuel energy. Environment International 146:106297;
doi:}\href{https://doi.org/10.1016/j.envint.2020.106297}{\DIFadd{10.1016/j.envint.2020.106297}}\DIFadd{.
}

\DIFaddend \bibitem[\citeproctext]{ref-lewington2012}
Lewington S, LiMing L, Sherliker P, Yu G, Millwood I, Zheng B, et al.
2012. Seasonal variation in blood pressure and its relationship with
outdoor temperature in 10 diverse regions of {China}: The {China
Kadoorie Biobank}. Journal of hypertension 30: 1383.

\bibitem[\citeproctext]{ref-li2022a}
Li X, Baumgartner J, Harper S, Zhang X, Sternbach T, Barrington-Leigh C,
et al. 2022. Field measurements of indoor and community air quality in
rural {Beijing} before, during, and after the {COVID-19} lockdown.
Indoor Air 32:e13095;
doi:\href{https://doi.org/10.1111/ina.13095}{10.1111/ina.13095}.

\bibitem[\citeproctext]{ref-lindemann2017}
Lindemann U, Stotz A, Beyer N, Oksa J, Skelton DA, Becker C, et al.
2017. Effect of indoor temperature on physical performance in older
adults during days with normal temperature and heat waves. International
journal of environmental research and public health 14;
doi:\href{https://doi.org/10.3390/ijerph14020186}{10.3390/ijerph14020186}.

\DIFaddbegin \bibitem[\citeproctext]{ref-liu2017}
\DIFadd{Liu B, Wu J, Zhang J, Wang L, Yang J, Liang D, et al. 2017.
Characterization and source apportionment of }{\DIFadd{PM2}}\DIFadd{.5 based on error
estimation from }{\DIFadd{EPA PMF}} \DIFadd{5.0 model at a medium city in }{\DIFadd{China}}\DIFadd{.
Environmental Pollution 222:10--22;
doi:}\href{https://doi.org/10.1016/j.envpol.2017.01.005}{\DIFadd{10.1016/j.envpol.2017.01.005}}\DIFadd{.
}

\DIFaddend \bibitem[\citeproctext]{ref-lowe2009}
Lowe A, Harrison W, El-Aklouk E, Ruygrok P, Al-Jumaily AM. 2009.
Non-invasive model-based estimation of aortic pulse pressure using
suprasystolic brachial pressure waveforms. Journal of Biomechanics
42:2111--2115;
doi:\href{https://doi.org/10.1016/j.jbiomech.2009.05.029}{10.1016/j.jbiomech.2009.05.029}.

\bibitem[\citeproctext]{ref-lv2022}
Lv Y, Zhu R, Xie J, Yoshino H. 2022. Indoor environment and the blood
pressure of elderly in the cold region of {China}. Indoor and Built
Environment 31:2482--2498;
doi:\href{https://doi.org/10.1177/1420326X221109510}{10.1177/1420326X221109510}.

\DIFaddbegin \bibitem[\citeproctext]{ref-manning2001}
\DIFadd{Manning WG, Mullahy J. 2001. Estimating log models: To transform or not
to transform? Journal of Health Economics 20:461--494;
doi:}\href{https://doi.org/10.1016/S0167-6296(01)00086-8}{\DIFadd{10.1016/S0167-6296(01)00086-8}}\DIFadd{.
}

\DIFaddend \bibitem[\citeproctext]{ref-mccracken2007}
McCracken JP, Smith KR, Díaz A, Mittleman MA, Schwartz J. 2007. Chimney
{Stove Intervention} to {Reduce Long-term Wood Smoke Exposure Lowers
Blood Pressure} among {Guatemalan Women}. Environmental Health
Perspectives 115:996--1001;
doi:\href{https://doi.org/10.1289/ehp.9888}{10.1289/ehp.9888}.

\bibitem[\citeproctext]{ref-mccracken2011}
McCracken J, Smith KR, Stone P, Díaz A, Arana B, Schwartz J. 2011.
Intervention to {Lower Household Wood Smoke Exposure} in {Guatemala
Reduces ST-Segment Depression} on {Electrocardiograms}. Environmental
Health Perspectives 119:1562--1568;
doi:\href{https://doi.org/10.1289/ehp.1002834}{10.1289/ehp.1002834}.

\bibitem[\citeproctext]{ref-mei2020}
Mei H, Han P, Wang Y, Zeng N, Liu D, Cai Q, et al. 2020. Field
{Evaluation} of {Low-Cost Particulate Matter Sensors} in {Beijing}.
Sensors 20:4381;
doi:\href{https://doi.org/10.3390/s20164381}{10.3390/s20164381}.

\bibitem[\citeproctext]{ref-meng2023}
Meng W, Zhu L, Liang Z, Xu H, Zhang W, Li J, et al. 2023. Significant
but {Inequitable Cost-Effective Benefits} of a {Clean Heating Campaign}
in {Northern China}. Environmental Science \& Technology 57:8467--8475;
doi:\href{https://doi.org/10.1021/acs.est.2c07492}{10.1021/acs.est.2c07492}.

\bibitem[\citeproctext]{ref-naimi2014}
Naimi AI, Kaufman JS, MacLehose RF. 2014. Mediation misgivings:
Ambiguous clinical and public health interpretations of natural direct
and indirect effects. International journal of epidemiology 43:1656--61;
doi:\href{https://doi.org/10.1093/ije/dyu107}{10.1093/ije/dyu107}.

\DIFaddbegin \bibitem[\citeproctext]{ref-ni2016}
\DIFadd{Ni K, Carter E, Schauer JJ, Ezzati M, Zhang Y, Niu H, et al. 2016.
Seasonal variation in outdoor, indoor, and personal air pollution
exposures of women using wood stoves in the }{\DIFadd{Tibetan Plateau}}\DIFadd{:
}{\DIFadd{Baseline}} \DIFadd{assessment for an energy intervention study. Environment
International 94:449--457;
doi:}\href{https://doi.org/10.1016/j.envint.2016.05.029}{\DIFadd{10.1016/j.envint.2016.05.029}}\DIFadd{.
}

\DIFaddend \bibitem[\citeproctext]{ref-niu2024}
Niu J, Chen X, Sun S. 2024. China's {Coal Ban} policy: {Clearing} skies,
challenging growth. Journal of Environmental Management 349:119420;
doi:\href{https://doi.org/10.1016/j.jenvman.2023.119420}{10.1016/j.jenvman.2023.119420}.

\bibitem[\citeproctext]{ref-olson2016}
Olson MR, Graham E, Hamad S, Uchupalanun P, Ramanathan N, Schauer JJ.
2016. Quantification of elemental and organic carbon in atmospheric
particulate matter using color space sensing---hue, saturation, and
value ({HSV}) coordinates. Science of The Total Environment
548--549:252--259;
doi:\href{https://doi.org/10.1016/j.scitotenv.2016.01.032}{10.1016/j.scitotenv.2016.01.032}.

\bibitem[\citeproctext]{ref-onakomaiya2019}
Onakomaiya D, Gyamfi J, Iwelunmor J, Opeyemi J, Oluwasanmi M,
Obiezu-Umeh C, et al. 2019. Implementation of clean cookstove
interventions and its effects on blood pressure in low-income and
middle-income countries: Systematic review. BMJ Open 9:e026517;
doi:\href{https://doi.org/10.1136/bmjopen-2018-026517}{10.1136/bmjopen-2018-026517}.

\bibitem[\citeproctext]{ref-pearl2000}
Pearl J. 2000. \emph{Causality: Models, reasoning, and inference}.
Cambridge University Press:Cambridge, U.K. ; New York.

\bibitem[\citeproctext]{ref-pearson2003}
Pearson TA, Mensah GA, Alexander RW, Anderson JL, Cannon RO 3rd, Criqui
M, et al. 2003.
\href{https://www.ncbi.nlm.nih.gov/pubmed/12551878}{Markers of
inflammation and cardiovascular disease: Application to clinical and
public health practice: {A} statement for healthcare professionals from
the {Centers} for {Disease Control} and {Prevention} and the {American
Heart Association}}. Circulation 107: 499--511.

\bibitem[\citeproctext]{ref-peel2015}
Peel JL, Baumgartner J, Wellenius GA, Clark ML, Smith KR. 2015. Are
{Randomized Trials Necessary} to {Advance Epidemiologic Research} on
{Household Air Pollution}? Current Epidemiology Reports 2:263--270;
doi:\href{https://doi.org/10.1007/s40471-015-0054-4}{10.1007/s40471-015-0054-4}.

\bibitem[\citeproctext]{ref-pope2004}
Pope III CA, Hansen ML, Long RW, Nielsen KR, Eatough NL, Wilson WE, et
al. 2004. Ambient particulate air pollution, heart rate variability, and
blood markers of inflammation in a panel of elderly subjects.
Environmental health perspectives 112:339--45;
doi:\href{https://doi.org/10.1289/ehp.6588}{10.1289/ehp.6588}.

\bibitem[\citeproctext]{ref-quansah2017}
Quansah R, Semple S, Ochieng CA, Juvekar S, Armah FA, Luginaah I, et al.
2017. Effectiveness of interventions to reduce household air pollution
and/or improve health in homes using solid fuel in low-and-middle income
countries: {A} systematic review and meta-analysis. Environment
International 103:73--90;
doi:\href{https://doi.org/10.1016/j.envint.2017.03.010}{10.1016/j.envint.2017.03.010}.

\DIFaddbegin \bibitem[\citeproctext]{ref-rahimi2021}
\DIFadd{Rahimi K, Bidel Z, Nazarzadeh M, Copland E, Canoy D, Ramakrishnan R, et
al. 2021. Pharmacological blood pressure lowering for primary and
secondary prevention of cardiovascular disease across different levels
of blood pressure: An individual participant-level data meta-analysis.
The Lancet 397:1625--1636;
doi:}\href{https://doi.org/10.1016/S0140-6736(21)00590-0}{\DIFadd{10.1016/S0140-6736(21)00590-0}}\DIFadd{.
}

\DIFaddend \bibitem[\citeproctext]{ref-rehfuess2014}
Rehfuess EA, Puzzolo E, Stanistreet D, Pope D, Bruce NG. 2014. Enablers
and {Barriers} to {Large-Scale Uptake} of {Improved Solid Fuel Stoves}:
{A Systematic Review}. Environmental Health Perspectives 122:120--130;
doi:\href{https://doi.org/10.1289/ehp.1306639}{10.1289/ehp.1306639}.

\bibitem[\citeproctext]{ref-rich2012}
Rich DQ, Kipen HM, Huang W, Wang G, Wang Y, Zhu P, et al. 2012.
Association {Between Changes} in {Air Pollution Levels During} the
{Beijing Olympics} and {Biomarkers} of {Inflammation} and {Thrombosis}
in {Healthy Young Adults}. JAMA 307;
doi:\href{https://doi.org/10.1001/jama.2012.3488}{10.1001/jama.2012.3488}.

\bibitem[\citeproctext]{ref-ridker2001}
Ridker PM. 2001.
\href{https://www.ncbi.nlm.nih.gov/pubmed/11282915}{High-sensitivity
{C-reactive} protein: Potential adjunct for global risk assessment in
the primary prevention of cardiovascular disease}. Circulation 103:
1813--8.

\bibitem[\citeproctext]{ref-ridker2000}
Ridker PM, Hennekens CH, Buring JE, Rifai N. 2000. C-reactive protein
and other markers of inflammation in the prediction of cardiovascular
disease in women. The New England journal of medicine 342:836--43;
doi:\href{https://doi.org/10.1056/NEJM200003233421202}{10.1056/NEJM200003233421202}.

\bibitem[\citeproctext]{ref-romieu2009}
Romieu I, Riojas-Rodríguez H, Marrón-Mares AT, Schilmann A,
Perez-Padilla R, Masera O. 2009. Improved {Biomass Stove Intervention}
in {Rural Mexico}: {Impact} on the {Respiratory Health} of {Women}.
American Journal of Respiratory and Critical Care Medicine 180:649--656;
doi:\href{https://doi.org/10.1164/rccm.200810-1556OC}{10.1164/rccm.200810-1556OC}.

\bibitem[\citeproctext]{ref-rosenthal2018}
Rosenthal J, Quinn A, Grieshop AP, Pillarisetti A, Glass RI. 2018. Clean
cooking and the {SDGs}: {Integrated} analytical approaches to guide
energy interventions for health and environment goals. Energy for
sustainable development : the journal of the International Energy
Initiative 42:152--159;
doi:\href{https://doi.org/10.1016/j.esd.2017.11.003}{10.1016/j.esd.2017.11.003}.

\DIFaddbegin \bibitem[\citeproctext]{ref-roth2022}
\DIFadd{Roth J. 2022. Pretest with }{\DIFadd{Caution}}\DIFadd{: }{\DIFadd{Event-Study Estimates}} \DIFadd{after
}{\DIFadd{Testing}} \DIFadd{for }{\DIFadd{Parallel Trends}}\DIFadd{. American Economic Review: Insights
4:305--322;
doi:}\href{https://doi.org/10.1257/aeri.20210236}{\DIFadd{10.1257/aeri.20210236}}\DIFadd{.
}

\DIFaddend \bibitem[\citeproctext]{ref-rtiinternational2009}
RTI International. 2009. Standard {Operating Procedure} for the {X-Ray
Fluorescence Analysis} of {Particulate Matter Deposits} on {Teflon
Filters}: {PM Xrf Analysis}.

\bibitem[\citeproctext]{ref-rubin1987}
Rubin DB. 1987.
\emph{\href{https://doi.org/10.1002/9780470316696}{Multiple {Imputation}
for {Nonresponse} in {Surveys}}}. 1st ed. Wiley.

\bibitem[\citeproctext]{ref-ruckerl2007}
Rückerl R, Greven S, Ljungman P, Aalto P, Antoniades C, Bellander T, et
al. 2007. Air pollution and inflammation (interleukin-6, {C-reactive}
protein, fibrinogen) in myocardial infarction survivors. Environmental
health perspectives 115:1072--80;
doi:\href{https://doi.org/10.1289/ehp.10021}{10.1289/ehp.10021}.

\bibitem[\citeproctext]{ref-ruiz-mercado2013}
Ruiz-Mercado I, Canuz E, Walker JL, Smith KR. 2013. Quantitative metrics
of stove adoption using {Stove Use Monitors} ({SUMs}). Biomass and
Bioenergy 57:136--148;
doi:\href{https://doi.org/10.1016/j.biombioe.2013.07.002}{10.1016/j.biombioe.2013.07.002}.

\bibitem[\citeproctext]{ref-scott2011}
Scott AJ, Scarrott C. 2011. Impacts of residential heating intervention
measures on air quality and progress towards targets in {Christchurch}
and {Timaru}, {New Zealand}. Atmospheric Environment 45:2972--2980;
doi:\href{https://doi.org/10.1016/j.atmosenv.2010.09.008}{10.1016/j.atmosenv.2010.09.008}.

\DIFaddbegin \bibitem[\citeproctext]{ref-secrest2016}
\DIFadd{Secrest MH, Schauer JJ, Carter EM, Lai AM, Wang Y, Shan M, et al. 2016.
The oxidative potential of }{\DIFadd{PM2}}\DIFadd{.5 exposures from indoor and outdoor
sources in rural }{\DIFadd{China}}\DIFadd{. The Science of the Total Environment
571:1477--1489;
doi:}\href{https://doi.org/10.1016/j.scitotenv.2016.06.231}{\DIFadd{10.1016/j.scitotenv.2016.06.231}}\DIFadd{.
}

\bibitem[\citeproctext]{ref-shakya2015}
\DIFadd{Shakya KM, Peltier RE. 2015. Non-sulfate sulfur in fine aerosols across
the }{\DIFadd{United States}}\DIFadd{: }{\DIFadd{Insight}} \DIFadd{for organosulfate prevalence. Atmospheric
environment (Oxford, England : 1994) 100:159--166;
doi:}\href{https://doi.org/10.1016/j.atmosenv.2014.10.058}{\DIFadd{10.1016/j.atmosenv.2014.10.058}}\DIFadd{.
}

\DIFaddend \bibitem[\citeproctext]{ref-shang2020}
Shang J, Zhang Y, Schauer JJ, Tian J, Hua J, Han T, et al. 2020.
Associations between source-resolved {PM2}.5 and airway inflammation at
urban and rural locations in {Beijing}. Environment International
139:105635;
doi:\href{https://doi.org/10.1016/j.envint.2020.105635}{10.1016/j.envint.2020.105635}.

\bibitem[\citeproctext]{ref-shankar2020}
Shankar AV, Quinn AK, Dickinson KL, Williams KN, Masera O, Charron D, et
al. 2020. Everybody stacks: {Lessons} from household energy case studies
to inform design principles for clean energy transitions. Energy Policy
141:111468;
doi:\href{https://doi.org/10.1016/j.enpol.2020.111468}{10.1016/j.enpol.2020.111468}.

\bibitem[\citeproctext]{ref-shen2017}
Shen H, Tao S, Chen Y, Ciais P, Güneralp B, Ru M, et al. 2017.
Urbanization-induced population migration has reduced ambient {PM}
{\textsubscript{2.5}} concentrations in {China}. Science Advances
3:e1700300;
doi:\href{https://doi.org/10.1126/sciadv.1700300}{10.1126/sciadv.1700300}.

\bibitem[\citeproctext]{ref-sinton2004}
Sinton JE, Smith KR, Peabody JW, Yaping L, Xiliang Z, Edwards R, et al.
2004. An assessment of programs to promote improved household stoves in
{China}. Energy for Sustainable Development 8:33--52;
doi:\href{https://doi.org/10.1016/S0973-0826(08)60465-2}{10.1016/S0973-0826(08)60465-2}.

\bibitem[\citeproctext]{ref-smith-sivertsen2009}
Smith-Sivertsen T, Díaz E, Pope D, Lie RT, Díaz A, McCracken J, et al.
2009. Effect of {Reducing Indoor Air Pollution} on {Women}'s
{Respiratory Symptoms} and {Lung Function}: {The RESPIRE Randomized
Trial}, {Guatemala}. American Journal of Epidemiology 170:211--220;
doi:\href{https://doi.org/10.1093/aje/kwp100}{10.1093/aje/kwp100}.

\bibitem[\citeproctext]{ref-snider2018}
Snider G, Carter E, Clark S, Tseng J(TzuW, Yang X, Ezzati M, et al.
2018. Impacts of stove use patterns and outdoor air quality on household
air pollution and cardiovascular mortality in southwestern {China}.
Environment International 117:116--124;
doi:\href{https://doi.org/10.1016/j.envint.2018.04.048}{10.1016/j.envint.2018.04.048}.

\bibitem[\citeproctext]{ref-song2023}
Song C, Liu B, Cheng K, Cole MA, Dai Q, Elliott RJR, et al. 2023.
Attribution of {Air Quality Benefits} to {Clean Winter Heating Policies}
in {China}: {Combining Machine Learning} with {Causal Inference}.
Environmental Science \& Technology 57:17707--17717;
doi:\href{https://doi.org/10.1021/acs.est.2c06800}{10.1021/acs.est.2c06800}.

\DIFaddbegin \bibitem[\citeproctext]{ref-steenland2018}
\DIFadd{Steenland K, Pillarisetti A, Kirby M, Peel J, Clark M, Checkley W, et
al. 2018. Modeling the potential health benefits of lower household air
pollution after a hypothetical liquified petroleum gas (}{\DIFadd{LPG}}\DIFadd{) cookstove
intervention. Environment International 111:71--79;
doi:}\href{https://doi.org/10.1016/j.envint.2017.11.018}{\DIFadd{10.1016/j.envint.2017.11.018}}\DIFadd{.
}

\DIFaddend \bibitem[\citeproctext]{ref-sternbach2022}
Sternbach TJ, Harper S, Li X, Zhang X, Carter E, Zhang Y, et al. 2022.
Effects of indoor and outdoor temperatures on blood pressure and central
hemodynamics in a wintertime longitudinal study of {Chinese} adults.
Journal of Hypertension 40:1950--1959;
doi:\href{https://doi.org/10.1097/HJH.0000000000003198}{10.1097/HJH.0000000000003198}.

\DIFaddbegin \bibitem[\citeproctext]{ref-sullivan2008}
\DIFadd{Sullivan AP, Holden AS, Patterson LA, McMeeking GR, Kreidenweis SM, Malm
WC, et al. 2008. A method for smoke marker measurements and its
potential application for determining the contribution of biomass
burning from wildfires and prescribed fires to ambient }{\DIFadd{PM2}}\DIFadd{.5 organic
carbon. Journal of Geophysical Research: Atmospheres 113;
doi:}\href{https://doi.org/10.1029/2008JD010216}{\DIFadd{10.1029/2008JD010216}}\DIFadd{.
}

\DIFaddend \bibitem[\citeproctext]{ref-sun2021}
Sun L, Abraham S. 2021. Estimating dynamic treatment effects in event
studies with heterogeneous treatment effects. Journal of Econometrics
225:175--199;
doi:\href{https://doi.org/10.1016/j.jeconom.2020.09.006}{10.1016/j.jeconom.2020.09.006}.

\bibitem[\citeproctext]{ref-tan2023}
Tan X, Chen G, Chen K. 2023. Clean heating and air pollution: {Evidence}
from {Northern China}. Energy Reports 9:303--313;
doi:\href{https://doi.org/10.1016/j.egyr.2022.11.166}{10.1016/j.egyr.2022.11.166}.

\DIFaddbegin \bibitem[\citeproctext]{ref-tang2020}
\DIFadd{Tang H, Cheng Z, Li N, Mao S, Ma R, He H, et al. 2020. The short- and
long-term associations of particulate matter with inflammation and blood
coagulation markers: }{\DIFadd{A}} \DIFadd{meta-analysis. Environmental Pollution
267:115630;
doi:}\href{https://doi.org/10.1016/j.envpol.2020.115630}{\DIFadd{10.1016/j.envpol.2020.115630}}\DIFadd{.
}

\bibitem[\citeproctext]{ref-tao2017}
\DIFadd{Tao J, Zhang L, Cao J, Zhang R. 2017. A review of current knowledge
concerning }{\DIFadd{PM2}}\DIFadd{.5 chemical composition, aerosol optical properties and
their relationships across }{\DIFadd{China}}\DIFadd{. Atmospheric Chemistry and Physics
17:9485--9518;
doi:}\href{https://doi.org/10.5194/acp-17-9485-2017}{\DIFadd{10.5194/acp-17-9485-2017}}\DIFadd{.
}

\DIFaddend \bibitem[\citeproctext]{ref-thompson2019}
Thompson RJ, Li J, Weyant CL, Edwards R, Lan Q, Rothman N, et al. 2019.
Field {Emission Measurements} of {Solid Fuel Stoves} in {Yunnan}, {China
Demonstrate Dominant Causes} of {Uncertainty} in {Household Emission
Inventories}. Environmental Science \& Technology 53:3323--3330;
doi:\href{https://doi.org/10.1021/acs.est.8b07040}{10.1021/acs.est.8b07040}.

\bibitem[\citeproctext]{ref-tuck2009}
Tuck MK, Chan DW, Chia D, Godwin AK, Grizzle WE, Krueger KE, et al.
2009. Standard {Operating Procedures} for {Serum} and {Plasma
Collection}: {Early Detection Research Network Consensus Statement}
{\emph{Standard Operating Procedure Integration Working Group}}. Journal
of Proteome Research 8:113--117;
doi:\href{https://doi.org/10.1021/pr800545q}{10.1021/pr800545q}.

\bibitem[\citeproctext]{ref-vanbuuren2011}
van Buuren S, Groothuis-Oudshoorn K. 2011. {\textbf{Mice}} :
{Multivariate Imputation} by {Chained Equations} in {\emph{R}}. Journal
of Statistical Software 45;
doi:\href{https://doi.org/10.18637/jss.v045.i03}{10.18637/jss.v045.i03}.

\bibitem[\citeproctext]{ref-vandonkelaar2021}
Van Donkelaar A, Hammer MS, Bindle L, Brauer M, Brook JR, Garay MJ, et
al. 2021. Monthly {Global Estimates} of {Fine Particulate Matter} and
{Their Uncertainty}. Environmental Science \& Technology
55:15287--15300;
doi:\href{https://doi.org/10.1021/acs.est.1c05309}{10.1021/acs.est.1c05309}.

\bibitem[\citeproctext]{ref-vanderweele2015}
VanderWeele TJ. 2015. \emph{Explanation in causal inference: Methods for
mediation and interaction}. Oxford University Press:New York.

\bibitem[\citeproctext]{ref-volckens2017}
Volckens J, Quinn C, Leith D, Mehaffy J, Henry CS, Miller-Lionberg D.
2017. Development and evaluation of an ultrasonic personal aerosol
sampler. Indoor air 27:409--416;
doi:\href{https://doi.org/10.1111/ina.12318}{10.1111/ina.12318}.

\bibitem[\citeproctext]{ref-wang2020}
Wang Q, Zhao Q, Wang G, Wang B, Zhang Y, Zhang J, et al. 2020. The
association between ambient temperature and clinical visits for
inflammation-related diseases in rural areas in {China}. Environmental
Pollution 261:114128;
doi:\href{https://doi.org/10.1016/j.envpol.2020.114128}{10.1016/j.envpol.2020.114128}.

\DIFaddbegin \bibitem[\citeproctext]{ref-wang2016}
\DIFadd{Wang Y, Zhang Y, Schauer JJ, De Foy B, Guo B, Zhang Y. 2016. Relative
impact of emissions controls and meteorology on air pollution mitigation
associated with the }{\DIFadd{Asia-Pacific Economic Cooperation}} \DIFadd{(}{\DIFadd{APEC}}\DIFadd{)
conference in }{\DIFadd{Beijing}}\DIFadd{, }{\DIFadd{China}}\DIFadd{. Science of The Total Environment
571:1467--1476;
doi:}\href{https://doi.org/10.1016/j.scitotenv.2016.06.215}{\DIFadd{10.1016/j.scitotenv.2016.06.215}}\DIFadd{.
}

\bibitem[\citeproctext]{ref-wasserstein2019}
\DIFadd{Wasserstein RL, Schirm AL, Lazar NA. 2019. Moving to a }{\DIFadd{World Beyond}}
{\DIFadd{``p }{\DIFadd{\(<\)}} \DIFadd{0.05.''}} \DIFadd{The American Statistician.
}

\DIFaddend \bibitem[\citeproctext]{ref-wen2023}
Wen H, Nie P, Liu M, Peng R, Guo T, Wang C, et al. 2023. Multi-health
effects of clean residential heating: {Evidences} from rural {China}'s
coal-to-gas/electricity project. Energy for Sustainable Development
73:66--75;
doi:\href{https://doi.org/10.1016/j.esd.2023.01.013}{10.1016/j.esd.2023.01.013}.

\bibitem[\citeproctext]{ref-wooldridge2021}
Wooldridge JM. 2021. Two-{Way Fixed Effects}, the {Two-Way Mundlak
Regression}, and {Difference-in-Differences Estimators}.;
doi:\href{https://doi.org/10.2139/ssrn.3906345}{10.2139/ssrn.3906345}.

\bibitem[\citeproctext]{ref-who2021}
World Health Organization. 2021. {WHO Global Air Quality Guidelines}:
{Particulate Matter PM2}.5 and {PM10}), {Ozone}, {Nitrogen Dioxide},
{Sulfur Dioxide} and {Carbon Monoxide}.

\bibitem[\citeproctext]{ref-xu2019}
Xu H, Brook RD, Wang T, Song X, Feng B, Yi T, et al. 2019. Short-term
effects of ambient air pollution and outdoor temperature on biomarkers
of myocardial damage, inflammation and oxidative stress in healthy
adults. Environmental Epidemiology 3:e078;
doi:\href{https://doi.org/10.1097/EE9.0000000000000078}{10.1097/EE9.0000000000000078}.

\DIFaddbegin \bibitem[\citeproctext]{ref-xu2009}
\DIFadd{Xu W, Collet J-P, Shapiro S, Lin Y, Yang T, Wang C, et al. 2009.
}\href{https://www.ncbi.nlm.nih.gov/pubmed/19146745}{\DIFadd{Validation and
clinical interpretation of the }{\DIFadd{St George}}\DIFadd{'s }{\DIFadd{Respiratory Questionnaire}}
\DIFadd{among }{\DIFadd{COPD}} \DIFadd{patients, }{\DIFadd{China}}}\DIFadd{. The International Journal of
Tuberculosis and Lung Disease: The Official Journal of the International
Union Against Tuberculosis and Lung Disease 13: 181--189.
}

\DIFaddend \bibitem[\citeproctext]{ref-yan2020}
Yan L, Carter E, Fu Y, Guo D, Huang P, Xie G, et al. 2020. Study
protocol: {The INTERMAP China Prospective} ({ICP}) study. Wellcome Open
Research 4:154;
doi:\href{https://doi.org/10.12688/wellcomeopenres.15470.2}{10.12688/wellcomeopenres.15470.2}.

\bibitem[\citeproctext]{ref-yangk2021}
Yang K. 2021. \DIFdelbegin \DIFdel{Power industry provides inexhaustible power for national
rejuvenation; 杨昆:}\DIFdelend \DIFaddbegin {\DIFaddend 电力工业为民族复兴提供不竭动力 \DIFdelbegin \DIFdel{.
}\DIFdelend \DIFaddbegin \DIFadd{(Power industry provides
inexhaustible power for national rejuvenation)}}\DIFadd{. 中国能源新闻网 (China
Energy News Network).
}\DIFaddend 

\bibitem[\citeproctext]{ref-yap2015}
Yap P-S, Garcia C. 2015. Effectiveness of {Residential Wood-Burning
Regulation} on {Decreasing Particulate Matter Levels} and
{Hospitalizations} in the {San Joaquin Valley Air Basin}. American
Journal of Public Health 105:772--778;
doi:\href{https://doi.org/10.2105/AJPH.2014.302360}{10.2105/AJPH.2014.302360}.

\bibitem[\citeproctext]{ref-ye2022}
Ye W, Steenland K, Quinn A, Liao J, Balakrishnan K, Rosa G, et al. 2022.
Effects of a {Liquefied Petroleum Gas Stove Intervention} on
{Gestational Blood Pressure}: {Intention-to-Treat} and
{Exposure-Response Findings From} the {HAPIN Trial}. Hypertension
79:1887--1898;
doi:\href{https://doi.org/10.1161/HYPERTENSIONAHA.122.19362}{10.1161/HYPERTENSIONAHA.122.19362}.

\bibitem[\citeproctext]{ref-young2015}
Young OR, Guttman D, Qi Y, Bachus K, Belis D, Cheng H, et al. 2015.
Institutionalized governance processes: {Comparing} environmental
problem solving in {China} and the {United States}. Global Environmental
Change 31:163--173;
doi:\href{https://doi.org/10.1016/j.gloenvcha.2015.01.010}{10.1016/j.gloenvcha.2015.01.010}.

\bibitem[\citeproctext]{ref-yu2021}
Yu C, Kang J, Teng J, Long H, Fu Y. 2021. Does coal-to-gas policy reduce
air pollution? {Evidence} from a quasi-natural experiment in {China}.
Science of The Total Environment 773:144645;
doi:\href{https://doi.org/10.1016/j.scitotenv.2020.144645}{10.1016/j.scitotenv.2020.144645}.

\bibitem[\citeproctext]{ref-yun2020}
Yun X, Shen G, Shen H, Meng W, Chen Y, Xu H, et al. 2020. Residential
solid fuel emissions contribute significantly to air pollution and
associated health impacts in {China}. Science Advances 6:eaba7621;
doi:\href{https://doi.org/10.1126/sciadv.aba7621}{10.1126/sciadv.aba7621}.

\bibitem[\citeproctext]{ref-zhang2007}
Zhang J(Jim), Smith KR. 2007. Household {Air Pollution} from {Coal} and
{Biomass Fuels} in {China}: {Measurements}, {Health Impacts}, and
{Interventions}. Environmental Health Perspectives 115:848--855;
doi:\href{https://doi.org/10.1289/ehp.9479}{10.1289/ehp.9479}.

\bibitem[\citeproctext]{ref-zhang2019}
Zhang Q, Zheng Y, Tong D, Shao M, Wang S, Zhang Y, et al. 2019. Drivers
of improved {PM} {\textsubscript{2.5}} air quality in {China} from 2013
to 2017. Proceedings of the National Academy of Sciences
116:24463--24469;
doi:\href{https://doi.org/10.1073/pnas.1907956116}{10.1073/pnas.1907956116}.

\bibitem[\citeproctext]{ref-zhang2023}
Zhang W, Xu H, Yu X, Li J, Zhang Y, Dai R, et al. 2023. Rigorous
{Regional Air Quality Standards} for {Substantial Health Benefits}.
Earth's Future 11:e2023EF003860;
doi:\href{https://doi.org/10.1029/2023EF003860}{10.1029/2023EF003860}.

\bibitem[\citeproctext]{ref-zhuang2009}
Zhuang Z, Li Y, Chen B, Guo J. 2009. Chinese kang as a domestic heating
system in rural northern {China}---{A} review. Energy and Buildings
41:111--119;
doi:\href{https://doi.org/10.1016/j.enbuild.2008.07.013}{10.1016/j.enbuild.2008.07.013}.

\bibitem[\citeproctext]{ref-zigler2016}
Zigler CM, Kim C, Choirat C, Hansen JB, Wang Y, Hund L, et al. 2016.
\emph{Causal inference methods for estimating long-term health effects
of air quality regulations. {Research} report 187.} Health Effects
Institute / Health Effects Institute:Boston, MA.

\DIFaddbegin \bibitem[\citeproctext]{ref-zikova2016}
\DIFadd{Zíková N, Wang Y, Yang F, Li X, Tian M, Hopke PK. 2016. On the source
contribution to }{\DIFadd{Beijing PM2}}\DIFadd{.5 concentrations. Atmospheric Environment
134:84--95;
doi:}\href{https://doi.org/10.1016/j.atmosenv.2016.03.047}{\DIFadd{10.1016/j.atmosenv.2016.03.047}}\DIFadd{.
}

\DIFaddend \end{CSLReferences}

\newpage
\appendix
\renewcommand{\thefigure}{A\arabic{figure}}
\renewcommand{\thetable}{A\arabic{table}}
\setcounter{figure}{0}
\setcounter{table}{0}

\section{Appendices}\label{appendices}

\subsection{Biomarker descriptives}\label{biomarker-descriptives}

Below we show boxplots for the \DIFdelbegin \DIFdel{logged values of the }\DIFdelend blood inflammatory and oxidative stress
markers.

\begin{figure}[H]

\caption{\label{fig-afig-biomarkers}Boxplots for markers of systemic
inflammation including C-reactive protein (CRP), interleukin-6 (IL-6),
tumour necrosis factor alpha (TNF-\(\alpha\)) and \DIFdelbeginFL \DIFdelFL{markers of oxidative
stress including 8-hydroxy-2'-deoxyguanosine (8-OHdG) and
}\DIFdelendFL malondialdehyde (MDA)}

\DIFdelbeginFL %DIFDELCMD < \centering{
%DIFDELCMD < 

%DIFDELCMD < \includegraphics[width=0.9\textwidth,height=\textheight]{images/biomarker-boxplot-log.jpg}
%DIFDELCMD < 

%DIFDELCMD < }
%DIFDELCMD < %%%
\DIFdelendFL \DIFaddbeginFL \centering{

\includegraphics[width=0.9\textwidth,height=\textheight]{images/Biomarker boxplot.jpg}

}
\DIFaddendFL 

\end{figure}%

\newpage

\subsection{\DIFdelbegin \DIFdel{Imputation results}\DIFdelend \DIFaddbegin \DIFadd{Missing data}\DIFaddend }\DIFaddbegin \label{missing-data}

\subsubsection{\DIFadd{Missingness across imputed
variables}}\label{missingness-across-imputed-variables}

\DIFadd{The reasons for missing data differ by variable:
}

\begin{itemize}
\item
  \DIFadd{`Valid response' indicates that either the variable is populated or
  that `missing' is an appropriate response. For example, participants
  without diagnosed hypertension were not eligible to answer the survey
  question on whether they were taking BP-lowering medication.
  Participants without physician-diagnosed hypertension should be
  missing this variable (i.e., missing is a valid response).
}\item
  \DIFadd{`Non-response' indicates a missing value that should have been
  populated but was not. For example, we measured weight, height, and
  waist circumference at the clinic visit rather than during household
  visits in W1 and W2. The clinic visits took place on a different day
  than household visits and thus some participants who completed the
  home visits were away from the village on the day of the clinic visit.
  Thus, participants with missing ``waist circumference'', ``height'' or
  ``weight'' are considered `non-responses'. As a second example, if an
  indoor PM\textsubscript{2.5} monitor was placed in the household but
  did not collect any data, this counts as a `non-response' since the
  measurement was attempted but not completed.
}\item
  \DIFadd{`Not sampled' indicates that a value is missing because our protocol
  was to measure some variables in a sub-sample of participants and the
  participant was never sampled. For example, indoor
  PM\textsubscript{2.5} was measured in a 30\% randomly selected
  sub-sample of households and thus the 70\% of households in each
  village were `not sampled' for an indoor PM\textsubscript{2.5}
  measurement.
}\end{itemize}

\begin{table}[H]

\caption{\label{tbl-a-mi}\DIFaddFL{Missing and valid values of variables used in
multiple imputation.}}

\centering{

\centering
\resizebox{\ifdim\width>\linewidth 0.9\linewidth\else\width\fi}{!}{
\begin{tblr}[         %% tabularray outer open
]                     %% tabularray outer close
{                     %% tabularray inner open
colspec={Q[]Q[]Q[]Q[]},
}                     %% tabularray inner close
\toprule
Variable & Valid (\%) & Non-response (\%) & Not sampled (\%) \\ \midrule %% TinyTableHeader
Study wave & 3082 (100) & 0 (0) & 0 (0) \\
District & 3082 (100) & 0 (0) & 0 (0) \\
Village & 3082 (100) & 0 (0) & 0 (0) \\
Household ID & 3082 (100) & 0 (0) & 0 (0) \\
Participant ID & 3082 (100) & 0 (0) & 0 (0) \\
Village-level policy treatment status & 3082 (100) & 0 (0) & 0 (0) \\
Systolic central BP & 3081 (100) & 1 (0) & 0 (0) \\
Diastolic central BP & 3081 (100) & 1 (0) & 0 (0) \\
Systolic brachial BP & 3082 (100) & 0 (0) & 0 (0) \\
Diastolic brachial BP & 3082 (100) & 0 (0) & 0 (0) \\
Participant sex & 3081 (100) & 1 (0) & 0 (0) \\
Participant age & 3078 (99.9) & 4 (0.1) & 0 (0) \\
Exposure to tobacco smoke & 3081 (100) & 1 (0) & 0 (0) \\
Frequency of alcohol consumption in the past 12 months & 3081 (100) & 1 (0) & 0 (0) \\
Physician diagnosis of high BP & 3081 (100) & 1 (0) & 0 (0) \\
Whether participant took medication for high BP & 1527 (49.5) & 39 (1.3) & 1516 (49.2) \\
Waist circumference & 2568 (83.3) & 514 (16.7) & 0 (0) \\
Weight & 2614 (84.8) & 468 (15.2) & 0 (0) \\
Height & 2610 (84.7) & 472 (15.3) & 0 (0) \\
Self-reported diagnosis of diabetes & 3082 (100) & 0 (0) & 0 (0) \\
Self-reported diagnosis of chronic kidney disease & 3082 (100) & 0 (0) & 0 (0) \\
Household wealth index & 2945 (95.6) & 137 (4.4) & 0 (0) \\
Whether BP was measured in AM or PM & 3082 (100) & 0 (0) & 0 (0) \\
Right arm circumference & 3073 (99.7) & 9 (0.3) & 0 (0) \\
Frequency of farming activities in past 6 months & 3081 (100) & 1 (0) & 0 (0) \\
Frequency of exercise in the past 6 months & 3081 (100) & 1 (0) & 0 (0) \\
If participant snores while sleeping & 3081 (100) & 1 (0) & 0 (0) \\
If participant quits breathing while sleeping & 3081 (100) & 1 (0) & 0 (0) \\
Marital status & 3054 (99.1) & 28 (0.9) & 0 (0) \\
Self-reported diagnosis for coronary heart disease or myocardial infarction & 3081 (100) & 1 (0) & 0 (0) \\
Self-reported diagnosis for stroke or transient ischemic attack (TIA) & 3081 (100) & 1 (0) & 0 (0) \\
Heating season (Jan 15 to Mar 15) mean indoor PM2.5 & 494 (16) & 122 (4) & 2466 (80) \\
Indoor temperature & 3074 (99.7) & 8 (0.3) & 0 (0) \\
Self-reported health status & 3081 (100) & 1 (0) & 0 (0) \\
Blood pressure cuff size & 3078 (99.9) & 4 (0.1) & 0 (0) \\
Participant's highest level of education & 3059 (99.3) & 23 (0.7) & 0 (0) \\
Participant's current occupation & 3053 (99.1) & 29 (0.9) & 0 (0) \\
Time-varying binary indicator for enrollment in policy & 3082 (100) & 0 (0) & 0 (0) \\
Year of first treatment with policy & 3082 (100) & 0 (0) & 0 (0) \\
\bottomrule
\end{tblr}
}

}

\end{table}%DIF > 

\subsubsection{\DIFadd{Missing data by enrollment cohort and
outcome}}\label{missing-data-by-enrollment-cohort-and-outcome}

\DIFadd{The number (N) and percent (Pct.) of missing observations are displayed
in the table below by the enrollment cohort and outcome. `Missing'
indicates that the variable should have been populated for an
observation but was not. `Valid' indicates that the variable is
populated. `Not sampled' indicates that a value is missing because our
protocol was to measure some variables in a sub-sample of participants
and the participant was never sampled for the variable.
}

\begin{table}[H]

\caption{\label{tbl-a-mi-cohort}\DIFaddFL{Missing values by enrollment cohort and
outcome}}

\centering{

\centering
\resizebox{\ifdim\width>\linewidth 0.95\linewidth\else\width\fi}{!}{
\begin{tblr}[         %% tabularray outer open
]                     %% tabularray outer close
{                     %% tabularray inner open
colspec={Q[]Q[]Q[]Q[]Q[]Q[]Q[]Q[]Q[]Q[]},
cell{1}{3}={c=2,}{halign=c,},
cell{1}{5}={c=2,}{halign=c,},
cell{1}{7}={c=2,}{halign=c,},
cell{1}{9}={c=2,}{halign=c,},
row{3}={,cmd=\bfseries,},
row{16}={,cmd=\bfseries,},
row{25}={,cmd=\bfseries,},
row{33}={,cmd=\bfseries,},
}                     %% tabularray inner close
\toprule
&  & Never enrolled (N=1880) &  & Enrolled 2019 (N=642) &  & Enrolled 2020 (N=446) &  & Enrolled 2021 (N=173) &  \\ \cmidrule[lr]{3-4}\cmidrule[lr]{5-6}\cmidrule[lr]{7-8}\cmidrule[lr]{9-10}
&    & N & Pct. & N & Pct. & N & Pct. & N & Pct. \\ \midrule %% TinyTableHeader
Respiratory symptoms:   &                &             &             &            &             &            &             &            &             \\
Any symptoms            & Missing        & \num{23}   & \num{1.2}  & \num{4}   & \num{0.6}  & \num{5}   & \num{1.1}  & \num{1}   & \num{0.6}  \\
& Valid response & \num{1857} & \num{98.8} & \num{638} & \num{99.4} & \num{441} & \num{98.9} & \num{172} & \num{99.4} \\
Cough                   & Missing        & \num{23}   & \num{1.2}  & \num{4}   & \num{0.6}  & \num{5}   & \num{1.1}  & \num{1}   & \num{0.6}  \\
& Valid response & \num{1857} & \num{98.8} & \num{638} & \num{99.4} & \num{441} & \num{98.9} & \num{172} & \num{99.4} \\
Phlegm                  & Missing        & \num{23}   & \num{1.2}  & \num{4}   & \num{0.6}  & \num{5}   & \num{1.1}  & \num{1}   & \num{0.6}  \\
& Valid response & \num{1857} & \num{98.8} & \num{638} & \num{99.4} & \num{441} & \num{98.9} & \num{172} & \num{99.4} \\
Wheezing                & Missing        & \num{23}   & \num{1.2}  & \num{4}   & \num{0.6}  & \num{5}   & \num{1.1}  & \num{1}   & \num{0.6}  \\
& Valid response & \num{1857} & \num{98.8} & \num{638} & \num{99.4} & \num{441} & \num{98.9} & \num{172} & \num{99.4} \\
Shortness of breath     & Missing        & \num{24}   & \num{1.3}  & \num{4}   & \num{0.6}  & \num{5}   & \num{1.1}  & \num{2}   & \num{1.2}  \\
& Valid response & \num{1856} & \num{98.7} & \num{638} & \num{99.4} & \num{441} & \num{98.9} & \num{171} & \num{98.8} \\
Chest trouble           & Missing        & \num{24}   & \num{1.3}  & \num{4}   & \num{0.6}  & \num{5}   & \num{1.1}  & \num{1}   & \num{0.6}  \\
& Valid response & \num{1856} & \num{98.7} & \num{638} & \num{99.4} & \num{441} & \num{98.9} & \num{172} & \num{99.4} \\
Blood pressure:         &                &             &             &            &             &            &             &            &             \\
Brachial SBP            & Missing        & \num{30}   & \num{1.6}  & \num{11}  & \num{1.7}  & \num{17}  & \num{3.8}  & \num{1}   & \num{0.6}  \\
& Valid response & \num{1850} & \num{98.4} & \num{631} & \num{98.3} & \num{429} & \num{96.2} & \num{172} & \num{99.4} \\
Central SBP             & Missing        & \num{30}   & \num{1.6}  & \num{12}  & \num{1.9}  & \num{17}  & \num{3.8}  & \num{1}   & \num{0.6}  \\
& Valid response & \num{1850} & \num{98.4} & \num{630} & \num{98.1} & \num{429} & \num{96.2} & \num{172} & \num{99.4} \\
Brachial DBP            & Missing        & \num{30}   & \num{1.6}  & \num{11}  & \num{1.7}  & \num{17}  & \num{3.8}  & \num{1}   & \num{0.6}  \\
& Valid response & \num{1850} & \num{98.4} & \num{631} & \num{98.3} & \num{429} & \num{96.2} & \num{172} & \num{99.4} \\
Central DBP             & Missing        & \num{30}   & \num{1.6}  & \num{12}  & \num{1.9}  & \num{17}  & \num{3.8}  & \num{1}   & \num{0.6}  \\
& Valid response & \num{1850} & \num{98.4} & \num{630} & \num{98.1} & \num{429} & \num{96.2} & \num{172} & \num{99.4} \\
Biomarkers:             &                &             &             &            &             &            &             &            &             \\
IL6                     & Missing        & 8           & 0.6         & 4          & 0.9         & 1          & 0.3         & 0          & 0           \\
TNF                     & Missing        & 8           & 0.6         & 4          & 0.9         & 1          & 0.3         & 0          & 0           \\
CRP                     & Missing        & 8           & 0.6         & 4          & 0.9         & 1          & 0.3         & 0          & 0           \\
MDA                     & Missing        & 11          & 0.9         & 4          & 0.9         & 1          & 0.3         & 0          & 0           \\
FeNO                    & Missing        & \num{15}   & \num{0.8}  & \num{0}   & \num{0.0}  & \num{0}   & \num{0.0}  & \num{0}   & \num{0.0}  \\
& Not sampled    & \num{1341} & \num{71.3} & \num{448} & \num{69.8} & \num{383} & \num{85.9} & \num{111} & \num{64.2} \\
& Valid response & \num{524}  & \num{27.9} & \num{194} & \num{30.2} & \num{63}  & \num{14.1} & \num{62}  & \num{35.8} \\
Environmental outcomes: &                &             &             &            &             &            &             &            &             \\
Personal PM             & Missing        & \num{13}   & \num{0.7}  & \num{4}   & \num{0.6}  & \num{6}   & \num{1.3}  & \num{0}   & \num{0.0}  \\
& Not sampled    & \num{984}  & \num{52.3} & \num{343} & \num{53.4} & \num{239} & \num{53.6} & \num{84}  & \num{48.6} \\
& Valid response & \num{883}  & \num{47.0} & \num{295} & \num{46.0} & \num{201} & \num{45.1} & \num{89}  & \num{51.4} \\
Indoor PM               & Missing        & \num{60}   & \num{3.2}  & \num{11}  & \num{1.7}  & \num{16}  & \num{3.6}  & \num{4}   & \num{2.3}  \\
& Not sampled    & \num{1502} & \num{79.9} & \num{518} & \num{80.7} & \num{360} & \num{80.7} & \num{135} & \num{78.0} \\
& Valid response & \num{318}  & \num{16.9} & \num{113} & \num{17.6} & \num{70}  & \num{15.7} & \num{34}  & \num{19.7} \\
Indoor temperature      & Missing        & \num{33}   & \num{1.8}  & \num{10}  & \num{1.6}  & \num{16}  & \num{3.6}  & \num{1}   & \num{0.6}  \\
& Valid response & \num{1847} & \num{98.2} & \num{632} & \num{98.4} & \num{430} & \num{96.4} & \num{172} & \num{99.4} \\
\bottomrule
\end{tblr}
}

}

\end{table}%DIF > 

\newpage

\subsubsection{\DIFadd{Imputation results}}\DIFaddend \label{imputation-results}

The figures below show density plots for the values of body mass index,
waist circumference, and indoor PM\textsubscript{2.5} from the multiple
imputation models. The red lines show the values for each of the 30
imputed datasets, and the black line shows the value for the observed
data.

\begin{figure}[H]

\caption{\label{fig-afig-mi}Kernal density plots showing distribution of
multiply-imputed values for body mass index (kg/m\textsuperscript{2}),
waist circumference (cm), and indoor PM\textsubscript{2.5}
(µg/m\textsuperscript{3}) (red lines) and observed values (heavy black
line)\DIFaddbeginFL \DIFaddFL{.}\DIFaddendFL }

\begin{minipage}{0.50\linewidth}

\DIFdelbeginFL %DIFDELCMD < \centering{
%DIFDELCMD < 

%DIFDELCMD < \includegraphics[width=0.5\textwidth,height=\textheight]{images/MI_BMI_density.png}
%DIFDELCMD < 

%DIFDELCMD < }
%DIFDELCMD < %%%
\DIFdelendFL \DIFaddbeginFL \centering{

\includegraphics[width=0.9\textwidth,height=\textheight]{images/MI_BMI_density.png}

}
\DIFaddendFL 

\subcaption{\label{fig-afig-mi-1}}

\end{minipage}%
%
\begin{minipage}{0.50\linewidth}

\DIFdelbeginFL %DIFDELCMD < \centering{
%DIFDELCMD < 

%DIFDELCMD < \includegraphics[width=0.5\textwidth,height=\textheight]{images/MI_waist_circ_density.png}
%DIFDELCMD < 

%DIFDELCMD < }
%DIFDELCMD < %%%
\DIFdelendFL \DIFaddbeginFL \centering{

\includegraphics[width=0.9\textwidth,height=\textheight]{images/MI_waist_circ_density.png}

}
\DIFaddendFL 

\subcaption{\label{fig-afig-mi-2}}

\end{minipage}%
\newline
\begin{minipage}{0.50\linewidth}

\DIFdelbeginFL %DIFDELCMD < \centering{
%DIFDELCMD < 

%DIFDELCMD < \includegraphics[width=0.5\textwidth,height=\textheight]{images/MI_indoorPM_density.png}
%DIFDELCMD < 

%DIFDELCMD < }
%DIFDELCMD < %%%
\DIFdelendFL \DIFaddbeginFL \centering{

\includegraphics[width=0.9\textwidth,height=\textheight]{images/MI_indoorPM_density.png}

}
\DIFaddendFL 

\subcaption{\label{fig-afig-mi-3}}

\end{minipage}%

\end{figure}%

\newpage

\subsection{Participant flow diagram}\label{participant-flow-diagram}

\begin{figure}[H]

\DIFdelbeginFL %DIFDELCMD < \centering{
%DIFDELCMD < 

%DIFDELCMD < \includegraphics[width=0.8\textwidth,height=\textheight]{images/Participation-flowchart-Apr12.png}
%DIFDELCMD < 

%DIFDELCMD < }
%DIFDELCMD < 

%DIFDELCMD < %%%
\DIFdelendFL \caption{\label{fig-flowchart}Flow chart of BHET study participation at
the participant, household, and village levels across study years.}

\DIFaddbeginFL \centering{

\includegraphics[width=0.8\textwidth,height=\textheight]{images/Participation-flowchart-Apr12.png}

}

\DIFaddendFL \end{figure}%

\newpage

\DIFaddbegin \subsection{\DIFadd{Sample sizes}}\label{sample-sizes}

\begin{table}[H]

\caption{\label{tbl-samples}\DIFaddFL{Sample sizes for health and environmental
measurements for participants (P), households (HH), and villages (V).}}

\centering{

\centering
\resizebox{\ifdim\width>\linewidth 0.95\linewidth\else\width\fi}{!}{
\begin{talltblr}[         %% tabularray outer open
entry=none,label=none,
note{}={Note: P = Participants, HH = Households, V = Villages},
note{a}={Sample size for seasonal measurements.},
note{b}={Measured in the 5 minutes before blood pressure.},
]                     %% tabularray outer close
{                     %% tabularray inner open
colspec={Q[]Q[]Q[]Q[]Q[]Q[]Q[]Q[]Q[]Q[]Q[]Q[]Q[]Q[]Q[]Q[]},
cell{1}{2}={c=3,}{halign=c,},
cell{1}{5}={c=3,}{halign=c,},
cell{1}{8}={c=3,}{halign=c,},
cell{1}{11}={c=3,}{halign=c,},
cell{1}{14}={c=3,}{halign=c,},
cell{4}{1}={c=16}{},cell{8}{1}={c=16}{},cell{13}{1}={c=16}{},cell{16}{1}={c=16}{},cell{20}{1}={c=16}{},cell{24}{1}={c=16}{},
cell{3}{1}={preto={\hspace{1em}}},
cell{5}{1}={preto={\hspace{1em}}},
cell{6}{1}={preto={\hspace{1em}}},
cell{7}{1}={preto={\hspace{1em}}},
cell{9}{1}={preto={\hspace{1em}}},
cell{10}{1}={preto={\hspace{1em}}},
cell{11}{1}={preto={\hspace{1em}}},
cell{12}{1}={preto={\hspace{1em}}},
cell{14}{1}={preto={\hspace{1em}}},
cell{15}{1}={preto={\hspace{1em}}},
cell{17}{1}={preto={\hspace{1em}}},
cell{18}{1}={preto={\hspace{1em}}},
cell{19}{1}={preto={\hspace{1em}}},
cell{21}{1}={preto={\hspace{1em}}},
cell{22}{1}={preto={\hspace{1em}}},
cell{23}{1}={preto={\hspace{1em}}},
cell{25}{1}={preto={\hspace{1em}}},
cell{26}{1}={preto={\hspace{1em}}},
cell{27}{1}={preto={\hspace{1em}}},
row{4}={,cmd=\bfseries,},
row{8}={,cmd=\bfseries,},
row{13}={,cmd=\bfseries,},
row{16}={,cmd=\bfseries,},
row{20}={,cmd=\bfseries,},
row{24}={,cmd=\bfseries,},
}                     %% tabularray inner close
\toprule
& All waves &  &  & Wave 1 &  &  & Wave 2 &  &  & Wave 3 &  &  & Wave 4 &  &  \\ \cmidrule[lr]{2-4}\cmidrule[lr]{5-7}\cmidrule[lr]{8-10}\cmidrule[lr]{11-13}\cmidrule[lr]{14-16}
Outcome & P & HH & V & P & HH & V & P & HH & V & P & HH & V & P & HH & V \\ \midrule %% TinyTableHeader
Total                 & 1438 & 1236 & 50 & 1003 & 977 & 50 & 1110 & 1055 & 50 & . & 530 & 41 & 1028 & 1012 & 50 \\
Health Measures &&&&&&&&&&&&&&& \\
BP                    & 1423 & .    & .  & 975  & .   & .  & 1103 & .    & .  & . & .   & .  & 1004 & .    & .  \\
Respiratory symptoms  & 1429 & .    & .  & 991  & .   & .  & 1107 & .    & .  & . & .   & .  & 1010 & .    & .  \\
FeNO                  & 511  & .    & .  & 268  & .   & .  & 323  & .    & .  & . & .   & .  & 252  & .    & .  \\
Inflammatory biomarkers &&&&&&&&&&&&&&& \\
IL6                   & 1064 & .    & .  & 732  & .   & .  & 874  & .    & .  & . & .   & .  & .    & .    & .  \\
TNF                   & 1064 & .    & .  & 732  & .   & .  & 874  & .    & .  & . & .   & .  & .    & .    & .  \\
CRP                   & 1064 & .    & .  & 732  & .   & .  & 874  & .    & .  & . & .   & .  & .    & .    & .  \\
MDA                   & 1064 & .    & .  & 729  & .   & .  & 874  & .    & .  & . & .   & .  & .    & .    & .  \\
Personal air pollution &&&&&&&&&&&&&&& \\
Filter-derived PM2.5  & 761  & .    & .  & 489  & .   & .  & 485  & .    & .  & . & .   & .  & 494  & .    & .  \\
Filter-derived BC     & 755  & .    & .  & 489  & .   & .  & 478  & .    & .  & . & .   & .  & 476  & .    & .  \\
Indoor air pollution &&&&&&&&&&&&&&& \\
Sensor-derived PM2.5 \textsuperscript{a} & .    & 330  & .  & .    & .   & .  & .    & 268  & .  & . & 246 & .  & .    & 245  & .  \\
Filter-derived PM2.5  & .    & 177  & .  & .    & .   & .  & .    & 147  & .  & . & .   & .  & .    & 148  & .  \\
Filter-derived BC     & .    & 176  & .  & .    & .   & .  & .    & 146  & .  & . & .   & .  & .    & 138  & .  \\
Outdoor air pollution &&&&&&&&&&&&&&& \\
Sensor-derived PM2.5 \textsuperscript{a} & .    & .    & 50 & .    & .   & 40 & .    & .    & 50 & . & .   & 50 & .    & .    & 50 \\
Filter-derived PM2.5  & .    & .    & 50 & .    & .   & 44 & .    & .    & 50 & . & .   & 50 & .    & .    & 50 \\
Filter-derived BC     & .    & .    & 50 & .    & .   & 44 & .    & .    & 50 & . & .   & 50 & .    & .    & 50 \\
Indoor temperature &&&&&&&&&&&&&&& \\
'Point' temperature  \textsuperscript{b} & .    & 1228 & .  & .    & 956 & .  & .    & 1050 & .  & . & .   & .  & .    & 1001 & .  \\
Long-term temperature & .    & 753  & .  & .    & 366 & .  & .    & 557  & .  & . & 454 & .  & .    & 458  & .  \\
\bottomrule
\end{talltblr}
}

}

\end{table}%DIF > 

\DIFaddend \newpage

\DIFaddbegin \subsection{\DIFadd{District-level statistics}}\label{district-level-statistics}

\begin{table}[H]

\caption{\label{tbl-dist-stats}\DIFaddFL{Descriptive characteristics by district.}}

\centering{

\centering
\begin{talltblr}[         %% tabularray outer open
entry=none,label=none,
note{}={Note: Number of villages given in parenthesis. SD = Standard deviation},
]                     %% tabularray outer close
{                     %% tabularray inner open
width={1\linewidth},
colspec={X[0.4]X[0.075]X[0.075]X[0.075]X[0.075]X[0.075]X[0.075]X[0.075]X[0.075]},
cell{1}{2}={c=2,}{halign=c,},
cell{1}{4}={c=2,}{halign=c,},
cell{1}{6}={c=2,}{halign=c,},
cell{1}{8}={c=2,}{halign=c,},
row{1}={,cmd=\bfseries,},
column{1}={halign=l,},
}                     %% tabularray inner close
\toprule
& Fangshan (n=11) &  & Huairo (n=18) &  & Miyun (n=12) &  & Mentougou (n=9) &  \\ \cmidrule[lr]{2-3}\cmidrule[lr]{4-5}\cmidrule[lr]{6-7}\cmidrule[lr]{8-9}
& Mean & SD & Mean & SD & Mean & SD & Mean & SD \\ \midrule %% TinyTableHeader
Number of households & 699.1 & 514.4 & 163.1 & 111.6 & 274.4 & 198.8 & 204.3 &   73.20 \\
Per capita income (RMB, 1000s) & 7.2 & 1.4 & 20.0 & 2.8 & 17.3 & 2.8 & 11.3 &    2.30 \\
Distance to Beijing center (km) & 67.2 & 2.3 & 88.6 & 8.9 & 83.0 & 3.7 & 45.0 &    5.70 \\
Altitude (m) & 146.0 & 36.9 & 353.9 & 121.3 & 283.1 & 85.9 & 312.4 &  133.90 \\
Winter briquette quantity (tonnes) & 3.7 & 1.2 & 4.0 & 1.6 & 2.8 & 1.2 & 2.4 &    1.00 \\
Winter wood quantity (kilograms) & 1180.0 & 1192.0 & 2411.0 & 3845.0 & 2309.0 & 1997.0 & 1493.0 & 3144.00 \\
Participant age (years) & 59.0 & 9.1 & 60.0 & 8.8 & 60.0 & 8.9 & 62.0 &    9.70 \\
Education (0-None or primary school; 1-secondary +) & 0.4 & 0.5 & 0.3 & 0.5 & 0.3 & 0.4 & 0.3 &    0.46 \\
Participant weight (kg) & 69.0 & 11.0 & 66.0 & 11.0 & 65.0 & 10.0 & 68.0 &   11.00 \\
Participant height (cm) & 161.0 & 8.6 & 161.0 & 8.2 & 159.0 & 8.4 & 161.0 &    8.60 \\
\bottomrule
\end{talltblr}

}

\end{table}%DIF > 

\newpage

\DIFaddend \subsection{Policy uptake}\label{policy-uptake-1}

\DIFdelbegin \DIFdel{Figure~\ref{fig-afig-coal} shows trends over time in self-reported coal
and biomass consumption over each season. }\DIFdelend Table~\ref{tbl-fuel-did} shows results from applying our extended
two-way fixed effects models (in separate analyses) to coal and biomass
consumption.

\DIFdelbegin %DIFDELCMD < \begin{figure}[H]
%DIFDELCMD < %%%
\DIFdelendFL \DIFaddbeginFL \begin{table}[H]
\DIFaddendFL 

\DIFdelbeginFL %DIFDELCMD < \centering{
%DIFDELCMD < 

%DIFDELCMD < \includegraphics[width=1\textwidth,height=\textheight]{images/coal-plot.png}
%DIFDELCMD < 

%DIFDELCMD < }
%DIFDELCMD < %%%
\DIFdelendFL \DIFaddbeginFL \caption{\label{tbl-fuel-did}\DIFaddFL{Policy impacts on self-reported fuel use
(kg)}}
\DIFaddendFL 

\DIFdelbeginFL %DIFDELCMD < \caption{%
{%DIFAUXCMD
%DIFDELCMD < \label{fig-afig-coal}%%%
\DIFdelFL{Trends in self-reported coal and biomass,
by treatment season}}
%DIFAUXCMD
\DIFdelendFL \DIFaddbeginFL \centering{

\centering
\begin{talltblr}[         %% tabularray outer open
entry=none,label=none,
note{}={Note: ATT = Average Treatment Effect on the Treated, CI = confidence interval},
note{a}={Joint test that all ATTs are equal: F(3, 2886)= 1.856, p= 0.135},
note{b}={Joint test that all ATTs are equal: F(3, 2886)= 5.545, p= 0.001},
]                     %% tabularray outer close
{                     %% tabularray inner open
colspec={Q[]Q[]Q[]Q[]Q[]Q[]},
cell{1}{3}={c=2,}{halign=c,},
cell{1}{5}={c=2,}{halign=c,},
cell{3}{1}={c=6}{},cell{5}{1}={c=6}{},
cell{4}{1}={preto={\hspace{1em}}},
cell{6}{1}={preto={\hspace{1em}}},
cell{7}{1}={preto={\hspace{1em}}},
cell{8}{1}={preto={\hspace{1em}}},
cell{9}{1}={preto={\hspace{1em}}},
cell{10}{1}={preto={\hspace{1em}}},
row{3}={,cmd=\bfseries,},
row{5}={,cmd=\bfseries,},
}                     %% tabularray inner close
\toprule
&  & Coal &  & Biomass &  \\ \cmidrule[lr]{3-4}\cmidrule[lr]{5-6}
Cohort & Time & ATT\textsuperscript{a} & (95\% CI) & ATT\textsuperscript{b} & (95\% CI) \\ \midrule %% TinyTableHeader
Average ATT &&&&& \\
All & All & -2361 & (-2677, -2044) & -487 & (-805, -168) \\
Cohort-Time ATTs &&&&& \\
2019 & 2019 & -2631 & (-2913, -2348) & -653 & (-991, -315) \\
2019 & 2021 & -2416 & (-2847, -1984) & -633 & (-1201, -64) \\
2020 & 2021 & -2018 & (-2474, -1562) & -350 & (-701, 0) \\
2021 & 2021 & -1961 & (-2895, -1027) &  338 & (-30, 705) \\
\bottomrule
\end{talltblr}

}
\DIFaddendFL 

\DIFdelbeginFL %DIFDELCMD < \end{figure}%%%
\DIFdelend \DIFaddbegin \end{table}\DIFaddend %

\newpage

\subsection{Heterogeneity in \DIFdelbegin \DIFdel{treament
}\DIFdelend \DIFaddbegin \DIFadd{treatment
}\DIFaddend effects}\DIFdelbegin %DIFDELCMD < \label{heterogeneity-in-treament-effects}
%DIFDELCMD < %%%
\DIFdelend \DIFaddbegin \label{heterogeneity-in-treatment-effects}
\DIFaddend 

\subsubsection{Personal exposure}\label{personal-exposure}

Table\DIFdelbegin \DIFdel{Table}\DIFdelend ~\ref{tbl-a-het-personal} shows limited evidence that the \(ATT\)s
across cohorts and time demonstrate meaningful heterogeneity.

\DIFaddbegin \begin{table}[H]

\caption{\label{tbl-a-het-personal}\DIFaddFL{Heterogenous treatment effects:
Personal exposures}}

\centering{

\centering
\begin{talltblr}[         %% tabularray outer open
entry=none,label=none,
note{}={Note: ATT = Average Treatment Effect on the Treated, CI = confidence interval},
note{a}={Joint test that all cohort-time ATTs are equal: F(3, 1260)= 0.501, p= 0.682},
note{b}={Joint test that all cohort-time ATTs are equal: F(3, 1151)= 0.965, p= 0.408},
]                     %% tabularray outer close
{                     %% tabularray inner open
colspec={Q[]Q[]Q[]Q[]Q[]Q[]},
cell{1}{3}={c=2,}{halign=c,},
cell{1}{5}={c=2,}{halign=c,},
cell{3}{1}={c=6}{},cell{5}{1}={c=6}{},
cell{4}{1}={preto={\hspace{1em}}},
cell{6}{1}={preto={\hspace{1em}}},
cell{7}{1}={preto={\hspace{1em}}},
cell{8}{1}={preto={\hspace{1em}}},
cell{9}{1}={preto={\hspace{1em}}},
cell{10}{1}={preto={\hspace{1em}}},
row{3}={,cmd=\bfseries,},
row{5}={,cmd=\bfseries,},
}                     %% tabularray inner close
\toprule
&  & PM2.5 &  & Black carbon &  \\ \cmidrule[lr]{3-4}\cmidrule[lr]{5-6}
Cohort & Time & ATT\textsuperscript{a} & (95\% CI) & ATT\textsuperscript{b} & (95\% CI) \\ \midrule %% TinyTableHeader
Average ATT &&&&& \\
All & All & 0.2 & (-19.6, 19.9) & -0.4 & (-1.5, 0.6) \\
Cohort-Time ATTs &&&&& \\
2019 & 2019 & 7.4 & (-24.4, 39.2) & -0.9 & (-1.9, 0.2) \\
2019 & 2021 & -13.0 & (-36.6, 10.6) & -0.4 & (-2.0, 1.2) \\
2020 & 2021 & 15.7 & (-34.8, 66.3) & -0.1 & (-1.8, 1.5) \\
2021 & 2021 & -13.7 & (-36.6, 9.3) & 0.1 & (-1.4, 1.6) \\
\bottomrule
\end{talltblr}

}

\end{table}%DIF > 

\newpage

\DIFaddend \subsubsection{\texorpdfstring{Indoor
PM\textsubscript{2.5}}{Indoor PM2.5}}\label{indoor-pm2.5-1}

Table Table~\ref{tbl-a-het-indoor} shows estimates for cohort-time
\(ATT\)s for \DIFdelbegin \DIFdel{daily }\DIFdelend \DIFaddbegin \DIFadd{24-hr }\DIFaddend and seasonal indoor PM\textsubscript{2.5}.

\DIFaddbegin \begin{table}[H]

\caption{\label{tbl-a-het-indoor}\DIFaddFL{Heterogenous treatment effects for
Indoor PM\textsubscript{2.5}.}}

\centering{

\centering
\begin{talltblr}[         %% tabularray outer open
entry=none,label=none,
note{}={Note: ATT = Average Treatment Effect on the Treated, CI = confidence interval},
note{a}={Joint test that all ATTs are equal: F(1, 393)= 0.057, p= 0.811},
note{b}={Joint test that all ATTs are equal: F(1, 360)= 0.675, p= 0.412},
]                     %% tabularray outer close
{                     %% tabularray inner open
colspec={Q[]Q[]Q[]Q[]Q[]Q[]},
cell{1}{3}={c=2,}{halign=c,},
cell{1}{5}={c=2,}{halign=c,},
cell{3}{1}={c=6}{},cell{5}{1}={c=6}{},
cell{4}{1}={preto={\hspace{1em}}},
cell{6}{1}={preto={\hspace{1em}}},
cell{7}{1}={preto={\hspace{1em}}},
cell{8}{1}={preto={\hspace{1em}}},
row{3}={,cmd=\bfseries,},
row{5}={,cmd=\bfseries,},
}                     %% tabularray inner close
\toprule
&  & 24-hr &  & Seasonal &  \\ \cmidrule[lr]{3-4}\cmidrule[lr]{5-6}
Cohort & Time & ATT\textsuperscript{a} & (95\% CI) & ATT\textsuperscript{b} & (95\% CI) \\ \midrule %% TinyTableHeader
Average ATT &&&&& \\
All & All & -20.0 & (-45.6, 5.5) & -20.3 & (-37.5, -3.0) \\
Cohort-Time ATTs &&&&& \\
2020 & 2021 & -13.3 & (-45.9, 19.3) & -17.6 & (-39.6, 4.4) \\
2021 & 2021 & -35.9 & (-58.9, -13.0) & -25.9 & (-42.8, -9.0) \\
\bottomrule
\end{talltblr}

}

\end{table}%DIF > 

\DIFaddend \newpage

\subsubsection{Indoor temperature}\label{indoor-temperature}

\DIFaddbegin \begin{table}[H]

\caption{\label{tbl-a-het-temp}\DIFaddFL{Heterogeneous treatment effects for
indoor temperature}}

\centering{

\centering
\begin{talltblr}[         %% tabularray outer open
entry=none,label=none,
note{}={Note: ATT = Average Treatment Effect on the Treated, CI = confidence interval},
note{a}={P-value for omnibus test of heterogeneity across cohort time groups.},
]                     %% tabularray outer close
{                     %% tabularray inner open
colspec={Q[]Q[]Q[]Q[]Q[]Q[]Q[]Q[]Q[]Q[]Q[]Q[]Q[]},
cell{1}{2}={c=4,}{halign=c,},
cell{1}{6}={c=4,}{halign=c,},
cell{1}{10}={c=4,}{halign=c,},
cell{3}{1}={c=13}{},cell{8}{1}={c=13}{},cell{13}{1}={c=13}{},cell{18}{1}={c=13}{},
cell{4}{1}={preto={\hspace{1em}}},
cell{5}{1}={preto={\hspace{1em}}},
cell{6}{1}={preto={\hspace{1em}}},
cell{7}{1}={preto={\hspace{1em}}},
cell{9}{1}={preto={\hspace{1em}}},
cell{10}{1}={preto={\hspace{1em}}},
cell{11}{1}={preto={\hspace{1em}}},
cell{12}{1}={preto={\hspace{1em}}},
cell{14}{1}={preto={\hspace{1em}}},
cell{15}{1}={preto={\hspace{1em}}},
cell{16}{1}={preto={\hspace{1em}}},
cell{17}{1}={preto={\hspace{1em}}},
cell{19}{1}={preto={\hspace{1em}}},
cell{20}{1}={preto={\hspace{1em}}},
cell{21}{1}={preto={\hspace{1em}}},
cell{22}{1}={preto={\hspace{1em}}},
cell{23}{1}={preto={\hspace{1em}}},
row{3}={halign=l,cmd=\bfseries,},
row{8}={halign=l,cmd=\bfseries,},
row{13}={halign=l,cmd=\bfseries,},
row{18}={halign=l,cmd=\bfseries,},
column{1}={font=\fontsize{0.8em}{1.1em}\selectfont,},
column{2}={font=\fontsize{0.8em}{1.1em}\selectfont,},
column{3}={font=\fontsize{0.8em}{1.1em}\selectfont,},
column{4}={font=\fontsize{0.8em}{1.1em}\selectfont,},
column{5}={font=\fontsize{0.8em}{1.1em}\selectfont,},
column{6}={font=\fontsize{0.8em}{1.1em}\selectfont,},
column{7}={font=\fontsize{0.8em}{1.1em}\selectfont,},
column{8}={font=\fontsize{0.8em}{1.1em}\selectfont,},
column{9}={font=\fontsize{0.8em}{1.1em}\selectfont,},
column{10}={font=\fontsize{0.8em}{1.1em}\selectfont,},
column{11}={font=\fontsize{0.8em}{1.1em}\selectfont,},
column{12}={font=\fontsize{0.8em}{1.1em}\selectfont,},
column{13}={font=\fontsize{0.8em}{1.1em}\selectfont,},
column{1}={halign=l,},
}                     %% tabularray inner close
\toprule
& Point temp (°C) &  &  &  & Mean temp (°C) &  &  &  & Min temp (°C) &  &  &  \\ \cmidrule[lr]{2-5}\cmidrule[lr]{6-9}\cmidrule[lr]{10-13}
Cohort Time & ATT & (95\%CI) & p\textsuperscript{a} & Obs & ATT & (95\%CI) & p\textsuperscript{a} & Obs & ATT & (95\% CI) & p\textsuperscript{a} & Obs \\ \midrule %% TinyTableHeader
All times &&&&&&&&&&&& \\
2019 2019 & 1.66 & (0.5, 2.8) &  &  & 0.33 & (-0.8, 1.5) &  &  & 1.96 & (0.5, 3.4) &  &  \\
2019 2021 & 2.17 & (0.5, 3.9) &  &  & 0.80 & (0.0, 1.6) &  &  & 5.04 & (2.3, 7.8) &  &  \\
2020 2021 & 2.39 & (0.7, 4.1) &  &  & 0.66 & (-0.5, 1.8) &  &  & 7.27 & (4.6, 9.9) &  &  \\
2021 2021 & 0.60 & (-1.2, 2.4) & 0.37 & 2999 & 1.60 & (0.5, 2.7) & 0.45 & 1350 & 2.37 & (0.2, 4.5) & 0 & 1350 \\
Daytime &&&&&&&&&&&& \\
2019 2019 &  &  &  &  & 0.36 & (-0.8, 1.5) &  &  &  &  &  &  \\
2019 2021 &  &  &  &  & 0.91 & (0.1, 1.7) &  &  &  &  &  &  \\
2020 2021 &  &  &  &  & 0.95 & (-0.2, 2.1) &  &  &  &  &  &  \\
2021 2021 &  &  &  &  & 1.67 & (0.5, 2.8) & 0.48 & 1346 &  &  &  &  \\
Daytime heating &&&&&&&&&&&& \\
2019 2019 &  &  &  &  & 0.92 & (-0.2, 2.0) &  &  & 1.94 & (0.5, 3.4) &  &  \\
2019 2021 &  &  &  &  & 2.02 & (0.9, 3.2) &  &  & 5.46 & (2.7, 8.2) &  &  \\
2020 2021 &  &  &  &  & 2.63 & (1.5, 3.7) &  &  & 6.69 & (4.2, 9.2) &  &  \\
2021 2021 &  &  &  &  & 2.72 & (0.8, 4.6) & 0 & 1350 & 2.53 & (0.4, 4.7) & 0 & 1350 \\
Heating season &&&&&&&&&&&& \\
2019 2019 &  &  &  &  & 0.95 & (-0.2, 2.1) &  &  &  &  &  &  \\
2019 2021 &  &  &  &  & 2.18 & (0.9, 3.4) &  &  &  &  &  &  \\
2020 2021 &  &  &  &  & 2.97 & (1.9, 4.0) &  &  &  &  &  &  \\
2021 2021 &  &  &  &  & 2.80 & (0.9, 4.7) & 0 & 1346 &  &  &  &  \\
\bottomrule
\end{talltblr}

}

\end{table}%DIF > 

\DIFaddend \newpage

\subsubsection{Blood pressure outcomes}\label{blood-pressure-outcomes}

Table~\ref{tbl-bp-het} shows \(ATT\)s by treatment cohort and time, as
well as the results of joint tests of heterogeneity across \(ATT\)s.

\DIFaddbegin \begin{table}

\caption{\label{tbl-bp-het}\DIFaddFL{Heterogeneous treatment effects for the total
effect of the CHP on blood pressure.}}

\centering{

\centering
\begin{talltblr}[         %% tabularray outer open
entry=none,label=none,
note{}={Note: ATT = Average Treatment Effect on the Treated, CI = confidence interval, DiD = Difference-in-Differences.},
note{a}={Adjusted for age, sex, waist circumference, smoking, alcohol consumption, and use of blood pressure medication.},
note{b}={F-statistics and p-values for joint tests of equality across cohort and time ATTs},
]                     %% tabularray outer close
{                     %% tabularray inner open
width={1\linewidth},
colspec={X[0.16]X[0.16]X[0.16]X[0.2]X[0.16]X[0.16]},
cell{1}{3}={c=2,}{halign=c,},
cell{1}{5}={c=2,}{halign=c,},
cell{3}{1}={c=6}{},cell{8}{1}={c=6}{},cell{13}{1}={c=6}{},cell{18}{1}={c=6}{},
cell{4}{1}={preto={\hspace{1em}}},
cell{5}{1}={preto={\hspace{1em}}},
cell{6}{1}={preto={\hspace{1em}}},
cell{7}{1}={preto={\hspace{1em}}},
cell{9}{1}={preto={\hspace{1em}}},
cell{10}{1}={preto={\hspace{1em}}},
cell{11}{1}={preto={\hspace{1em}}},
cell{12}{1}={preto={\hspace{1em}}},
cell{14}{1}={preto={\hspace{1em}}},
cell{15}{1}={preto={\hspace{1em}}},
cell{16}{1}={preto={\hspace{1em}}},
cell{17}{1}={preto={\hspace{1em}}},
cell{19}{1}={preto={\hspace{1em}}},
cell{20}{1}={preto={\hspace{1em}}},
cell{21}{1}={preto={\hspace{1em}}},
cell{22}{1}={preto={\hspace{1em}}},
cell{23}{1}={preto={\hspace{1em}}},
row{3}={,cmd=\bfseries,},
row{8}={,cmd=\bfseries,},
row{13}={,cmd=\bfseries,},
row{18}={,cmd=\bfseries,},
}                     %% tabularray inner close
\toprule
&  & Adjusted DiD &  & Heterogeneity tests &  \\ \cmidrule[lr]{3-4}\cmidrule[lr]{5-6}
Cohort & Time & ATT\textsuperscript{a} & (95\% CI) & F-Statistic\textsuperscript{b} & p-value \\ \midrule %% TinyTableHeader
Brachial SBP &&&&& \\
2019 & 2019 & -2.4 & (-5.2, 0.5) &  &  \\
2019 & 2021 & -1.5 & (-4.0, 1.0) &  &  \\
2020 & 2021 & -1.3 & (-5.0, 2.5) &  &  \\
2021 & 2021 & 2.4 & (-0.5, 5.3) & 2.3 & 0.080 \\
Central SBP &&&&& \\
2019 & 2019 & -2.0 & (-4.7, 0.6) &  &  \\
2019 & 2021 & -2.0 & (-4.5, 0.5) &  &  \\
2020 & 2021 & -1.8 & (-5.1, 1.5) &  &  \\
2021 & 2021 & 2.1 & (-1.1, 5.3) & 1.9 & 0.140 \\
Brachial DBP &&&&& \\
2019 & 2019 & -2.7 & (-4.7, -0.7) &  &  \\
2019 & 2021 & -2.4 & (-4.0, -0.7) &  &  \\
2020 & 2021 & 0.2 & (-1.5, 1.9) &  &  \\
2021 & 2021 & 0.8 & (-0.5, 2.0) & 6.8 & 0.000 \\
Central DBP &&&&& \\
2019 & 2019 & -2.7 & (-4.6, -0.8) &  &  \\
2019 & 2021 & -2.5 & (-4.2, -0.9) &  &  \\
2020 & 2021 & 0.1 & (-1.7, 1.9) &  &  \\
2021 & 2021 & 1.1 & (-0.1, 2.2) & 10.0 & 0.000 \\
\bottomrule
\end{talltblr}

}

\end{table}%DIF > 

\DIFaddend \newpage

\subsubsection{Mediation analyses for blood
pressure}\label{mediation-analyses-for-blood-pressure}

Table~\ref{tbl-a-bp-med-het} shows the cohort-time treatment effects for
the mediation model for blood pressure.

\DIFaddbegin \begin{table}[H]

\caption{\label{tbl-a-bp-med-het}\DIFaddFL{Heterogeneous treatment effects for
blood pressure mediation model.}}

\centering{

\centering
\begin{talltblr}[         %% tabularray outer open
entry=none,label=none,
note{}={Note: Results combined across 30 multiply-imputed datasets. ATT = Average Treatment Effect on the Treated, CDE = Controlled Direct Effect, DBP = Diastolic blood pressure, SBP = Systolic blood pressure. Median p-values for heterogeneity tests for multiple mediation models: bSBP = 0.78, cSBP = 0.85, bDBP = 0.11, cDBP = 0.04.},
note{a}={Adjusted for age, sex, waist circumference, smoking, alcohol consumption, and use of blood pressure medication.},
note{b}={Mediators were set to the mean value for untreated participants at baseline.},
]                     %% tabularray outer close
{                     %% tabularray inner open
colspec={Q[]Q[]Q[]Q[]Q[]Q[]Q[]Q[]Q[]Q[]},
cell{2}{5}={c=2,}{halign=c,},
cell{2}{7}={c=2,}{halign=c,},
cell{2}{9}={c=2,}{halign=c,},
cell{1}{3}={c=2,}{halign=c,},
cell{1}{5}={c=6,}{halign=c,},
cell{4}{1}={c=10}{},cell{9}{1}={c=10}{},cell{14}{1}={c=10}{},cell{19}{1}={c=10}{},
cell{5}{1}={preto={\hspace{1em}}},
cell{6}{1}={preto={\hspace{1em}}},
cell{7}{1}={preto={\hspace{1em}}},
cell{8}{1}={preto={\hspace{1em}}},
cell{10}{1}={preto={\hspace{1em}}},
cell{11}{1}={preto={\hspace{1em}}},
cell{12}{1}={preto={\hspace{1em}}},
cell{13}{1}={preto={\hspace{1em}}},
cell{15}{1}={preto={\hspace{1em}}},
cell{16}{1}={preto={\hspace{1em}}},
cell{17}{1}={preto={\hspace{1em}}},
cell{18}{1}={preto={\hspace{1em}}},
cell{20}{1}={preto={\hspace{1em}}},
cell{21}{1}={preto={\hspace{1em}}},
cell{22}{1}={preto={\hspace{1em}}},
cell{23}{1}={preto={\hspace{1em}}},
cell{24}{1}={preto={\hspace{1em}}},
column{1}={halign=l,},
column{2}={halign=l,},
column{3}={halign=c,},
column{4}={halign=c,},
column{5}={halign=c,},
column{6}={halign=c,},
column{7}={halign=c,},
column{8}={halign=c,},
column{9}={halign=c,},
column{10}={halign=c,},
column{1}={font=\fontsize{0.8em}{1.1em}\selectfont,},
column{2}={font=\fontsize{0.8em}{1.1em}\selectfont,},
column{3}={font=\fontsize{0.8em}{1.1em}\selectfont,},
column{4}={font=\fontsize{0.8em}{1.1em}\selectfont,},
column{5}={font=\fontsize{0.8em}{1.1em}\selectfont,},
column{6}={font=\fontsize{0.8em}{1.1em}\selectfont,},
column{7}={font=\fontsize{0.8em}{1.1em}\selectfont,},
column{8}={font=\fontsize{0.8em}{1.1em}\selectfont,},
column{9}={font=\fontsize{0.8em}{1.1em}\selectfont,},
column{10}={font=\fontsize{0.8em}{1.1em}\selectfont,},
row{4}={,cmd=\bfseries,},
row{9}={,cmd=\bfseries,},
row{14}={,cmd=\bfseries,},
row{19}={,cmd=\bfseries,},
}                     %% tabularray inner close
\toprule
&  & Adjusted Total Effect &  & CDE Mediated By: &  &  &  &  &  \\ \cmidrule[lr]{3-4}\cmidrule[lr]{5-10}
&  &  &  & Indoor PM &  & Indoor Temp &  & PM + Temp &  \\ \cmidrule[lr]{5-6}\cmidrule[lr]{7-8}\cmidrule[lr]{9-10}
Cohort & Time & ATT\textsuperscript{a} & (95\% CI) & ATT\textsuperscript{b} & (95\% CI) & ATT\textsuperscript{b} & (95\% CI) & ATT\textsuperscript{b} & (95\% CI) \\ \midrule %% TinyTableHeader
Brachial SBP &&&&&&&&& \\
2019 & 2019 & -2.16 & (-5.0, 0.7) & -1.67 & (-4.7, 1.3) & -1.17 & (-4.1, 1.8) & -0.63 & (-3.7, 2.5) \\
2019 & 2021 & -1.54 & (-4.0, 1.0) & -0.86 & (-3.6, 1.9) & -0.13 & (-2.7, 2.5) & 0.56 & (-2.4, 3.5) \\
2020 & 2021 & -1.45 & (-5.2, 2.3) & -0.68 & (-4.6, 3.2) & -0.01 & (-3.6, 3.6) & 0.83 & (-3.2, 4.9) \\
2021 & 2021 & 2.28 & (-0.5, 5.1) & 2.48 & (-0.8, 5.8) & 1.55 & (-2.1, 5.2) & 1.76 & (-2.2, 5.7) \\
Central SBP &&&&&&&&& \\
2019 & 2019 & -1.81 & (-4.4, 0.8) & -1.21 & (-4.1, 1.7) & -0.83 & (-3.6, 1.9) & -0.19 & (-3.2, 2.8) \\
2019 & 2021 & -1.82 & (-4.3, 0.7) & -1.05 & (-3.9, 1.8) & -0.47 & (-2.8, 1.9) & 0.31 & (-2.5, 3.1) \\
2020 & 2021 & -1.85 & (-5.2, 1.5) & -1.12 & (-4.7, 2.5) & -0.52 & (-3.9, 2.8) & 0.28 & (-3.6, 4.2) \\
2021 & 2021 & 2.15 & (-1.0, 5.3) & 2.32 & (-1.2, 5.9) & 1.37 & (-2.1, 4.8) & 1.57 & (-2.2, 5.3) \\
Brachial DBP &&&&&&&&& \\
2019 & 2019 & -2.51 & (-4.5, -0.6) & -2.03 & (-4.2, 0.2) & -2.01 & (-3.9, -0.1) & -1.47 & (-3.6, 0.7) \\
2019 & 2021 & -2.34 & (-3.9, -0.7) & -1.71 & (-3.8, 0.4) & -1.74 & (-3.2, -0.3) & -1.08 & (-3.0, 0.9) \\
2020 & 2021 & 0.08 & (-1.7, 1.8) & 0.23 & (-1.6, 2.1) & 0.92 & (-1.0, 2.8) & 1.12 & (-1.0, 3.2) \\
2021 & 2021 & 0.73 & (-0.5, 1.9) & 1.08 & (-0.7, 2.8) & 0.11 & (-1.3, 1.5) & 0.47 & (-1.4, 2.3) \\
Central DBP &&&&&&&&& \\
2019 & 2019 & -2.54 & (-4.4, -0.7) & -1.97 & (-4.2, 0.2) & -2.26 & (-4.1, -0.4) & -1.62 & (-3.8, 0.5) \\
2019 & 2021 & -2.45 & (-4.0, -0.9) & -1.72 & (-3.8, 0.4) & -2.05 & (-3.5, -0.6) & -1.29 & (-3.3, 0.7) \\
2020 & 2021 & 0.08 & (-1.7, 1.9) & 0.23 & (-1.7, 2.2) & 0.94 & (-1.0, 2.9) & 1.14 & (-1.0, 3.3) \\
2021 & 2021 & 1.12 & (0.0, 2.2) & 1.51 & (-0.1, 3.2) & 0.48 & (-0.8, 1.8) & 0.88 & (-0.9, 2.7) \\
\bottomrule
\end{talltblr}

}

\end{table}%DIF > 

\DIFaddend \newpage

\subsubsection{\DIFdelbegin \DIFdel{Respiratory }\DIFdelend \DIFaddbegin \DIFadd{Self-reported respiratory
}\DIFaddend outcomes}\DIFdelbegin %DIFDELCMD < \label{respiratory-outcomes}
%DIFDELCMD < %%%
\DIFdelend \DIFaddbegin \label{self-reported-respiratory-outcomes}
\DIFaddend 

Appendix tables \ref{tbl-a-het-resp}, \ref{tbl-a-het-cough},
\ref{tbl-a-het-phlegm}, \ref{tbl-a-het-wheeze}, \ref{tbl-a-het-breath},
\ref{tbl-a-het-nochest} below show Average Treatment Effect on the
Treated (\(ATT\)s) by treatment cohort and time. ATTs are derived from
estimating marginal effects from extended two-way fixed effects models
with additional adjustment for age, sex, and smoking status.

\DIFaddbegin \begin{table}[H]

\caption{\label{tbl-a-het-resp}\DIFaddFL{Heterogenous treatment effects for
self-reported respiratory outcomes: Any respiratory symptom.}}

\centering{

\centering
\begin{talltblr}[         %% tabularray outer open
entry=none,label=none,
note{}={Note: Joint test that all ATTs are equal: F(3, 3050)= 0.998, p= 0.393.},
]                     %% tabularray outer close
{                     %% tabularray inner open
width={0.7\linewidth},
colspec={X[]X[]X[]X[]},
cell{2}{1}={c=4}{},cell{4}{1}={c=4}{},
cell{3}{1}={preto={\hspace{1em}}},
cell{5}{1}={preto={\hspace{1em}}},
cell{6}{1}={preto={\hspace{1em}}},
cell{7}{1}={preto={\hspace{1em}}},
cell{8}{1}={preto={\hspace{1em}}},
cell{9}{1}={preto={\hspace{1em}}},
row{2}={,cmd=\bfseries,},
row{4}={,cmd=\bfseries,},
}                     %% tabularray inner close
\toprule
Cohort & Time & ATT & (95\% CI) \\ \midrule %% TinyTableHeader
Average ATT &&& \\
All & All & -7.5 & (-12.7, -2.3) \\
Cohort-Time ATTs &&& \\
2019 & 2019 & -11.3 & (-18.4, -4.2) \\
2019 & 2021 & -9.3 & (-16.7, -1.9) \\
2020 & 2021 & 0.9 & (-10.8, 12.6) \\
2021 & 2021 & -6.7 & (-12.7, -0.7) \\
\bottomrule
\end{talltblr}

}

\end{table}%DIF > 

\begin{table}[H]

\caption{\label{tbl-a-het-cough}\DIFaddFL{Heterogenous treatment effects for
self-reported respiratory outcomes: Coughing.}}

\centering{

\centering
\begin{talltblr}[         %% tabularray outer open
entry=none,label=none,
note{}={Note: Joint test that all ATTs are equal: F(3, 3050)= 0.482, p= 0.695.},
]                     %% tabularray outer close
{                     %% tabularray inner open
width={0.7\linewidth},
colspec={X[]X[]X[]X[]},
cell{2}{1}={c=4}{},cell{4}{1}={c=4}{},
cell{3}{1}={preto={\hspace{1em}}},
cell{5}{1}={preto={\hspace{1em}}},
cell{6}{1}={preto={\hspace{1em}}},
cell{7}{1}={preto={\hspace{1em}}},
cell{8}{1}={preto={\hspace{1em}}},
cell{9}{1}={preto={\hspace{1em}}},
row{2}={,cmd=\bfseries,},
row{4}={,cmd=\bfseries,},
}                     %% tabularray inner close
\toprule
Cohort & Time & ATT & (95\% CI) \\ \midrule %% TinyTableHeader
Average ATT &&& \\
All & All & -2.7 & (-7.1, 1.7) \\
Cohort-Time ATTs &&& \\
2019 & 2019 & -4.9 & (-10.5, 0.7) \\
2019 & 2021 & -0.8 & (-8.1, 6.5) \\
2020 & 2021 & -2.2 & (-8.8, 4.3) \\
2021 & 2021 & -2.3 & (-10.0, 5.4) \\
\bottomrule
\end{talltblr}

}

\end{table}%DIF > 

\begin{table}

\caption{\label{tbl-a-het-phlegm}\DIFaddFL{Heterogenous treatment effects for
self-reported respiratory outcomes: Phlegm}}

\centering{

\centering
\begin{talltblr}[         %% tabularray outer open
entry=none,label=none,
note{}={Note: Joint test that all ATTs are equal: F(3, 3050)= 3.415, p= 0.017.},
]                     %% tabularray outer close
{                     %% tabularray inner open
width={0.7\linewidth},
colspec={X[]X[]X[]X[]},
cell{2}{1}={c=4}{},cell{4}{1}={c=4}{},
cell{3}{1}={preto={\hspace{1em}}},
cell{5}{1}={preto={\hspace{1em}}},
cell{6}{1}={preto={\hspace{1em}}},
cell{7}{1}={preto={\hspace{1em}}},
cell{8}{1}={preto={\hspace{1em}}},
cell{9}{1}={preto={\hspace{1em}}},
row{2}={,cmd=\bfseries,},
row{4}={,cmd=\bfseries,},
}                     %% tabularray inner close
\toprule
Cohort & Time & ATT & (95\% CI) \\ \midrule %% TinyTableHeader
Average ATT &&& \\
All & All & -1.6 & (-5.6, 2.4) \\
Cohort-Time ATTs &&& \\
2019 & 2019 & -5.3 & (-13.1, 2.6) \\
2019 & 2021 & -2.9 & (-8.8, 3.0) \\
2020 & 2021 & 3.1 & (-2.9, 9.1) \\
2021 & 2021 & 6.1 & (0.8, 11.5) \\
\bottomrule
\end{talltblr}

}

\end{table}%DIF > 

\begin{table}[H]

\caption{\label{tbl-a-het-wheeze}\DIFaddFL{Heterogenous treatment effects for
self-reported respiratory outcomes: Wheezing attacks.}}

\centering{

\centering
\begin{talltblr}[         %% tabularray outer open
entry=none,label=none,
note{}={Note: Joint test that all ATTs are equal: F(3, 3050)= 2.474, p= 0.06.},
]                     %% tabularray outer close
{                     %% tabularray inner open
width={0.7\linewidth},
colspec={X[]X[]X[]X[]},
cell{2}{1}={c=4}{},cell{4}{1}={c=4}{},
cell{3}{1}={preto={\hspace{1em}}},
cell{5}{1}={preto={\hspace{1em}}},
cell{6}{1}={preto={\hspace{1em}}},
cell{7}{1}={preto={\hspace{1em}}},
cell{8}{1}={preto={\hspace{1em}}},
cell{9}{1}={preto={\hspace{1em}}},
row{2}={,cmd=\bfseries,},
row{4}={,cmd=\bfseries,},
}                     %% tabularray inner close
\toprule
Cohort & Time & ATT & (95\% CI) \\ \midrule %% TinyTableHeader
Average ATT &&& \\
All & All & 1.0 & (-1.9, 3.9) \\
Cohort-Time ATTs &&& \\
2019 & 2019 & -0.9 & (-3.5, 1.8) \\
2019 & 2021 & 2.3 & (-1.7, 6.4) \\
2020 & 2021 & -1.0 & (-6.8, 4.7) \\
2021 & 2021 & 8.6 & (-2.3, 19.5) \\
\bottomrule
\end{talltblr}

}

\end{table}%DIF > 

\begin{table}

\caption{\label{tbl-a-het-breath}\DIFaddFL{Heterogenous treatment effects for
self-reported respiratory outcomes: Trouble breathing}}

\centering{

\centering
\begin{talltblr}[         %% tabularray outer open
entry=none,label=none,
note{}={Note: Joint test that all ATTs are equal: F(3, 3050)= 0.916, p= 0.432.},
]                     %% tabularray outer close
{                     %% tabularray inner open
width={0.7\linewidth},
colspec={X[]X[]X[]X[]},
cell{2}{1}={c=4}{},cell{4}{1}={c=4}{},
cell{3}{1}={preto={\hspace{1em}}},
cell{5}{1}={preto={\hspace{1em}}},
cell{6}{1}={preto={\hspace{1em}}},
cell{7}{1}={preto={\hspace{1em}}},
cell{8}{1}={preto={\hspace{1em}}},
cell{9}{1}={preto={\hspace{1em}}},
row{2}={,cmd=\bfseries,},
row{4}={,cmd=\bfseries,},
}                     %% tabularray inner close
\toprule
Cohort & Time & ATT & (95\% CI) \\ \midrule %% TinyTableHeader
Average ATT &&& \\
All & All & -3.4 & (-9.2, 2.4) \\
Cohort-Time ATTs &&& \\
2019 & 2019 & -5.0 & (-12.6, 2.6) \\
2019 & 2021 & -5.1 & (-13.1, 2.9) \\
2020 & 2021 & 2.4 & (-6.4, 11.1) \\
2021 & 2021 & -5.1 & (-18.7, 8.5) \\
\bottomrule
\end{talltblr}

}

\end{table}%DIF > 

\begin{table}

\caption{\label{tbl-a-het-nochest}\DIFaddFL{Heterogenous treatment effects for
self-reported respiratory outcomes: Chest trouble}}

\centering{

\centering
\begin{talltblr}[         %% tabularray outer open
entry=none,label=none,
note{}={Note: Joint test that all ATTs are equal: F(3, 3050)= 3.176, p= 0.023.},
]                     %% tabularray outer close
{                     %% tabularray inner open
width={0.7\linewidth},
colspec={X[]X[]X[]X[]},
cell{2}{1}={c=4}{},cell{4}{1}={c=4}{},
cell{3}{1}={preto={\hspace{1em}}},
cell{5}{1}={preto={\hspace{1em}}},
cell{6}{1}={preto={\hspace{1em}}},
cell{7}{1}={preto={\hspace{1em}}},
cell{8}{1}={preto={\hspace{1em}}},
cell{9}{1}={preto={\hspace{1em}}},
row{2}={,cmd=\bfseries,},
row{4}={,cmd=\bfseries,},
}                     %% tabularray inner close
\toprule
Cohort & Time & ATT & (95\% CI) \\ \midrule %% TinyTableHeader
Average ATT &&& \\
All & All & -3.4 & (-8.1, 1.3) \\
Cohort-Time ATTs &&& \\
2019 & 2019 & -2.7 & (-8.9, 3.5) \\
2019 & 2021 & -2.7 & (-10.2, 4.7) \\
2020 & 2021 & -2.0 & (-8.3, 4.2) \\
2021 & 2021 & -11.3 & (-17.5, -5.1) \\
\bottomrule
\end{talltblr}

}

\end{table}%DIF > 

\DIFaddend \newpage

\DIFaddbegin \subsubsection{\DIFadd{FeNO}}\label{feno}

\DIFadd{Figure~\ref{fig-feno-het} shows the cohort-time treatment effects for
FeNO for both basic and covariate-adjusted DiD models.
}

\begin{figure}[H]

\caption{\label{fig-feno-het}\DIFaddFL{Heterogenous treatment effects for FeNO.}}

\centering{

\includegraphics[width=0.9\textwidth,height=\textheight]{images/ETWFE cohort-time results.png}

}

\end{figure}%DIF > 

\newpage

\DIFaddend \subsubsection{Outdoor and personal mixed
combustion}\label{outdoor-and-personal-mixed-combustion}

\begin{figure}[H]

\DIFdelbeginFL %DIFDELCMD < \centering{
%DIFDELCMD < 

%DIFDELCMD < \includegraphics[width=1\textwidth,height=\textheight]{images/did-mixed-ct-wave.png}
%DIFDELCMD < 

%DIFDELCMD < }
%DIFDELCMD < %%%
\DIFdelendFL \DIFaddbeginFL \centering{

\includegraphics[width=0.9\textwidth,height=\textheight]{images/did-mixed-ct-wave.png}

}
\DIFaddendFL 

\caption{\label{fig-afig-mixed-ct}Adjusted and unadjusted treatment
effect for outdoor and personal exposure (µg/m\textsuperscript{3}) to
the mixed combustion source by treatment year.}

\end{figure}%

\newpage

\DIFdelbegin \subsection{\DIFdel{Impact of sample composition on FeNO
results}}%DIFAUXCMD
\addtocounter{subsection}{-1}%DIFAUXCMD
%DIFDELCMD < \label{impact-of-sample-composition-on-feno-results}
%DIFDELCMD < %%%
\DIFdelend \DIFaddbegin \subsection{\DIFadd{Impact of adjustment for district on air pollution
estimates}}\label{impact-of-adjustment-for-district-on-air-pollution-estimates}
\DIFaddend 

\DIFdelbegin \DIFdel{Table~\ref{tbl-a-feno} shows differences in the \(ATT\)s for
the impact
of the CBHP policy on FeNO depending on whether the estimation sample
includes all individuals or is limited to those with repeated measures
across campaigns.
}\DIFdelend \DIFaddbegin \begin{table}[H]
\DIFaddendFL 

\DIFaddbeginFL \caption{\label{tbl-dist-fe}\DIFaddFL{Impact of adding district-level fixed
effects to models for personal and indoor air pollution.}}

\centering{

\centering
\begin{talltblr}[         %% tabularray outer open
entry=none,label=none,
note{}={Note: (a) excludes and (b) includes district fixed effects. All models adjusted for household size, smoking, outdoor temperature, and outdoor dewpoint. Standard errors (in parenthesis) clustered by village.},
]                     %% tabularray outer close
{                     %% tabularray inner open
colspec={Q[]Q[]Q[]Q[]Q[]Q[]Q[]Q[]Q[]},
cell{1}{2}={c=2,}{halign=c,},
cell{1}{4}={c=2,}{halign=c,},
cell{1}{6}={c=2,}{halign=c,},
cell{1}{8}={c=2,}{halign=c,},
row{1}={,cmd=\bfseries,},
column{2}={halign=c,},
column{3}={halign=c,},
column{4}={halign=c,},
column{5}={halign=c,},
column{6}={halign=c,},
column{7}={halign=c,},
column{8}={halign=c,},
column{9}={halign=c,},
}                     %% tabularray inner close
\toprule
& Personal PM2.5 &  & Black carbon &  & 24-hr indoor &  & Seasonal indoor &  \\ \cmidrule[lr]{2-3}\cmidrule[lr]{4-5}\cmidrule[lr]{6-7}\cmidrule[lr]{8-9}
& (a) & (b) & (a)  & (b)  & (a)   & (b)   & (a)    & (b)    \\ \midrule %% TinyTableHeader
ATT & 0.2 & 0.0 & -0.4 & -0.3 & -20.0 & -21.8 & -20.3 & -22.0 \\
& (10.1) & (9.4) & (0.5) & (0.5) & (12.4) & (13.3) & (7.8) & (8.6) \\
Observations & 1,270 & 1,270 & 1,161 & 1,161 & 399 & 399 & 366 & 366 \\
Year fixed effects & X & X & X & X & X & X & X & X \\
Cohort fixed effects & X & X & X & X & X & X & X & X \\
District fixed effects &  & X &  & X &  & X &  & X \\
\bottomrule
\end{talltblr}

}

\end{table}%DIF > 

\DIFaddend \newpage

\DIFaddbegin \subsection{\DIFadd{Impact of sample composition on brachial and central blood
pressure
results.}}\label{impact-of-sample-composition-on-brachial-and-central-blood-pressure-results.}

\begin{table}[H]

\caption{\label{tbl-a-bp-sample}\DIFaddFL{Effects of excluding later-enrolled
participants on estimates of the CHP on systolic and diastolic blood
pressure.}}

\centering{

\centering
\begin{talltblr}[         %% tabularray outer open
entry=none,label=none,
note{}={Note: ATT = Average Treatment Effect on the Treated, BP = blood pressure, CI = confidence interval, DiD = Difference-in-Differences, ETWFE = Extended Two-Way Fixed Effects.},
note{a}={Marginal effect from ETWFE models adjusted for age, sex, waist circumference, smoking, alcohol consumption, and use of blood pressure medication. Results combined across 30 multiply-imputed datasets.},
]                     %% tabularray outer close
{                     %% tabularray inner open
width={1\linewidth},
colspec={X[0.3]X[0.2]X[0.2]X[0.1]X[0.2]},
cell{1}{3}={c=3,}{halign=c,},
cell{3}{1}={c=5}{},cell{8}{1}={c=5}{},
cell{4}{1}={preto={\hspace{1em}}},
cell{5}{1}={preto={\hspace{1em}}},
cell{6}{1}={preto={\hspace{1em}}},
cell{7}{1}={preto={\hspace{1em}}},
cell{9}{1}={preto={\hspace{1em}}},
cell{10}{1}={preto={\hspace{1em}}},
cell{11}{1}={preto={\hspace{1em}}},
cell{12}{1}={preto={\hspace{1em}}},
cell{13}{1}={preto={\hspace{1em}}},
cell{4}{1}={r=2,}{valign=h,},
cell{6}{1}={r=2,}{valign=h,},
cell{9}{1}={r=2,}{valign=h,},
cell{11}{1}={r=2,}{valign=h,},
row{3}={,cmd=\bfseries,},
row{8}={,cmd=\bfseries,},
}                     %% tabularray inner close
\toprule
&  & Adjusted DiD &  &  \\ \cmidrule[lr]{3-5}
&   & Individuals & ATT\textsuperscript{a} & (95\% CI) \\ \midrule %% TinyTableHeader
All participants &&&& \\
Systolic BP (mmHg) & Brachial & 1423 & -1.4 & (-3.3, 0.5) \\
Systolic BP (mmHg) & Central & 1423 & -1.6 & (-3.4, 0.3) \\
Diastolic BP (mmHg) & Brachial & 1423 & -1.6 & (-3.0, -0.3) \\
Diastolic BP (mmHg) & Central & 1423 & -1.7 & (-3.0, -0.3) \\
Excluding participants enrolled after W1 &&&& \\
Systolic BP (mmHg) & Brachial &  992 & -1.6 & (-3.3, -0.0) \\
Systolic BP (mmHg) & Central &  992 & -1.6 & (-3.1, -0.1) \\
Diastolic BP (mmHg) & Brachial &  992 & -1.7 & (-2.9, -0.4) \\
Diastolic BP (mmHg) & Central &  992 & -1.7 & (-2.9, -0.5) \\
\bottomrule
\end{talltblr}

}

\end{table}%DIF > 

\newpage

\DIFaddend \subsection{Impact of including Season 3
data}\label{impact-of-including-season-3-data}

Table~\ref{tbl-a-ind-s3} shows differences in the \(ATT\)s for the
impact of seasonal indoor PM\textsubscript{2.5} when \DIFdelbegin \DIFdel{season }\DIFdelend \DIFaddbegin \DIFadd{wave }\DIFaddend 3 data
(collected in 41 villages during COVID-19) are included versus excluded.

\DIFdelbegin \DIFdel{Figure~\ref{fig-afig-did-opm-w3} }\DIFdelend \DIFaddbegin \begin{table}[H]

\caption{\label{tbl-a-ind-s3}\DIFaddFL{Effects of the CHP policy on indoor
seasonal PM\textsubscript{2.5} based on whether Wave 3 data are included
vs.~excluded.}}

\centering{

\centering
\begin{talltblr}[         %% tabularray outer open
entry=none,label=none,
note{}={Note: ATT = Average Treatment Effect on the Treated, CI = confidence interval.},
]                     %% tabularray outer close
{                     %% tabularray inner open
width={1\linewidth},
colspec={X[0.166666666666667]X[0.166666666666667]X[0.0833333333333333]X[0.0833333333333333]X[0.166666666666667]X[0.0833333333333333]X[0.0833333333333333]X[0.166666666666667]},
cell{1}{3}={c=3,}{halign=c,},
cell{1}{6}={c=3,}{halign=c,},
cell{3}{1}={c=8}{},cell{5}{1}={c=8}{},
cell{4}{1}={preto={\hspace{1em}}},
cell{6}{1}={preto={\hspace{1em}}},
cell{7}{1}={preto={\hspace{1em}}},
cell{8}{1}={preto={\hspace{1em}}},
cell{9}{1}={preto={\hspace{1em}}},
row{3}={,cmd=\bfseries,},
row{5}={,cmd=\bfseries,},
}                     %% tabularray inner close
\toprule
&  & With Season 3 data &  &  & Without Season 3 data &  &  \\ \cmidrule[lr]{3-5}\cmidrule[lr]{6-8}
Cohort & Time & Obs & ATT & (95\% CI) & Obs & ATT & (95\% CI) \\ \midrule %% TinyTableHeader
Average ATT (µg/m3) &&&&&&& \\
All & All & 546 & -33.5 & (-55.0, -12.0) & 389 & -27.9 & (-49.8, -6.1) \\
Cohort-Time ATTs (µg/m3) &&&&&&& \\
2020 & 2020 & 546 & -36.1 & (-60.8, -11.4) &  &  &  \\
2020 & 2021 & 546 & -28.8 & (-59.1, 1.4) & 389 & -25.8 & (-55.4, 3.8) \\
2021 & 2021 & 546 & -36.3 & (-50.5, -22.1) & 389 & -32.0 & (-49.6, -14.4) \\
\bottomrule
\end{talltblr}

}

\end{table}%DIF > 

\DIFadd{Table~\ref{tbl-a-out-s3} }\DIFaddend shows the impact of including Wave 3 data on
the estimates of the impact of the policy on \DIFaddbegin \DIFadd{24-hr and }\DIFaddend seasonal outdoor
PM\textsubscript{2.5}.

\DIFaddbegin \begin{table}[H]

\caption{\label{tbl-a-out-s3}\DIFaddFL{Effects of the CHP policy on outdoor 24-hr
and seasonal PM\textsubscript{2.5} based on whether Wave 3 data are
included vs.~excluded.}}

\centering{

\centering
\begin{talltblr}[         %% tabularray outer open
entry=none,label=none,
note{}={Note: ATT = Average Treatment Effect on the Treated, CI = confidence interval.},
]                     %% tabularray outer close
{                     %% tabularray inner open
colspec={Q[]Q[]Q[]Q[]Q[]Q[]Q[]},
cell{1}{2}={c=3,}{halign=c,},
cell{1}{5}={c=3,}{halign=c,},
}                     %% tabularray inner close
\toprule
& With Season 3 data &  &  & Without Season 3 data &  &  \\ \cmidrule[lr]{2-4}\cmidrule[lr]{5-7}
& Obs & ATT & (95\% CI) & Obs & ATT & (95\% CI) \\ \midrule %% TinyTableHeader
24-hr PM2.5 & 14831 & -1.5 & (-6.5, 3.6) & 11174 & -2.1 & (-10.0, 5.8) \\
Seasonal PM2.5 &   189 & 0.8 & (-4.5, 6.0) &   139 & 0.5 & (-4.8, 5.9) \\
\bottomrule
\end{talltblr}

}

\end{table}%DIF > 

\newpage

\subsection{\DIFadd{Impact of sample composition on FeNO
results}}\label{impact-of-sample-composition-on-feno-results}

\DIFadd{Table~\ref{tbl-a-feno} shows differences in the \(ATT\)s for the impact
of the CBHP policy on FeNO depending on whether the estimation sample
includes all individuals or is limited to those with repeated measures
across campaigns.
}

\begin{table}[H]

\caption{\label{tbl-a-feno}\DIFaddFL{Effects of the CHP on FeNO (ppb) based on the
number of individuals with repeated measurements.}}

\centering{

\centering
\begin{talltblr}[         %% tabularray outer open
entry=none,label=none,
note{}={Note: ATT = Average Treatment Effect on the Treated, CI = Confidence Interval, DiD = Difference-in-Differences},
]                     %% tabularray outer close
{                     %% tabularray inner open
colspec={Q[]Q[]Q[]Q[]Q[]Q[]Q[]},
cell{1}{2}={c=2,}{halign=c,},
cell{1}{4}={c=2,}{halign=c,},
cell{1}{6}={c=2,}{halign=c,},
column{1}={halign=l,},
column{2}={halign=c,},
column{3}={halign=c,},
column{4}={halign=c,},
column{5}={halign=c,},
column{6}={halign=c,},
column{7}={halign=c,},
}                     %% tabularray inner close
\toprule
& All participants &  & Participants with >1 measure &  & Participants with 3 measures &  \\ \cmidrule[lr]{2-3}\cmidrule[lr]{4-5}\cmidrule[lr]{6-7}
& ATT & (95\% CI) & ATT & (95\% CI) & ATT & (95\% CI) \\ \midrule %% TinyTableHeader
DiD & 0.9 & (-1.6, 3.3) & -0.4 & (-2.9, 2.0) & 0.3 & (-2.8, 3.3) \\
Adjusted DiD & 0.3 & (-2.2, 2.8) & -0.6 & (-3.2, 2.0) & 0.3 & (-2.9, 3.4) \\
Observations & 793 &  & 541 &  & 272 &  \\
\bottomrule
\end{talltblr}

}

\end{table}%DIF > 

\newpage

\subsection{\DIFadd{Alternative PMF analyses}}\label{alternative-pmf-analyses}

\subsubsection{\DIFadd{Disaggregated analyses}}\label{disaggregated-analyses}

\DIFadd{Figure~\ref{fig-pmf-sup} shows results for the 3-, 4-, and 5-factor
solutions for both personal and outdoor exposure samples.
}

\DIFadd{For the 3-factor solution, Factor 1, represented in dark pink, is
interpreted as transported dust. This conclusion is supported by
moderate loadings of mineral dust-related species such as silicon (Si)
and water-soluble calcium (Ca). Calcium and aluminum are typically
associated with crustal materials, especially in arid or semi-arid
regions like the Gobi Desert and the Taklamakan Desert, which are known
sources of transported dust to northern China, including Beijing. Factor
2, represented in light pink, likely represents local dust, with
significant contributions from metals typically associated with crustal
materials. Factor 3, represented in beige, is interpreted as a mixture
of combustion-related sources, with contributions from elemental and
organic carbon, as well as evidence of secondary particulate matter
formation. This includes carbon species indicative of combustion,
contributions from Pb and water-soluble potassium (K), and secondary
formation markers such as sulfate
(SO\textsubscript{4}\textsuperscript{2-}).
}

\DIFadd{In the 4-factor solution, Factor 1 (dark pink) remains identified as
transported dust. Factor 2 (light pink) continues to represent local
dust. Factor 3 (beige) now represents mixed combustion, capturing
contributions from both biomass and coal-related sources. A new Factor 4
(light green) is interpreted as secondary sulfur, given its strong
association with sulfate and ammonium
(NH\textsubscript{4}\textsuperscript{+}), both markers for secondary
particulate matter formation.
}

\DIFadd{In the 5-factor solution, Factor 1 (dark pink) continues to represent
transported dust. Factor 2 (light pink) shifts its primary attribution
to secondary sulfur. Factor 3 (beige) is interpreted as mixed
combustion, now dominated by biomass burning contributions. Factor 4
(light green) likely represents coal combustion, based on species such
as arsenic (As) and selenium (Se), which are typically associated with
coal burning. Finally, Factor 5 (dark green) represents local dust,
primarily associated with crustal elements.
}

\DIFaddend \begin{figure}[H]

\caption{\DIFdelbeginFL %DIFDELCMD < \label{fig-afig-did-opm-w3}%%%
\DIFdelFL{Effects of the CBHP policy on
}\DIFdelendFL \DIFaddbeginFL \label{fig-pmf-sup}\DIFaddFL{3-, 4-, and 5- factor solutions separately
for }\DIFaddendFL outdoor \DIFdelbeginFL \DIFdelFL{seasonal PM\textsubscript{2.5} based on whether Season 3 data
are included vs}\DIFdelendFL \DIFaddbeginFL \DIFaddFL{and personal exposure samples}\DIFaddendFL .\DIFdelbeginFL \DIFdelFL{~excluded.}\DIFdelendFL }

\DIFdelbeginFL %DIFDELCMD < \centering{
%DIFDELCMD < 

%DIFDELCMD < \includegraphics[width=0.6\textwidth,height=\textheight]{images/did-outdoor-w3.png}
%DIFDELCMD < 

%DIFDELCMD < }
%DIFDELCMD < %%%
\DIFdelendFL \DIFaddbeginFL \centering{

\includegraphics[width=0.95\textwidth,height=\textheight]{images/sp_sup_fig.png}

}
\DIFaddendFL 

\end{figure}%

\DIFaddbegin \subsubsection{\DIFadd{PMF results disaggregated by
day}}\label{pmf-results-disaggregated-by-day}

\begin{figure}[H]

\caption{\label{fig-pmf-sup-day}\DIFaddFL{Mass concentrations (ug/m3) of
contributions to PM2.5 mass by each of the four named sources identified
in the source analysis. From left to right, the source contributions
represented are `sulfur secondary', `mixed combustion', `dust', and
`transported dust'. Source contribution mass concentrations are shown by
day and color-coded by district, with purple for Fangshan, blue for
Huairou, green for Mentougou, and yellow for Miyun.}}

\centering{

\includegraphics[width=0.95\textwidth,height=\textheight]{images/pe_time_all_fig.png}

}

\end{figure}%DIF > 

\subsubsection{\DIFadd{PMF results disaggregated by
month}}\label{pmf-results-disaggregated-by-month}

\begin{figure}[H]

\caption{\label{fig-pmf-sup-month}\DIFaddFL{Mass concentrations (ug/m3) of
contributions to PM2.5 mass by each of the four named sources identified
in the source analysis. From left to right, the source contributions
represented are `sulfur secondary', `mixed combustion', `dust', and
`transported dust'. Source contribution mass concentrations are shown by
month (November, December, January) and color-coded by district, with
purple for Fangshan, blue for Huairou, green for Mentougou, and yellow
for Miyun.}}

\centering{

\includegraphics[width=0.95\textwidth,height=\textheight]{images/out_time_all_fig.png}

}

\end{figure}%DIF > 

\DIFaddend \newpage

\DIFdelbegin \subsection{\DIFdel{Pre-trends for blood
pressure}}%DIFAUXCMD
\addtocounter{subsection}{-1}%DIFAUXCMD
%DIFDELCMD < \label{pre-trends-for-blood-pressure}
%DIFDELCMD < %%%
\DIFdelend \DIFaddbegin \subsubsection{\DIFadd{Personal exposure sample
diagnostics}}\label{personal-exposure-sample-diagnostics}
\DIFaddend 

\DIFaddbegin \DIFadd{Table~\ref{tbl-pmf-personal} shows diagnostics for the PMF analysis for
outdoor exposure samples.
}

\begin{table}[H]

\caption{\label{tbl-pmf-personal}\DIFaddFL{PMF error estimation diagnostics for
personal exposure samples.}}

\centering{

\centering
\begin{tblr}[         %% tabularray outer open
]                     %% tabularray outer close
{                     %% tabularray inner open
width={1\linewidth},
colspec={X[0.142857142857143]X[0.214285714285714]X[0.214285714285714]X[0.214285714285714]X[0.214285714285714]},
cell{1}{2}={c=4,}{halign=c,},
column{1}={halign=c,},
column{2}={halign=c,},
column{3}={halign=c,},
column{4}={halign=c,},
column{5}={halign=c,},
row{2}={halign=c,},
cell{7}{1}={}{halign=l,},
cell{10}{1}={}{halign=l,},
cell{7}{2}={}{halign=l,},
cell{10}{2}={}{halign=l,},
cell{7}{3}={}{halign=l,},
cell{10}{3}={}{halign=l,},
cell{7}{4}={}{halign=l,},
cell{10}{4}={}{halign=l,},
cell{7}{5}={}{halign=l,},
cell{10}{5}={}{halign=l,},
}                     %% tabularray inner close
\toprule
& Potential Factor Solution &  &  &  \\ \cmidrule[lr]{2-5}
Diagnostic & 3 & 4 & 5 & 6 \\ \midrule %% TinyTableHeader
Qexp & 17301 & 16126 & 14951 & 13776 \\
Qtrue & 1225294 & 101524 & 84622 & 70122 \\
Qrobust & 117329 & 96215 & 80219 & 66865 \\
Qr/Qexp & 6.78 & 5.97 & 5.37 & 4.85 \\
Q/Qexp >6 & ns-S, NH3, NO3, SO4, ws-Na, Al, Cl, Pb & ns-S, ws-Na, Al, Cl, Pb & NO3, ws-Na, Al, Cl & NO3, ws-Na, Al, ws-K \\
DISP \% dQ & <0.01\% & <0.01\% & <0.01\% & <0.01\% \\
DISP Swaps & 0 & 0 & 0 & 0 \\
Bootstrap mapping < 100\% & mixed combustion - 95\% & sulfur secondary - 85\% & coal - 80\% & chloride - 45\% \\
\bottomrule
\end{tblr}

}

\end{table}%DIF > 

\newpage

\subsubsection{\DIFadd{Outdoor exposure sample
diagnostics}}\label{outdoor-exposure-sample-diagnostics}

\DIFadd{Table~\ref{tbl-pmf-outdoor} shows diagnostics for the PMF analysis for
outdoor exposure samples.
}

\begin{table}[H]

\caption{\label{tbl-pmf-outdoor}\DIFaddFL{PMF error estimation diagnostics for
outdoor samples.}}

\centering{

\centering
\begin{tblr}[         %% tabularray outer open
]                     %% tabularray outer close
{                     %% tabularray inner open
width={1\linewidth},
colspec={X[0.142857142857143]X[0.214285714285714]X[0.214285714285714]X[0.214285714285714]X[0.214285714285714]},
cell{1}{2}={c=4,}{halign=c,},
column{1}={halign=c,},
column{2}={halign=c,},
column{3}={halign=c,},
column{4}={halign=c,},
column{5}={halign=c,},
row{2}={halign=c,},
cell{7}{1}={}{halign=l,},
cell{10}{1}={}{halign=l,},
cell{7}{2}={}{halign=l,},
cell{10}{2}={}{halign=l,},
cell{7}{3}={}{halign=l,},
cell{10}{3}={}{halign=l,},
cell{7}{4}={}{halign=l,},
cell{10}{4}={}{halign=l,},
cell{7}{5}={}{halign=l,},
cell{10}{5}={}{halign=l,},
}                     %% tabularray inner close
\toprule
& Potential Factor Solution &  &  &  \\ \cmidrule[lr]{2-5}
Diagnostic & 3 & 4 & 5 & 6 \\ \midrule %% TinyTableHeader
Qexp & 10716 & 9980 & 9244 & 8508 \\
Qtrue & 43280 & 34603 & 28467 & 21947 \\
Qrobust & 41521 & 33209 & 26432 & 21347 \\
Qr/Qexp & 3.87 & 3.33 & 2.86 & 2.51 \\
Q/Qexp >6 & ws-Ca, Cl & Cl & Pb & None \\
DISP \% dQ & <0.01\% & <0.01\% & <0.01\% & <0.01\% \\
DISP Swaps & 0 & 0 & 0 & 0 \\
Bootstrap mapping < 100\% & None & sulfur secondary - 55\% & None & coal - 90\%, EC + OC - 90\%, chloride - 90\% \\
\bottomrule
\end{tblr}

}

\end{table}%DIF > 

\newpage

\subsubsection{\texorpdfstring{PM\textsubscript{2.5} constituents for
outdoor
samples}{PM2.5 constituents for outdoor samples}}\label{pm2.5-constituents-for-outdoor-samples}

\begin{table}[H]

\caption{\label{tbl-species-outdoor}\DIFaddFL{Mean (95\% confidence interval)
species concentrations (x100 µg/m\textsuperscript{3}) for outdoor
samples included in PMF.}}

\centering{

\centering
\begin{talltblr}[         %% tabularray outer open
entry=none,label=none,
note{}={Note: CI = Confidence interval.},
]                     %% tabularray outer close
{                     %% tabularray inner open
colspec={Q[]Q[]Q[]Q[]Q[]Q[]Q[]},
cell{1}{2}={c=2,}{halign=c,},
cell{1}{4}={c=2,}{halign=c,},
cell{1}{6}={c=2,}{halign=c,},
}                     %% tabularray inner close
\toprule
& Wave 1 &  & Wave 2 &  & Wave 4 &  \\ \cmidrule[lr]{2-3}\cmidrule[lr]{4-5}\cmidrule[lr]{6-7}
Species & Mean & 95\% CI & Mean & 95\% CI & Mean & 95\% CI \\ \midrule %% TinyTableHeader
Al & 110  & (108-113) & 66.7  & (65.5-67.8) & 60.9  & (59.8-62.1) \\
Fe & 85.2  & (82.9-87.5) & 57.9  & (57-58.9) & 64.1  & (63-65.2) \\
Pb & 6.26  & (5.82-6.71) & 4.32  & (4.06-4.57) & 7.69  & (7.27-8.11) \\
Si & 139  & (136-142) & 76.7  & (75.4-77.9) & 128  & (126-129) \\
Chloride & 56.1  & (54.5-57.8) & 32.2  & (31.5-32.9) & 15.6  & (14.9-16.4) \\
EC & 123  & (121-125) & 112  & (111-113) & 103  & (101-104) \\
ws-Na & 11.4  & (10.8-12) & 9.28  & (8.88-9.68) & 9.23  & (8.76-9.7) \\
Ammonium & 136  & (133-138) & 152  & (151-154) & 78.3  & (77.3-79.4) \\
Nitrate & 225  & (222-227) & 259  & (257-261) & 146  & (145-148) \\
ns-S & 87  & (85.3-88.7) & 70  & (69.2-70.9) & 75.9  & (74.9-76.9) \\
OC & 1080  & (1070-1080) & 901  & (898-904) & 649  & (645-653) \\
Sulfate & 223  & (220-225) & 206  & (205-208) & 132  & (131-134) \\
wi-Ca & 79.4  & (77.2-81.7) & 40.7  & (39.8-41.6) & 79  & (77.6-80.4) \\
wi-K & 81.4  & (79.5-83.3) & 33.7  & (33-34.3) & 37.8  & (37-38.6) \\
wi-Mg & 39.8  & (38.6-41) & 21.2  & (20.6-21.7) & 38.6  & (37.7-39.4) \\
ws-Ca & 33.5  & (32.5-34.5) & 23.5  & (22.8-24.2) & 15.1  & (14.3-15.9) \\
ws-K & 52.4  & (50.8-54) & 25.4  & (24.9-25.9) & 17.9  & (17.4-18.4) \\
ws-Mg & 6.98  & (6.41-7.55) & 3.19  & (2.97-3.41) & 2.96  & (2.72-3.2) \\
\bottomrule
\end{talltblr}

}

\end{table}%DIF > 

\newpage

\subsubsection{\texorpdfstring{PM\textsubscript{2.5} constituents for
personal
samples}{PM2.5 constituents for personal samples}}\label{pm2.5-constituents-for-personal-samples}

\begin{table}[H]

\caption{\label{tbl-species-personal}\DIFaddFL{Mean (95\% confidence interval)
species concentrations (x100 µg/m\textsuperscript{3}) for personal
samples included in PMF.}}

\centering{

\centering
\begin{talltblr}[         %% tabularray outer open
entry=none,label=none,
note{}={Note: CI = Confidence interval.},
]                     %% tabularray outer close
{                     %% tabularray inner open
colspec={Q[]Q[]Q[]Q[]Q[]Q[]Q[]},
cell{1}{2}={c=2,}{halign=c,},
cell{1}{4}={c=2,}{halign=c,},
cell{1}{6}={c=2,}{halign=c,},
}                     %% tabularray inner close
\toprule
& Wave 1 &  & Wave 2 &  & Wave 4 &  \\ \cmidrule[lr]{2-3}\cmidrule[lr]{4-5}\cmidrule[lr]{6-7}
Species & Mean & 95\% CI & Mean & 95\% CI & Mean & 95\% CI \\ \midrule %% TinyTableHeader
Al & 203  & (202-204) & 213  & (212-214) & 27.4  & (26.7-28.1) \\
Fe & 40  & (39.4-40.6) & 57.1  & (55.9-58.4) & 33.6  & (32.9-34.3) \\
Pb & 15.8  & (15.4-16.1) & 13.7  & (13.5-14) & 12.8  & (12.4-13.1) \\
Si & 77.2  & (75.9-78.4) & 93.5  & (92-95) & 52.1  & (51.2-53) \\
Chloride & 70.2  & (69-71.5) & 43.8  & (42.7-45) & 34.6  & (33.6-35.5) \\
EC & 241  & (239-243) & 219  & (218-221) & 194  & (192-196) \\
ws-Na & 8.75  & (8.36-9.14) & 12.3  & (11.8-12.7) & 10.8  & (10.3-11.2) \\
Ammonium & 60.5  & (59.5-61.5) & 87.8  & (86.4-89.1) & 31.1  & (30.2-31.9) \\
Nitrate & 151  & (149-152) & 235  & (232-237) & 59.1  & (58.1-60.1) \\
ns-S & 46.3  & (45.4-47.2) & 44.1  & (43.2-45) & 27.5  & (26.7-28.3) \\
OC & 3740  & (3730-3740) & 3430  & (3420-3440) & 2680  & (2680-2690) \\
Sulfate & 129  & (127-130) & 126  & (125-128) & 77.6  & (76.5-78.7) \\
wi-Ca & 46.8  & (45.7-47.9) & 81.6  & (79.8-83.5) & 48.4  & (47.6-49.2) \\
wi-K & 88.2  & (87-89.4) & 72.8  & (71.6-74) & 67.7  & (66.5-68.9) \\
wi-Mg & 25.6  & (25-26.2) & 30.4  & (29.7-31.1) & 26.5  & (26-27) \\
ws-Ca & 57.1  & (56.4-57.9) & 42.8  & (41.9-43.8) & 15.7  & (15.1-16.2) \\
ws-K & 49  & (48-50) & 43.8  & (42.9-44.7) & 28.7  & (27.9-29.5) \\
ws-Mg & 4  & (3.75-4.24) & 3.77  & (3.49-4.04) & 2.06  & (1.81-2.3) \\
\bottomrule
\end{talltblr}

}

\end{table}%DIF > 

\newpage

\subsection{\DIFadd{Pre-trends}}\label{pre-trends}

\subsubsection{\DIFadd{Blood pressure}}\label{blood-pressure-1}

\DIFaddend \begin{figure}[H]

\caption{\DIFdelbeginFL %DIFDELCMD < \label{fig-afig-pt3}%%%
\DIFdelendFL \DIFaddbeginFL \label{fig-afig-pt-bp}\DIFaddendFL Comparison of pre-interventions trends in
blood pressure between waves 1 and 2 for never treated and villages
\DIFdelbeginFL \DIFdelFL{later }\DIFdelendFL treated \DIFdelbeginFL \DIFdelFL{in wave 3}\DIFdelendFL \DIFaddbeginFL \DIFaddFL{later.}\DIFaddendFL }

\DIFdelbeginFL %DIFDELCMD < \centering{
%DIFDELCMD < 

%DIFDELCMD < \includegraphics[width=0.8\textwidth,height=\textheight]{images/never_Y3.png}
%DIFDELCMD < 

%DIFDELCMD < }
%DIFDELCMD < %%%
\DIFdelendFL \DIFaddbeginFL \centering{

\includegraphics[width=0.95\textwidth,height=\textheight]{images/BP-pretrends_FT20orFT21.png}

}
\DIFaddendFL 

\end{figure}%

\DIFaddbegin \subsubsection{\DIFadd{Personal exposure and black
carbon}}\label{personal-exposure-and-black-carbon}

\DIFaddend \begin{figure}[H]

\caption{\DIFdelbeginFL %DIFDELCMD < \label{fig-afig-pt4}%%%
\DIFdelendFL \DIFaddbeginFL \label{fig-afig-pt-pe-bc}\DIFaddendFL Comparison of \DIFdelbeginFL \DIFdelFL{pre-interventions }\DIFdelendFL \DIFaddbeginFL \DIFaddFL{pre-intervention }\DIFaddendFL trends
in \DIFdelbeginFL \DIFdelFL{blood pressure }\DIFdelendFL \DIFaddbeginFL \DIFaddFL{personal exposure and black carbon }\DIFaddendFL between waves 1 and 2 for never
treated and villages \DIFdelbeginFL \DIFdelFL{later }\DIFdelendFL treated \DIFdelbeginFL \DIFdelFL{in wave 4}\DIFdelendFL \DIFaddbeginFL \DIFaddFL{later.}\DIFaddendFL }

\DIFdelbeginFL %DIFDELCMD < \centering{
%DIFDELCMD < 

%DIFDELCMD < \includegraphics[width=0.7\textwidth,height=\textheight]{images/never_Y4.png}
%DIFDELCMD < 

%DIFDELCMD < }
%DIFDELCMD < %%%
\DIFdelendFL \DIFaddbeginFL \centering{

\includegraphics[width=0.8\textwidth,height=\textheight]{images/pe-bc-pretrends.png}

}
\DIFaddendFL 

\end{figure}%

\DIFdelbegin %DIFDELCMD < \newpage
%DIFDELCMD < %%%
\DIFdelend \DIFaddbegin \subsubsection{\DIFadd{Self-reported respiratory
outcomes}}\label{self-reported-respiratory-outcomes-1}
\DIFaddend 

\DIFaddbegin \begin{figure}[H]

\caption{\label{fig-afig-pt-resp}\DIFaddFL{Comparison of pre-intervention trends
in self-reported respiratory outcomes between waves 1 and 2 for never
treated and villages treated later.}}

\centering{

\includegraphics[width=0.9\textwidth,height=\textheight]{images/resp-pretrends.png}

}

\end{figure}%DIF > 

\DIFaddend \newpage

\subsection{Impact of group and time fixed
effects}\label{impact-of-group-and-time-fixed-effects}

\DIFaddbegin \DIFadd{Table~\ref{tbl-a-fe} shows the impact of adding different sets of cohort
and time fixed effects to the model for personal exposure.
}

\begin{table}[H]

\caption{\label{tbl-a-fe}\DIFaddFL{Effects of the CHP on personal exposure
(\(\mu g / m^{3}\)) with variations in fixed effects for treatment group
and time.}}

\centering{

\centering
\begin{talltblr}[         %% tabularray outer open
entry=none,label=none,
note{}={Note: DiD = Difference-in-Differences.  FE = Fixed effects. All models adjusted for household size, smoking, outdoor temperature, and outdoor humidity. Standard errors clustered by village and 95\% confidence intervals shown in brackets.},
]                     %% tabularray outer close
{                     %% tabularray inner open
colspec={Q[]Q[]Q[]Q[]Q[]},
row{1}={,cmd=\bfseries,},
column{2}={halign=c,},
column{3}={halign=c,},
column{4}={halign=c,},
column{5}={halign=c,},
}                     %% tabularray inner close
\toprule
& Adjusted DiD & No time FE & No group FE & No FE \\ \midrule %% TinyTableHeader
ATT & 0.2 & -27.5 & -4.9 & -22.0 \\
& [-19.6, 19.9] & [-49.4, -5.5] & [-24.0, 14.3] & [-40.1, -3.8] \\
Observations & 1,270 & 1,270 & 1,270 & 1,270 \\
Year fixed effects & X &  & X &  \\
Cohort fixed effects & X & X &  &  \\
\bottomrule
\end{talltblr}

}

\end{table}%DIF > 

\DIFaddend \newpage

\DIFaddbegin \subsection{\DIFadd{Retrospective design
analysis}}\label{retrospective-design-analysis}

\subsubsection{\DIFadd{All outcomes}}\label{all-outcomes}

\DIFadd{Table~\ref{tbl-rd} shows the results of a design-based analysis (Gelman
and Carlin 2014) for different hypothetical effect sizes, conditional on
our design and sample size.
}

\begin{table}[H]

\caption{\label{tbl-rd}\DIFaddFL{Design-based analysis for various hypothetical
effect sizes.}}

\centering{

\centering
\resizebox{\ifdim\width>\linewidth 0.9\linewidth\else\width\fi}{!}{
\begin{talltblr}[         %% tabularray outer open
entry=none,label=none,
note{}={Note: BP = Blood Pressure, pp = percentage points, ppb = parts per billion, SE = Standard Error},
note{a}={Assuming the true effect size and our study design and precision, the probability that an observed estimate from a replication will have the wrong sign.},
note{b}={Assuming the true effect size and our study design and precision, the ratio by which an observed estimate from a replication will exaggerate the true effect.},
]                     %% tabularray outer close
{                     %% tabularray inner open
colspec={Q[]Q[]Q[]Q[]Q[]Q[]Q[]Q[]},
cell{1}{3}={c=2,}{halign=c,},
cell{1}{5}={c=4,}{halign=c,},
cell{3}{1}={c=8}{},cell{12}{1}={c=8}{},cell{20}{1}={c=8}{},
cell{4}{1}={preto={\hspace{1em}}},
cell{5}{1}={preto={\hspace{1em}}},
cell{6}{1}={preto={\hspace{1em}}},
cell{7}{1}={preto={\hspace{1em}}},
cell{8}{1}={preto={\hspace{1em}}},
cell{9}{1}={preto={\hspace{1em}}},
cell{10}{1}={preto={\hspace{1em}}},
cell{11}{1}={preto={\hspace{1em}}},
cell{13}{1}={preto={\hspace{1em}}},
cell{14}{1}={preto={\hspace{1em}}},
cell{15}{1}={preto={\hspace{1em}}},
cell{16}{1}={preto={\hspace{1em}}},
cell{17}{1}={preto={\hspace{1em}}},
cell{18}{1}={preto={\hspace{1em}}},
cell{19}{1}={preto={\hspace{1em}}},
cell{21}{1}={preto={\hspace{1em}}},
cell{22}{1}={preto={\hspace{1em}}},
cell{23}{1}={preto={\hspace{1em}}},
cell{24}{1}={preto={\hspace{1em}}},
cell{25}{1}={preto={\hspace{1em}}},
row{3}={halign=l,cmd=\bfseries,},
row{12}={halign=l,cmd=\bfseries,},
row{20}={halign=l,cmd=\bfseries,},
cell{4}{1}={r=2,}{valign=h,},
cell{6}{1}={r=2,}{valign=h,},
cell{8}{1}={r=2,}{valign=h,},
cell{10}{1}={r=2,}{valign=h,},
cell{13}{1}={r=6,}{valign=h,},
cell{21}{1}={r=4,}{valign=h,},
column{1}={halign=l,},
column{2}={halign=l,},
column{3}={halign=c,},
column{4}={halign=c,},
column{5}={halign=c,},
column{6}={halign=c,},
column{7}={halign=c,},
}                     %% tabularray inner close
\toprule
&  & Observed Results &  & Hypothetical Design Analysis &  &  &  \\ \cmidrule[lr]{3-4}\cmidrule[lr]{5-8}
&   & Estimate & SE & Effect & Power (\%) & S-bias\textsuperscript{a} & M-bias\textsuperscript{b} \\ \midrule %% TinyTableHeader
Blood pressure (mmHg) &&&&&&& \\
Systolic BP (mmHg) & Brachial & -1.4 & 1.0 & -2.5 & 73.2 & 0.02 & 0.5 \\
Systolic BP (mmHg) & Central & -1.6 & 0.9 & -2.5 & 75.4 & 0.02 & 0.5 \\
Diastolic BP (mmHg) & Brachial & -1.6 & 0.7 & -2.0 & 80.0 & 0.00 & 1.1 \\
Diastolic BP (mmHg) & Central & -1.7 & 0.7 & -2.0 & 82.7 & 0.00 & 1.1 \\
Pulse Pressure & Brachial & 0.2 & 0.6 & 0.5 & 12.9 & 0.23 & 1.0 \\
Pulse Pressure & Central & 0.1 & 0.6 & 0.5 & 14.5 & 0.22 & 1.0 \\
BP Amplification x100 & Pulse pressure & -0.0 & 0.6 & 0.1 & 5.3 & 0.44 & 4.5 \\
BP Amplification x100 & Systolic BP & 0.1 & 0.2 & 0.1 & 10.0 & 0.28 & 1.2 \\
Respiratory outcomes &&&&&&& \\
Self-reported (pp) & Any symptom & -7.5 & 2.7 & -5.0 & 47.0 & 0.00 & 1.4 \\
Self-reported (pp) & Coughing & -2.7 & 2.2 & -2.0 & 14.5 & 0.21 & 1.0 \\
Self-reported (pp) & Phlegm & -1.6 & 2.0 & -3.0 & 31.2 & 0.10 & 0.7 \\
Self-reported (pp) & Wheezing attacks & 1.0 & 1.5 & -1.0 & 10.4 & 0.27 & 1.2 \\
Self-reported (pp) & Trouble breathing & -3.4 & 3.0 & -3.0 & 17.4 & 0.18 & 0.9 \\
Self-reported (pp) & Chest trouble & -3.4 & 2.4 & -1.0 & 7.0 & 0.36 & 1.8 \\
Measured & FeNO (ppb) & 0.3 & 1.3 & -0.5 & 6.8 & 0.36 & 1.9 \\
Inflammatory markers (\%) &&&&&&& \\
Measured & IL6 (pg/mL) & 0.8 & 0.6 & -0.2 & 6.0 & 0.39 & 2.5 \\
Measured & TNF-alpha (pg/mL) & 0.8 & 0.5 & -0.4 & 15.0 & 0.21 & 0.9 \\
Measured & CRP (mg/L) & 0.1 & 0.3 & -0.1 & 6.5 & 0.37 & 2.1 \\
Measured & MDA (µM) & 0.2 & 0.2 & -0.2 & 13.2 & 0.23 & 1.0 \\
\bottomrule
\end{talltblr}
}

}

\end{table}%DIF > 

\newpage

\subsubsection{\DIFadd{Blood pressure}}\label{blood-pressure-2}

\DIFadd{Figure~\ref{fig-rd-bp} show the range of estimates for power, sign-bias,
and effect exaggeration for several hypothetical effect sizes for the
impact of the program on blood pressure, conditional on our study design
and sample size (standard error of 1 mmHg).
}

\begin{figure}[H]

\caption{\label{fig-rd-bp}\DIFaddFL{Power, sign-bias, and exaggeration ratios for
various hypothetical effects of the policy on blood pressure, given our
study design.}}

\centering{

\includegraphics[width=0.8\textwidth,height=\textheight]{images/rd-plot.png}

}

\end{figure}%DIF > 

\newpage

\section*{\DIFadd{Abbreviations and other
terms}}\label{abbreviations-and-other-terms}
\addcontentsline{toc}{section}{\DIFadd{Abbreviations and other terms}}

\begin{longtable}[]{@{}
  >{\raggedright\arraybackslash}p{(\columnwidth - 2\tabcolsep) * \real{0.2500}}
  >{\raggedright\arraybackslash}p{(\columnwidth - 2\tabcolsep) * \real{0.7500}}@{}}
\toprule\noalign{}
\endhead
\bottomrule\noalign{}
\endlastfoot
\DIFadd{ANMB }& \DIFadd{Absolute Normalized Mean Bias }\\
\DIFadd{ATT }& \DIFadd{Average Treatment Effect on the Treated }\\
\DIFadd{BAM }& \DIFadd{Beta Attenuation Monitor }\\
\DIFadd{BC }& \DIFadd{Black carbon }\\
\DIFadd{BP }& \DIFadd{Blood pressure }\\
\DIFadd{CI }& \DIFadd{Confidence Interval }\\
\DIFadd{CIE }& \DIFadd{International Commission on Illumination }\\
\DIFadd{CHP }& \DIFadd{Clean Heating Policy }\\
\DIFadd{cDBP }& \DIFadd{Central diastolic blood pressure }\\
\DIFadd{CRP }& \DIFadd{C-reactive protein }\\
\DIFadd{cSBP }& \DIFadd{Central systolic blood pressure }\\
\DIFadd{DAG }& \DIFadd{Directed acyclic graph }\\
\DIFadd{DiD }& \DIFadd{Difference-in-Differences }\\
\DIFadd{DISP }& \DIFadd{Displacement of Factor Elements }\\
\DIFadd{EC }& \DIFadd{Elemental carbon }\\
\DIFadd{EDXRF }& \DIFadd{Evo energy-dispersive X-ray fluorescence }\\
\DIFadd{ETWFE }& \DIFadd{Extended Two-Way Fixed Effects }\\
\DIFadd{FEM }& \DIFadd{Federal equivalent method }\\
\DIFadd{FID }& \DIFadd{Flame ionization detector }\\
\DIFadd{FeNO }& \DIFadd{Fractional exhaled nitric oxide }\\
\DIFadd{HAPIN }& \DIFadd{Household Air Pollution Intervention Network }\\
\DIFadd{HPLC }& \DIFadd{High-performance liquid chromatography }\\
\DIFadd{IL-6 }& \DIFadd{Interleukin-6 }\\
\DIFadd{MDA }& \DIFadd{Malondialdehyde }\\
\DIFadd{NISP }& \DIFadd{National Improved Stove Program }\\
\DIFadd{NIST }& \DIFadd{National Institute of Standards and Technology }\\
\DIFadd{ns-S }& \DIFadd{Non-Sulfate Sulfur }\\
\DIFadd{OC }& \DIFadd{Organic Carbon }\\
\DIFadd{OD }& \DIFadd{Optic densities }\\
\DIFadd{PKU }& \DIFadd{Peking University }\\
\DIFadd{PM\textsubscript{2.5} }& \DIFadd{Particulate matter less than 2.5 microns in
aerodynamic diameter }\\
\DIFadd{RMSE }& \DIFadd{Root mean square error }\\
\DIFadd{SRM }& \DIFadd{Standard reference material }\\
\DIFadd{TNF-\(\alpha\) }& \DIFadd{Tumour necrosis factor alpha }\\
\DIFadd{UCAS }& \DIFadd{University of Chinese Academy of Sciences }\\
\DIFadd{UPAS }& \DIFadd{Ultrasonic Personal Aerosol Samplers }\\
\DIFadd{W1, W2, W3, W4 }& \DIFadd{Wave 1, Wave 2, Wave 3, Wave 4 }\\
\DIFadd{wi }& \DIFadd{Water Insoluble Species }\\
\DIFadd{ws }& \DIFadd{Water Soluble Species }\\
\end{longtable}

\DIFaddend \section*{About the authors}\label{about-the-authors}
\addcontentsline{toc}{section}{About the authors}
\DIFaddbegin 

\textbf{\DIFadd{Jill Baumgartner, PhD}} \DIFadd{is a professor jointly appointed in the
Department of Equity, Ethics, and Policy and the Department of
Epidemiology, Biostatistics \& Occupational Health at McGill University.
Her work evaluates the health impacts of air pollution, household energy
use, and climate change..
}

\textbf{\DIFadd{Sam Harper, PhD}} \DIFadd{is a professor in the Department of
Epidemiology, Biostatistics \& Occupational Health at McGill University.
His research focuses on evaluating the impacts of social and economic
policies on health.
}

\textbf{\DIFadd{Chris Barrington-Leigh, PhD}} \DIFadd{is an associate professor jointly
appointed in the Department of Equity, Ethics, and Policy and the Bieler
School of Environment at McGill University. His research evaluates
}

\textbf{\DIFadd{Collin Brehmer}} \DIFadd{is a PhD student in the Department of Civil and
Environmental Engineering at Colorado State University.
}

\textbf{\DIFadd{Ellison M. Carter, PhD}} \DIFadd{is an associate professor appointed in
the Department of Civil and Environmental Engineering at Colorado State
University. Her research combines interests and expertise in air
quality, exposure science, and chemistry and aims to answer questions
relevant to energy policy and their impacts on air pollution exposures
and human health.
}

\textbf{\DIFadd{Xiaoying Li, PhD}} \DIFadd{is a Research Scientist in the Department of
Mechanical Engineering at Colorado State University and formerly a
postdoctoral fellow at McGill University. Her research interests include
investigating the interactions of indoor, outdoor air pollution, and
personal exposures to air pollution and evaluating the impacts of clean
energy interventions on air quality and human health.
}

\textbf{\DIFadd{Brian E. Robinson, PhD}} \DIFadd{is an associate professor in the
Department of Geography at McGill University who studies how people's
livelihoods are influenced by ecosystem services and resource use,
particularly in developing regions. His interdisciplinary research
explores the interactions between livelihoods, the environment, and the
institutions that govern resource management.
}

\textbf{\DIFadd{Guofeng Shen, PhD}} \DIFadd{is an assistant professor of environmental
science at Peking University. His research interests and experiences are
in sustainable household energy and environment, largely focusing on
fates, impacts and controls of hazardous pollutants produced from indoor
solid fuel use that is an important indicator of sustainable
development.
}

\textbf{\DIFadd{Talia J. Sternbach}} \DIFadd{is a PhD student in the Department of
Epidemiology, Biostatistics \& Occupational Health at McGill University.
}

\textbf{\DIFadd{Shu Tao, PhD}}\DIFadd{, is a professor of environmental science at Peking
University who focuses on measuring and modeling clean energy
transition, air pollution emissions, population exposures, and their
estimated health impacts in China. His research integrates environmental
science, atmospheric modeling, field measurements, and exposure
assessment to better understand the sources, distribution, and health
effects of air pollutants.
}

\textbf{\DIFadd{Kaibing Xue}} \DIFadd{is a PhD student in atmospheric science at the
University of the Chinese Academy of Sciences
}

\textbf{\DIFadd{Wenlu Yuan}} \DIFadd{is a PhD student in the Department of Epidemiology,
Biostatistics \& Occupational Health at McGill University.
}

\textbf{\DIFadd{Xiang Zhang}} \DIFadd{is a PhD student in the Department of Geography at
McGill University.
}

\textbf{\DIFadd{Yuanxun Zhang, PhD}}\DIFadd{, is a professor of atmospheric sciences at
the University of the Chinese Academy of Sciences and directs the
Yanshan Earth Critical Zone at the National Observation and Research
Station in China. He has expertise in atmospheric chemistry and
measurement of air pollution and its chemical composition. His research
focuses on understanding the chemical processes and mechanisms driving
the formation of air pollutants and their impacts on climate and human
health.
}\DIFaddend 

\section*{Other publications}\label{other-publications}
\addcontentsline{toc}{section}{Other publications}

Li X, Baumgartner J, Barrington-Leigh C, Harper S, Robinson B, Shen G,
et al.~2022a. Socioeconomic and Demographic Associations with Wintertime
Air Pollution Exposures at Household, Community, and District Scales in
Rural Beijing, China. Environ Sci Technol 56:8308--8318;
doi:10.1021/acs.est.1c07402.

Li X, Baumgartner J, Harper S, Zhang X, Sternbach T, Barrington-Leigh C,
et al.~2022b. Field measurements of indoor and community air quality in
rural Beijing before, during, and after the COVID-19 lockdown. Indoor
Air 32:e13095; doi:10.1111/ina.13095.

Sternbach TJ, Harper S, Li X, Zhang X, Carter E, Zhang Y, et al.~2022.
Effects of indoor and outdoor temperatures on blood pressure and central
hemodynamics in a wintertime longitudinal study of Chinese adults. J
Hypertension 40:1950--1959; doi:10.1097/HJH.0000000000003198.




\end{document}

% Options for packages loaded elsewhere
\PassOptionsToPackage{unicode}{hyperref}
\PassOptionsToPackage{hyphens}{url}
\PassOptionsToPackage{dvipsnames,svgnames,x11names}{xcolor}
%
\documentclass[
  letterpaper,
  DIV=11,
  numbers=noendperiod]{scrartcl}

\usepackage{amsmath,amssymb}
\usepackage{iftex}
\ifPDFTeX
  \usepackage[T1]{fontenc}
  \usepackage[utf8]{inputenc}
  \usepackage{textcomp} % provide euro and other symbols
\else % if luatex or xetex
  \usepackage{unicode-math}
  \defaultfontfeatures{Scale=MatchLowercase}
  \defaultfontfeatures[\rmfamily]{Ligatures=TeX,Scale=1}
\fi
\usepackage{lmodern}
\ifPDFTeX\else  
    % xetex/luatex font selection
\fi
% Use upquote if available, for straight quotes in verbatim environments
\IfFileExists{upquote.sty}{\usepackage{upquote}}{}
\IfFileExists{microtype.sty}{% use microtype if available
  \usepackage[]{microtype}
  \UseMicrotypeSet[protrusion]{basicmath} % disable protrusion for tt fonts
}{}
\makeatletter
\@ifundefined{KOMAClassName}{% if non-KOMA class
  \IfFileExists{parskip.sty}{%
    \usepackage{parskip}
  }{% else
    \setlength{\parindent}{0pt}
    \setlength{\parskip}{6pt plus 2pt minus 1pt}}
}{% if KOMA class
  \KOMAoptions{parskip=half}}
\makeatother
\usepackage{xcolor}
\usepackage[right=1in,left=1in]{geometry}
\setlength{\emergencystretch}{3em} % prevent overfull lines
\setcounter{secnumdepth}{5}
% Make \paragraph and \subparagraph free-standing
\ifx\paragraph\undefined\else
  \let\oldparagraph\paragraph
  \renewcommand{\paragraph}[1]{\oldparagraph{#1}\mbox{}}
\fi
\ifx\subparagraph\undefined\else
  \let\oldsubparagraph\subparagraph
  \renewcommand{\subparagraph}[1]{\oldsubparagraph{#1}\mbox{}}
\fi


\providecommand{\tightlist}{%
  \setlength{\itemsep}{0pt}\setlength{\parskip}{0pt}}\usepackage{longtable,booktabs,array}
\usepackage{calc} % for calculating minipage widths
% Correct order of tables after \paragraph or \subparagraph
\usepackage{etoolbox}
\makeatletter
\patchcmd\longtable{\par}{\if@noskipsec\mbox{}\fi\par}{}{}
\makeatother
% Allow footnotes in longtable head/foot
\IfFileExists{footnotehyper.sty}{\usepackage{footnotehyper}}{\usepackage{footnote}}
\makesavenoteenv{longtable}
\usepackage{graphicx}
\makeatletter
\def\maxwidth{\ifdim\Gin@nat@width>\linewidth\linewidth\else\Gin@nat@width\fi}
\def\maxheight{\ifdim\Gin@nat@height>\textheight\textheight\else\Gin@nat@height\fi}
\makeatother
% Scale images if necessary, so that they will not overflow the page
% margins by default, and it is still possible to overwrite the defaults
% using explicit options in \includegraphics[width, height, ...]{}
\setkeys{Gin}{width=\maxwidth,height=\maxheight,keepaspectratio}
% Set default figure placement to htbp
\makeatletter
\def\fps@figure{htbp}
\makeatother
\newlength{\cslhangindent}
\setlength{\cslhangindent}{1.5em}
\newlength{\csllabelwidth}
\setlength{\csllabelwidth}{3em}
\newlength{\cslentryspacingunit} % times entry-spacing
\setlength{\cslentryspacingunit}{\parskip}
\newenvironment{CSLReferences}[2] % #1 hanging-ident, #2 entry spacing
 {% don't indent paragraphs
  \setlength{\parindent}{0pt}
  % turn on hanging indent if param 1 is 1
  \ifodd #1
  \let\oldpar\par
  \def\par{\hangindent=\cslhangindent\oldpar}
  \fi
  % set entry spacing
  \setlength{\parskip}{#2\cslentryspacingunit}
 }%
 {}
\usepackage{calc}
\newcommand{\CSLBlock}[1]{#1\hfill\break}
\newcommand{\CSLLeftMargin}[1]{\parbox[t]{\csllabelwidth}{#1}}
\newcommand{\CSLRightInline}[1]{\parbox[t]{\linewidth - \csllabelwidth}{#1}\break}
\newcommand{\CSLIndent}[1]{\hspace{\cslhangindent}#1}

\usepackage{booktabs}
\usepackage{longtable}
\usepackage{array}
\usepackage{multirow}
\usepackage{wrapfig}
\usepackage{float}
\usepackage{colortbl}
\usepackage{pdflscape}
\usepackage{tabu}
\usepackage{threeparttable}
\usepackage{threeparttablex}
\usepackage[normalem]{ulem}
\usepackage{makecell}
\usepackage{xcolor}
\usepackage{colortbl}
\makeatletter
\renewenvironment{table}%
  {\renewcommand\familydefault\sfdefault
   \@float{table}}
  {\end@float}
\makeatother
\KOMAoption{captions}{tableheading}
\makeatletter
\makeatother
\makeatletter
\makeatother
\makeatletter
\@ifpackageloaded{caption}{}{\usepackage{caption}}
\AtBeginDocument{%
\ifdefined\contentsname
  \renewcommand*\contentsname{Table of contents}
\else
  \newcommand\contentsname{Table of contents}
\fi
\ifdefined\listfigurename
  \renewcommand*\listfigurename{List of Figures}
\else
  \newcommand\listfigurename{List of Figures}
\fi
\ifdefined\listtablename
  \renewcommand*\listtablename{List of Tables}
\else
  \newcommand\listtablename{List of Tables}
\fi
\ifdefined\figurename
  \renewcommand*\figurename{Figure}
\else
  \newcommand\figurename{Figure}
\fi
\ifdefined\tablename
  \renewcommand*\tablename{Table}
\else
  \newcommand\tablename{Table}
\fi
}
\@ifpackageloaded{float}{}{\usepackage{float}}
\floatstyle{ruled}
\@ifundefined{c@chapter}{\newfloat{codelisting}{h}{lop}}{\newfloat{codelisting}{h}{lop}[chapter]}
\floatname{codelisting}{Listing}
\newcommand*\listoflistings{\listof{codelisting}{List of Listings}}
\makeatother
\makeatletter
\@ifpackageloaded{caption}{}{\usepackage{caption}}
\@ifpackageloaded{subcaption}{}{\usepackage{subcaption}}
\makeatother
\makeatletter
\@ifpackageloaded{tcolorbox}{}{\usepackage[skins,breakable]{tcolorbox}}
\makeatother
\makeatletter
\@ifundefined{shadecolor}{\definecolor{shadecolor}{rgb}{.97, .97, .97}}
\makeatother
\makeatletter
\makeatother
\makeatletter
\makeatother
\ifLuaTeX
  \usepackage{selnolig}  % disable illegal ligatures
\fi
\IfFileExists{bookmark.sty}{\usepackage{bookmark}}{\usepackage{hyperref}}
\IfFileExists{xurl.sty}{\usepackage{xurl}}{} % add URL line breaks if available
\urlstyle{same} % disable monospaced font for URLs
\hypersetup{
  pdftitle={How Do Household Energy Transitions Work?},
  pdfauthor={Jill Baumgartner (Co-PI); Sam Harper (Co-PI)},
  colorlinks=true,
  linkcolor={blue},
  filecolor={Maroon},
  citecolor={Blue},
  urlcolor={Blue},
  pdfcreator={LaTeX via pandoc}}

\title{How Do Household Energy Transitions Work?}
\author{Jill Baumgartner (Co-PI) \and Sam Harper (Co-PI)}
\date{2024-04-02}

\begin{document}
\maketitle
\ifdefined\Shaded\renewenvironment{Shaded}{\begin{tcolorbox}[sharp corners, frame hidden, boxrule=0pt, borderline west={3pt}{0pt}{shadecolor}, enhanced, interior hidden, breakable]}{\end{tcolorbox}}\fi

\renewcommand*\contentsname{Table of contents}
{
\hypersetup{linkcolor=}
\setcounter{tocdepth}{3}
\tableofcontents
}
\hypertarget{abstract}{%
\subsection*{Abstract}\label{abstract}}

\hypertarget{introduction}{%
\subsubsection*{Introduction}\label{introduction}}

\hypertarget{methods}{%
\subsubsection*{Methods}\label{methods}}

\hypertarget{results}{%
\subsubsection*{Results}\label{results}}

\hypertarget{conclusions}{%
\subsubsection*{Conclusions}\label{conclusions}}

\hypertarget{introduction-1}{%
\section{Introduction}\label{introduction-1}}

China is deploying an ambitious policy to transition up to 70\% of
households in northern China to clean space heating, including a
large-scale roll out across rural and peri-urban Beijing. To meet this
target the Beijing municipal government announced a two-pronged program
that designates coal-restricted areas and simultaneously offers
subsidies to night-time electricity rates and for the purchase and
installation of electric-powered, air-source heat pumps to replace
traditional coal-heating stoves. The policy was piloted in 2015 and,
starting in 2016, was rolled out on a village-by-village basis; however
there is uncertainty as to when villages will receive the program. The
variability in when the policy is applied to each village allows us to
treat the roll-out of the program as a quasi-randomized intervention.
Households may also be differentially affected by this program due to
factors such as financial constraints, preferences and social capital,
and there is uncertainty about whether and how this intervention may
affect indoor and outdoor air pollution, as well as heating behaviors
and health outcomes.

\hypertarget{background}{%
\section{Background}\label{background}}

\hypertarget{context-for-the-policy}{%
\subsection{Context for the policy}\label{context-for-the-policy}}

Beijing has a temperate continental monsoon climate characterized by
cold, dry winters and hot, humid summers. Access to central heating is
limited to urban areas and households in most rural and peri-urban areas
of Beijing historically heated their homes using mostly coal and
sometimes biomass-fueled heaters or kangs (a traditional Chinese
combined cooking and heating stove). Household coal burning was a major
contributor to indoor and outdoor air pollution in northern China,
especially in winter. Prior to 2016, coal fuel was used to meet over
80\% of northern China's space heating demand (Dispersed Coal Management
Research Group 2023). At that time, household coal-fuelled heaters
burned approximately half of the over 400 million tons of coal used for
space heating (Group 2016) and contributed to \textasciitilde30\% of
northern China's wintertime air pollution. In 2013, exposure to ambient
fine particulate matter from coal combustion - from industry,
electricity, and domestic sources - was the largest estimated
contributor to population exposure to PM\textsubscript{2.5} and
contributed to an estimated 366,000 premature deaths annually in China
(Group 2016).

Replacing household coal stoves with clean heating alternatives was
considered a potentially impactful intervention to reduce outdoor
PM\textsubscript{2.5} across the region and mitigate its health impacts.
A number of clean heating options including electric heat pumps, gas
heaters, and electric resistance heaters with thermal storage were
widely promoted by the Chinese government (Dispersed Coal Management
Research Group 2023). By 2021, over 36 million households in northern
China were treated by the policy and an estimated 21 million additional
households expected to be treated by 2025. Whether this large-scale
energy policy yielded air quality and health benefits remains a critical
and unresolved question.

\hypertarget{prior-evidence-of-household-energy-interventions-and-air-pollution}{%
\subsection{Prior evidence of household energy interventions and air
pollution}\label{prior-evidence-of-household-energy-interventions-and-air-pollution}}

Household energy interventions, mostly cooking-related, that replace
solid fuel stoves with cleaner-burning alternatives have been
implemented and studied extensively in countries including China over
the past several decades. While their introduction is expected to reduce
air pollution emissions and subsequent exposures, there is still no
consensus about their effectiveness in achieving health-relevant air
pollution reductions in real-world settings (Quansah et al. 2017). In
particular, the effectiveness of large-scale household energy programs
like China's Coal Ban and Heat Pump (CBHP) subsidy program Clean Heating
Plan has been rarely empirically investigated, especially at sub-city
spatial resolution. In Ireland, county-level residential coal bans in
the 1990s were associated with 40-70\% decreases in black smoke
concentrations in ban-affected areas (Dockery et al. 2013). In
Australia, a wood-burning stove exchange lower daily wintertime PM10
from 44 to 27 µg/m3 (Johnston et al. 2013), and clean energy policies in
New Zealand were associated with 11-36\% reductions in winter PM10
(Scott and Scarrott 2011). The few evaluations of the Clean Heating Plan
observed small decreases in outdoor PM\textsubscript{2.5} (-7 to -2.4
µg/m3) in municipalities or prefectures in the policy compared with
neighboring areas not affected by the policy (Niu et al. 2024; Song et
al. 2023; Tan et al. 2023; Yu et al. 2021), and a recent modeling study
estimated 36\% lower personal exposure to PM\textsubscript{2.5} based on
household-reported changes in fuel use (Meng et al. 2023). However, none
of these studies included field-based measurements of air pollution or
personal exposures, which are known to differ considerably from modeled
estimates (Thompson et al. 2019), and few accounted for secular changes
in air quality over time, limiting any conclusions about the air quality
benefits of the Clean Heating Plan.

\hypertarget{prior-evidence-on-clean-energy-interventions-and-cardiovascular-outcomes}{%
\subsection{Prior evidence on clean energy interventions and
cardiovascular
outcomes}\label{prior-evidence-on-clean-energy-interventions-and-cardiovascular-outcomes}}

Most previous health assessments of household energy interventions have
focused on cookstoves rather than heating. Randomized trials of less
polluting cookstoves generally indicate a potential cardiovascular
benefit. In older Guatemalan women, a chimney stove intervention lowered
exposure to air pollution and reduced the occurrence of nonspecific
ST-segment depression. That same study and other trials in Nigeria and
Ghana observed blood pressure reductions (range: −3.7 to −1.3mmHg) in
women assigned to gas, ethanol, or improved combustion biomass stoves,
and are supported by non-randomized, controlled intervention studies in
Nicaragua and Bolivia (blood pressure reductions from −5.9 to −5.5mmHg).
A recent multi-country randomized trial did not observe a protective
effect of gas stoves on gestational blood pressure despite large
reductions in air pollution, though the participants were younger (mean
age: 25y) than in intervention studies showing a blood pressure benefit
(mean age range: 28 to 53y).

The few population-based evaluations of household energy policies also
indicate a cardio-respiratory benefit. Residential wood-burning bans
were associated with reductions in cardiovascular hospitalizations
(-7\%) in California and with reduced cardiovascular (-17.9\%) and
respiratory (−22.8\%) mortality in Australia, though neither study fully
controlled for secular improvements in health. Most relevant to our
study are two quasi-experimental assessments of coal replacement
policies. In Ireland, reductions in respiratory not but cardiovascular
mortality were observed following a coal ban. A multi-city study of
Chinese adults in cities where the Clean Heating Policy was piloted
compared with adults in cities not in the pilot observed small decreases
in chronic lung diseases (-3.0 to -1.1\%) but no change in
physician-diagnosed cardiovascular diseases, potentially due to the
short (one-year) post-policy evaluation period or confounding by other
unmeasured city-wide air quality or health-related policies.

\hypertarget{assessing-dynamic-and-heterogeneous-treatment-effects}{%
\subsection{Assessing dynamic and heterogeneous treatment
effects}\label{assessing-dynamic-and-heterogeneous-treatment-effects}}

Since 2015, thousands of villages across Beijing and northern China
entered the policy prior to the start of the heating season each year.
Given the many behavioral, social, or economic factors that might affect
both new heater use and coal stove suspension (e.g., energy prices and
availability, wintertime temperature, COVID-19 pandemic, user
preferences), it is possible that the effect of the policy on air
pollution and health may be dynamic over time and/or heterogeneous
across treatment cohorts. Thus, it may be important to study both the
overall and group-time effects of the policy.

\hypertarget{evaluating-the-mechanisms-through-which-policies-may-affect-health-outcomes.}{%
\subsection{Evaluating the mechanisms through which policies may affect
health
outcomes.}\label{evaluating-the-mechanisms-through-which-policies-may-affect-health-outcomes.}}

With several notable exceptions, decades of household energy
intervention studies indicating limited or even no benefit to air
quality or health very clearly demonstrate that intervention is not
simple when studying an exposure that is as central to daily life as
household energy use (add refs). Household energy intervention and
policies, particularly those implemented at the household- or
village-scales, can produce multiple behavioral, environmental, and
health-related changes, making it important to investigate the
mechanisms through which such policies exert their health impacts
(Dominici et al. 2014). The health benefits achievable with transition
from traditional coal stoves to a new electric home heating system, for
example, may be influenced by factors including outdoor air quality (Lai
2019), the desirability and usage patterns of new and traditional stoves
(Ezzati and Baumgartner 2017) indoor temperature (Lewington et al.
2012), or behaviors including physical activity (Lindemann et al. 2017).
Only recently were these mediating factors considered in assessments of
household energy interventions and health, and rarely in any
comprehensive way (Rosenthal et al. 2018). Understanding these
mechanisms can provide valuable scientific insight into the success (or
failure) of clean energy policies like the Clean Heating Policy in
meeting its air quality and health goals, and may answer questions that
can inform the design of more effective future energy interventions. For
example, is there successful uptake of the intervention or policy? Does
the policy lead to heating behavior changes that result in colder homes
and thus offsets any cardiovascular-enhancing effects of improved air
quality? Answers to these questions are facilitated by the analysis of
mediating pathways.

\hypertarget{specific-aims-and-overarching-approach}{%
\section{Specific Aims and Overarching
Approach}\label{specific-aims-and-overarching-approach}}

This study used three data collection campaigns in winter 2018/19,
winter 2019/20, and winter 2021/22, as well as a partial campaign in
winter 2020/21 to advance the following aims:

\begin{enumerate}
\def\labelenumi{\arabic{enumi}.}
\item
  Estimate how much of the policy's overall effect on health, including
  respiratory symptoms and cardiovascular outcomes (blood pressure,
  central hemodynamics, blood inflammatory and oxidative stress
  markers), can be attributed to its impact on changes in
  PM\textsubscript{2.5};
\item
  Quantify the impact of the policy on outdoor air quality and personal
  air pollution exposures, and specifically the source contribution from
  household coal burning (Previously Aim 3);
\item
  Quantify the contribution of changes in the chemical composition of
  PM\textsubscript{2.5} from different sources to the overall effect on
  health outcomes (Previously Aim 2).
\end{enumerate}

\hypertarget{study-design-and-methods}{%
\section{Study Design and Methods}\label{study-design-and-methods}}

\hypertarget{location-context-and-recruitment}{%
\subsection{Location, context, and
recruitment}\label{location-context-and-recruitment}}

In December 2018 we recruited 50 villages across 4 administrative
districts (Fangshan, Huairou, Mentougou, and Miyun) in the Beijing
Municipality Region in northern China. Villages were selected on the
basis of their residents primarily using household coal stoves for
heating, and because roughly half of the villages were expected to enter
into the policy over the course of our study. We used local guides in
each village to help determine a roster of households that were not
vacant during the winter months, from which we randomly selected
households to recruit for participation.

\begin{figure}[H]

{\centering \includegraphics[width=0.8\textwidth,height=\textheight]{images/policy-implementation-map.png}

}

\caption{\label{fig-map}Map of village implementation of CBHP policy}

\end{figure}

We recruited approximately 20 households in each village and randomly
selected one eligible person from each household to participate.
Participants were eligible to participate if they were over 40 years
old, lived in the study villages, were not planning to move out of the
village in the next year, and were not on current immunotherapy or
treatment with corticosteroids. Research staff introduced the study and
its measurements to an eligible person in each household and answered
any questions related to the study. All participants provided written
informed consent prior to joining the study. The study protocols were
approved by research ethics boards at Peking University
(IRB00001052-18090) and McGill University (A08-E53-18B).

\hypertarget{data-collection-overview}{%
\subsection{Data Collection Overview}\label{data-collection-overview}}

Trained staff traveled to participants' homes to conduct tablet-based
household and individual questionnaires, measure their blood pressure,
and place air pollution and temperature monitors in their homes.
Anthropometrics (height, weight, and waist circumference), measurement
of airway inflammation, and whole blood samples were obtained no more
than a month later at a village clinic in the first and second
campaigns. In the fourth campaign, which occurred during the COVID-19
pandemic, anthropometric measurements and airway inflammation were
assessed in participant homes to avoid group contact and blood samples
were not collected.

\hypertarget{air-pollution}{%
\subsubsection{Air Pollution}\label{air-pollution}}

\hypertarget{outdoor-air-pollution}{%
\paragraph{Outdoor air pollution}\label{outdoor-air-pollution}}

In each village, two sensors were set up to measure community
PM\textsubscript{2.5} at different locations in each village. One sensor
was placed near the center of the village, and the other was placed no
less than 500m away from the centrally-located sensor. Sensors were
placed at least 1.5m above the ground and in a location without a
visible point source of PM\textsubscript{2.5}.

We collected filter-based community PM\textsubscript{2.5} samples to
calibrate the sensor-based PM\textsubscript{2.5} measurements.
Ultrasonic Personal Aerosol Samplers (UPAS, Access Sensor Technologies,
Fort Collins, CO, USA) were used to collect filter-based
PM\textsubscript{2.5} samples with a flow rate of 1.0 L/min (Volckens et
al. 2017). Samplers housed 37mm PTFE filters (VWR, 2.0-μm pore size) and
were equipped with a cyclone inlet with a 2.5μm cut point designed to
perform under the sampling flow rate. For community outdoor
measurements, a UPAS was co-located with each PM\textsubscript{2.5}
sensor in each village in rotation. Every week, the used filters were
removed and replaced with a new filter. In total, 126, 371, and 289
filter-based, community outdoor PM\textsubscript{2.5} samples were
collected in seasons 1, 2, and 4, respectively. Field blank filters were
collected at a rate of \textasciitilde10\%, subject to the same field
conditions as samples.

For PM\textsubscript{2.5} sensor calibration and quality control, all PM
sensors were co-located with a reference-grade PM\textsubscript{2.5}
instrument (Model 5030 Synchronized Hybrid Ambient Realtime Particulate
(SHARP) Monitor, Thermo Fisher Scientific, United States) on the rooftop
of a building at Peking University campus for 7 to 10 days before and
after each field campaign. Sensor-measured PM\textsubscript{2.5}
concentrations were highly correlated with those measured by the SHARP
(Spearman correlation coefficients (rho) of 0.95 and 0.82 in pre- and
post-calibration, respectively).

We established linear regression models between the filter-based
PM\textsubscript{2.5} mass concentrations (i.e., the reference
concentrations) and the sensor-based PM\textsubscript{2.5}
concentrations averaged over the same sampling period as the
filter-based samples. The slopes of the models were used as the
adjustment factors for the sensor-based PM\textsubscript{2.5}
concentrations. Separate regression models were conducted for indoor and
outdoor sensors given the sensitivity of the sensors to relative
humidity, temperature, and particle sources, which may differ for indoor
versus outdoor conditions.

\hypertarget{indoor-pm2.5}{%
\paragraph{\texorpdfstring{Indoor
PM\textsubscript{2.5}}{Indoor PM2.5}}\label{indoor-pm2.5}}

In the second and fourth field seasons (i.e., Season 2 and Season 4), we
randomly selected six households from the 20 recruited in each village
to measure indoor concentrations of PM\textsubscript{2.5}. In Season 4,
we aimed to monitor indoor PM\textsubscript{2.5} in the same households
where we measured indoor PM\textsubscript{2.5} in Season 2. If a
household dropped out of the project or declined indoor
PM\textsubscript{2.5} monitoring, we then recruited another household
already enrolled in this study to measure indoor PM\textsubscript{2.5}.
In total, indoor measurements were conducted in 300 households in both
Season 2 and Season 4 (Table~\ref{tbl-pm-sample}).

\hypertarget{tbl-pm-sample}{}
\begin{table}
\caption{\label{tbl-pm-sample}Household recruitment for overall and indoor air quality measurements. }\tabularnewline

\centering
\begin{tabular}{lrrrrrr}
\toprule
\multicolumn{1}{c}{ } & \multicolumn{3}{c}{Overall} & \multicolumn{3}{c}{Indoor} \\
\cmidrule(l{3pt}r{3pt}){2-4} \cmidrule(l{3pt}r{3pt}){5-7}
Sample & Season 1 & Season 2 & Season 4 & Season 1 & Season 2 & Season 4\\
\midrule
New recruitment & 977 & 0 & 196 & 300 & 68 & 52\\
Households from Season 1 & \textbackslash{} & \textbackslash{} & 866 & 0 & 780 & 0\\
Households from Season 2 & \textbackslash{} & \textbackslash{} & \textbackslash{} & \textbackslash{} & 162 & 248\\
Total recruitment & 977 & 0 & 1062 & 300 & 1010 & 300\\
\bottomrule
\end{tabular}
\end{table}

Time-resolved indoor PM\textsubscript{2.5} concentrations were measured
using the same commercially available sensor (PMS7003 Plantower, Zefan,
Inc.) as was used for outdoor sensor-based PM\textsubscript{2.5}
measurements and recorded PM\textsubscript{2.5} concentrations every 1
min. The sensor was placed on a table in a room where participants
reported spending most of their time when awake, e.g., a living room or
bedroom. Indoor PM\textsubscript{2.5} sensors were deployed between late
November and mid January within field seasons (i.e., Season 2 and Season
4), depending on the village and household visit schedule. The
measurement continued from the time of deployment until sensors were
recollected from homes in late April.

We randomly selected three households from the six households in which
we deployed PM\textsubscript{2.5} sensors to co-locate a filter-based
PM\textsubscript{2.5} sampler with the PM\textsubscript{2.5} sensor. We
collected a 24-h PM\textsubscript{2.5} filter sample at the first 24-h
of indoor PM\textsubscript{2.5} sensor measurements. Filter-based
PM\textsubscript{2.5} samples were collected using Ultrasonic Personal
Aerosol Samplers (UPAS, Access Sensor Technologies) or Personal Exposure
Monitors (PEMs, Apex Pro) operating with flow rates of 1.0 and 1.8
L/min, respectively. Both samplers housed 37 mm PTFE filters (VWR,
2.0-μm pore size) and were equipped with a cyclone inlet with a 2.5 μm
cut point designed to perform under the corresponding sampling flow
rate. After 24-h, the samplers were retrieved and loaded with new
filters for measurements in other villages, once the previous sample
filters were removed and stored for later analysis. In total, we
successfully collected 149 and 148 indoor PM\textsubscript{2.5} filter
samples in Seasons 2 and 4, respectively.

\hypertarget{personal-exposure-to-pm2.5-and-black-carbon}{%
\paragraph{\texorpdfstring{Personal exposure to PM\textsubscript{2.5}
and black
carbon}{Personal exposure to PM2.5 and black carbon}}\label{personal-exposure-to-pm2.5-and-black-carbon}}

To measure personal exposure we used two types of samplers: Personal
Exposure Monitors (PEMs, Apex Pro; Casella, UK) and Ultrasonic Personal
Aerosol Samplers (UPAS, Access Sensor Technologies, Fort Collins, CO,
USA). PEMs actively sampled air at a flow rate of 1.8 L/min, and UPAS
sampled air at 1.0 L/min (Volckens et al. 2017). Both samplers housed 37
mm PTFE filters (VWR, 2.0-μm pore size) and were equipped with a cyclone
inlet with a 2.5 μm cutpoint. Sampler flow rates were calibrated the
night before deployment and also measured after the sampling period.
Very few post-sampling measurements (\textless2\%) deviated from the
target flow rate by \textgreater{} +/-10\%. Participants were instructed
to wear a small waistpack (for the PEM and sampling pump) or an arm band
or cross-body sling (for the UPAS) for 24 hours, which they could remove
from their body and place within 2 meters while sleeping, sitting, or
bathing. Field blanks for personal air pollution exposure measurements
were collected at a rate of \textasciitilde10\% in each village. All
filters were placed in individually labeled cases, sealed in plastic
bags, and then transported to a field laboratory and immediately stored
in a -20°C freezer. Following completion of the field sampling campaign,
the samples and blanks were transported to Colorado State University,
where they were stored in a -20°C freezer prior to PM\textsubscript{2.5}
mass measurement and chemical analysis of PM.

All filters were placed in an environmentally-controlled equilibration
chamber (21-22 °C, 30-34\% relative humidity) for at least 24 hours
before tare and gross weighing. Before each weight was taken, filters
were discharged by a polonium-210 strip. Filters were weighed on a
microbalance (Mettler Toledo Inc., XS3DU, USA) with 1-μg resolution in
triplicate or more, until the differences among three weights were less
than 3 μg. The average of three readings was used to determine filter
mass, which was then blank-corrected using the median value of blank
filters {[}3 μg for UPAS-collected filters (53\% of samples); 33 μg for
PEM-collected filters (47\% of filter samples){]}, and PM2.5
concentrations were calculated by dividing the mass by the sampled air
volume.

Filters were analyzed for black carbon (BC) using an optical
transmissometer data acquisition system (SootScan\^{}TM OT21 Optical
Transmissometer; Magee Scientific, Berkeley, CA, USA). Light attenuation
through each filter was measured before and after sampling in the field.
To calculate BC mass, the difference between the pre- and post- light
attenuation was converted to a mass surface loading using the classical
Magee mass absorption cross-sections of 16.6 m2/g for the 880 nm channel
optical BC (Ahmed et al. 2009). BC concentrations were calculated by
multiplying surface loadings by the sampled surface area of the filters
(8.6 cm2 for UPAS-collected filters; 7.1 cm2 for PEM-collected filters),
correcting for the field blank mass using the median value of blanks
(0.31 μg for UPAS-collected filters; 0.01 μg for PEM-collected filters),
and finally dividing by the sampled air volume.

\emph{Field equipment (Figure) to be added}

\hypertarget{outdoor-and-indoor-household-air-temperature}{%
\subsubsection{Outdoor and indoor (household) air
temperature}\label{outdoor-and-indoor-household-air-temperature}}

Hourly outdoor temperature and relative humidity data were obtained from
the extensive network of meteorological stations in Beijing
(http://beijingair.sinaapp.com). We measured indoor temperature in all
participant homes prior to blood pressure measurement. In a random 75\%
subsample of households in each campaign, we also placed a real-time
temperature sensor (iButton DS1921G-F5; Thermochron, Maxim Inc., USA) in
the room where participants report spending most of their daytime hours
when indoors. Sensors were wall-mounted at a standardized height
(\textasciitilde1.5 to 2 meters), away from major heating sources,
windows, and doors, and were programmed to log temperature every 125
minutes for up to 4 months to capture the full winter and early spring
when heaters may still be intermittently used.

\hypertarget{household-stove-use-using-sensors}{%
\subsubsection{Household stove use using
sensors}\label{household-stove-use-using-sensors}}

{[}\ldots to be completed\ldots{]}

\hypertarget{questionnaires}{%
\subsubsection{Questionnaires}\label{questionnaires}}

Field staff administered household and individual-level questionnaires
to assess household demographic information, household assets, house
structure, stove and fuel use patterns (including a complete roster of
heating methods and their contributions in each room), and smoking. We
used Surveybe computer-assisted personal interview (CAPI) software to
collect survey data via handheld electronic tablets. Questions were read
to participants in Mandarin-Chinese, and their responses were recorded
into tablets.

The questionnaire and other study measurements were tested prior to the
start of data collection for this study in 12 households located in a
Beijing village that was eligible for our study but was instead selected
for testing. We used the test village to assess whether the questions
were understandable and interpreted as intended and to identify any
problems with the study measurements or their implementation. Study
protocols were subsequently adapted prior to the start of data
collection.

In addition to household and individual participant questionnaires, we
also conducted village surveys with one representative from each village
committee to inquire about any other policies or programs being
implemented in the village (e.g., biomass burning bans) and to
understand how the policy was implemented in that village. Committee
members answered questions about assignment versus application to the
policy, any renovations required by the upper-level government, level of
subsidies provided for heating stoves and electricity, and technical and
logistic guidance to villagers.

\hypertarget{blood-pressure}{%
\subsubsection{Blood pressure}\label{blood-pressure}}

Following 5 min of quiet rest, at least three brachial and central
systolic (bSBP/cSBP) and diastolic (bDBP/cDBP) blood pressures (BPs)
were taken by trained staff at 1 min apart on the participant's
supported right arm. We used an automated oscillometric device (BP+;
Uscom Ltd, New Zealand) that estimates central pressures from the
brachial cuff pressure fluctuations. Central pressures were previously
validated against invasive cBP measurements (Costello et al. 2015; Lowe
et al. 2009). The BP devices were factory calibrated by the manufacturer
prior to the start of the first and fourth campaigns. Up to five
measurements were taken if the difference between the last two was
\textgreater5 mmHg or staff were unable to obtain a reading. The BP
measurements were conducted in the participant's home and staff were
trained to follow strict quality control procedures, including use of an
appropriately sized cuff, correct positioning on the arm, both feet on
the ground, and ensuring 5 min of quiet rest before measurement (SOP:
https://osf.io/gmka5). The average of the final two measurements was
used for statistical analysis unless only one BP measurement was
obtained (n = 13 observations), in which case, a single measurement was
used. The time of day, day of the week, and indoor temperature prior to
BP measurement were also recorded.

\hypertarget{self-reported-respiratory-symptoms-and-airway-inflammation}{%
\subsubsection{Self-reported respiratory symptoms and airway
inflammation}\label{self-reported-respiratory-symptoms-and-airway-inflammation}}

During questionnaire assessment, participants were asked about recent
airway symptoms including cough, phlegm, wheeze, and tightness in the
chest using standard American Thoracic Society (ATS) questions that were
validated for use in Mandarin-Chinese. In a 25\% random subsample of
participants in each season, we also measure exhaled nitric oxide, a
non-invasive marker of airway inflammation, using a handheld device
(NIOX VERO®, Aerocrine, Solna, Sweden), following ATS recommendations
and guidelines (ATS/ERS 2005).

\hypertarget{blood-inflammatory-and-oxidative-stress-markers}{%
\subsubsection{Blood inflammatory and oxidative stress
markers}\label{blood-inflammatory-and-oxidative-stress-markers}}

Trained nurses collected 4 ml of whole blood in a labeled vacutainer via
venipuncture using standard techniques (Tuck 2009). Fasting blood
samples were collected in the morning and stored at 4-10°C prior to
centrifugation. Two serum aliquots were placed a -30°C freezer for
temporary storage. Collection-to-storage time was \textless4 hrs for all
samples in both campaigns with blood collection. Within 3-5 days of
collection, the samples were transported in styrofoam containers with
dry ice to a -80°C freezer with a backup generator and alarm system at
Peking University. The first aliquot was analyzed for glucose and a
complete lipid profile within two months of collection, the results of
which were given to participants. The second aliquot was stored in the
-80°C freezer for analysis of biomarkers of systemic inflammation
(C-reactive protein, interleukin-6, tumour necrosis factor alpha) and
oxidative stress (8-hydroxy-2'-deoxyguanosine and malondialdehyde) at
the University of the Chinese Academy of Sciences in summer of 2023.
These biomarkers were selected because they are associated with the
development of cardiovascular disease and events (e.g., Danesh 2008;
Pearson 2003; Ridker 2000; 2001; ERF 2012), and both acute and
longer-term exposures to air pollution have been associated with changes
in inflammatory and oxidative stress markers (e.g., Pope 2004; Rückerl
2007; Rich 2012; Kipen 2010; Huang 2012).

We followed standard methods for analysis (FDA Guidance, 2018). For
inflammatory markers (IL-6, TNF-α, CRP), the optic densities (OD) of all
samples were measured using an automated ELISA reader. Every plate had 8
standard samples used to generate a standard curve that related OD and
standard inflammatory marker concentration. A standard curve for each
microplate was generated by a computer software program based on a
4-parameter method. Each plate included at least 3 control samples to
ensure the stability of standard curves. All samples, standards, and
controls were measured in duplicate, and the average was used for
statistical analysis. For oxidative stress biomarkers (MDA and 8-OHdG),
the chromatographic peak areas of all samples were measured using HPLC
with UV detector and HPLC-MS/MS. Every plate had 7 standard samples used
to generate a standard curve that related peak area and concentration of
each standard oxidative stress marker. A standard curve for each plate
was generated using a computer software program based on a linear
method. Each plate included at least 3 control samples to ensure the
stability of standard curves. Standards and controls were measured in
duplicate and samples were measured once due to high precision in a
pilot study (Food and Drug Administration 2018).

\hypertarget{anthropometric-measurements.}{%
\subsubsection{Anthropometric
measurements.}\label{anthropometric-measurements.}}

Body weight, height, and waist circumference were measured at the clinic
visit in the first two campaigns and in participant homes in the last
campaign. Weight was measured in light indoor clothing without shoes in
kilograms to one decimal place, using standing scales supported on a
steady surface. Height was measured without shoes in centimeters to one
decimal place with a stadiometer. Waist circumference was measured at 1
cm above the navel at minimal respiration in centimeters to one decimal
place.

\hypertarget{measuring-policy-impacts}{%
\subsection{Measuring policy impacts}\label{measuring-policy-impacts}}

To understand how Beijing's policy works we used a
difference-in-differences (DiD) design (Callaway 2020), leveraging the
staggered rollout of the policy across multiple villages to estimate its
impact on health outcomes and understand the mechanisms through which it
works. Simple comparisons of treated and untreated (i.e., control)
villages after the CBHP policy has been implemented are likely to be
biased by unmeasured village-level characteristics (e.g., migration,
average winter temperature) that are associated with health outcomes.
Similarly, comparisons of only treated villages before and after
exposure to the program are susceptible to bias by other factors
associated with changes in outcomes over time (i.e., secular trends,
impacts of the COVID-19 pandemic). By comparing \emph{changes} in
outcomes among treated villages to \emph{changes} in outcomes among
untreated villages, we can control for any unmeasured time-invariant
characteristics of villages as well as any general secular trends
affecting all villages that are unrelated to the policy.

\begin{figure}[H]

{\centering \includegraphics[width=0.5\textwidth,height=\textheight]{hei-report_files/figure-pdf/fig-didfig-1.pdf}

}

\caption{\label{fig-didfig}Stylized example of
difference-in-differences}

\end{figure}

The DiD design compares outcomes before and after an intervention in a
treated group relative to the same outcomes measured in a control group.
The control group trend provides the crucial ``counterfactual'' estimate
of what would have happened in the treated group had it not been
treated. By comparing each group to itself, this approach helps to
control for both measured and unmeasured fixed differences between the
treated and control groups. By measuring changes over time in outcomes
in the control group unaffected by the treatment, this approach also
controls for any unmeasured factors affecting outcome trends in both
treated and control groups. This is important since there are often many
potential factors affecting outcome trends that cannot be disentangled
from the policy if one only studies the treated group (as in a
traditional pre-post design).

The canonical DiD design (Card and Krueger 1994) compares two groups
(treated and control) at two different time periods (pre- and
post-intervention, Figure~\ref{fig-didfig}). In the first time period
both groups are untreated, and in the second time period one group is
exposed to the intervention. If we assume that the differences between
the groups would have remained constant in the absence of the
intervention (parallel trends assumption), then an unbiased estimate of
the impact of the intervention in the post period can be calculated by
subtracting the pre-post difference in the untreated group from the
pre-post difference in the treated group.

However, when multiple groups are treated at different time periods, the
most common approach has been to use a two-way fixed effects model to
estimate the impact of the intervention which controls for secular
trends and differences between districts. However, recent evidence
suggests that the traditional two-way fixed effects estimation of the
treatment effect may be biased in the context of heterogeneous treatment
effects (Callaway and Sant'Anna 2021; Goodman-Bacon 2021)

\hypertarget{measuring-pathways-and-mechanisms}{%
\subsection{Measuring pathways and
mechanisms}\label{measuring-pathways-and-mechanisms}}

To estimate how much of the CBHP intervention may work through different
mechanisms, we used causal mediation analysis. Causal approaches to
mediation attempt to discern between, and clarify the necessary
assumptions for identifying, different kinds of mediated effects. Taking
as an example the DAG in Figure~\ref{fig-dag1}, with \(T\) as the
policy, \(M\) as PM\textsubscript{2.5}, and \(Y\) as systolic blood
pressure, we can define the controlled direct effect (\(CDE\)) as the
effect of the CBHP policy on systolic blood pressure if we fix the value
of PM\textsubscript{2.5} to a certain reference level for the entire
population. For example, we can estimate the impact of the policy on
health outcomes while holding PM\textsubscript{2.5} at a uniform level
of average background exposure, or some other hypothetical level.

\begin{figure}[H]

{\centering \includegraphics[width=0.5\textwidth,height=\textheight]{hei-report_files/figure-pdf/fig-dag1-1.pdf}

}

\caption{\label{fig-dag1}Example of direct and indirect effects with
outcome (\(Y\)), treatment (\(T\)), and mediator (\(M\))}

\end{figure}

Although other mediated effects such as ``natural'' direct and indirect
effects are theoretically estimable (VanderWeele 2015), they involve
challenging ``cross-world'' assumptions that are difficult to anchor in
policy (Naimi et al. 2014). Other approaches to mechanisms have focused
on principal stratification (e.g., Zigler et al. 2016), although
conceptual difficulties with identifying the (unverifiable) principal
strata make it challenging for questions of mediation. Because
controlled direct effects are considered more directly policy relevant
for public health, we focus on estimating these mediated quantities.

\hypertarget{data-analysis}{%
\section{Data Analysis}\label{data-analysis}}

\hypertarget{total-effect}{%
\subsection{Total Effect}\label{total-effect}}

To estimate the total effect of the policy we used a DiD analysis that
accommodates staggered treatment rollout. To allow for heterogeneity in
the context of staggered rollout we used `extended' two-way fixed
effects (ETWFE) models (Wooldridge 2021) to estimate the total effect of
the CBHP policy. The mean outcome (replaced by a suitable link function
\(g(\cdot)\) for binary or count outcomes) was defined using a set of
linear predictors:

\begin{equation}\protect\hypertarget{eq-etwfe}{}{Y_{ijt}=g(\mu_{ijt}) = \alpha + \sum_{r=q}^{T} \beta_{r} d_{r} + \sum_{s=r}^{T} \gamma_{s} fs_{t}+ \sum_{r=q}^{T} \sum_{s=r}^{T} \tau_{rt} (d_{r} \times fs_{t}) + \varepsilon_{ijt}}\label{eq-etwfe}\end{equation}

where \(Y_{ijt}\) is the outcome for individual \(i\) in village \(j\)
at time \(t\), \(d_{r}\) represent treatment cohort dummies, i.e., fixed
effects for cohorts of villages that were first exposed to the policy at
the same time \(q\) (e.g., in 2019, 2020, or 2021), \(fs_{t}\) are time
fixed effects corresponding to different winter data collection
campaigns (2018-19, 2019-20, or 2021-22), and \(\tau_{rt}\) are the
cohort-time \emph{ATTs}. The ETWFE and other approaches that allow for
several (potentially heterogenous) treatment effects may also be
averaged to provide a weighted \(ATT\). Several potential possibilities
are feasible, including weighting by treatment cohorts or time since
policy adoption (Goin and Riddell 2023)

\hypertarget{mediation-analysis}{%
\subsection{Mediation Analysis}\label{mediation-analysis}}

As noted above, with respect to the mediation analysis we are chiefly
interested in the \(CDE\), which can be derived by adding relevant
mediators \(M\) to this model. If we also allow for exposure-mediator
interaction and potentially allow for adjustment for confounders \(W\)
of the mediator-outcome effect, we can extend equation
Equation~\ref{eq-etwfe} as follows:

\begin{equation}\protect\hypertarget{eq-etwfem}{}{
\begin{aligned}
Y_{ijt}=g(\mu_{ijt}) = \alpha + \sum_{r=q}^{T} \beta_{r} d_{r} + \sum_{s=r}^{T} \gamma_{s} fs_{t}+ \sum_{r=q}^{T} \sum_{s=r}^{T} \tau_{rt} (d_{r} \times fs_{t}) \\ + \delta M_{it} + \sum_{r=q}^{T} \sum_{s=r}^{T} \eta_{rt} (d_{r} \times fs_{t} \times M_{it}) + \zeta \mathbf{W} + \varepsilon_{ijt}
\end{aligned}
}\label{eq-etwfem}\end{equation}

where now \(\delta\) is the conditional effect of the mediator \(M\) at
the reference level of the treatment (again, represented via the series
of group-time interaction terms), and the collection of \(\eta\) terms
are coefficients for the product terms allowing for mediator-treatment
interaction. Finally, \(\zeta\) is a vector of coefficients for the set
of confounders contained within \(\mathbf{W}\).

As noted above, in the staggered DiD framework that allows for
heterogeneity we do not have a single treatment effect but a collection
of group-time treatment effects that may be averaged in different ways.
This extends to the estimation of the \(CDE\), in which case we will
also have several \(CDE\)s that can be averaged to make inferences about
the extent to which the policy's impact is mediated by
\emph{PM\textsubscript{2.5}}. Based on the setup in
Equation~\ref{eq-etwfem} the \(CDE\) is estimated as:
\(\delta + \eta_{rt}MT\). In the absence of interaction between the
exposure and the mediator (i.e., \(\eta_{rt}=0\)) the \(CDE\) will
simply be the estimated treatment effects
\(\sum_{r=q}^{T} \sum_{s=r}^{T} \tau_{rt}\), i.e., the effect of the
policy holding \(M\) constant. For a valid estimate of the \(CDE\) we
must account for confounding of the mediator-outcome effect, represented
by \(W\) in the equation above. Baseline measures of both the outcome
and the proposed mediators inherent in our DiD strategy will help to
reduce the potential for unmeasured confounding of the mediator-outcome
effect (Keele et al. 2015).

\hypertarget{results-1}{%
\section{Results}\label{results-1}}

Study flowchart

\begin{figure}[H]

{\centering \includegraphics[width=0.8\textwidth,height=\textheight]{images/participation-flow-chart-Mar18.png}

}

\caption{\label{fig-flowchart}Flow chart of BHET study participation at
the participant, household, and village levels across study years.
Participation (number of units) in each study year is shown in the white
boxes. Additions (+) and losses (-) to the study sample between years
are indicated by the light blue arrows. Data collection was limited to
household- and village-level environmental measurements due to the
COVID-19 pandemic in year 3. We visited 530 households in 41 villages
before travel restrictions limited further data collection. This affects
the additions and losses to the study sample reported from years 2 to 3
and years 3 to 4.}

\end{figure}

Figure~\ref{fig-flowchart} shows the participation across waves of data
collection.

\hypertarget{description-of-study-sample}{%
\subsection{Description of study
sample}\label{description-of-study-sample}}

\hypertarget{tbl-table1}{}
\begin{table}
\caption{\label{tbl-table1}Descriptive statistics for selected demographic, health, and
environmental measures at baseline, by treatment status }\tabularnewline

\centering\centering
\fontsize{9}{11}\selectfont
\begin{tabular}[t]{rrrrrrr}
\toprule
\multicolumn{1}{c}{ } & \multicolumn{2}{c}{Never treated (N=603)} & \multicolumn{2}{c}{Ever treated (N=400)} & \multicolumn{2}{c}{ } \\
\cmidrule(l{3pt}r{3pt}){2-3} \cmidrule(l{3pt}r{3pt}){4-5}
  & Mean & Std. Dev. & Mean & Std. Dev. & Diff. in Means & Std. Error\\
\midrule
\textbf{Demographics:} & \textbf{} & \textbf{} & \textbf{} & \textbf{} & \textbf{} & \textbf{}\\
Age (years) & 59.9 & 9.4 & 60.4 & 9.2 & 0.5 & 0.6\\
Female (\%) & 59.5 & 49.1 & 59.1 & 49.2 & -0.4 & 3.2\\
No education (\%) & 11.5 & 31.9 & 12.3 & 32.9 & 0.9 & 2.1\\
Primary education (\%) & 75.5 & 43.0 & 77.6 & 41.7 & 2.1 & 2.8\\
Secondary+ education (\%) & 12.6 & 33.2 & 9.8 & 29.7 & -2.9 & 2.0\\
\textbf{Health measures:} & \textbf{} & \textbf{} & \textbf{} & \textbf{} & \textbf{} & \textbf{}\\
Never smoker (\%) & 61.9 & 48.6 & 59.5 & 49.1 & -2.4 & 3.2\\
Former smoker (\%) & 11.9 & 32.4 & 15.1 & 35.8 & 3.2 & 2.2\\
Current smoker (\%) & 26.2 & 44.0 & 25.4 & 43.6 & -0.8 & 2.8\\
Never drinker (\%) & 55.9 & 49.7 & 52.5 & 50.0 & -3.4 & 3.2\\
Occasional drinker (\%) & 26.0 & 43.9 & 25.5 & 43.6 & -0.5 & 2.8\\
Daily drinker (\%) & 17.8 & 38.3 & 21.9 & 41.4 & 4.1 & 2.6\\
Systolic (mmHg) & 131.4 & 16.8 & 128.7 & 14.3 & -2.7 & 1.0\\
Diastolic (mmHg) & 82.7 & 11.6 & 82.1 & 11.3 & -0.6 & 0.8\\
Waist circumference (cm) & 87.7 & 10.5 & 85.4 & 9.5 & -2.3 & 0.8\\
Body mass index (kg/m2) & 26.3 & 3.7 & 25.8 & 3.6 & -0.5 & 0.3\\
Frequency of coughing (\%) & 18.7 & 39.0 & 19.7 & 39.8 & 1.0 & 2.6\\
Frequency of wheezing (\%) & 6.2 & 24.2 & 6.6 & 24.8 & 0.3 & 1.6\\
Shortness of breath (\%) & 29.2 & 45.5 & 34.3 & 47.5 & 5.1 & 3.0\\
Chest trouble (\%) & 11.6 & 32.0 & 14.1 & 34.9 & 2.5 & 2.2\\
Any respiratory problem (\%) & 50.6 & 50.0 & 54.3 & 49.9 & 3.7 & 3.2\\
\textbf{Environmental measures:} & \textbf{} & \textbf{} & \textbf{} & \textbf{} & \textbf{} & \textbf{}\\
Temperature (°C) & 13.8 & 3.6 & 13.5 & 3.3 & -0.3 & 0.2\\
Personal PM2.5 (ug/m3) & 150.2 & 300.3 & 103.8 & 107.3 & -46.3 & 19.1\\
\bottomrule
\multicolumn{7}{l}{\rule{0pt}{1em}Includes all individuals sampled at each of 3 waves.}\\
\end{tabular}
\end{table}

Table~\ref{tbl-table1} shows the distribution of selected demographic,
health, and environmental characteristics from the baseline survey,
prior to any villages being enrolled in the ban. We provide means and
standard deviations separately for villages that eventually enter into
the ban with those that never do so. As noted above, although our DiD
identification strategy allows for fixed differences between treated and
untreated villages, overall the differences at baseline are generally
small and the groups seem well balanced on most measures, with the
exception of personal PM\textsubscript{2.5} exposure, which was lower in
villages that were eventually treated.

\hypertarget{summary-of-pm-and-bc-measurements}{%
\subsection{Summary of PM and BC
measurements}\label{summary-of-pm-and-bc-measurements}}

At baseline, fine particulate matter (PM\textsubscript{2.5}) and black
carbon (BC) concentrations were highest, on average, for personal
measurements compared to indoor and outdoor measurements, with indoor
levels being higher than outdoor levels (Figure~\ref{fig-pm-baseline}).
This trend (personal \textgreater{} indoor \textgreater{} outdoor) was
observed among households in treated and control villages. Personal,
indoor, and outdoor geometric mean (95\% confidence interval)
concentrations of PM\textsubscript{2.5} were 72 (65,80), 45 (39,53), and
31 (28,35), respectively, and elevated relative to health-based
guidelines. The current World Health Organization (WHO) guidelines state
that annual average concentrations of PM2.5 should not exceed 5 µg/m3,
while 24-hour average exposures should not exceed 15 µg/m3 more than 3 -
4 days per year. Interim targets have been set to support the planning
of incremental milestones toward cleaner air, particularly for cities,
regions and countries that are struggling with high air pollution
levels. For PM2.5 these are: 35 µg/m3 annual mean, 75 µg/m3 24-hour
mean; 25 µg/m3 annual mean, 50 µg/m3 24-hour mean; 15 µg/m3 annual mean,
37.5 µg/m3 24-hour mean; 10 µg/m3 annual mean, 25 µg/m3 24-hour mean.

\begin{figure}[H]

{\centering \includegraphics[width=0.8\textwidth,height=\textheight]{images/fig-pm-baseline.png}

}

\caption{\label{fig-pm-baseline}Geometric mean and 95\% Confidence
Intervals for filter-based air pollutant concentrations at baseline.}

\end{figure}

\hypertarget{policy-uptake}{%
\subsection{Policy uptake}\label{policy-uptake}}

Each year of the study, participants reported what energy types they
used for space heating in winter. Based on these data, heating energy
types were classified into four categories: exclusive use of a heat pump
(`heat pump exclusively'), use of a heat pump and a kang (`heat pump
with kang'), use of solid fuel with use of electric heating devices that
were not heat pumps (`solid fuel with electricity (not heat pump)'), and
exclusive use of solid fuel. In villages treated under the policy,
Figure~\ref{fig-sankey} shows meaningful transitions from solid fuel to
heat pumps was observed. For example, in villages treated in S2, over
90\% of households used heat pumps in S2, increasing to 96\% in S4,
while only 3\% used heat pumps in S1. Conversely, the use of solid fuel
decreased from 97\% in S1 to 8\% in S2 and 3\% in S4. Villages treated
in S4 initially relied heavily on solid fuel, with approximately 90\% of
households reporting use of solid fuels in S1 and S2. However, after
treatment, 94\% of households in these same villages reported using heat
pumps in S4.

In untreated villages, an active transition from solid fuel to clean
energy was also observed. The use of heat pumps increased gradually from
5\% in S1 to 10\% in S2 and 25\% in S4. The percentage of households
using solid fuel with electric devices remained relatively stable,
ranging from 64\% to 70\%. However, the exclusive use of solid fuel
decreased from 30\% in S1 to 7\% in S4. We also observed a substantial
decline in self-reported coal use when villages entered into the CBHP
policy (Appendix Figure~\ref{fig-afig-coal})

\begin{figure}[H]

{\centering \includegraphics[width=1\textwidth,height=\textheight]{images/sankey.png}

}

\caption{\label{fig-sankey}Transitions to different energy sources
across study seasons}

\end{figure}

\hypertarget{aim-1-policy-impacts-and-potential-mediation}{%
\subsection{Aim 1: Policy impacts and potential
mediation}\label{aim-1-policy-impacts-and-potential-mediation}}

\hypertarget{impact-of-policy-on-pm-mass}{%
\subsubsection{Impact of policy on PM
mass}\label{impact-of-policy-on-pm-mass}}

In estimating the treatment effect on indoor and outdoor air pollution,
we evaluated both 24-h mean values (specifically, the same 24-h period
when personal exposure samples were collected in each village) and
seasonal mean values (with `season' defined from Jan.~15th to Mar.~15th)
of PM2.5 data collected in each village. For estimating the treatment
effect on personal exposure to PM2.5 and black carbon (BC), the results
from the filter-based measurements that were collected for a 24-h period
were used for analysis. We estimated the basic ETWFE models, as well as
ETWFE models further adjusted for covariates, including temperature,
relative humidity, wind speed, boundary layer height, wind direction,
and the mean quantity of wood burned in each village).

\begin{figure}[H]

{\centering \includegraphics[width=0.4\textwidth,height=\textheight]{images/did-pm.png}

}

\caption{\label{fig-did-pm}Treatment effect on outdoor and indoor PM2.5,
as well as personal exposure to PM2.5 and black carbon. Outdoor and
indoor PM2.5 were derived from sensor measurements after being adjusted
based on co-located gravimetric PM2.5 measurements. `Daily' indicates
the mean PM2.5 concentrations during the 24 hours when personal exposure
samples were collected in each village. `Seasonal' indicates the
seasonal mean PM2.5 concentrations in each village, from Jan.~15th to
Mar.~15th. `Did' represents the DiD analysis without any covariates,
while `Adjusted DiD' represents DiD analysis with covariates.}

\end{figure}

Treatment was associated with reductions in both seasonal and 24-h
indoor PM2.5 means (Figure~\ref{fig-did-pm}). On average, treatment was
associated with a decrease in 24-h indoor PM2.5 of 38 {[}75, 1{]} μg/m3.
After adjusting for covariates such as outdoor temperature, dewpoint,
household smoking status, and the number of residents in each household,
the treatment effect decreased to 31 {[}64, -2{]} μg/m3. The treatment
effect on seasonal indoor PM2.5 (39 {[}55, 23{]} μg/m3) remained
consistent even after covariate adjustment. This finding likely reflects
the direct benefit of the policy in replacing coal stoves, thereby
improving indoor air quality.

Overall we found little evidence of an impact of the CBHP policy on 24-h
and seasonal outdoor (local community-level) PM\textsubscript{2.5}, as
well as limited evidence of any impact on personal PM2.5 and BC.
Treatment was associated with lower, but statistically imprecise,
personal 24-h BC exposures. This finding would be consistent with the
expectation that the policy contributed to reducing air pollutant
emissions from solid fuel burning, as BC serves as a potential indicator
of such combustion, particularly in our study settings.

\hypertarget{impact-of-policy-on-health-outcomes}{%
\subsection{Impact of policy on health
outcomes}\label{impact-of-policy-on-health-outcomes}}

\begin{figure}[H]

{\centering \includegraphics[width=0.6\textwidth,height=\textheight]{images/did-bp.png}

}

\caption{\label{fig-did-bp}Overall impacts of the `coal-to-clean energy'
policy on BP, pulse pressure, and BP amplification}

\end{figure}

Figure~\ref{fig-did-bp} shows the impacts of the policy in basic ETWFE
models and models further adjusted for age, sex, waist circumference,
smoking, alcohol consumption, and use of blood pressure medication.
Overall exposure to the CBHP policy demonstrated reductions in blood
pressure of approximately 1.5mmHg on systolic and diastolic BP, but we
found little evidence of a meaningful impact on pulse pressure or BP
amplification.

Table~\ref{tbl-table_resp} shows the impacts on respiratory outcomes. In
both basic and covariate-adjusted ETWFE models we found that exposure to
the CBHP policy reduced self-reports of any poor respiratory symptoms by
around 7 percentage points. This was largely through reductions in
reports of having chest trouble or difficulty breathing several or most
days of the week.

\hypertarget{tbl-table_resp}{}
\begin{table}[H]
\caption{\label{tbl-table_resp}Marginal effect of policy on self-reported respiratory symptoms from
basic and adjusted DiD models }\tabularnewline

\centering
\begin{tabular}{lrrrrrr}
\toprule
\multicolumn{1}{c}{ } & \multicolumn{3}{c}{Basic DiD} & \multicolumn{3}{c}{Adjusted DiD} \\
\cmidrule(l{3pt}r{3pt}){2-4} \cmidrule(l{3pt}r{3pt}){5-7}
Frequency of: & Estimate & LL & UL & Estimate & LL & UL\\
\midrule
Any respiratory symptom & -0.07 & -0.14 & -0.01 & -0.08 & -0.15 & -0.01\\
Coughing & -0.02 & -0.06 & 0.03 & -0.02 & -0.07 & 0.03\\
Phlegm & -0.01 & -0.06 & 0.03 & -0.02 & -0.06 & 0.03\\
Wheezing attacks & 0.00 & -0.04 & 0.04 & 0.00 & -0.04 & 0.04\\
Trouble breathing & -0.05 & -0.12 & 0.02 & -0.05 & -0.12 & 0.02\\
\addlinespace
Chest trouble & -0.07 & -0.13 & -0.01 & -0.06 & -0.12 & -0.01\\
\bottomrule
\end{tabular}
\end{table}

\hypertarget{mediated-impact-on-blood-pressure}{%
\subsection{Mediated impact on blood
pressure}\label{mediated-impact-on-blood-pressure}}

As noted above, we aimed to assess whether any health impacts of the
CBHP policy may work specifically through pathways involving changes in
PM\textsubscript{2.5} and indoor temperature. Below we show in results
from several mediation models. We evaluated potential mediation for each
mediator separately and in a single model accounting for multiple
mediators, and we set the values of both mediators to the WHO mean
annual interim PM\textsubscript{2.5} and indoor temperature guidelines.
Given that we did not find evidence of any policy impact on BP outcomes
other than systolic and diastolic BP, we did not undertake mediation
analysis for other outcomes. In Table~\ref{tbl-bp-med} we show that
conditioning on indoor PM and indoor temperature largely explains the
entire total effect of the CBHP policy on blood pressure for systolic
BP, and roughly half of the total effect for diastolic BP.

\hypertarget{tbl-bp-med}{}
\begin{table}[H]
\caption{\label{tbl-bp-med}Controlled direct effects for the CBHP policy }\tabularnewline

\centering
\begin{tabular}{lllll}
\toprule
\multicolumn{2}{c}{ } & \multicolumn{3}{c}{CDE Mediated By:} \\
\cmidrule(l{3pt}r{3pt}){3-5}
Outcome & Adjusted Total Effect & Indoor PM & Indoor Temp & PM + Temp\\
\midrule
Brachial SBP & -1.4 (-3.31, 0.51) & -0.93 (-3.05, 1.2) & -0.43 (-2.35, 1.49) & 0.13 (-2.02, 2.28)\\
Central SBP & -1.56 (-3.4, 0.28) & -1.02 (-3.13, 1.08) & -0.54 (-2.39, 1.3) & 0.07 (-2.05, 2.19)\\
Brachial DBP & -1.6 (-2.96, -0.25) & -1.3 (-2.81, 0.21) & -1.02 (-2.45, 0.42) & -0.65 (-2.25, 0.94)\\
Central DBP & -1.66 (-2.97, -0.34) & -1.31 (-2.81, 0.19) & -1.08 (-2.48, 0.32) & -0.68 (-2.27, 0.9)\\
\bottomrule
\end{tabular}
\end{table}

\hypertarget{aim-2-source-contributions}{%
\subsection{Aim 2: Source
contributions}\label{aim-2-source-contributions}}

The model diagnostics for the 3 to 6-factor PMF solutions are given in
Figure~\ref{fig-source-table}. Model fit was assessed using Q/Qexp (how
our model fit divided by the expected fit). As the change in Q/Qexp
decreases as more factors are added, the model may be fitting additional
sources that do not improve the overall fit. The largest change in
Q/Qexp was from 3 to 4 sources (6.24 to 5.37) while the changes moving
from 4 to 5 and 5 to 6 are similar which suggests that 4 factors is
sufficient to explain the variation in our data. We assessed the random
error in our model by randomly sampling blocks of data, fitting new
models with the blocks, and comparing how the source profiles compared
to that of the original model (BS mapping). The 3 and 4 factor solutions
had high BS mapping (all factors found in \textgreater{} 96.5\% of
bootstrap runs). The additional sources identified in the 5 (lead) and 6
(chloride) factor solutions have low bootstrap mapping (\textgreater{}
72\%) which means those solutions are not as consistent as the 3 and 4
factor solutions. The possibility that multiple, different, solutions
could result in the same Q value was assessed using displacement. The
displacement approach takes the original factor profiles and modifies
the values for each species up or down to maintain a small change in Q,
reruns the solution with the new species values, then compares the
profiles of the new model to the original. Any swaps indicate that small
changes in the species values could result in factor profiles that look
different from the original solution, and that the original solution is
unstable. None of the factors in any of the solutions discussed were
swapped during displacement which indicates that all of the potential
solutions are stable. Based on the Q/Qexp, BS mapping, and
interpretability of the factors, the 4 factor solution was chosen for
this dataset.

\begin{figure}[H]

{\centering \includegraphics[width=0.8\textwidth,height=\textheight]{images/source-table.png}

}

\caption{\label{fig-source-table}PMF error estimation diagnostics.}

\end{figure}

The source profiles for the four-factor solution are presented in
Figure~\ref{fig-source-figure}. The first source was identified as dust
by high percentages of crustal elements like wi-Ca, Si, and wi-Mg. The
second source was constituted of non-sulfate sulfur as well as secondary
inorganic ions (ammonium, nitrate, and sulfate). Non-sulfate sulfur is a
tracer for primary coal combustion, while secondary inorganic ions
indicate a secondary source. Since coal combustion is a major source of
energy in our study area, it is likely that the second source is a
mixture of primary and secondary emissions that originate from coal and
other sulfurous fuel combustion. Additionally, the mean source
contribution of the second source is higher in outdoor than personal
exposure measurements. Secondary formation occurs outdoors in the
presence of sunlight, so higher outdoor concentrations compared to
personal exposure further support our naming the second source and
sulfur secondary. The third source had high percentages of ws-Ca nd Al,
which in our study region, has been found to be indicative of
transported dust from dust storms that can occur in the spring. While
our samples were collected during winter months only, it is possible
that transported dust from previous years still remained. The fourth
source was characterized by high percentages of tracers for both coal
(OC, wi-K, chloride, Pb) and biomass combustion (EC, ws-K). Coal and
biomass combustion is common in our study setting so this source is
likely a mixture of the two combustion sources.

\begin{figure}[H]

{\centering \includegraphics[width=0.8\textwidth,height=\textheight]{images/source-figure.png}

}

\caption{\label{fig-source-figure}Source profiles for the 4-factor PMF
solution to the sum of elements, ions, elemental carbon, and organic
carbon for outdoor and personal PM2.5 exposure measurements. The lines
separate the major contributing species to each source}

\end{figure}

We extend the source profiles across the different treatment cohorts in
Figure~\ref{fig-source-season}.

\begin{figure}[H]

{\centering \includegraphics[width=0.8\textwidth,height=\textheight]{images/source-season.png}

}

\caption{\label{fig-source-season}Arithmetic mean dispersion normalized
source contributions found from the 4-factor PMF solution for A outdoor
and B personal PM2.5 exposure samples by year the group received
treatment.}

\end{figure}

\hypertarget{aim-3}{%
\subsection{Aim 3}\label{aim-3}}

\begin{itemize}
\tightlist
\item
  Table of mediated health effects by source contribution (coal and
  biomass)
\end{itemize}

\hypertarget{discussion-and-conclusions}{%
\section{Discussion and Conclusions}\label{discussion-and-conclusions}}

\begin{itemize}
\tightlist
\item
  Generally describing high take-up of the policy
\item
  Reductions in indoor PM\textsubscript{2.5} but not personal or outdoor
\item
  Reduction in blood pressure and self-reported respiratory symptoms
\end{itemize}

Other relevant results (Tables or figures in SI)

Policy impacts on other relevant outcomes:

\begin{itemize}
\tightlist
\item
  Temperature
\item
  Heating room
\item
  Well-being
\end{itemize}

\hypertarget{implications-of-findings}{%
\section{Implications of Findings}\label{implications-of-findings}}

\hypertarget{data-availability-statement}{%
\section{Data Availability
Statement}\label{data-availability-statement}}

\begin{itemize}
\tightlist
\item
  Description of datasets and code available on our project page at the
  Open Science Foundation
\end{itemize}

\hypertarget{acknowledgements}{%
\section{Acknowledgements}\label{acknowledgements}}

To come\ldots{}

\hypertarget{references}{%
\section{References}\label{references}}

\hypertarget{refs}{}
\begin{CSLReferences}{1}{0}
\leavevmode\vadjust pre{\hypertarget{ref-ahmed2009}{}}%
Ahmed T, Dutkiewicz VA, Shareef A, Tuncel G, Tuncel S, Husain L. 2009.
Measurement of black carbon ({BC}) by an optical method and a
thermal-optical method: {Intercomparison} for four sites. Atmospheric
Environment 43:6305--6311;
doi:\href{https://doi.org/10.1016/j.atmosenv.2009.09.031}{10.1016/j.atmosenv.2009.09.031}.

\leavevmode\vadjust pre{\hypertarget{ref-callaway2020}{}}%
Callaway B. 2020.
\href{https://doi.org/10.1007/978-3-319-57365-6_352-1}{Difference-in-{Differences}
for {Policy Evaluation}}. In: \emph{Handbook of {Labor}, {Human
Resources} and {Population Economics}} (K.F. Zimmermann, ed). Springer
International Publishing:Cham. 1--61.

\leavevmode\vadjust pre{\hypertarget{ref-callaway2021}{}}%
Callaway B, Sant'Anna PHC. 2021. Difference-in-{Differences} with
multiple time periods. Journal of Econometrics 225:200--230;
doi:\href{https://doi.org/10.1016/j.jeconom.2020.12.001}{10.1016/j.jeconom.2020.12.001}.

\leavevmode\vadjust pre{\hypertarget{ref-card1994}{}}%
Card D, Krueger AB. 1994. Minimum {Wages} and {Employment}: {A Case
Study} of the {Fast-Food Industry} in {New Jersey} and {Pennsylvania}.
American Economic Review 84: 772--93.

\leavevmode\vadjust pre{\hypertarget{ref-costello2015}{}}%
Costello BT, Schultz MG, Black JA, Sharman JE. 2015. Evaluation of a
{Brachial Cuff} and {Suprasystolic Waveform Algorithm Method} to
{Noninvasively Derive Central Blood Pressure}. American Journal of
Hypertension 28:480--486;
doi:\href{https://doi.org/10.1093/ajh/hpu163}{10.1093/ajh/hpu163}.

\leavevmode\vadjust pre{\hypertarget{ref-cdcgr2023}{}}%
Dispersed Coal Management Research Group
北京大学能源研究院气候变化与能源转型项目. 2023. 中国散煤综合治理研究报告
{China Dispersed Coal Governance Report}.

\leavevmode\vadjust pre{\hypertarget{ref-dockery2013}{}}%
Dockery DW, Rich DQ, Goodman PG, Clancy L, Ohman-Strickland P, George P,
et al. 2013. \href{https://www.ncbi.nlm.nih.gov/pubmed/24024358}{Effect
of air pollution control on mortality and hospital admissions in
{Ireland}}. Research Report (Health Effects Institute) 3--109.

\leavevmode\vadjust pre{\hypertarget{ref-dominici2014}{}}%
Dominici F, Greenstone M, Sunstein CR. 2014. Science and regulation.
{Particulate} matter matters. Science (New York, NY) 344:257--9;
doi:\href{https://doi.org/10.1126/science.1247348}{10.1126/science.1247348}.

\leavevmode\vadjust pre{\hypertarget{ref-ezzati2017}{}}%
Ezzati M, Baumgartner JC. 2017. Household energy and health: Where next
for research and practice? Lancet (London, England) 389:130--132;
doi:\href{https://doi.org/10.1016/S0140-6736(16)32506-5}{10.1016/S0140-6736(16)32506-5}.

\leavevmode\vadjust pre{\hypertarget{ref-fda2018}{}}%
Food and Drug Administration. 2018. Bioanalytical {Method Validation
Guidance} for {Industry}.

\leavevmode\vadjust pre{\hypertarget{ref-goin2023}{}}%
Goin DE, Riddell CA. 2023. Comparing {Two-way Fixed Effects} and {New
Estimators} for {Difference-in-Differences}: {A Simulation Study} and
{Empirical Example}. Epidemiology 34:535;
doi:\href{https://doi.org/10.1097/EDE.0000000000001611}{10.1097/EDE.0000000000001611}.

\leavevmode\vadjust pre{\hypertarget{ref-goodman-bacon2021}{}}%
Goodman-Bacon A. 2021. Difference-in-differences with variation in
treatment timing. Journal of Econometrics 225:254--277;
doi:\href{https://doi.org/10.1016/j.jeconom.2021.03.014}{10.1016/j.jeconom.2021.03.014}.

\leavevmode\vadjust pre{\hypertarget{ref-gbdmaps2016}{}}%
Group GMW. 2016. Burden of disease attributable to coal-burning and
other air pollution sources in {China}.

\leavevmode\vadjust pre{\hypertarget{ref-johnston2013}{}}%
Johnston FH, Hanigan IC, Henderson SB, Morgan GG. 2013. Evaluation of
interventions to reduce air pollution from biomass smoke on mortality in
{Launceston}, {Australia}: Retrospective analysis of daily mortality,
1994-2007. BMJ 346:e8446--e8446;
doi:\href{https://doi.org/10.1136/bmj.e8446}{10.1136/bmj.e8446}.

\leavevmode\vadjust pre{\hypertarget{ref-keele2015}{}}%
Keele L, Tingley D, Yamamoto T. 2015. Identifying mechanisms behind
policy interventions via causal mediation analysis. Journal of Policy
Analysis and Management 34: 937--963.

\leavevmode\vadjust pre{\hypertarget{ref-lai2019}{}}%
Lai. 2019. Relative contributions of household solid fuel use and
outdoor air pollution to chemical components of personal {PM2}.5
exposures. Indoor Air-international Journal of Indoor Air Quality and
Climate.

\leavevmode\vadjust pre{\hypertarget{ref-lewington2012}{}}%
Lewington S, LiMing L, Sherliker P, Yu G, Millwood I, Zheng B, et al.
2012. Seasonal variation in blood pressure and its relationship with
outdoor temperature in 10 diverse regions of {China}: The {China
Kadoorie Biobank}. Journal of hypertension 30: 1383.

\leavevmode\vadjust pre{\hypertarget{ref-lindemann2017}{}}%
Lindemann U, Stotz A, Beyer N, Oksa J, Skelton DA, Becker C, et al.
2017. Effect of indoor temperature on physical performance in older
adults during days with normal temperature and heat waves. International
journal of environmental research and public health 14;
doi:\href{https://doi.org/10.3390/ijerph14020186}{10.3390/ijerph14020186}.

\leavevmode\vadjust pre{\hypertarget{ref-lowe2009}{}}%
Lowe A, Harrison W, El-Aklouk E, Ruygrok P, Al-Jumaily AM. 2009.
Non-invasive model-based estimation of aortic pulse pressure using
suprasystolic brachial pressure waveforms. Journal of Biomechanics
42:2111--2115;
doi:\href{https://doi.org/10.1016/j.jbiomech.2009.05.029}{10.1016/j.jbiomech.2009.05.029}.

\leavevmode\vadjust pre{\hypertarget{ref-meng2023}{}}%
Meng W, Zhu L, Liang Z, Xu H, Zhang W, Li J, et al. 2023. Significant
but {Inequitable Cost-Effective Benefits} of a {Clean Heating Campaign}
in {Northern China}. Environmental Science \& Technology 57:8467--8475;
doi:\href{https://doi.org/10.1021/acs.est.2c07492}{10.1021/acs.est.2c07492}.

\leavevmode\vadjust pre{\hypertarget{ref-naimi2014}{}}%
Naimi AI, Kaufman JS, MacLehose RF. 2014. Mediation misgivings:
Ambiguous clinical and public health interpretations of natural direct
and indirect effects. International journal of epidemiology 43:1656--61;
doi:\href{https://doi.org/10.1093/ije/dyu107}{10.1093/ije/dyu107}.

\leavevmode\vadjust pre{\hypertarget{ref-niu2024}{}}%
Niu J, Chen X, Sun S. 2024. China's {Coal Ban} policy: {Clearing} skies,
challenging growth. Journal of Environmental Management 349:119420;
doi:\href{https://doi.org/10.1016/j.jenvman.2023.119420}{10.1016/j.jenvman.2023.119420}.

\leavevmode\vadjust pre{\hypertarget{ref-quansah2017}{}}%
Quansah R, Semple S, Ochieng CA, Juvekar S, Armah FA, Luginaah I, et al.
2017. Effectiveness of interventions to reduce household air pollution
and/or improve health in homes using solid fuel in low-and-middle income
countries: {A} systematic review and meta-analysis. Environment
International 103:73--90;
doi:\href{https://doi.org/10.1016/j.envint.2017.03.010}{10.1016/j.envint.2017.03.010}.

\leavevmode\vadjust pre{\hypertarget{ref-rosenthal2018}{}}%
Rosenthal J, Quinn A, Grieshop AP, Pillarisetti A, Glass RI. 2018. Clean
cooking and the {SDGs}: {Integrated} analytical approaches to guide
energy interventions for health and environment goals. Energy for
sustainable development : the journal of the International Energy
Initiative 42:152--159;
doi:\href{https://doi.org/10.1016/j.esd.2017.11.003}{10.1016/j.esd.2017.11.003}.

\leavevmode\vadjust pre{\hypertarget{ref-scott2011}{}}%
Scott AJ, Scarrott C. 2011. Impacts of residential heating intervention
measures on air quality and progress towards targets in {Christchurch}
and {Timaru}, {New Zealand}. Atmospheric Environment 45:2972--2980;
doi:\href{https://doi.org/10.1016/j.atmosenv.2010.09.008}{10.1016/j.atmosenv.2010.09.008}.

\leavevmode\vadjust pre{\hypertarget{ref-song2023}{}}%
Song C, Liu B, Cheng K, Cole MA, Dai Q, Elliott RJR, et al. 2023.
Attribution of {Air Quality Benefits} to {Clean Winter Heating Policies}
in {China}: {Combining Machine Learning} with {Causal Inference}.
Environmental Science \& Technology 57:17707--17717;
doi:\href{https://doi.org/10.1021/acs.est.2c06800}{10.1021/acs.est.2c06800}.

\leavevmode\vadjust pre{\hypertarget{ref-tan2023}{}}%
Tan X, Chen G, Chen K. 2023. Clean heating and air pollution: {Evidence}
from {Northern China}. Energy Reports 9:303--313;
doi:\href{https://doi.org/10.1016/j.egyr.2022.11.166}{10.1016/j.egyr.2022.11.166}.

\leavevmode\vadjust pre{\hypertarget{ref-thompson2019}{}}%
Thompson RJ, Li J, Weyant CL, Edwards R, Lan Q, Rothman N, et al. 2019.
Field {Emission Measurements} of {Solid Fuel Stoves} in {Yunnan}, {China
Demonstrate Dominant Causes} of {Uncertainty} in {Household Emission
Inventories}. Environmental Science \& Technology 53:3323--3330;
doi:\href{https://doi.org/10.1021/acs.est.8b07040}{10.1021/acs.est.8b07040}.

\leavevmode\vadjust pre{\hypertarget{ref-vanderweele2015}{}}%
VanderWeele TJ. 2015. \emph{Explanation in causal inference: Methods for
mediation and interaction}. Oxford University Press:New York.

\leavevmode\vadjust pre{\hypertarget{ref-volckens2017}{}}%
Volckens J, Quinn C, Leith D, Mehaffy J, Henry CS, Miller-Lionberg D.
2017. Development and evaluation of an ultrasonic personal aerosol
sampler. Indoor air 27:409--416;
doi:\href{https://doi.org/10.1111/ina.12318}{10.1111/ina.12318}.

\leavevmode\vadjust pre{\hypertarget{ref-wooldridge2021}{}}%
Wooldridge JM. 2021. Two-{Way Fixed Effects}, the {Two-Way Mundlak
Regression}, and {Difference-in-Differences Estimators}.;
doi:\href{https://doi.org/10.2139/ssrn.3906345}{10.2139/ssrn.3906345}.

\leavevmode\vadjust pre{\hypertarget{ref-yu2021}{}}%
Yu C, Kang J, Teng J, Long H, Fu Y. 2021. Does coal-to-gas policy reduce
air pollution? {Evidence} from a quasi-natural experiment in {China}.
Science of The Total Environment 773:144645;
doi:\href{https://doi.org/10.1016/j.scitotenv.2020.144645}{10.1016/j.scitotenv.2020.144645}.

\leavevmode\vadjust pre{\hypertarget{ref-zigler2016}{}}%
Zigler CM, Kim C, Choirat C, Hansen JB, Wang Y, Hund L, et al. 2016.
\emph{Causal inference methods for estimating long-term health effects
of air quality regulations. {Research} report 187.} Health Effects
Institute / Health Effects Institute:Boston, MA.

\end{CSLReferences}

\newpage
\appendix
\renewcommand{\thefigure}{A\arabic{figure}}
\renewcommand{\thetable}{A\arabic{table}}
\setcounter{figure}{0}
\setcounter{table}{0}

\hypertarget{appendices}{%
\section*{Appendices}\label{appendices}}
\addcontentsline{toc}{section}{Appendices}

\begin{figure}[H]

{\centering \includegraphics[width=1\textwidth,height=\textheight]{images/coal-plot.png}

}

\caption{\label{fig-afig-coal}Trends in self-reported coal and biomass,
by treatment season}

\end{figure}

\hypertarget{about-the-authors}{%
\section*{About the authors}\label{about-the-authors}}
\addcontentsline{toc}{section}{About the authors}

\hypertarget{other-publications}{%
\section*{Other publications}\label{other-publications}}
\addcontentsline{toc}{section}{Other publications}



\end{document}
